% !TEX root = creche.tex
\chapter{Some relational structures}
\label{relational}

\def\ceq#1#2#3{\parbox[b]{20ex}{$\displaystyle #1$}\parbox[b]{4ex}{\hfil$#2$}$\displaystyle #3$}

In the first section we prove that theory of dense linear orders without endpoints is $\omega$-cate\-gorical.
That is, any two such countable orders are isomorphic.
This is an easy classical result of Cantor.
In this chapter we examine Cantor's construction (a so-called \textit{back-and-forth\/} construction) in great detail.
In the second section we apply the same technique to prove that the theory of the random graph\ is $\omega$-categorical.

%%%%%%%%%%%%%%%%%%%%%%%%%%%%%%%%%%
%%%%%%%%%%%%%%%%%%%%%%%%%%%%%%%%%%
%%%%%%%%%%%%%%%%%%%%%%%%%%%%%%%%%%
%%%%%%%%%%%%%%%%%%%%%%%%%%%%%%%%%%
%%%%%%%%%%%%%%%%%%%%%%%%%%%%%%%%%%
%%%%%%%%%%%%%%%%%%%%%%%%%%%%%%%%%%
\section{Dense linear orders}
\label{dlo}
The \emph{language of strict orders,} which in this section we denote by $L$, contains only a binary relation symbol $<$.
A structure $M$ of signature $L$ is a \emph{strict order\/} if it models (the universal closure of) the following formulas
\begin{itemize}
\item[1.] $x \not< x$\hfill irreflexive;
\item[2.] $x< z< y\ \imp\ x< y$\hfill transitive.
\end{itemize}
Note that the following is an immediate consequence of \ssf{1} and \ssf{2}.
\begin{itemize}
\item[] $x<y\ \imp\ y\not<x$\hfill antisymmetric.
\end{itemize}
We say that the order is \emph{total\/} or \emph{linear\/} if
\begin{itemize}
\item[li.] $x< y\ \vee\ y< x\ \vee\ x=y$\hfill linear or total.
\end{itemize}
An order is \emph{dense\/} if
\begin{itemize}
\item[nt.]  $\E x,y\ (x<y)$\hfill non trivial
\item[d.] $x<y\ \imp\ \E z\;(x<z<y)$\hfill dense.
\end{itemize}
We need to require the existence of two comparable elements, then \ssf{d} implies that dense orders are in fact infinite.
We say that the ordering has no \emph{endpoints\/} if
\begin{itemize}
\item[e.] $\E y\ (x<y)\ \wedge\ \E y\ (y<x)$\hfill without endpoints.
\end{itemize}

We denote by \emph{$T_{\textrm{lo}}$} the theory strict linear orders and by \emph{$T_{\textrm{dlo}}$} the theory of dense linear orders without endpoints.
Clearly, these are consistent theories: $\QQ$ with the usual ordering is a model of $T_{\rm dlo}$.

We introduce some notation to improve readability of the proof of the following theorem.
Let $A$ and $B$ be subsets of an ordered set.
We write $A<B$ if $a<b$ for every $a\in A$ and $b\in B$.
We write $a<B$ e $A<b$ for $\{a\}<B$, respectively $A<\{b\}$.
Let $M\models T_{\rm lo}$.
Then $M\models T_{\rm dlo}$ if and only if for every finite $A, B\subseteq M$ such that $A<B$ there is a $c$ such that  $A<c<B$.
In fact axiom \ssf{d} is evident and axioms \ssf{nt} and \ssf{e} are obtained taking replacing $A$ and/or $B$ by the empty set.

Now we prove the first of a series of lemmas that we call \emph{extension lemmas.}
Recall that in the language of strict orders an injective map $k:M\to N$ is a partial isomorphism if 

\ceq{\hfill M\models a<b}{\IFF}{N\models ka<kb}\hfill for every $a,b\in\dom k$.

When $M,N\models T_{\rm lo}$ the direction $\IMP$ suffices.

\begin{lemma}\label{lem_ordinericco}
Fix $M\models T_{\rm lo}$ and $N\models T_{\rm dlo}$.
Let $k:M\imp N$ be a finite partial isomorphism and let $b\in M$.
Then there is a partial isomorphism $h:M\to N$ that extends $k$ and is defined in $b$.
\end{lemma}

\begin{proof}
Given a finite partial isomorphism $k:M\to N$ define

\ceq{\hfill A^-}{=}{\big\{a\in\dom(k)\;:\;a<b\big\};}

\ceq{\hfill A^+}{=}{\big\{a\in\dom(k)\;:\;b<a\big\}.}

The sets $A^-$ and $A^+$ are finite and partition $\dom k$, and  $A^-<A^+$.
As $k:M\imp N$ is a partial isomorphism, $k[A^-]<k[A^+]$.
Then in $N$ there is an element $c$ such that $k[A^-]<c<k[A^+]$.
It is easy to check that setting $h=k\cup\big\{\<b,c\>\big\}$ gives the required extension.
\end{proof}

The following is an equivalent version of Lemma~\ref{lem_ordinericco}.

\begin{corollary}\label{coroll_ordinericco}
Let $M\models T_{\rm lo}$ be countable and let $N\models T_{\rm dlo}$.
Let $k:M\imp N$ be a finite partial isomorphism.
Then there is a (total) embedding $h:M\hookrightarrow N$ that extends $k$.
\end{corollary}

\begin{proof}
Let $\<a_i:i<\omega\>$ be an enumeration of $M$.
Define by induction a chain of finite partial isomorphisms $h_i:M\imp N$ such that $a_i\in\dom h_{i+1}$.
The construction starts with $h_0=k$.
At stage $i+1$ we chose any finite partial isomorphism $h_{i+1}:M\imp N$ that extends $h_i$ and is defined in $a_i$.
This is possible by Lemma~\ref{lem_ordinericco}.
In the end we set

\ceq{\hfill h}{=}{\bigcup_{i\in\omega} h_i.}

It is immediate to verify that $h:M\hookrightarrow N$ is the required embedding.
\end{proof}



%We say that a theory is \emph{$\omega$-categorical\/} if it has a unique countable model up to isomorphism.

We are now ready to prove that any two countable models of $T_{\rm dlo}$ are isomorphic which is a classical result of Cantor's.
Actually what we prove is slightly more general than that.
In fact Cantor's theorem is obtained from the theorem below by setting $k=\0$, which we are allowed to, because all models have the same empty characteristic (cfr.\@ Remark~\ref{rmk_characteristic}).

\begin{theorem}\label{thm_zigzagcantor}
Every finite partial isomorphism $k:M\imp N$ between countable models of $T_{\rm dlo}$ extends to an isomorphism $g:M\isomap N$.
\end{theorem}

The following is the archetypal \emph{back-and-forth\/} construction.
It is important to note that it does not mention linear orders at all.
It only uses the extension Lemma~\ref{lem_ordinericco}.
The same construction can be applied in many other contexts where an extension lemma holds (cfr.\@ Theorem~\ref{thm_riccozigzag}).

\vspace*{-\parskip}
\begin{proof}
Let $\<a_i:i<\omega\>$ and $\<b_i:i<\omega\>$ be enumerations of $M$ and $N$ respectively.
We define by induction a chain of finite partial isomorphisms $g_i:M\imp N$ such that $a_i\in\dom g_{i+1}$ and $b_i\in\range g_{i+1}$.
In the end we set

\ceq{\hfill g}{=}{\bigcup_{i\in\omega}g_i}

We begin by letting $g_0=k$.
The inductive step consists of two half-steps that we call the \textit{forth step} and \textit{back step}.
In the forth step we define $g_{i+^1\!\!/\!_2}$ such that $a_i\in\dom g_{i+^1\!\!/\!_2}$.
In the back step to define $g_{i+1}$ such that $b_i\in\range g_{i+1}$.

By the extension lemma~\ref{lem_ordinericco} there is a finite partial isomorphism $g_{i+^1\!\!/\!_2}:M\imp N$ that extends $g_i$ and is defined in $a_i$.
Now apply the same lemma to extend $\big(g_{i+^1\!\!/\!_2}\big)^{-1}:N\to M$ to a finite partial isomorphism $(g_{i+1})^{-1}:N\to M$ defined in $b_i$.
\end{proof}

Let $\lambda$ be an infinite cardinal.
We say that a theory is \emph{$\lambda$-categorical\/} if any two models of $T$ of cardinality $\lambda$ are isomorphic.
From Theorem~\ref{thm_zigzagcantor}, taking $k=\0$, we obtain the following.

\begin{corollary}\label{corol_DLO_omegacat}
The theory $T_{\rm dlo}$ is $\omega$-categorical.
\end{corollary}

We also obtain that $T_{\rm dlo}$ is a complete theory.
This is consequence of the following general fact.

\begin{proposition}\label{thm_categorical->complete}
If $T$ has no finite models and is $\lambda$-categorical for some $\lambda\ge|L|$, then $T$ is complete.
\end{proposition}

\begin{proof}
Let $M$ and $N$ be any two models of $T$.
Applying the upward and/or downward L\"owenheim-Skolem theorem, we may assume they both have cardinality $\lambda$ (here we use that $M$ and $N$ are both infinite and that $\lambda\ge|L|$).
Hence $M\simeq N$ and in particular $M\equiv N$.
\end{proof}

\begin{exercise}\label{ex_ricco->dlo}
    Prove that the extension Lemma~\ref{lem_ordinericco} characterizes models of $T_{\rm dlo}$ among models of $T_{\rm lo}$.
    That is, if $N$ is a model of $T_{\rm lo}$ such that the conclusion of Lemma~\ref{lem_ordinericco} holds, then $M\models  T_{\rm dlo}$.
\end{exercise}

\begin{exercise}
Prove that $T_{\rm dlo}$ is not $\lambda$-categorical for any uncountable $\lambda$.
\end{exercise}

\begin{exercise}
Prove that, in the language of strict orders, $\QQ\preceq\RR$.
\end{exercise}

\begin{exercise}\label{VaughtEsempio3modelli}
Let $L$ be the language of strict orders expanded with countably many constants $\big\{c_i: i\in\omega\big\}$.
Let $T$ be the theory that extends $T_{\rm dlo}$ by the axioms $c_i<c_{i+1}$ for all $i$.
Prove that $T$ is complete.
Find three non isomorphic countable models of this theory.
For a suitably chosen model $N$ of $T$, prove the statement in Lemma~\ref{lem_ordinericco}, where $M$ any model of $T$.
\end{exercise}


\begin{exercise}\label{categorica->completa}
Show that in Theorem~\ref{thm_categorical->complete} the assumption $\lambda\ge|L|$ is necessary.
Hint: let $\nu$ be an uncountable cardinal.
The language contains only the ordinals $i<\nu$ as constants.
The theory $T$ says that there are infinitely many elements and either $i=0$ for every $i<\nu$, or $i\neq j$ for every $i<j<\nu$.
Prove that $T$ is incomplete.
Prove that $T$ has countable models and that these are all isomorphic.
\end{exercise}

%%%%%%%%%%%%%%%%%%%%%%%%%%%%%%%%%%
%%%%%%%%%%%%%%%%%%%%%%%%%%%%%%%%%%
%%%%%%%%%%%%%%%%%%%%%%%%%%%%%%%%%%
%%%%%%%%%%%%%%%%%%%%%%%%%%%%%%%%%%
\section{Random graphs}
\label{randomgraph}

Recall that the \emph{language of graphs}, which in this section we denote by $L$, contains only a binary relation $r$.
A \emph{graph\/} structure of signature $L$ such that

\begin{itemize}
\item[1.] $\neg r(x,x)$\hfill irreflexive
\item[2.] $r(x,y)\imp r(y,x)$\hfill symmetric.
\end{itemize}

An element of a graph $M$ is called a \emph{vertex\/} or a \emph{node}.
An \emph{edge\/} is an unordered pair of vertices $\{a,b\}\subseteq M$ such that $M\models r(a,b)$.
In words we may say that $a$ is adjacent to $b$.

A \emph{random graph\/} is a graph that also satisfies the following axioms for every $n$
\begin{itemize}
\item[nt.] $\E x,y\ \ (x\neq y)$\hfill non trivial
\item[r$_n$.] $\displaystyle\bigwedge^n_{i,j=1} x_i\neq y_j\ \ \imp\ \ \E z \bigwedge^n_{i=1} \big[r(x_i,z) \wedge \neg r(z,y_i) \wedge z\neq y_i\big]$\hfill for every $n\in\ZZ^+$.
\end{itemize}

The theory of graphs is denoted by \emph{$T_{\textrm{gph}}$} and the theory of random graphs is denoted by \emph{$T_{\textrm{rg}}$}.
The scheme of axioms \ssf{r$_n$} plays the same role as density in the previous section.
It says that given two disjoint sets $A^+$ and $A^-$ of cardinality $\le n$ there is a vertex $z$ that is adjacent to all vertices in $A^+$ and to no vertex in $A^-$.
We explicitly required that $z\notin A^-$, by \ssf{1} it is clear that $z\notin A^+$.

Strictly speaking, the axioms \ssf{r$_n$} do not mention the cases when $A^+$ or $A^-$ are empty.
But as it is evident that random graphs are infinite, we can deal with them by adding redundant elements.

The following is the analogous of Lemma~\ref{lem_ordinericco} for random graphs.
Recall that in the language of graphs a map $k:M\to N$ is a partial isomorphism if it is injective and

\ceq{\hfill M\models r(a,b)}{\IFF}{N\models r(ka,kb)}\hfill for every $a,b\in\dom k$.

\begin{lemma}\label{lem_graforicco}
Fix $M\models T_{\rm gph}$ and $N\models T_{\rm rg}$.
Let $k:M\imp N$ be a finite partial isomorphism and let $b\in M$.
Then there is a partial isomorphism $h:M\imp N$ that extends $k$ and is defined in $b$.
\end{lemma}
\begin{proof} The structure of the proof is the same as in Lemma~\ref{lem_ordinericco}, so we use the same notation.
Assume $b\notin\dom k$ and define\medskip

\hfil$A^+=\big\{x\in\dom k\ :\ M\models r(x,b)\big\}\ \ \ \textrm{e}\ \ \ A^-=\big\{y\in\dom k\ :\  M\models \neg r(y,b)\big\}$.\medskip

These two sets are finite and disjoint, then so are $k[A^+]$ and $k[A^-]$.
Then there is a $c\notin\range k$ such that\medskip

\hfil$\displaystyle\bigwedge_{a\in A^+}r(ka,c)\ \ \wedge\ \bigwedge_{a\in A^-}\neg r(ka,c)$.

As $k[A^+]\cup k[A^-]=\range k$, it is immediate to verify that $h=k\cup\big\{\<b,c\>\big\}$ is the required extension.
\end{proof}

Some readers may doubt that $T_{\rm rg}$ is consistent.

\begin{proposition}
There exists a random graph.
\end{proposition}
\begin{proof}
The domain of is the set of natural numbers.
Let $r(n,m)$ hold if the $n$-th prime number divides $m$ or, conversely, the $m$-th prime number divides $n$.
\end{proof}

The same proof as that of Corollary~\ref{coroll_ordinericco} gives the following.

\begin{corollary}\label{coroll_graforicco}
Let $M\models T_{\rm gph}$ be countable and let $N\models T_{\rm rg}$.
Let $k:M\imp N$ be a finite partial isomorphism.
Then there is a (total) embedding $h:M\hookrightarrow N$ that extends $k$.
\end{corollary}

The proof of Theorem~\ref{thm_zigzagcantor} gives the following theorem and its corollary.

\begin{theorem}\label{gaomegacat}
Every finite partial isomorphism $k:M\imp N$ between countable models of $T_{\rm rg}$ extends to an isomorphism $g:M\isomap N$.
\end{theorem}

\begin{corollary}\label{corol_RG_omegacat}
The theory $T_{\rm rg}$ is $\omega$-categorical (and therefore complete).
\end{corollary}

\begin{exercise}
Let $a,b,c\in N\models T_{\rm rg}$.
Prove that $r(a,N)=r(b,N)\cap r(c,N)$ occurs only in the trivial case $a=b=c$.
\end{exercise}

\begin{exercise}\label{ex_r(b,N)}
Let $N\models T_{\rm rg}$ prove that for every $b\in N$ the set $r(b,N)$ is a random graph.
Is every random graph $M\subseteq N$ of the form $\phi(N)$ for some $\phi(x)\in L(N)$~?
\end{exercise}

% \begin{exercise}
% Let $A\subseteq N\models  T_{\rm rg}$. What are the possible cardinalities of $\big\{b:r(b,N){=}A\big\}$~?
% \end{exercise}

\begin{exercise}\label{unionedisgiunta}
Let $N$ be free union of two random graphs $N_1$ and $N_2$.
That is, $N=N_1\sqcup N_2$ and $r^N= r^{N_1}\sqcup r^{N_2}$, where  $\sqcup$ denotes the disjoint union.
Prove that $N$ is not a random graph.
Show that $N_1$ is not definable without parameters.
Write a first order formula $\psi(x,y)$ true if $x$ and $y$ belong to the same connected component of $N$.
Axiomatize the class of graphs that are free union of two random graphs.
\end{exercise}

% \begin{exercise}\label{N1N2}
% Siano $N_1$ ed $N_2$ due grafi aleatori numerabili. Sia $N$ un grafo che ha per dominio l'unione disgiunta di $N_1$ ed $N_2$ e come archi quelli di $N_1$ pi\`u quelli di $N_2$ pi\`u quelli che congiungono tutti i vertici di $N_1$ con tutti i vertici di $N_2$. Si dimostri concisamente che $N$ non \`e un grafo aleatorio. Esiste un immersione di $N$ in un grafo aleatorio? Mostrare che l'insieme $N_1$ \`e definibile con parametri in $N$ ma non \`e definibile senza parametri. %Suggerimento: \`e simile all'esercizio~\ref{unionedisgiunta} per quanto visto all'esercizio~\ref{grafocomplementare}.
% \end{exercise}
%
% %$\E x,y\; \Big[ r(x,b)\wedge r(y,b)\wedge \neg\E z\,\big[z\neq b\wedge r(x,z)\wedge r(x,z)\big]\Big]$
%
% \begin{exercise}
% Siano $N_1$ ed $N_2$ due grafi aleatori numerabili e sia $c\in N_2$. Sia $N$ un grafo che ha per dominio l'unione disgiunta di $N_1$ ed $N_2$ e come archi quelli di $N_1$ pi\`u quelli di $N_2$ pi\`u quelli che congiungono tutti i vertici di $N_1$ con $c$. Si dimostri concisamente che $N$ non \`e un grafo aleatorio. \`E l'insieme $N_1$ definibile con una formula pura? Si dimostri che due qualsiasi di questi grafi sono tra loro isomorfi.
% \end{exercise}
%
% \begin{exercise}
% Sia $T$ la teoria definita nell'esercizio~\ref{unionedisgiunta2} e sia $N\models T$. Si dimostri che per ogni grafo numerabile $M$, ogni mappa finita $k:M\to N$ che preserva la verit\`a delle formule esistenziali ha un'estensione ad una mappa totale $h:M\to N$ che preserva la verit\`a le formule esistenziali.
% \end{exercise}

\begin{exercise}
Prove that $T_{\rm rg}$ is not $\lambda$-categorical for any uncountable $\lambda$.
Hint: prove that there is a random graph $N$ of cardinality $\lambda$ where every vertex is adjacent to $<\lambda$ vertices.
Compare it with its complement graph (the graph that has edges between pairs that are non adjacent in $N$).
\end{exercise}

\begin{exercise}
Prove that $T_{\rm rg}$ is not finitely axiomatizable.
Hint: given a random graph $N$ and a set $P$ of cardinality $n+1$ show that you can add edges to $M\sqcup P$ and make it satisfy axiom \ssf{r$_n$} but not \ssf{r$_{n+1}$}.
\end{exercise}

\begin{exercise}[ (Peter J.~Cameron)]\label{ex_CameronRG}
Prove that for every infinite countable graph $M$ the following are equivalent
\begin{itemize}
\item[1.] $M$ is either random, complete, or empty (i.e.\@ $r^M=M^2$ or $r^M=\0$)
\item[2.] if $M_1,M_2\subseteq M$ are such that $M_1\sqcup M_2=M$, then $M_1\simeq M$ or $M_2\simeq M$.
\end{itemize}
Hint: \ssf{2}$\IMP$\ssf{1}. Assume \ssf2 and show that if $r(a,M)=\0$ for some $a\in M$ then the graph is null; if $\{b\}\cup r(b,M)=M$ for some $b$, then the graph is complete.
Clearly, \ssf{2} implies that any finite partition of $M$ contains an element isomorphic to $M$.
Then the claim above generalizes as follows: if there is a finite $A$ such that $\bigcap_{a\in A}r(a, M)=\0$ then $M$ is the empty graph and if there is a finite $B$ such that $B\cup\bigcup_{b\in B}r(b,M)=M$ then $M$ is the complete graph.

Suppose $M$ is not a random graph.
Fix some finite, disjoint $A$ and $B$ such that no $c$ satisfies both $r(A,c)= A$ and $r(B,c)=\0$.
Let $M_1=\{c:r(A,c)\neq A\}$ and $M_2=M\sm M_1$.
Now note that 

\hfil$\displaystyle\bigcap_{a\in A}r(a, M_1)=\0$\hfil and\hfil $\displaystyle\bigcup_{b\in B}r(b, M_2)=M_2$.
\end{exercise}

\section{Notes and references}

We refer the reader to~\cite{Cameron} for a well-written accessible survey on the amazing model theoretic properties of the random graph.

\begin{biblist}[]\normalsize
  \bib{Cameron}{article}{
    author={Cameron, Peter J.},
    title={The random graph},
    note={t.a. in The Mathematics of Paul Erdös III},
    status={\href{https://arxiv.org/abs/1301.7544}{arXiv:1301.7544}},
    %eprint={https://arxiv.org/abs/1301.7544},
    date={2013},
   }
\end{biblist}