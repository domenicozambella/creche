% !TEX root = creche.tex
\chapter{Group actions on models}
\label{actions}

\def\medrel#1{\parbox[t]{5ex}{$\displaystyle\hfil #1$}}
\def\ceq#1#2#3{\parbox[t]{17ex}{$\displaystyle #1$}\medrel{#2}{$\displaystyle #3$}}

\noindent\llap{\textcolor{red}{\Large\warning}\kern1.5ex}\ignorespaces
Chapter under revision.

In this chapter, $L$ is a signature, $T$ is a complete theory without finite models, and $\U$ is a saturated model of inaccessible cardinality $\kappa$ strictly larger than $|L|$.
We use the same notation and make the same implicit assumptions as in Section~\ref{monster}.

Assumption~\ref{notation_GXphi} below is in force throughout the chapter. 

%%%%%%%%%%%%%%%%%%%%%%%%%%%%%%%%%%%%%%%%%%%%
%%%%%%%%%%%%%%%%%%%%%%%%%%%%%%%%%%%%%%%%%%%%
%%%%%%%%%%%%%%%%%%%%%%%%%%%%%%%%%%%%%%%%%%%%
%%%%%%%%%%%%%%%%%%%%%%%%%%%%%%%%%%%%%%%%%%%%
\section{The dual perspective on invariance}\label{dual_perspective}
% \def\BDelta{\LDelta\kern-.4ex_{{\rm qf}}}

The notions of invariance introduceted in Chapter~\ref{invariant} extend seamlessly to the action of a group $G$ on $\U^{\gr z}$.
It is convenient to work in a more general setting.

In this chapter $\Delta\subseteq L_{{\mr x}\,{\gr z}}(\U)$.
Let $\mrX\subseteq\U^{\mr x}$ and $\grZ\subseteq\U^{\gr z}$.
We define a useful auxiliary structure $\UDelta=\big\<\mrX\,;\grZ\big\>$.
This is a 2-sorted structure whose signature $\LDelta$ contains relation symbols for every formula $\phi({\mr x}\,;{\gr z})\in\Delta$.
As there is little risk of confusion, these relations symbols are also denoted by $\phi({\mr x}\,;{\gr z})$, and we identify $\LDelta$ with a subset of $L$.

When $\mrX$ and $\grZ$ are type definable and $\Delta$ is small, $\UDelta$ is a saturated structure.
But, as for the time being we do not need compactness, until further notice these sets are arbitrary.

In the first few sections we only work a subset of the quantifier-free fragment of $\LDelta_{\,\mr x}$.
Let $\BDelta(\grZ)$ be the set of Boolean combinations of the formulas $\phi({\mr x}\,;{\gr b})$, where $\phi({\mr x}\,;{\gr z})\in\Delta$ and ${\gr b}\in\grZ$.
%  $\LDelta_{{\rm\,at^\pm}\!,\,{\mr x}}(\grZ)$ and 
% $\LDelta_{{\rm\,qf},\,{\mr x}}(\grZ)$ with 
% $\pmDelta(\grZ)$, respectively 
We write $S_\Delta(\grZ)$ for the set of complete $\BDelta(\grZ)$-types~--~which we may conveniently think as subsets of $\pmDelta(\grZ)$.
In this chapter the types in $S_\Delta(\grZ)$ are required to be finitely consistent $\mrX$~--~a piece of information that is not displayed by the notation as there is little risk of ambiguity.

A set $\mrD\subseteq\mrX$ is (relatively) \emph{$\BDelta({\grZ})$-definable\/} if $\mrD=\phi(\mrX)$ for some $\phi({\mr x})\in\BDelta({\grZ})$.
Similarly, we define (relatively) \emph{$\BDelta({\grZ})$-type-definable\/} sets.

\begin{assumption}\label{notation_GXphi}
  Let $G$ be a group that acts on the sets ${\mrX}$ and ${\grZ}$.
  We require that for every $\phi({\mr x}\,;{\gr z})\in\Delta$ the set $\phi(\mrX\,;\grZ)$ is invariant under the action of $G$.
  In other words, $G\le\Aut(\UDelta)$.\smallskip

  When $p({\mr x})\subseteq\BDelta(\grZ)$ and ${\mrD}\subseteq\mrX$ we write $p({\mr x})\proves{\mr x}\in{\mrD}$ if the inclusion $\psi(\mrX)\subseteq{\mrD}$ holds for some $\psi({\mr x})$ that is conjunctions of formulas in $p({\mr x})$.
\end{assumption}

Let $\grD\subseteq\grZ$.
We say that $\grD$ is \emph{invariant\/} under the action of $G$, or \emph{$G$-invariant,} if  $\grD$ is fixed setwise by $G$.
That is, $g\,\grD=\grD$ for every $g\in G$.
Yet in other words, if

\ceq{\ssf{is.}\hfill{\gr a}\in\grD}{\iff}{g\,{\gr a}\in\grD}\hfill for every ${\gr a}\in\grZ$ and every $g\in G$.

A formula is $G$-invariant if the set it defines is $G$-invariant.
We say that the type $p({\mr x})\subseteq\BDelta(\grZ)$ is \emph{invariant\/} under the action of $G$, or \emph{$G$-invariant,} if 

\ceq{\ssf{it.}\hfill\theta({\mr x}\,;{\gr\bar a})\in p}{\IFF}{\theta({\mr x}\,;g\,{\gr\bar a})\in p}\hfill for every $g\in G$ and every $\theta({\mr x}\,;{\gr\bar a})\in\BDelta(\grZ)$.

Note that $p({\mr x})$ is $G$-invariant exactly when the sets ${\gr\D_{p,\theta}}$ are.
In this section we want to discuss the invariance of types using the sets $\mrD\subseteq\mrX$.

An immediate consequence of the invariance of $\phi(\mrX\,;\grZ)$ is that any $G$-translate of a $\BDelta({\grZ})$-definable set is again $\BDelta({\grZ})$-definable.
In particular, by reasoning as in Remark~\ref{rem_image_def_set} we obtain that for every $\BDelta({\grZ})$-formula $\theta({\mr x}\,;{\gr\bar b})$ and every $g\in G$

\ceq{\hfill g[\theta(\mrX\,;{\gr\bar b})]}{=}{\theta(\mrX\,;g\,{\gr\bar b}).}

Therefore, a type $p({\mr x})\subseteq\BDelta(\grZ)$ is $G$-invariant if

\ceq{\hfill p({\mr x})\proves{\mr x}\in\mrD}{\IFF}{p({\mr x})\proves{\mr x}\in g{\cdot}\mrD}\hfill for every $\mrD\subseteq\mrX$ and $g\in G$.

% At the first reading the reader may assume that $G=\Aut(\U)$, $\mrX=\U^{\mr x}$, $\grZ=\U^{\gr z}$ and $\Delta=L_{{\mr x}\,;{\gr z}}$, where $|{\gr z}|=\omega$.
% Then $\pmDelta(\grZ)=L_{\mr x}(\U)$ and  $S_\Delta(\grZ)=S_{\mr x}(\U)$. 
% Note that when $G$ acts by automorphisms $\phi(\U^{\mr x};\U^{\gr z})$ is $G$-invariant for all $\phi({\mr x}\,;{\gr z})\in L$.

A set $\mrD\subseteq\mrX$ is \emph{generic\/} under the action of $G$, or \emph{$G$-generic\/} for short, if finitely many $G$-translates of $\mrD$ cover $\mrX$; we say \emph{$n$-$G$-generic\/} if $\le n$ translates suffices.
Dually, we say that $\mrD$ is \emph{persistent\/} under the action of $G$, or \emph{$G$-persistent\/} for short, if the intersection of any finitely many $G$-translates of $\mrD$ is nonempty; we say \emph{$n$-$G$-persistent\/} when the request is limited to $\le n$ translates.
When $\mrX$ and/or $\grZ$ are not clear from the context, we say that these notions are \emph{relative\/} to $\mrX$ and $\grZ$.

The same properties may be attributed to formulas (as these are identified with the set they define).
When these properties are attributed to a type $p({\mr x})$, we understand that they hold for every conjunction of formulas in $p({\mr x})$.
See Exercise~\ref{ex_persistent_types} for an alternative characterization when $p({\mr x})$ is small.

\noindent\llap{\textcolor{red}{\Large\warning}\kern1.5ex}\ignorespaces
The terminology above is non-standard.
In~\cite{CK} the authors write \textit{quasi-non-dividing\/} for \textit{persistent\/} under the action of $\Aut(\U/A)$.
Their terminology has good motivations, but it would be a mouthful if adapted to our context.
In topological dynamics similar notions have been introduced with different terminology: \textit{syndetic\/} corresponds to \textit{generic\/} and \textit{thick\/} corresponds to \textit{persistent.}

Notation: for $\mrD\subseteq\mrX$ and $H\subseteq G$ we write \emph{$H{\cdot}\mrD$\/} for $\{h{\cdot}\mrD: h\in H\}$.

In this chapter many proofs require some juggling with negations as epitomized by the following fact.

\begin{fact}\label{fact_fip}
  The following are equivalent
  \begin{itemize}
    \item[1.] $\mrD$ is not $G$-generic
    \item[2.] $\neg\mrD$ is $G$-persistent.
  \end{itemize}
\end{fact}

\begin{proof}
  Immediate by spelling out the definitions
  \begin{itemize}
    \item[1.] there are no finite $H\subseteq G$ such that $\mrX\ \subseteq\ \cup\, H{\cdot}{\mr\D}$.
    \item[2.] $\varnothing\ \neq\ \mrX\,\cap\,\big(\cap\, H{\cdot}\neg{\mr\D}\big)$ for every finite $H\subseteq G$.\qedhere
  \end{itemize} 
\end{proof}

Define the following type

\ceq{\hfill\emph{$\gamma_G({\mr x})$}}{=}{\{\theta({\mr x})\in \BDelta({\grZ})\ :\ \theta({\mr x})\textrm{ is }G\textrm{-generic}\}}

\begin{corollary}\label{corol_q_pers}
  Let $p({\mr x})\subseteq\BDelta(\grZ)$ be such that $\gamma_G({\mr x})\cup p({\mr x})$ is finitely consistent.
  Then $p({\mr x})$ is $G$-persistent.
\end{corollary}

\begin{proof}
  Let $\theta({\mr x})$ be a conjunction of formulas in $p({\mr x})$.
  As $\gamma_G({\mr x})\cup\{\theta({\mr x})\}$ is finitely consistent, it cannot be that $\neg\theta({\mr x})$ is $G$-generic.
  From Fact~\ref{fact_fip}, we obtain that $\theta({\mr x})$ is $G$-persistent.
\end{proof}

The converse implication holds for complete types.

\begin{theorem}\label{thm_generic_invariant}
  Let $p({\mr x})\in S_\Delta(\grZ)$.
  Then the following are equivalent
  \begin{itemize}
    \item[1.] $p({\mr x})$ is $G$-invariant
    \item[2.] $p({\mr x})\proves\gamma_G({\mr x})$
    \item[3.] $p({\mr x})$ is $G$-persistent.
  \end{itemize}
\end{theorem}

\begin{proof}
  \ssf1$\IMP$\ssf2.
  Let $\mrD$ be a $G$-generic $\BDelta(\grZ)$-definable set.
  Pick $H\subseteq G$ be finite such that $\mrX\subseteq\cup\,H{\cdot}\mrD$.
  Then $p({\mr x})\proves {\mr x}\in\cup\,H{\cdot}\mrD$.
  By completeness, $p({\mr x})\proves {\mr x}\in h\,\mrD$ for some $h\in H$.
  Finally, by invariance, $p({\mr x})\proves{\mr x}\in\mrD$.
  
  \ssf2$\IMP$\ssf3.
  By Corollary~\ref{corol_q_pers}.

  \ssf3$\IMP$\ssf1.
  % uppose $p({\mr x})$ is not $(n+2)$-persistent and let $n$ be minimal.
  % Let $g_1,\dots,g_{n+2}\in G$ and $\theta({\mr x})\,;{\gr\bar b}$ witness this.
  % Then 
  Suppose $p({\mr x})$ is not $G$-invariant.
  % Then, by completeness, $p({\mr x})\proves\phi({\mr x}\,;{\gr b})\wedge\neg\phi({\mr x}\,;g\,{\gr b})$ for some $g\in G$ and $\phi({\mr x}\,;{\gr b})\in\BDelta(\grZ)$.
  % Clearly $\phi({\mr x}\,;{\gr b})\wedge\neg\phi({\mr x}\,;g\,{\gr b})$ is not $2$-$G$-persistent as it is inconsistent with its $g$-translate.
  By completeness $p({\mr x})\proves{\mr x}\in(\neg\mrD\cap g{\cdot}\mrD)$ for some $g\in G$ and some $\BDelta(\grZ)$-definable set $\mrD$.
  Clearly $\neg\mrD\cap g{\cdot}\mrD$ is not $2$-$G$-persistent as it is inconsistent with its $g$-translate.
\end{proof}

% \begin{corollary}\label{corol_gammaG_invaqriancer}
%   The following are equivalent for every $\BDelta(\grZ)$-definable set $\mrD$
%   \begin{itemize}
%     \item [1.] $\gamma_G({\mr x})\proves{\mr x}\in\mrD$
%     \item [2.] $p({\mr x})\proves{\mr x}\in\mrD$ for every $G$-persistent $p({\mr x})\in S_\Delta(\grZ)$.
%   \end{itemize}
% \end{corollary}

% \begin{proof}
%   \ssf1$\IMP$\ssf2.
%   This is an immediate consequence of Theorem~\ref{thm_generic_invariant}.

%   \ssf2$\IMP$\ssf1.
%   Suppose $\gamma_G({\mr x})\not\proves{\mr x}\in\mrD$.
%   Then there is a type $p({\mr x})\in S_\Delta(\grZ)$ finitely consistent with $\gamma_G({\mr x})\cup\{{\mr x}\notin\mrD\}$.
%   By Corollary~\ref{corol_q_pers} $p({\mr x})$ is $G$-persistent.
%   Then $\neg\ssf2$.
% \end{proof}

The theorem yields a necessary condition for the existence of $G$-persistent global $\BDelta({\grZ})$-types.

\begin{corollary}\label{corol_def_mu}
  Assume there exists a $G$-persistent type $p({\mr x})\in S_\Delta(\grZ)$.
  Then for every $\BDelta({\grZ})$-definable set $\mrD$
  \begin{itemize}
    \item[1.] $\mrD$ and $\neg\mrD$ are not both $G$-generic
    \item[2.] if $\mrD$ is $G$-generic then it is $G$-persistent
    \item[3.] $\gamma_G({\mr x})$ is finitely consistent.
  \end{itemize}
\end{corollary}

\begin{proof}
  Clearly, \ssf1 and \ssf2 are equivalent by Fact~\ref{fact_fip} and follow from \ssf3.
  Finally, \ssf3 is an immediate consequence of \ssf2 of Theorem~\ref{thm_generic_invariant}.
\end{proof}

The following theorem gives a necessary and sufficient condition for the  existence of global $G$-invariant $\BDelta({\grZ})$-type.
Ideally, we would like that every $G$-persistent $\BDelta({\grZ})$-type extends to a global persistent type.
Unfortunately this is not true in general (it is a strong assumption, see Section~\ref{tame_landscape}).
A set $\mrD$ is \emph{$G$-wide\/} if every finite cover of $\mrD$ by $\BDelta({\grZ})$-definable sets contains a $G$-persistent set.
In~\cite{CK} a similar property is called \textit{quasi-non-forking.}
Our use of the term \textit{wide\/} is consistent with~\cite{Hr}, though we apply it to a narrow context.
A type is $G$-wide if every conjunction of formulas in the type is $G$-wide.

\begin{theorem}\label{thm_generic_invariant2}
  Let $\mrD$ be a $\BDelta(\grZ)$-definable set.
  Then the following are equivalent 
  \begin{itemize}
    \item[1.] $\gamma_G({\mr x})\cup\{{\mr x}\in\mrD\}$ is finitely consistent
    \item[2.] there exists a $G$-persistent type $p({\mr x})\in S_\Delta(\grZ)$ containing ${\mr x}\in\mrD$
    \item[3.] $\mrD$ is $G$-wide.
  \end{itemize}
\end{theorem}

\begin{proof}
  \ssf1$\IMP$\ssf2.
  By Corollary~\ref{corol_q_pers}, it suffices to pick any $p({\mr x})\in S_\Delta(\grZ)$ consistent with $\gamma_G({\mr x})\cup\{{\mr x}\in\mrD\}$.

  \ssf2$\IMP$\ssf1.
  By Theorem~\ref{thm_generic_invariant}.

  \ssf2$\IMP$\ssf3.
  Let ${\mr\C_1},\dots,{\mr\C_n}$ be $\BDelta({\grZ})$-definable sets that cover $\mrD$.
  Pick $p({\mr x})$ as in \ssf2.
  By completeness, $p({\mr x})\proves {\mr x}\in{\mr\C_i}$ for some $i$.
  Then, by Theorem~\ref{thm_generic_invariant}, $\neg{\mr\C_i}$ is not $G$-generic.
  Therefore, by Fact~\ref{fact_fip}, ${\mr\C_i}$ is $G$-persistent.

  \ssf3$\IMP$\ssf2.
  Let $p({\mr x})$ be maximal among the $\BDelta({\grZ})$-types that are finitely satisfiable in $\mrX\cap\mrD$ and are such that $\theta(\U^{\mr x})$ is $G$-wide for every $\theta({\mr x})$ that is conjunction of formulas in $p({\mr x})$.
  We claim that $p({\mr x})$ is a complete $\BDelta({\grZ})$-type.
  Suppose for a contradiction that $\theta({\mr x}),\neg\theta({\mr x})\notin p$.
  By maximality there is some formula $\psi({\mr x})$, a conjunction of formulas in $p({\mr x})$, and some $\BDelta({\grZ})$-definable sets ${\mr\C_1},\dots,{\mr\C_n}$ that cover both $\psi(\U^{\mr x})\cap\theta(\U^{\mr x})$ and $\psi(\U^{\mr x})\smallsetminus\theta(\U^{\mr x})$ and such that no ${\mr\C_i}$ is $G$-persistent.
  As ${\mr\C_1},\dots,{\mr\C_n}$ cover $\psi(\U^{\mr x})$ this is a contradiction.
  It is only left to show that $p({\mr x})$ is $G$-persistent.
  This follows from completeness and Theorem~\ref{thm_generic_invariant}.
\end{proof}

\begin{corollary}\label{corol_intersectionGwide}
  Let $\mrD$ be a $G$-wide $\BDelta(\grZ)$-definable set.
  Then $\mrD\cap g{\cdot}\mrD$ is $G$-wide for every $g\in G$.
\end{corollary}

\begin{proof}
  Let $p({\mr x})\in S_\Delta(\grZ)$ be a $G$-persistent type such that $p({\mr x})\proves{\mr x}\in\mrD$.
  By $G$-invariance $p({\mr x})\proves{\mr x}\in g{\cdot}\mrD$.
\end{proof}

% Let $q({\mr x})\subseteq\BDelta(\grZ)$.
% We say that $q({\mr x})$ is \emph{$G$-prime\/} if for every $\BDelta(\grZ)$-definable set $\mrD$ and every $g_1,g_2\in G$ if $q({\mr x})\proves{\mr x}\in(g_1{\cdot}\mrD\cup g_2{\cdot}\mrD)$ then $q({\mr x})\proves{\mr x}\in g_i{\cdot}\mrD$ for some $i$.
% The following fact is immediate.

% \begin{fact}
%   The following are equivalent for every $\BDelta(\grZ)$-type $q({\mr x})$
%   \begin{itemize}
%     \item [1.] $q({\mr x})$ is $G$-prime and $G$-invariant
%     \item [2.] if $\mrD$ is $\BDelta(\grZ)$-definable, $g\in G$, and $q({\mr x})\proves{\mr x}\in(\mrD\cup g{\cdot}\mrD)$, then $q({\mr x})\proves{\mr x}\in\mrD$.
%   \end{itemize}
% \end{fact}

% \begin{proposition}
%   For every type $q({\mr x})\subseteq\BDelta(\grZ)$ the following are equivalent
%   \begin{itemize}
%     \item [1.] $q({\mr x})\proves\gamma_G({\mr x})$
%     \item [2.] $q({\mr x})$ is $G$-prime and $G$-invariant.
%   \end{itemize}
% \end{proposition}

% \begin{proof}
%   \ssf1$\IMP$\ssf2.
%   Suppose $q({\mr x})\proves{\mr x}\in(\mrD\cup g{\cdot}\mrD)$.
%   Let $p({\mr x})\in S_\Delta(\grZ)$ extend $q({\mr x})$.
%   By completeness $p({\mr x})\proves{\mr x}\in\mrD$ or $p({\mr x})\proves{\mr x}\in g{\cdot}\mrD$.
%   As $p({\mr x})$ is $G$-invariant by Theorem~\ref{thm_generic_invariant}, it follows that
%   $p({\mr x})\proves{\mr x}\in\mrD$.
%   Finally, as $p({\mr x})$ is arbitrary, $q({\mr x})\proves{\mr x}\in\mrD$. 

%   \ssf2$\IMP$\ssf1.
%   As $\mrD$ is $G$-generic and $q({\mr x})$ is $G$-prime, $q({\mr x})\proves{\mr x}\in g{\cdot}\mrD$ for some $g\in G$.
%   Finally $q({\mr x})\proves{\mr x}\in\mrD$ by invariance.
% \end{proof}

\begin{exercise}
  Prove that if $p(x)\in S(\U)$ is finitely satisfiable in every $M\supseteq A$ then it is persistent under the action of $\Autf(\U/A)$.
  Is the same true for incomplete types?
\end{exercise}

\begin{exercise}
  Let $p(x)\in S_\Delta(\U)$.
  Prove that the following are equivalent
  \begin{itemize}
    \item[1.] $p(x)$ is $G$-invariant
    \item[2.] $p(x)\proves x\in\D$ for every 2-$G$-generic definable set $\D$
    \item[3.] $p(x)$ is 2-$G$-persistent.
  \end{itemize}
\end{exercise}

\begin{exercise}
  Let $\D$ be a $\BDelta(\Z)$-type-definable set.
  Prove that the following are equivalent 
  \begin{itemize}
    \item[1.] there is a $G$-invariant type $p(x)\in S_\Delta(\Z)$ such that $p(x)\proves x\in\D$.
    \item[2.] every finite cover of $\D$ by $\BDelta(\Z)$-definable sets contains a 2-$G$-persistent set.
  \end{itemize}
\end{exercise}

% \begin{exercise}\label{ex_gen_sat}
%   Let $p(x)\subseteq \BDelta(\Z)$ be $G$-generic.
%   Prove that it is finitely satisfiable in $\X$.
% \end{exercise}

\begin{exercise}\label{ex_persistent_types}
  Prove that for every $p(x)\subseteq \BDelta(A)$ following are equivalent
  \begin{itemize}
    \item[1.] $p(x)$ is $G$-persistent
    \item[2.] $p(\U^x)$ is $G$-persistent.
  \end{itemize}
\end{exercise}

\begin{exercise}
  Give an example of a persistent set that is not wide.
  Hint: find inspiration in Example~\ref{ex_cyclic_order}.
\end{exercise}

\begin{exercise}\label{ex_generic_type_vs_formulas}
  Let $\D\subseteq\X$ be a $G$-generic $\BDelta(\Z)$-type-definable set.
  Prove that $\D$ is definable by a $\BDelta(\Z)$-type containing only $G$-generic formulas.
\end{exercise}

% \begin{exercise}\label{ex_Gprime}
%   Prove that if $q(x)$ is $G$-prime and $q(x)\proves x\in \cup\,H{\cdot}\D$ for some finite $H\subseteq G$, then $q(x)\proves x\in\D$.
% \end{exercise}

% \begin{exercize}
%   Let $\mrD$ be a $\BDelta(\grZ)$-definable set.
%   Prove that the following are equivalent
%   \begin{itemize}
%     \item [1.] $\mrD$ is $G$-wide;
%     \item [2.] every finite cover of $\mrD$ by $\pmDelta({\grZ})$-definable sets contains a $G$-persistent set.
%   \end{itemize}
% \end{exercize}

%%%%%%%%%%%%%%%%%%%%%%%%%%
%%%%%%%%%%%%%%%%%%%%%%%%%%
%%%%%%%%%%%%%%%%%%%%%%%%%%
%%%%%%%%%%%%%%%%%%%%%%%%%%
%%%%%%%%%%%%%%%%%%%%%%%%%%
\section{The action of normal subgroups}\label{normalsubgroups}

Let $H\trianglelefteq G$.
The following is an immediate consequence of normality.

\begin{remark}\label{rem_invariance_normalsubg}
% \newlength{\ceqlength}
% \settowidth{\ceqlength}{p(x) is H-invariant\ }
% \def\medrel#1{\parbox[t]{5ex}{$\displaystyle\hfil #1$}}
% \def\ceq#1#2#3{\parbox[t]{\ceqlength}{$\displaystyle #1$}\medrel{#2}{$\displaystyle #3$}}
%
  For every $\mrD\subseteq\mrX$ and every $g\in G$
  
  \ceq{\hfill\mrD\textrm{ is }H\textrm{-foo}}{\IFF}{g{\cdot}\mrD\textrm{ is }H\textrm{-foo},}
  
  where \textit{foo\/} can be replaced by \textit{generic,} \textit{invariant,} \textit{persistent,} \textit{wide.}
  In particular, the type $\gamma_H({\mr x})$ is $G$-invariant.
  % It also follows that for any $p({\mr x})\in S_\Delta(\grZ)$
  % \begin{itemize}
  %   \item [3.] \ceq{\hfill p({\mr x})\textrm{ is }H\textrm{-invariant}}{\IFF}{g\cdot p({\mr x})\textrm{ is }H\textrm{-invariant}.}
  % \end{itemize}
\end{remark}

Recall that if $\gamma_H({\mr x})$ is finitely consistent, then $H$-generic sets are $H$-wide, cf.\@ Theorem~\ref{thm_generic_invariant2}.
As it happens, we can slightly strengthen this fact (because $G$-genericity is a weaker assumption than $H$-genericity).

\begin{proposition}\label{prop_Ggeneric_Hpersistent}
  Assume that $\gamma_H({\mr x})$ is finitely consistent.
  Let $\mrD$ be a $\BDelta(\grZ)$-definable set.
  Then if $\mrD$ is $G$-generic it is also $H$-wide.
\end{proposition}

\begin{proof}
  Let $p({\mr x})\in S_\Delta(\grZ)$ be finitely consistent with $\gamma_H({\mr x})$.
  As $\mrD$ is $G$-generic, by completeness $p({\mr x})\proves{\mr x}\in g{\cdot}\mrD$ for some $g\in G$.
  Equivalently, $g^{-1}\!\cdot p({\mr x})\proves{\mr x}\in\mrD$.
  As $p({\mr x})$ is $H$-persistent, from Remark~\ref{rem_invariance_normalsubg} we obtain that also $g^{-1}\!\cdot p({\mr x})$ is $H$-persistent.
  Then the proposition follows from Theorem~\ref{thm_generic_invariant2}.
\end{proof}

As $\gamma_G({\mr x})$ may be inconsistent, the following notion is relevant

\ceq{\hfill Q}{=}{\big\{q({\mr x})\subseteq\gamma_G({\mr x})\;:\textrm{ maximally finitely consistent set}\big\}}.

It is easy to see that $Q$ is closed under the action of $G$.
Define 

\ceq{\hfill H}{=}{\bigcap_{q\in Q}\Stab(q)}.

Then $H\trianglelefteq G$.
It is easy to verify that $\gamma_H({\mr x})\subseteq q({\mr x})$ for every $q\in Q$.
In particular, $\gamma_H({\mr x})$ is consistent.
In fact, if $\mrD$ is $H$-generic, then $q({\mr x})\proves {\mr x}\in h{\cdot}\mrD$ for some $h\in H$.
% Equivalently $h^{-1}{\cdot}q({\mr x})\proves{\mr x}\in \mrD$.
Then $q({\mr x})\proves\mrD$ follows because $h\in\Stab(q)$.

\begin{example}
  Assume that $G=\Aut(\U/A)$ and $\mrX=O({\mr a}/A)$ for some ${\mr a}\in\U^{\mr x}$.
  Let $\grZ=\U^z$ and $\Delta=L_{{\mr x},{\gr z}}(A)$.
  If $\epsilon({\mr x}\,;{\mr y})\in L(A)$ is a finite equivalence relation then, for every ${\mr b}\in\mrX$, the sets $\epsilon(\mrX\,;{\mr b})$ are $G$-generic.
  Therefore, all sets in $\acl^\eq\!A$ that intersect $\mrX$ are $G$-generic.
  Let $Q$ be as above.\
  Then any $q({\mr x})\in Q$ decides all formulas ${\mr x}\in\mrD$ for $\mrD\in\acl^\eq\!A$.
  Therefore restriction to $\acl^\eq\!A$ is a surjective map from $Q$ onto $S_{\mr x}(\acl^\eq\!A)$.
  It follows that $H\le\Aut(\U/\acl^\eq\!A)$, where $H$ the group $H$ defined above.

  Ideally, we would like to have that $H=\Aut(\U/\acl^\eq\varnothing)$.
  This happens when the types in $S_{\mr x}(\acl^\eq\!A)$ have a unique extension to a type in $Q$.
  We will prove that this is the case under the assumption of stability.
  However it is not true in general, see Exercise~\ref{ex_H_neq_acleq}.
\end{example}

\begin{exercise}\label{ex_H_neq_acleq}
  With the notation of the example above.
  Show that $\Aut(\U/\acl^\eq\varnothing)$ differs from $H$ when $T=T_{\rm dlo}$.
  Hint: the algebraic closure of the empty set is trivial, while $Q$ is not.
\end{exercise}

\begin{comment}
The following normal subgroup of $G$ is particularly interesting

{
\def\medrel#1{\parbox[t]{5ex}{$\displaystyle\hfil #1$}}
\def\ceq#1#2#3{\parbox[t]{8ex}{$\displaystyle #1$}\medrel{#2}{$\displaystyle #3$}}

\ceq{\hfill G^0}{=}{\big\{g\in G\ :\ g{\cdot}\mrD=\mrD\textrm{ for every }G\textrm{-generic }\BDelta(\grZ)\textrm{-type-definable set }\mrD\big\}}.
}

Note that by Exercise~\ref{ex_generic_type_vs_formulas} we can replace type-definable with definable in the definition of $G^0$.

\begin{proposition}
  Assume the action of $G$ is regular.
  Let $K\le G$ be such that all $K$-orbits (that is, $K{\cdot}{\mr a}$, for ${\mr a}\in\mrX$) are $\BDelta(\grZ)$-type-definable.
  Then, if $K$ has finite index, $G^0\le K$.
\end{proposition}

\begin{proof}
  Let $g\in G^0$.
  As $K$ has finite index and the action is transitive $K{\cdot}{\mr a}$ is $G$-generic for every ${\mr a}\in\mrX$.
  Then $g{\cdot}K{\cdot}{\mr a}=K{\cdot}{\mr a}$.
  By regularity, $g{\cdot}K=K$, then $g\in K$.
\end{proof}

%%%%%%%%%%%%%%%%%%%%
%%%%%%%%%%%%%%%%%%%%
%%%%%%%%%%%%%%%%%%%%
%%%%%%%%%%%%%%%%%%%%
%%%%%%%%%%%%%%%%%%%%
\section{An example}
  \def\medrel#1{\parbox[t]{5ex}{$\displaystyle\hfil #1$}}
  \def\ceq#1#2#3{\parbox[t]{13ex}{$\displaystyle #1$}\medrel{#2}{$\displaystyle #3$}}

% In this section $G=\Aut(\UDelta)$ and $\mrX=p_0(\U^{\mr x})$, for some $p_0({\mr x})\in S_\Delta(\grZ)$.
Let ${\mr y}$ be a variable of the same sort of ${\mr x}$.
Below $\BDelta(\grZ)$ may also denote $\LDelta_{{\rm\,qf},\,{\mr x},{\mr y}}(\grZ)$.
In this section a $\BDelta(\grZ)$-definable finite equivalence relation is a formula $\epsilon({\mr x}\,;{\mr y})\in\BDelta(\grZ)$ such that $\epsilon(\mrX\,;\mrX)$ is an equivalence relation with finitely many classes.
Note that when $G$ acts transitively on $\mrX$, equivalence classes of $G$-invariant finite equivalence relations are $G$-generic sets.
% \begin{remark}\label{rem_trans_action}
  % Assume that $G$ acts transitively on $\mrX$.
  Define
  
  \ceq{\hfill e({\mr x}\,;{\mr y})}{=}{\big\{\epsilon({\mr x}\,;{\mr y})\in\BDelta(\grZ)\ :\ G\textrm{-invariant, finite equivalence relation}\big\}}.

  The $e({\mr x}\,;{\mr y})$-equivalence classes are those defined by the types
  
  \ceq{\hfill q({\mr x})}{=}{\big\{\epsilon({\mr x}\,;{\mr a})\in\BDelta(\grZ)\ :\ G\textrm{-invariant, finite equivalence relation}\big\}.}

% \end{remark}

\begin{fact}
  Assume that $G$ acts transitively on $\mrX$.
  Let $\mrD$ be a $\BDelta(\grZ)$-definable set that intersects $\mrX$ and has finite $G$-orbit.
  Then $\mrD$ is $G$-generic.
\end{fact}

\begin{proof}
  Let $g_i{\cdot}\mrD$, for $i=1,\dots,n$, enumerate the $G$-orbit of $\mrD$.
  Define 

  \ceq{\hfill\epsilon({\mr x}\,;{\mr y})}{=}{\bigwedge_{i=1}^n\big[{\mr x}\in g_i{\cdot}\mrD\ \iff\ {\mr y}\in g_i{\cdot}\mrD\big]}

  Clearly, $\epsilon({\mr x}\,;{\mr y})$ is $G$-invariant.
  Pick any ${\mr a}\in\mrD$.  
  Then $\epsilon(\U^{\mr x};{\mr a})\subseteq\mrD$.
  As the action is transitive, $\epsilon(\U^{\mr x};{\mr a})$, and a fortiori $\mrD$, is $G$-generic.
\end{proof}

\end{comment}


The following lemma refine Theorem~\ref{thm_Shelah_strong_types} and Proposition~\ref{prop_trans_action} by adding a requirement on the language.
The proof is essentially the same (but it would be rude to leave it as an exercise).

\begin{lemma}\label{lem_trans_action}
  \def\medrel#1{\parbox[t]{5ex}{$\displaystyle\hfil #1$}}
  \def\ceq#1#2#3{\parbox[t]{15ex}{$\displaystyle #1$}\medrel{#2}{$\displaystyle #3$}}

  Let $\Delta\subseteq L_{{\mr x}\,{\gr z}}$.
  Let $p({\mr x})$ be a complete $\BDelta(A)$-type.
  Finally, let\smallskip
  
  \ceq{\hfill Q}{=}{\big\{q({\mr x})\subseteq\BDelta(\acl^\eq\!A)\ :\ q({\mr x})\proves p({\mr x})\textrm{ and is complete}\big\}}.\smallskip
  
  Then $\Aut(\U/A)$ acts on $Q$ transitively, i.e.\@ any two types in $Q$ are conjugated.
  Moreover, for any ${\mr a}\models p({\mr x})$ there is a unique $q({\mr x})\in Q$ such that

  \ceq{\hfill q({\mr x})}{\iff}{\big\{\epsilon({\mr x}\,;{\mr a})\ :\ \epsilon({\mr x}\,;{\mr y})\in\BDelta(A)\textrm{ finite equivalence relation}\big\}}.

\end{lemma}

\begin{proof}
  We prove that any two types $q_1,q_2\in Q$ are realized be some $a_1\equiv_Aa_2$, respectively.
  The claim follows because any automorphism in $\Aut(\U/A)$ that takes $a_1$ to $a_2$ takes also $q_1$ in $q_2$.
  It suffices to prove that every type $p({\mr x})\in S(A)$ is consistent with every $q\in Q$.
  Suppose not, then $p({\mr x})\imp\neg\phi({\mr x})$ for some $\phi({\mr x})\in q$.
  Let $\psi({\mr x})$ be the disjunction of all the (finitely many) conjugates of $\phi({\mr x})$ over $A$.
  Then $p({\mr x})\imp\neg\psi({\mr x})$.
  But $\psi({\mr x})\in\BDelta(\acl^\eq\!A)$ and is invariant over $A$.
  Therefore $\psi({\mr x})\in\BDelta(A)$.
  Then $\psi({\mr x})\in p$, a contradiction.
  
  Let $\<q_i:i<\lambda\>$ be an enumeration without repetitions of $Q$.
  For every $i<j<\lambda$ pick $\theta_{i,j}({\mr x}\,;w)\in L$ and $c\in\acl^\eq\!A$ such that  $\theta_{i,j}({\mr x}\,;c)$ separates $q_i$ and $q_j$.
  Define 
  {
  \def\medrel#1{\parbox[t]{5ex}{$\displaystyle\hfil #1$}}
  \def\ceq#1#2#3{\parbox[t]{15ex}{$\displaystyle #1$}\medrel{#2}{$\displaystyle #3$}}
  
  \ceq{\hfill\epsilon_{i,j}({\mr x},{\mr y})}{=}{\bigwedge_{f\in\Aut(\U/A)}\big[\theta_{i,j}({\mr x}\,;fc)\iff\theta_{i,j}({\mr y}\,;fc)\big]}
  }

  Clearly $\epsilon_{i,j}({\mr x},{\mr y})$ defines a finite equivalence relation. 
  As this is definable and invariant over $A$, it belongs to $\BDelta(A)$.
  The equivalence in the lemma holds if we take for $q({\mr x})$, the unique $q_i({\mr x})$ satisfied by ${\mr a}$.
\end{proof}

\begin{comment}

\begin{corollary}
  Let $G=\Aut(\U/A)$.
  Let $\mrX=p(\U^{\mr x})$ where $p({\mr x})$ is a complete $\GDelta(A)$-type.
  Then for every $\BDelta(\grZ)$-definable set $\mrD$ that intersects $\mrX$ the following are equivalent
  \begin{itemize}
    \item [1.]  $\mrC\subseteq\mrD$ for some $\mrC\in\acl^\eq\!A$ that intersects $\mrX$
    \item [2.]  $\mrD$ is $G$-generic.
  \end{itemize}
\end{corollary}

\begin{proof}
  \ssf1$\IMP$\ssf2.
  Pick any ${\mr a}\in\mrC\cap\mrX$.
  We can assume that $\mrC=\epsilon(\U^{\mr x}\,;{\mr a})$ is an equivalence class of some finite equivalence relation $\epsilon({\mr x}\,;{\mr y})\in L(A)$ 
  As $G$ acts transitively on $\mrX$ the orbit of $\mrC$ coves $\mrX$.
  Therefore $\mrC\subseteq\mrD$, and a fortiori $\mrD$, is $G$-generic.

  \ssf2$\IMP$\ssf1.
  Let $Q$ be as in Lemma~\ref{lem_trans_action}.
  Pick any $q({\mr x})\in Q$.
  By genericity $q({\mr x})\proves{\mr x}\in g{\cdot}\mrD$ for some $g\in G$.
  Then $g^{-1}\!\cdot q({\mr x})\proves{\mr x}\in \mrD$.
  Pick any ${\mr a}\models g^{-1}\!\cdot q({\mr x})$.
  As $g^{-1}\!\cdot q({\mr x})\in Q$, by the second claim in the lemma and compactness,  $\epsilon(\U^{\mr x}\,;{\mr a})\subseteq\mrD$ for some finite equivalence relation $\epsilon({\mr x}\,;{\mr y})\in\GDelta(A)$.
\end{proof}
\end{comment}

%%%%%%%%%%%%%%%%%%%%%%%%%%%
%%%%%%%%%%%%%%%%%%%%%%%%%%%
%%%%%%%%%%%%%%%%%%%%%%%%%%%
%%%%%%%%%%%%%%%%%%%%%%%%%%%
%%%%%%%%%%%%%%%%%%%%%%%%%%%
\section{Strong genericity}\label{strong_genericity}

Unfortunately, $G$-genericity is not preserved under intersection.
To obtain closure under intersection, we need to push the concept to a higher level of complexity.

A set $\mrD\subseteq\U^{\mr x}$ is \emph{strongly $G$-generic\/} if for every finite $H\subseteq G$ the set $\cap\,H{\cdot}\mrD$ is generic (recall that $H{\cdot}\mrD$ stands for $\{h{\cdot}\mrD: h\in H\}$).
Dually, we say that $\mrD$ is \emph{weakly $G$-persistent\/} if for some finite $H\subseteq G$ the set $\cup\,H{\cdot}\mrD$ is persistent.
Again, the same properties may be attributed to formulas and types when every conjunction of formulas in the type has the property.

\begin{lemma}\label{lem_strongly_generic}
  The intersection of two strongly $G$-generic sets is strongly $G$-generic.
\end{lemma}

\begin{proof}
  We may assume that all sets mentioned below are subsets of $\mrX$.
  Let ${\mr\D}$ and ${\mr\C}$ be strongly $G$-generic and let $K\subseteq G$ be an arbitrary finite set.
  It suffices to prove that $\mrB=\cap\, K{\cdot}({\mr\C}\cap{\mr\D})$ is $G$-generic. 
  Clearly $\mrB={\mr\C'}\cap{\mr\D'}$, where ${\mr\C'}=\cap\, K{\cdot}{\mr\C}$ and ${\mr\D'}=\cap\, K{\cdot}{\mr\D}$.
  Note that ${\mr\C'}$ and ${\mr\D'}$ are both strongly $G$-generic.
  In particular $\mrX=\cup\,H{\cdot}\mr\D'$ for some finite $H\subseteq G$.
  Now, from
  
  \ceq{\hfill\cup\,H{\cdot}\mrB}{=}{\cup\,H\Big[{\mr\C'}\ \cap\ {\mr\D'}\Big]}

  \ceq{\hfill\cup\,H{\cdot}\mrB}{\supseteq}{\cup\,H\Big[\big(\cap\, H{\cdot}{\mr\C'}\big)\ \cap\ {\mr\D'}\Big]}
  
  \ceq{}{=}{ \big(\cap\, H{\cdot}{\mr\C'}\big)\ \cap\ \big(\cup\,H{\cdot}{\mr\D'}\big)}
  
  \ceq{}{=}{\cap\, H{\cdot}{\mr\C'}}
  
  As ${\mr\C'}$ is strongly $G$-generic, $\cap\, H{\cdot}{\mr\C'}$ is $G$-generic.
  Therefore $\cup\,H{\cdot}\mrB$ is also $G$-generic.
  The $G$-genericity of $\mrB$ follows.
\end{proof}

Define the following type

\ceq{\hfill\emph{${}^{\rm s}\kern-.2ex\gamma_G({\mr x})$}}{=}{\{\theta({\mr x})\in \BDelta({\grZ})\ :\ \theta({\mr x})\textrm{ is strongly }G\textrm{-generic}\}}

\begin{corollary}\label{corol_str_gen}
  ${}^{\rm s}\kern-.2ex\gamma_G({\mr x})$ is finitely consistent, strongly $G$-generic, and $G$-invariant.
\end{corollary}

\begin{proof}
  The strong $G$-genericity is an immediate consequence of Lemma~\ref{lem_strongly_generic}.
  The finite satisfiability is a consequence of $G$-genericity.% , see Exercise~\ref{ex_gen_sat}.
  As for invariance, note that any translate of a strongly $G$-generic formula is also strongly $G$-generic.
\end{proof}

\begin{corollary}\label{corol_q_w_pers}
  Let $p({\mr x})\subseteq L(\U)$.
  Then the following are equivalent
  \begin{itemize}
    \item [1.] ${}^{\rm s}\kern-.2ex\gamma({\mr x})\cup p({\mr x})$ is finitely consistent
    \item [2.] $p({\mr x})$ is weakly $G$-persistent.
  \end{itemize}
\end{corollary}

\begin{proof}
  \ssf1$\IMP$\ssf2. Similar to Corollary~\ref{corol_q_pers}.
  Let $\theta({\mr x})$ be a conjunction of formulas in $p({\mr x})$.
  As ${}^{\rm s}\kern-.2ex\gamma({\mr x})\cup\{\theta({\mr x})\}$ is finitely consistent, it cannot be that $\neg\theta({\mr x})$ is strongly $G$-generic.
  From Fact~\ref{fact_fip}, we obtain that $\neg\theta({\mr x})$ not being strongly $G$-generic is equivalent to $\theta({\mr x})$ being weakly $G$-persistent.

  \ssf2$\IMP$\ssf1. 
  Suppose $\mrD$ is strongly $G$-generic and $p({\mr x})\proves{\mr x}\notin\mrD$.
  From Fact~\ref{fact_fip}, $\mrD$ is not weakly $G$-persistent, and $\neg$\ssf2 follows.
\end{proof}

% Note that 
% %from Corollary~\ref{corol_str_gen} and~\ref{corol_q_w_pers} 
% the $G$-translate of a weakly $G$-persistent type is weakly $G$-persistent.

\begin{exercise}
  Show that $G$-generic sets need not be closed under intersection.
  Hint: find inspiration in Example~\ref{ex_cyclic_order}.
\end{exercise}

\begin{exercise}
  Prove that the following are equivalent
  \begin{itemize}
    \item [1.] $\D$ is weakly $G$-persistent
    \item [2.] there is a non $G$-generic set $\C$ such that $\D\cup\C$ is generic.
  \end{itemize}
\end{exercise}

%%%%%%%%%%%%%%%%%%%%%%%%%%%%%%%%%%%
%%%%%%%%%%%%%%%%%%%%%%%%%%%%%%%%%%%
%%%%%%%%%%%%%%%%%%%%%%%%%%%%%%%%%%%
%%%%%%%%%%%%%%%%%%%%%%%%%%%%%%%%%%%
%%%%%%%%%%%%%%%%%%%%%%%%%%%%%%%%%%%
\section{The diameter of a Lascar type}\label{newelski}

As an application we prove an interesting property of the Lascar types.
Recall that $\Ll({\mr a}/A)$, the Lascar strong type of ${\mr a}\in\U^{\mr x}$, is the union of a chain of type-definable sets of the form $\big\{{\mr x}\ :\ d_A({\mr a},{\mr x})\le n\big\}$.
In this section we prove that $\Ll({\mr a}/A)$ is type-definable (if and) only this chain is finite.
In other words, only if the connected component of ${\mr a}$ in the Lascar graph has finite diameter.

It is convenient to address the problem in more general terms.
We work under Assumption~\ref{notation_GXphi} and also assume that $G$ acts transitively on $\mrX$ i.e.\@ $G\,{\mr a}=\mrX$ for every ${\mr a}\in\mrX$.
Let $K\subseteq G$ be a set of generators that is
\begin{itemize}
  \item[1.] symmetric i.e.\@ it contains the unit and is closed under inverse
  \item[2.] conjugacy invariant i.e.\@ $g\,Kg^{-1}=K$ for every $g\in G$
\end{itemize}

We define a discrete metric on $\mrX$.
For ${\mr a},{\mr b}\in\mrX$ let $d({\mr a},{\mr b})$ be the minimal $n$ such that ${\mr a}\in K^n{\mr b}$.
This defines a metric which is $G$-invariant by \ssf2.
The \emph{diameter\/} of a set $\mrC\subseteq\mrX$ is the supremum of $d({\mr a},{\mr b})$ for ${\mr a},{\mr b}\in\mrC$.

We are interested in sufficient conditions for $\mrX$ to have finite diameter.
The notions introduced in Section~\ref{strong_genericity} offer some hint.

\begin{proposition}\label{prop_wpers_finite_diameter}
  If $\mrX$ has a weakly persistent subset of finite diameter, then $\mrX$ itself has finite diameter.
\end{proposition}

\begin{proof}
  Let $\mrC\subseteq\mrX$ be a weakly persistent set of diameter $n$.
  Let $H\subseteq G$ be finite such that $\cup\,H{\cdot}\mrC$ is persistent.
  We claim that also $\cup\,H{\cdot}\mrC$ has finite diameter.
  Let ${\mr a}\in\mrC$ be arbitrary.
  Let $m$ be larger than $d(h{\mr a}, k{\mr a})$ for all $h,k,\in H$.
  Now, let $h{\mr b}$ and $k{\mr c}$, for some $h,k,\in H$ and ${\mr b},{\mr c}\in\mrC$, be two arbitrary elements of $\cup\,H{\cdot}\mrC$.
  As $h\mrC$ and $k\mrC$ have the same diameter of $\mrC$, 

  \ceq{\hfill d(h{\mr b},\, k{\mr c})}{\le}{d(h{\mr b},\,h{\mr a})\ +\ d(h{\mr a},\, k{\mr a})\ +\ d(k{\mr a},\,k{\mr c})}

  \ceq{}{\le}{n+m+n.}

  This proves that $\cup\,H{\cdot}\mrC$ has finite diameter.
  Therefore, without loss of generality, we may assume that $\mrC$ itself is persistent.
  
  By the transitivity of the action, any two elements of $\mrX$ are of the form $h{\mr a}$, $k{\mr a}$ for some $h,k\in G$ and some ${\mr a}\in\mrC$.
  By persistency, there are ${\mr c}\in\mrC\cap h\mrC$ and ${\mr d}\in\mrC\cap k\mrC$.
  Then 

  \ceq{\hfill d(h{\mr a},\, k{\mr a})}{\le}{d(h{\mr a},\,{\mr c})\ +\ d({\mr c},\, {\mr d})\ +\ d({\mr d},\,k{\mr a})}

  \ceq{}{\le}{n+n+n.}

  Therefore the diameter of $\mrX$ does not exceed $3n$.
\end{proof}

\begin{theorem}\label{thm_newelski}
  Suppose that $\mrX$ and the sets ${\mr\X_n}=K^n{\mr a}$, for some ${\mr a}\in\mrX$, are type-definable.
  Then $\mrX$ has finite diameter.
\end{theorem}

\begin{proof}
  By Proposition~\ref{prop_wpers_finite_diameter}, it suffices to prove that ${\mr\X_n}$ is weakly persistent.
  By Corollary~\ref{corol_q_w_pers} it suffices to show that for some $n$ the type ${}^{\rm s}\kern-.2ex\gamma_G({\mr x})$ is finitely satisfiable in ${\mr\X_n}$.
  Suppose not.
  Let $\psi_n({\mr x})\in{}^{\rm s}\kern-.2ex\gamma_G$ be a formula that is not satisfied in $\mr\X_n$.
  Then the type $p({\mr x})=\{\psi_n({\mr x}):n\in\omega\}$ is finitely consistent.
  From the type-definablity of $\mrX$ it follows that $p({\mr x})$ has a realization in $\mrX$.
  As this realization belongs to some ${\mr\X_n}$ we contradict the definition of $\psi_n({\mr x})$.
\end{proof}

% \ceq{\hfill p_n({\mr\X},{\mr\X})}{=}{\{\<{\mr a},{\mr b}\>\in{\mr\X^2}\ :\ {\mr a}\in K^n{\mr b}\}}

% Below we write ${\mr x}\in K^n{\mr y}$ for the type $p_n({\mr x},{\mr y})$ and ${\mr x}\in G\,{\mr y}$ for the disjunction of all these types (n.b.\@ an infinite disjunction of types need not be a type).

\begin{example}\label{ex_newelski}
  Let $K\subseteq\Aut(\U/A)$ be the set of automorphisms that fix a model containing $A$.
  Then the group $G$  generated by $K$ is $\Autf(\U/A)$ and $G\cdot{\mr a}=\mrX$ is $\Ll({\mr a}/A)$.
  Let $\Delta=L_{{\mr x}\,{\gr z}}$ and $\grZ=\U^{\gr z}$.
  Then $d({\mr a},{\mr b})$ coincides with the distance in the Lascar graph.
  As shown in Proposition~\ref{prop_Lascar_distance_type_def} the sets $K^n\cdot{\mr a}=\{{\mr x}:d({\mr x},{\mr a})\le n\}$ are type definable.
  Then from Theorem~\ref{thm_newelski} it follows that $\Ll({\mr a}/A)$ is type definable (if and) only if it has finite diameter.
\end{example} 

%%%%%%%%%%%%%%%%%%%%%%%%%%%%%%%%%%%
%%%%%%%%%%%%%%%%%%%%%%%%%%%%%%%%%%%
%%%%%%%%%%%%%%%%%%%%%%%%%%%%%%%%%%%
%%%%%%%%%%%%%%%%%%%%%%%%%%%%%%%%%%%
%%%%%%%%%%%%%%%%%%%%%%%%%%%%%%%%%%%
\section{A tamer landscape}\label{tame_landscape}

Under suitable assumptions -- e.g.\@ see Theorem~\ref{thm_persistent_finsat} -- some  notions introduced in this chapter coalesce and we are left with a tamer landscape.
In general, we have the following.


\begin{theorem}\label{thm_coalesce}
  Assume 
  \begin{itemize}
    \item[1.] $G$-persistent $\BDelta({\grZ})$-definable sets are $G$-wide.
  \end{itemize}
  Then the following hold
  \begin{itemize}
    \item[2.] $G$-generic $\BDelta({\grZ})$-definable sets are closed under intersection 
    \item[3.] $G$-generic $\BDelta({\grZ})$-definable sets are strongly $G$-generic
    \item[4.] weakly $G$-persistent $\BDelta({\grZ})$-definable sets are $G$-persistent
    \item[5.] $G$-generic $\BDelta({\grZ})$-definable sets are $G$-wide.
  \end{itemize}
\end{theorem}

\begin{proof}
  Clearly \ssf2$\IFF$\ssf3$\IFF$\ssf4.

  \ssf1$\IMP$\ssf2.
  Let $\mrC$ and $\mrD$ be $G$-generic $\BDelta({\grZ})$-definable sets.
  Suppose for a contradiction that $\mrC\cap\mrD$ is not $G$-generic.
  Then $\neg(\mrC\cap\mrD)$ is $G$-persistent.
  By \ssf1 and Theorem~\ref{thm_generic_invariant2} there is a $G$-invariant type $p({\mr x})\in S_\Delta(\grZ)$ such that $p({\mr x})\proves{\mr x}\notin\mrC\cap\mrD$.
  By completeness either $p({\mr x})\proves{\mr x}\notin\mrC$ or $p({\mr x})\proves{\mr x}\notin\mrD$.
  This is a contradiction because by Theorem~\ref{thm_generic_invariant} $p({\mr x})\proves{\mr x}\in\mrC$ and $p({\mr x})\proves{\mr x}\in\mrD$.

  \ssf1$\IMP$\ssf5.
  By~\ssf{3}, because strongly $G$-generic sets are $G$-persistent.
  %
  % \ssf4$\IMP$\ssf1.
  % Note that, by \ssf3, the type ${}^{\rm s}\kern-.2ex\gamma_G({\mr x})$ coincides with $\gamma_G({\mr x})$, in particular $\gamma_G({\mr x})$ is finitely consistent.
  % Let $\mrD$ be a $G$-persistent $\BDelta({\grZ})$-definable set.
  % We show that $\gamma_G({\mr x})={}^{\rm s}\kern-.2ex\gamma_G({\mr x})$ is finitely satisfiable in $\mrX\cap\mrD$.
  % Then, by \ssf4 and Corollary~\ref{corol_q_w_pers}, any global extension of $\gamma_G({\mr x})\cup\{{\mr x}\in\mrD\}$ witness \ssf2 of Theorems~\ref{thm_generic_invariant2}.
  % Suppose not, then $\gamma_G({\mr x})\proves {\mr x}\notin\mrD$.
  % Therefore $\neg\mrD$ is $G$-generic, contradicting the consistency of ${}^{\rm s}\kern-.2ex\gamma_G({\mr x})$.
\end{proof}

% \begin{assumption}\label{notation_2}
%   For $G$, $\mrX$, $\grZ$ and $\Delta$ as in Assumption~\ref{notation_GXphi} we also require that the equivalent conditions in Theorem~\ref{thm_coalesce} hold.
% \end{assumption}

\begin{remark}\label{rem_coalesce} 
  Assume that conditions \ssf1 in Theorem~\ref{thm_coalesce} holds.
  Then the types $\gamma_G({\mr x})$ and ${}^{\rm s}\kern-.2ex\gamma_G({\mr x})$ coincide and are finitely consistent.
  Also, $G$-invariant global types exist.
  It is also worth mentioning that every positive Boolean combination of $G$-generic sets is $G$-generic.
\end{remark}

%%%%%%%%%%%%%%%%%%%%%%%%%%
%%%%%%%%%%%%%%%%%%%%%%%%%%
%%%%%%%%%%%%%%%%%%%%%%%%%%
%%%%%%%%%%%%%%%%%%%%%%%%%%
%%%%%%%%%%%%%%%%%%%%%%%%%%
\section{Definable groups}\label{definablegroups}

\def\medrel#1{\parbox[t]{5ex}{$\displaystyle\hfil #1$}}
\def\ceq#1#2#3{\parbox[t]{25ex}{$\displaystyle #1$}\medrel{#2}{$\displaystyle #3$}}

In this section we assume that $\grZ$ and $\mrX$ are type-definable over some set of parameters $A$.
Moreover we assume that $\grZ$ is a group that acts on $\mrX$.
The group operations and the group action are assumed definable over $A$.
We use the symbol $\,\cdot\,$ for both the group multiplication and the group action.

Let $\Psi\subseteq L_{\mr x}(A)$.
In this section $\Delta$ contains formulas $\phi({\mr x}\,;{\gr z})$ of the form  $\psi({\gr z^{-1}}\!\cdot{\mr x})$ for $\psi({\mr x})\in\Psi$.
The sets $\phi(\mrX\,;\grZ)$ are $\grZ$-invariant.
We write ${\gr 1}$ for the identity of $\grZ$.
Note that $\phi(\mrX\,;{\gr g})={\gr g}\cdot\phi(\mrX\,;{\gr 1})$.

Note that as $\mrX$ and $\grZ$ are assumed to be type-definable, $\UDelta$ is a saturated $\LDelta$-structure.
In this section \emph{$G$}$~=~\Aut(\UDelta)$.
It is worth noticing that automorphisms of $\UDelta$ do not preserve the group operations nor the group action.

To each ${\gr h}\in\grZ$ we associate the $\LDelta$-automorphism $\<{\mr a}\,;{\gr g}\>\mapsto\<{\gr h}{\cdot}{\mr a}\,;{\gr h}{\cdot}{\gr g}\>$.
Therefore $\grZ$ is, up to isomorphism, a subgroup of $G$.
For any ${\gr g}\in\grZ$, the orbit of $\phi(\mrX\,;{\gr g})$ under the action of $\grZ$ is $\{\phi(\mrX\,;{\gr h}) : {\gr h}\in\grZ\}$.
Therefore it coincides with the orbit under the action of $G$.
% The following fact follows immediately.

% \begin{fact}
%   For every  $\BDelta(\grZ)$-type-definable set $\mrD$ the following are 
%   \begin{itemize}
%     \item [1.] the $G$-orbit of $\mrD$ is bounded
%     \item [2.] the $\grZ$-orbit of $\mrD$ is bounded.
%   \end{itemize}
% \end{fact}

We also consider the action some other normal subgroup \emph{$H$\/}$\,\trianglelefteq G$.
% In the applications $H$ will be either $\Autf(\UDelta)$ or $\Aut(\UDelta/M)$.

\def\medrel#1{\parbox[t]{5ex}{$\displaystyle\hfil #1$}}
\def\ceq#1#2#3{\parbox[t]{6ex}{$\displaystyle #1$}\medrel{#2}{$\displaystyle #3$}}

\begin{proposition}\label{prop_Ggeneric_persistent}
  Let $\mrD$ be a $\BDelta(\grZ)$-definable set.
  Assume that $\gamma_H({\mr x})$ is consistent.
  Then \ssf1$\IMP$\ssf2 holds, where
  \begin{itemize}
    \item [1.] $\mrD$ is $\grZ$-generic
    \item [2.] ${\gr g}{\cdot}\mrD$ is $H$-wide for every ${\gr g}\in\grZ$.
  \end{itemize}
\end{proposition}

We will revisit this proposition under the assumption of stability and with $\Autf(\U^\Delta)$ for $H$~--~then the consistency of $\gamma_H({\mr x})$ is guaranteed, and also the converse implication holds.
See Theorem~\ref{thm_Ggeneric_persistent}.\vspace*{-0.5\baselineskip}
%
\begin{proof}
  Let ${\gr g}$ be given.
  If $\mrD$ is $\grZ$-generic, then so is ${\gr g}{\cdot}\mrD$.
  Then ${\gr g}{\cdot}\mrD$ is, a fortiori, $G$-generic.
  Therefore \ssf2 follows from Proposition~\ref{prop_Ggeneric_Hpersistent}.
\end{proof}

We write \emph{$({\gr g})_H$\/} for the $H$-orbit of ${\gr g}$, that is, the set $\{f({\gr g})\ :\ f\in H\}$.

\begin{proposition}\label{prop_wideHcojugate}
  Let $\theta({\mr x}\,;{\gr z_1},\dots,{\gr z_n})$ be a Boolean combination of formulas $\phi_i({\mr x}\,;{\gr z_i})$ for some $\phi_i({\mr x}\,;{\gr z})\in\Delta$.
  Then for every ${\gr h_i}\in({\gr g_i})_H$ the following are equivalent
  \begin{itemize}
    \item [1.] $\theta({\mr x}\,;{\gr g_1},\dots,{\gr g_n})$ is $H$-wide
    \item [2.] $\theta({\mr x}\,;{\gr h_1},\dots,{\gr h_n})$ is $H$-wide.
  \end{itemize}
\end{proposition}

\begin{proof}
  Without loss of generality we can assume that only conjunctions occur in $\theta({\mr x}\,;{\gr g_1},\dots,{\gr g_n})$.
  Let ${\mr\C_i}=\phi_i(\mrX\,;{\gr 1})$.
  Then \ssf1 says that $\mrC={\gr g_1}{\cdot}{\mr\C_1}\cap\dots\cap{\gr g_n}{\cdot}{\mr\C_n}$ is $H$-wide.
  Let $f_i\in H$ be such that ${\gr h_i}=f_i({\gr g_i})$.
  Then, by Corollary~\ref{corol_intersectionGwide} also the intersection of the sets $f_i[\mrC]$ is $H$-wide.
  A fortiori the intersection of the sets $f_i[{\gr g_i}{\cdot}{\mr\C_i}]$ is $H$-wide.
  As $f_i[{\gr g_i}{\cdot}{\mr\C_i}]=f_i({\gr g_i}){\cdot}{\mr\C_i}$ the equivalence follows.
\end{proof}

If $\grA\subseteq\grZ$, we write \emph{$\<\grA\>$} for the subgroup generated by $\grA$.

\begin{proposition}\label{prop_stabilizer1}
  \def\medrel#1{\parbox[t]{5ex}{$\displaystyle\hfil #1$}}
  \def\ceq#1#2#3{\parbox[t]{15ex}{$\displaystyle #1$}\medrel{#2}{$\displaystyle #3$}}
  Let $\theta({\mr x}\,;{\gr z_1},\dots,{\gr z_n})$ be a Boolean combination of formulas $\phi_i({\mr x}\,;{\gr z_i})$ for some $\phi_i({\mr x}\,;{\gr z})\in\Delta$.
  Let ${\gr g}\in\grZ$ be arbitrary.
  Assume that $\theta({\mr x}\,;{\gr 1},\dots,{\gr 1})$ is $H$-wide.
  Then $\theta({\mr x}\,;{\gr h_1},\dots,{\gr h_n})$ is $H$-wide for every 
  
  \ceq{\hfill{\gr h_1},\dots,{\gr h_n}}{\in}{\Big\<\bigcup_{{\gr g}\in\grZ}({\gr g})_H^{-1}\!\cdot({\gr g})_H\Big\>.}
\end{proposition}

\begin{proof}
  We proceed by induction on the number of factors of the form ${\gr a^{-1}}\!\cdot{\gr b}$, for some ${\gr a},{\gr b}\in({\gr g})_H$, that occur in ${\gr h_1},\dots,{\gr h_n}$.
  Without loss of generality we can assume that only conjunctions occur in $\theta({\mr x}\,;{\gr g_1},\dots,{\gr g_n})$.
  Let ${\mr\C_i}=\phi_i(\mrX\,;{\gr 1})$.
  Assume inductively that ${\gr h_1}{\cdot}{\mr\C_1}\cap\dots\cap{\gr h_n}{\cdot}{\mr\C_n}$ is $H$-wide.
  Pick two arbitrary ${\gr a},{\gr b}\in({\gr g})_H$.
  By Remark~\ref{rem_invariance_normalsubg}
  
  \hspace*{7ex}${\gr a}\cdot{\mr\C_1}\ \cap\ {\gr a}\cdot{\gr h_1^{-1}}\!\cdot{\gr h_2}\cdot{\mr\C_2}\ \cap\ \dots\dots\ \cap\ {\gr a}\cdot{\gr h_1^{-1}}\!\cdot{\gr h_n}\cdot{\mr\C_n}$ is $H$-wide.
  
  By Proposition~\ref{prop_wideHcojugate}, in this intersection we can replace ${\gr a}\cdot{\mr\C_1}$ by ${\gr b}\cdot{\mr\C_1}$.
  Then finally 

  \hspace*{7ex}${\gr h_1}\cdot{\gr a^{-1}}\!\cdot{\gr b}\cdot{\mr\C_1}\ \cap\ {\gr h_2}\cdot{\mr\C_2}\ \cap\ \dots\dots\ \cap\ {\gr h_n}\cdot{\mr\C_n}$ is $H$-wide.
\end{proof}

% In Section~\ref{stable_groups} we will give an interesting characterization of this group under the assumption of stability.

\begin{comment}
\section{Tentative}

In this section we work over a given $M\preceq\U^\Delta$.

Let $\mrD\subseteq\mrX$ be definable by some $p({\mr x})\subseteq\BDelta(M)$.
We say that $\mrD$ has 

A formula $\psi({\mr x})\in L$ is \emph{symmetric\/} if $\psi({\gr g}\cdot{\mr x})\iff\psi({\gr g^{-1}}\!\cdot{\mr x})$ holds for every ${\gr g}\in\grZ$.
We say that $\Psi\subseteq L_{\mr x}$ is symmetric if every formula in $\Psi$ is symmetric.
When $\Psi$ is symmetric, the map $\iota:\<{\mr a}\,;{\gr g}\>\mapsto\<{\mr a}\,;{\gr g^{-1}}\>$ is an automorphism of $\U^\Delta$.
As $H$ is normal and $\iota$ is an involution, $\iota\,H=H{\cdot}\iota$ and $H\cup\iota\,H$ is a subgroup of $G$.
In particular, if $f\in H$ and ${\gr g}\in\grZ$ then $f({\gr g^{-1}})=f'({\gr g})^{\gr -1}$ for some $f'\in H$.
We write \emph{$K$\/} for $H\cup\iota\,H$.

When $\Psi$ is symmetric, we can strengthen Fact~\ref{prop_wideHcojugate} as follows.

\begin{fact}\label{prop_wideHcojugate_symm}
  Let $\mrD=\phi(\mrX\,;{\gr 1})$ for some $\phi({\mr x}\,;{\gr z})\in\pmDelta$.
  For $i=1,2$ let ${\gr h_i}\in({\gr g_i})_K$. 
  Then \ssf1 and \ssf2 of Fact~\ref{prop_wideHcojugate} are equivalent
\end{fact}

\begin{proof}
  Let $f_i\in K$ be such that ${\gr h_i}\in f_i{\gr g_i})$.
  By Remark~\ref{rem_invariance_normalsubg} it suffices to consider the case $f_1\in H$ and $f_2\in\iota\,H$.
  Then $f_2({\gr g_2})=f_3({\gr g_2^{-1}})$.
  By Fact~\ref{prop_wideHcojugate}, \ssf2 is equivalent to claiming that ${\gr g_1}{\cdot}\mrD\ \cap\ {\gr g_2^{-1}}{\cdot}\mrD$ is $H$-wide. 
  By symmetry ${\gr g_2^{-1}}{\cdot}\mrD={\gr g_2}{\cdot}\mrD$, so the fact follows.
\end{proof}


For any $\grW\subseteq\grZ$, we write $\<\grW\>$ for the subgroup generated by $\grW$.

\begin{proposition}
  Assume $\Psi$ is symmetric.
  Let $\mrD=\phi(\mrX\,;{\gr 1})$ for some $\phi({\mr x}\,;{\gr z})\in\pmDelta$.
  Let ${\gr g}$ be such that $\mrD\cap{\gr g}{\cdot}\mrD$ is $H$-wide.
  Then $\mrD\cap{\gr g'}{\cdot}\mrD$ is $H$-wide for every ${\gr g'}\in\<\,({\gr g})_{K}^2\>$.
\end{proposition}

\begin{proof}
  Assume inductively that $\mrD\cap{\gr g'}{\cdot}\mrD$ is $H$-wide, where ${\gr g'}\in\<\,({\gr g})_{K}^2\>$.
  Pick two arbitrary ${\gr a},{\gr b}\in({\gr g})_K$.
  We claim that $\mrD\cap{\gr a}{\cdot}{\gr b}{\cdot}{\gr g'}{\cdot}\mrD$ is $H$-wide.
  From the induction hypothesis it follows that ${\gr b}{\cdot}\mrD\cap{\gr b}{\cdot}{\gr g'}{\cdot}\mrD$ is $H$-wide.
  By Fact~\ref{prop_wideHcojugate_symm}, ${\gr a^{-1}}{\cdot}\mrD\cap{\gr b}{\cdot}{\gr g'}{\cdot}\mrD$ is $H$-wide and the claim follows.
\end{proof}
\end{comment}

%%%%%%%%%%%%%%%%%%%%%%%%%%%%%%
%%%%%%%%%%%%%%%%%%%%%%%%%%%%%%
%%%%%%%%%%%%%%%%%%%%%%%%%%%%%%
%%%%%%%%%%%%%%%%%%%%%%%%%%%%%%
%%%%%%%%%%%%%%%%%%%%%%%%%%%%%%
%%%%%%%%%%%%%%%%%%%%%%%%%%%%%%
\section{Connected components}

\def\medrel#1{\parbox[t]{5ex}{$\displaystyle\hfil #1$}}
\def\ceq#1#2#3{\parbox[t]{5ex}{$\displaystyle #1$}\medrel{#2}{$\displaystyle #3$}}

For $\mrD\subseteq\mrX$, the stabilizers of $\mrD$ is $\grW=\{{\gr g}\in\grZ\ :\ {\gr g}{\cdot}\mrD=\mrD\}$, a subgroup of $\grZ$.
% Let \emph{$G$\/} = $\Aut(\UDelta)$.

\begin{definition}\label{def_G0}\noindent\smallskip

  \ceq{\hfill{\gr\Z^0}}{=}{\bigcap\big\{\grW\le\grZ\ :\ \textrm{ stabilizer of a }\BDelta(\grZ)\textrm{-definable set with finite $\grZ$-orbit}\big\}}.

  % \smallskip
  % \ceq{\ssf2.\hfill{\gr\Z^{00}}}{=}{\bigcap\big\{\grW\le\grZ\ :\textrm{ stabilizer of a }\BDelta(\grZ)\textrm{-type-definable set with small $G$-orbit}\big\}}.
\end{definition}

% The following fact is immediate.

% \begin{fact}\label{fact_G0}\noindent\vspace*{0.1ex}

%   \ceq{\ssf2.\hfill{\gr\Z^0}}{=}{\bigcap\big\{\grW\le\grZ\ :\ \BDelta(\mrX\,;\grZ)\textrm{-definable subgroup with finite index}\big\}}.\smallskip

%   % \ceq{\ssf3.\hfill{\gr\Z^0}}{=}{\bigcap\big\{\grW\le\grZ\ :\ \textrm{ stabilizer of a }\BDelta(\mrX,\grZ)\textrm{-definable $\grZ$-generic set}\big\}}.\smallskip
% \end{fact}

% Note that we can replace \textit{definable\/} with \textit{type-definable\/} (essentially, by Exercise~\ref{ex_generic_type_vs_formulas}).

Note that $\gr\Z^0$ is a normal subgroup of $\grZ$.
Note also that every $\grW\le\grZ$ on the r.h.s.\@ has finite index in $\grZ$.

% Recall that \emph{$\bdd A$\/} for the set of type-definable sets with bounded orbit over $A$, equivalently, invariant over $\Autf A$.

% \begin{fact}\noindent\vspace*{0.1ex}
  
%   \ceq{\ssf1.\hfill{\gr\Z^0}}{=}{\bigcap\big\{\grW\le\grZ\ :\ \grW\in\BDelta\textrm{-}\acl^\eq\varnothing\big\}}.\smallskip

%   \ceq{\ssf2.\hfill{\gr\Z^{00}}}{=}{\bigcap\big\{\grW\le\grZ\ :\ \grW\in\BDelta\textrm{-}\bdd\varnothing\big\}}.

% \end{fact}

% \begin{proof}
%   The index of a subgroup is the cardinality of its $\grZ$-orbit. 
%   By Remark~\ref{rem_H} the $\grZ$-orbit of an $\BDelta(\grZ)$-type-definable set coincides with its $G$-orbit.
% \end{proof}

\begin{proposition}\label{prop_Z0_inclusione1}
  % Assume $\gamma_{\Z^0}({\mr x})$ finitely consistent.
  Then\smallskip         

  \ceq{\hfill{\gr\Z^0}}{\subseteq}{\big\{{\gr g}\in\grZ\ :\ \mrD\cap{\gr g}{\cdot}\mrD\neq\varnothing\textrm{ for every }\BDelta(\grZ)\textrm{-definable }\grZ\textrm{-generic set }\mrD\big\}}.
 
  % Let ${\gr g}$ be such that $\ $ for every $\grZ$-generic $\BDelta(\grZ)$-definable set $\mrD$.
  % Then ${\gr g}\in{\gr\Z^0}$.
\end{proposition}

\begin{proof}
  
  % We apply Proposition~\ref{prop_Ggeneric_Hpersistent} with $H=\gr\Z^0$, $G=\grZ$, and $\Delta=\BDelta_{\;{\mr x},{\gr z}}$.
  % It follows that if $\mrD$ is $\grZ$-generic then it is $\gr\Z^0$-wide.
  % In particular, $\mrD$ is $\gr\Z^0$-persistent, hence $\mrD\cap{\gr g}{\cdot}\mrD\neq\varnothing$ for every ${\gr g}\in\gr\Z^0$.
\end{proof}

\begin{proposition}\label{prop_Z0_inclusione2}
  Assume $\mrX=\grZ$.
  Then\smallskip

  \ceq{\hfill{\gr\Z^0}}{\supseteq}{\big\{{\gr g}\in\grZ\ :\ \mrD\cap{\gr g}{\cdot}\mrD\neq\varnothing\textrm{ for every }\BDelta(\grZ)\textrm{-definable }\grZ\textrm{-generic set }\mrD\big\}}.
  
\end{proposition}

% We will revisit this proposition under the assumption of stability and prove the converse inclusion.
% See Theorem~\ref{thm_Z0_stable}.\vspace*{-0.5\baselineskip}

\begin{proof}
  Let ${\gr g}$ be an element of the set on the r.h.s.
  Let $\grW$ be as in Definition~\ref{def_G0}.
  % A fortiori $\grW$ is the stabilizer of a set with finite  $\grZ$-orbit.
  Then $\grW$ has finite index in $\grZ$.
  Then $\grW$ (viewed as an $\BDelta(\grZ)$-definable subset of $\mrX$) is $\grZ$-generic.
  Then $\grW\cap{\gr g}{\cdot}\grW\neq\varnothing$; which implies ${\gr g}\in\grW$.
 \end{proof}

%%%%%%%%%%%%%%%%%%%%%%%%%%
%%%%%%%%%%%%%%%%%%%%%%%%%%
%%%%%%%%%%%%%%%%%%%%%%%%%%
%%%%%%%%%%%%%%%%%%%%%%%%%%
%%%%%%%%%%%%%%%%%%%%%%%%%%
\section{Notes and references}

In Example~\ref{ex_newelski} we prove a theorem of Newelski's~\cite{Newelski}.
The original proof is rather long and complex.
A simplified proof (also due, reportedly, to Newelski) appears in~\cite{Pelaez}*{Section 3.3} and~\cite{Casanovas}*{Chapter 9}.
The proof here is a streamlined and generalized version of the latter~--~inspired by~\cite{Z16}.

\begin{biblist}[]\normalsize
\bib{Casanovas}{book}{
  author={Casanovas, Enrique},
  title={Simple theories and hyperimaginaries},
  series={Lecture Notes in Logic},
  volume={39},
  publisher={Cambridge
  University Press},
  date={2011},
  % pages={xiv+169},
  % isbn={978-0-521-11955-9},
  % review={\MR{2814891}},
  % doi={10.1017/CBO9781139003728},
}\smallskip
\bib{CK}{article}{
  author={Chernikov, Artem},
  author={Kaplan, Itay},
  title={Forking and dividing in ${\rm NTP}_2$ theories},
  journal={J. Symbolic Logic},
%  volume={77},
  date={2012},
%  number={1},
%  pages={1--20},
%  issn={0022-4812},
%  review={\MR{2951626}},
%  doi={10.2178/jsl/1327068688},
}\smallskip
\bib{Hr}{article}{
  label={Hr},
  author={Hrushovski, Ehud},
  title={Stable group theory and approximate subgroups},
  journal={J. Amer. Math. Soc.},
  volume={25},
  date={2012},
  number={1},
  pages={189--243},
  % issn={0894-0347},
  % doi={10.1090/S0894-0347-2011-00708-S},
}\smallskip
\bib{Newelski}{article}{
  author={Newelski, Ludomir},
  title={The diameter of a Lascar strong type},
  journal={Fund. Math.},
%  volume={176},
  date={2003},
%  number={2},
%  pages={157--170},
%  issn={0016-2736},
%  review={\MR{1971306}},
%  doi={10.4064/fm176-2-4},
}\smallskip
\bib{Pelaez}{book}{
  author={Pel\'aez, Rodrigo},
  title={\href{http://www.ub.edu/modeltheory/documentos/ThesisRPP.pdf}{About the Lascar group}},
  series={PhD Thesis},
  publisher={Universitat de Barcelona, Departament de L\'ogica, Hist\'oria i Filosofia de la Ci\'encia},
  date={2008},
  }\smallskip
\bib{Z16}{article}{
  author={Zambella, Domenico},
  title={On the diameter of Lascar strong types after Ludomir Newelski},
  conference={
  title={A tribute to Albert Visser},
  },
  book={
  % series={Tributes},
  % volume={30},
  publisher={Coll. Publ., [London]},
  },
  date={2016},
  status={\href{https://arxiv.org/abs/1605.00218}{arXiv:1605.00218}},
  %  pages={231--236},
  %  review={\MR{3559880}},
}\smallskip
\end{biblist}

  