% !TEX root = creche.tex
\chapter{Actions of groups}
\label{actions}

\def\medrel#1{\parbox[t]{5ex}{$\displaystyle\hfil #1$}}
\def\ceq#1#2#3{\parbox[t]{17ex}{$\displaystyle #1$}\medrel{#2}{$\displaystyle #3$}}

In this chapter, $L$ is a signature, $T$ is a complete theory without finite models, and $\U$ is a saturated model of inaccessible cardinality $\kappa$ strictly larger than $|L|$.
We use the same notation and make the same implicit assumptions as in Section~\ref{monster}.

%%%%%%%%%%%%%%%%%%%%%%%%%%%%%%%%%%%%%%%%%%%%
%%%%%%%%%%%%%%%%%%%%%%%%%%%%%%%%%%%%%%%%%%%%
%%%%%%%%%%%%%%%%%%%%%%%%%%%%%%%%%%%%%%%%%%%%
%%%%%%%%%%%%%%%%%%%%%%%%%%%%%%%%%%%%%%%%%%%%
\section{The dual perspective on invariance}\label{dual_perspective}

The notions of invariance introduceted in the previous chapters extend seemlessly to the action of a group $G$ on $\U^{\gr z}$.
Let $\grD\subseteq\U^{\gr z}$.
We say that $\grD$ is \emph{invariant\/} under the action of $G$, or \emph{$G$-invariant,} if it is fixed setwise by $G$.
That is, $g\grD=\grD$ for every $g\in G$.
Yet in other words, if

\ceq{\ssf{is1.}\hfill {\gr a}\in\grD}{\iff}{g{\gr a}\in\grD}\hfill for every ${\gr a}\in\U^{\gr z}$ and every $g\in G$.

A formula is invariant if the set it defines is invariant.
We say that $p({\mr x})\subseteq L(\U)$ is \emph{invariant\/} under the action of $G$, or \emph{$G$-invariant,} if for every formula $\phi({\mr x}\,;{\gr z})\in L$ 

\ceq{\ssf{it1.}\hfill\phi({\mr x}\,;{\gr a})\in p}{\IFF}{\phi({\mr x}\,;g{\gr a})\in p}\hfill for every ${\gr a}\in\U^{\gr z}$ and every $g\in G$.

It should be clear that invariant under the action of $\Aut(\U/A)$ coincides with invariant over $A$ and Lascar invariant over $A$ coincides with invariant under the action of $\Autf(\U/A)$.

Note that $p({\mr x})$ is invariant exactly when the sets $\gr\D_{p,\phi}\subseteq\U^{\gr z}$ are.
In this section we want to discuss invariance using sets $\mrD\subseteq\U^{\mr x}$.
We need stronger assumptions on the action of $G$.
But first we introduce a more flexible framework.

In this chapter $\Delta\subseteq L_{{\mr x}\,{\gr z}}(\U)$, ${\mrX}\subseteq\U^{\mr x}$, and ${\grZ}\subseteq\U^{\gr z}$ are some arbitrary nonempty sets (at some point we will require that $\mrX$ and $\grZ$ are type-definable).
We write $L_\Delta({\grZ})$ for the set of formulas $\theta({\mr x})$ that are Boolean combination of formulas $\phi({\mr x}\;{\gr b})$ for some ${\gr b}\in{\grZ}$.
Such formulas a called $\Delta$-formulas.
A (relatively) $\Delta$-definable set is a set of the form $\theta(\mrX)$ for some $\Delta$-formula $\theta({\mr x})$.
Subsets of $L_\Delta({\grZ})$ are called $\Delta$-types.
We write $S_\Delta({\grZ})$ for the set of complete $\Delta$-types.

\begin{assumption}\label{notation_GXphi}
  Let $G$ be a group that acts on $\mrX$ and on $\grZ$.
  We require that for every $\phi({\mr x}\,;{\gr z})\in\Delta$ the set $\phi(\mrX;\grZ)$ is invariant under the action of $G$.
\end{assumption}

An immediate consequence of Assumption~\ref{notation_GXphi} is that any $G$-translate of a $\Delta$-definable set is again $\Delta$-definable.
In particular, by reasoning as in Remark~\ref{rem_image_def_set} we obtain that for every $\Delta$-formula $\theta({\mr x}\,;{\gr\bar b})$ and every $g\in G$

\ceq{\hfill g[\theta(\mrX\,;{\gr\bar b})]}{=}{\theta(\mrX\,;g{\gr\bar b}).}

Therefore $p({\mr x})\subseteq L_\Delta(\grZ)$ is invariant if

\ceq{\hfill p({\mr x})\proves{\mr x}\in\mrD}{\IFF}{p({\mr x})\proves{\mr x}\in g\mrD}\hfill for every $\Delta$-definable $\mrD\subseteq\mrX$ and $g\in G$.

At the first reading the reader may assume that $G=\Aut(\U)$, $\mrX=\U^{\mr x}$, $\grZ=\U^{\gr z}$ and $\Delta=L_{{\mr x}\,;{\gr z}}$, where $|{\gr z}|=\omega$.
Then $L_\Delta(\grZ)=L_{\mr x}(\U)$ and  $S_\Delta(\grZ)=S_{\mr x}(\U)$. 
% Note that when $G$ acts by automorphisms $\phi(\U^{\mr x};\U^{\gr z})$ is invariant for all $\phi({\mr x}\,;{\gr z})\in L$.

A set $\mrD\subseteq\mrX$ is \emph{generic\/} under the action of $G$, or \emph{$G$-generic\/} for short, if finitely many $G$-translates of $\mrD$ cover $\mrX$; we say \emph{$n$-$G$-generic\/} if $\le n$ translates suffices.
Dually, we say that $\mrD$ is \emph{persistent\/} under the action of $G$, or \emph{$G$-persistent\/} for short, if the intersection of any finitely many $G$-translates of $\mrD$ is nonempty; we say \emph{$n$-$G$-persistent\/} when the request is limited to $\le n$ translates.
We may drop reference to $G$ when it is clear from the context.

The same properties may be attributed to formulas (as these are identified with the set they define).
When these properties are attributed to a type $p({\mr x})$, we understand that they hold for every conjunction of formulas in $p({\mr x})$.
See Exercise~\ref{ex_persistent_types} for an alternative characterization when $p({\mr x})$ is small.

\noindent\llap{\textcolor{red}{\Large\warning}\kern1.5ex}\ignorespaces
The terminology above non-standard.
In~\cite{CK} the authors write \textit{quasi-non-dividing\/} for \textit{persistent\/} when $G=\Aut(\U/A)$.
Their terminology has good motivations, but it would be a mouthful if adapted to our context.

\begin{example}
  If $p({\mr x})\subseteq L(\U)$ is finitely satisfiable in $A$ then $p({\mr x})$ is persistent (in any $\mrX\supseteq A^{\mr x}$) under the action of $\Aut(\U/A)$.
  In fact, the same ${\mr a}\in A^{\mr x}$ that satisfies $\phi({\mr x})$ also satisfies every $\Aut(\U/A)$-translate of $\phi({\mr x})$.
\end{example}

In this chapter many proofs require some juggling with negations.

\begin{fact}\label{fact_fip}
  (Assume~\ref{notation_GXphi})\ \  
  The following are equivalent
  \begin{itemize}
    \item[1.] $\mrD$ is not $n$-generic
    \item[2.] $\neg\mrD$ is $n$-persistent.
  \end{itemize}
\end{fact}

\begin{proof}
  Immediate by spelling out the definitions\smallskip
  \begin{itemize}
    \item[1.] there are no $g_1,\dots,g_n\in G$ such that \smash{$\displaystyle\mrX\ \subseteq\ \bigcup_{i=1}^n g_i{\mr\D}$}
    \item[2.]  $\displaystyle\0\ \neq\ \mrX\,\cap\,\bigcap_{i=1}^n\neg g_i{\mr\D}$ for every $g_1,\dots,g_n\in G$.\qedhere
  \end{itemize} 
\end{proof}

\begin{theorem}\label{thm_generic_invariant}
  (Assume~\ref{notation_GXphi})\ \  
  Let $p({\mr x})\in S_\Delta(\grZ)$ be finitely satisfiable in $\mrX$.
  Then the following are equivalent
  \begin{itemize}
    \item[1.] $p({\mr x})$ is invariant
    \item[2.] $p({\mr x})\proves{\mr x}\in\mrD$ for every generic $\Delta$-definable set $\mrD$
    \item[5.] $p({\mr x})$ is persistent.
  \end{itemize}
\end{theorem}

\begin{proof}
  \ssf1$\IMP$\ssf2.
  Let $g_1{\mr\D},\dots,g_n{\mr\D}$ be translations of $\mrD$ that cover $\mrX$.
  Negate \ssf2.
  By completeness, $p({\mr x})\proves {\mr x}\notin\mrD$.
  Hence, from invariance we obtain

  \ceq{\hfill p({\mr x})}{\proves}{{\mr x}\ \notin\ \bigcup_{i=1}^ng_i\mrD.}

  This contradicts the finite satisfiability of $p({\mr x})$ in $\mrX$.
  
  \ssf2$\IMP$\ssf3.
  Let $\mrD$ be defined by a conjunction of formulas in $p({\mr x})$.
  If $\mrD$ is not persistent then, by Fact~\ref{fact_fip}, $\neg\mrD$ is generic. 
  By \ssf2, $p({\mr x})\proves{\mr x}\notin\mrD$, a contradiction.

  \ssf3$\IMP$\ssf1.
  % uppose $p({\mr x})$ is not $(n+2)$-persistent and let $n$ be minimal.
  % Let $g_1,\dots,g_{n+2}\in G$ and $\theta({\mr x})\,;{\gr\bar b}$ witness this.
  % Then 
  If $p({\mr x})$ is not invariant then, by completeness, $p({\mr x})\proves\phi({\mr x}\,;{\gr b})\wedge\neg\phi({\mr x}\,;g{\gr b})$ for some $g\in G$.
  Clearly $\phi({\mr x}\,;{\gr b})\wedge\neg\phi({\mr x}\,;g{\gr b})$ is not persistent as it is inconsistent with its $g$-translate.
\end{proof}

The theorem yields an immediate necessary condition for the existence of an invariant global $\Delta$-type.

\begin{corollary}
  (Assume~\ref{notation_GXphi})\ \  
  If there exists an invariant global $\Delta$-type then for every $\Delta$-definable set $\mrD$
  \begin{itemize}
    \item[1.] $\mrD$ and $\neg\mrD$ cannot be both generic
    \item[2.] if $\mrD$ is generic than it is persistent.
  \end{itemize}
\end{corollary}

\begin{proof}
  By Fact~\ref{fact_fip}, \ssf1 and \ssf2 are equivalent; \ssf1 is an immediate consequenc of \ssf2 of Theorem~\ref{thm_generic_invariant}.
\end{proof}

The following theorem gives a necessary and sufficient condition for the  existence of global invariant $\Delta$-type.
Ideally, we would like to prove that every persistent $\Delta$-type extends to a global persitent type.
Unfortunately this is not true -- we need a stronger property.
A $\Delta$-definable set $\mrD$ is \emph{hereditarely persistent\/} if every finite cover of $\mrD$ by $\Delta$-definable sets contains a persistent set.
In~\cite{CK} a similar property is called \textit{quasi-non-forking,} but we persist in the unorthodoxy.
A type is hereditarely persistent if every conjunction of formulas in the type is hereditarely persistent.

\begin{theorem}\label{thm_generic_invariant2}
  (Assume~\ref{notation_GXphi})\ \  
  Let $q({\mr x})\subseteq L(\U)$.
  Then the following are equivalent 
  \begin{itemize}
    \item[1.] $q({\mr x})$ extends to an invariant type $p({\mr x})\in S_\Delta(\grZ)$ finitely satisfiable in $\mrX$
    \item[2.] $q({\mr x})$ is hereditarely persistent.
  \end{itemize}
\end{theorem}

\begin{proof}
  \ssf1$\IMP$\ssf2.
  Let $\theta({\mr x})$ be a conjunction of formulas in $q({\mr x})$.
  Suppose ${\mr\C_1},\dots,{\mr\C_n}$ cover $\theta(\U^{\mr x})$ and pick $p({\mr x})$ as in \ssf1.
  By completeness, $p({\mr x})\proves {\mr x}\in{\mr\C_i}$ for some $i$.
  Then, by Theorem~\ref{thm_generic_invariant}, $\neg{\mr\C_i}$ is not generic.
  Therefore, by Fact~\ref{fact_fip}, ${\mr\C_i}$ is persistent.

  \ssf2$\IMP$\ssf1.
  Let $p({\mr x})$ be maximal among the $\Delta$-types that contain $q({\mr x})$ and are such that $\theta(\U^{\mr x})$ is hereditarely persistent for every $\theta({\mr x})$ that is conjunction of formulas in $p({\mr x})$.
  We claim that $p$ is a complete $\Delta$-type.
  Suppose for a contradiction that $\theta({\mr x}),\neg\theta({\mr x})\notin p$.
  By maximality there is some formula $\psi({\mr x})$, a conjunction of formulas in $p({\mr x})$ and some ${\mr\C_1},\dots,{\mr\C_n}$ that cover both $\psi(\U^{\mr x})\cap\theta(\U^{\mr x})$ and $\psi(\U^{\mr x})\smallsetminus\theta(\U^{\mr x})$ and such that no ${\mr\C_i}$ is persistent.
  As ${\mr\C_1},\dots,{\mr\C_n}$ cover $\psi(\U^{\mr x})$ this is a contradiction.
  It is only left to show that $p({\mr x})$ is finitely satisfiable in $\mrX$ and invariant.
  Finite satisfiability follows from persistency.
  From completeness and Theorem~\ref{thm_generic_invariant} we obtain invariance.
\end{proof}

\begin{exercise}
  Prove that if $p({\mr x})\in S(\U)$ is finitely satisfiable in every $M\supseteq A$ then it is persistent under the action of $\Autf(\U/A)$.
  Is the same true for incomplete types?
\end{exercise}

\begin{exercise}
  Let $p(x)\in S(\U)$ be persistent.
  Prove that the following are equivalent
  \begin{itemize}
    \item[1.] $p(x)$ is invariant and finitely satisfiable in $\X$
    \item[2.] $p(x)\proves x\in\D$ for every 2-generic definable set $\D$
    \item[3.] $p(x)$ is 2-persistent.
  \end{itemize}
\end{exercise}

\begin{exercise}
  Let $\D\subseteq\U^x$ be $\Delta$-definable.
  Prove that the following are equivalent 
  \begin{itemize}
    \item[1.] there is an invariant type $p(x)\in S_\Delta(\grZ)$ finitely satisfiable in $\X\cap\D$
    \item[2.] every finite cover of $\D$ by $\Delta$-definable sets contains a 2-persistent set.
  \end{itemize}
\end{exercise}

\begin{exercise}\label{ex_gen_sat}
  Let $p(x)\subseteq L_\phi(\U)$ be generic.
  Prove that it is finitely satisfiable.
\end{exercise}

\begin{exercise}\label{ex_persistent_types}
  Prove that for every $p(x)\subseteq L_\phi(A)$ following are equivalent
  \begin{itemize}
    \item[1.] $p(x)$ is persistent
    \item[2.] $p(\U^x)$ is persistent.
  \end{itemize}
\end{exercise}

%%%%%%%%%%%%%%%%%%%%%%%%%%
%%%%%%%%%%%%%%%%%%%%%%%%%%
%%%%%%%%%%%%%%%%%%%%%%%%%%
%%%%%%%%%%%%%%%%%%%%%%%%%%
%%%%%%%%%%%%%%%%%%%%%%%%%%
\section{Strong genericity}\label{strong_genericity}

Unfortunatelly, genericy is not preserved under intersection.
To obtain closure under intersection, we need to push the concept to a higher level of complexity.
This is only required in Section~\ref{newelski}.

A set $\mrD\subseteq\U^{\mr x}$ is \emph{strongly generic\/} if the intersection of $\mrD$ with any of its tranlations $\mrD$ is generic.
Dually, we say that $\mrD$ is \emph{weakly persistent\/} if the union of $\mrD$ with one of its translations is peristent.
Again, the same properties may be attributed to formulas and types.

Notation: for $\mrB\subseteq\mrX$ and $H\subseteq G$ we write $H\,\mrB$ for $\{h\mrB: h\in H\}$.

\begin{lemma}\label{lem_strongly_generic}
  (Assume~\ref{notation_GXphi})\ \  
  The intersection of strongly generic sets is strongly generic.
\end{lemma}

\begin{proof}
  We may assume that all sets mentioned below are subsets of $\mrX$.
  Let ${\mr\D}$ and ${\mr\C}$ be strongly generic and let $K\subseteq G$ be an arbitrary finite set.
  It suffices to prove that $\mrB=\cap\, K\,({\mr\C}\cap{\mr\D})$ is generic. 
  Clearly $\mrB={\mr\C'}\cap{\mr\D'}$, where ${\mr\C'}=\cap\, K\,{\mr\C}$ and ${\mr\D'}=\cap\, K\,{\mr\D}$.
  Note that ${\mr\C'}$ and ${\mr\D'}$ are both strongly generic.
  In particular $\mrX=\cup\,H\,\mr\D'$ for some finite $H\subseteq G$.
  Now, from
  
  \ceq{\hfill\cup\,H\,\mrB}{=}{\cup\,H\Big[{\mr\C'}\ \cap\ {\mr\D'}\Big]}

  \ceq{\hfill\cup\,H\,\mrB}{\supseteq}{\cup\,H\Big[\big(\cap\, H\,{\mr\C'}\big)\ \cap\ {\mr\D'}\Big]}
  
  \ceq{}{=}{ \big(\cap\, H\,{\mr\C'}\big)\ \cap\ \big(\cup\,H\,{\mr\D'}\big)}
  
  \ceq{}{=}{\cap\, H\,{\mr\C'}}
  
  As ${\mrC'}$ is strongly generic, $\cap\, H\,{\mr\C'}$ is generic.
  Therefore $\cup\,H\,\mrB$ is also generic.
  The genericity of $\mrB$ follows.
\end{proof}

\begin{corollary}\label{corol_str_gen}
  (Assume~\ref{notation_GXphi})\ \  
  Let $q({\mr x})=\{\theta({\mr x})\in L_\phi(\U)\,:\, \theta({\mr x})\textrm{ strongly generic}\}$.
  Then $q({\mr x})$ is finitely satisfiable in $\mrX$, strongly generic, and invariant.
\end{corollary}

\begin{proof}
  Strong genericity is an immediate consequence of Lemma~\ref{lem_strongly_generic}.
  Finite satisfiability is a consequence of genericity, see Exercise~\ref{ex_gen_sat}.
  As for invariance, note that any translate of a strongly generic formula is also strongly generic.
\end{proof}

\begin{corollary}\label{corol_q_w_pers}
  (Assume~\ref{notation_GXphi})\ \  
  Let $q({\mr x})$ be as in Corollary~\ref{corol_str_gen}.
  Let $p({\mr x})\subseteq L(\U)$ be such that $p({\mr x})\cup q({\mr x})$ is finitely satisfied in $\mrX$.
  Then $p({\mr x})$ is weakly persistent.
\end{corollary}

\begin{proof}
  Let $\theta({\mr x})\in p$.
  As $q({\mr x})$ is finitely satisfiable in $\theta(\U^{\mr x})$, we cannot have that $\neg\theta({\mr x})$ is strongly generic.
  Reasoning as for Fact~\ref{fact_fip} we obtain that for every set $\mrD$ the following are equivalent
  \begin{itemize}
    \item[1.] $\mrD$ is not strongly generic
    \item[2.] $\neg\mrD$ is weakly persistent.
  \end{itemize}
  By applying this equivalence with $\neg\theta(\U^{\mr x})$ for $\mrD$, we conclude that $\theta({\mr x})$ weakly persistent.
\end{proof}

\begin{exercise}
  Show that generic sets need not be closed under intersection.
  Hint: try e.g.\@ with the cyclic order in Example~\ref{ex_cyclic_order}.
\end{exercise}

\begin{exercise}
  Prove that the following are equivalent
  \begin{itemize}
    \item[1.] every persistent $\phi$-formula is hereditarely persistent
    \item[2.] for every $\Delta$-definable sets $\C$ and $\D$, if $\C$ is not generic then $\D\cup\C$ and $\neg\D\cup\C$ cannot be both generic.
  \end{itemize}
\end{exercise}


%%%%%%%%%%%%%%%%%%%%%%%%%%%%%%%%%%%
%%%%%%%%%%%%%%%%%%%%%%%%%%%%%%%%%%%
%%%%%%%%%%%%%%%%%%%%%%%%%%%%%%%%%%%
%%%%%%%%%%%%%%%%%%%%%%%%%%%%%%%%%%%
%%%%%%%%%%%%%%%%%%%%%%%%%%%%%%%%%%%
\section{The diameter of a Lascar type}\label{newelski}

The Lascar strong type $\Ll({\mr a}/A)$ is the union of a chain of type-definable sets of the form $\big\{{\mr x}\ :\ d_A({\mr a},{\mr x})\le n\big\}$.
In this section we prove that $\Ll({\mr a}/A)$ is type-definable if and only this chain is finite.
In other words, if the connected component of $a$ in the Lascar graph has finite diameter.
It is convenient to address the problem in more general terms.

Let $G$ be a normal subgroup of $\Aut(\U)$.
Assume $G$ is the union of a countable chain of sets $\<G_n:n\in\omega\>$ with the following properties
\begin{itemize}
  \item[1.] every $G_n$ is symmetric i.e.\@ it contains the unit and is closed under inverse
  \item[2.] every $G_n$ is conjugancy invariant i.e.\@ $g\,G_ng^{-1}=G_n$ for every $g\in G$
  \item[3.] $G_nG_m\subseteq  G_{n+m}$ for every $n,m\in\omega$.
\end{itemize}

Assume $G$ acts transitively on $\mrX$ i.e., $G\,{\mr a}=\mrX$ for every ${\mr a}\in\mrX$.
We define a discrete metric on $\mrX$.
For ${\mr a},{\mr b}\in\mrX$ let $d({\mr a},{\mr b})$ be the minimal $n$ such that ${\mr a}\in G_n{\mr b}$.
This defines a metric by \ssf1 and \ssf3.
By \ssf2, this metric in $G$-invariant.
The \emph{diameter\/} of a set $\mrC\subseteq\mrX$ is the supremum of $d({\mr a},{\mr b})$ for ${\mr a},{\mr b}\in\mrC$.

\begin{example}
  Let $G$ be $\Autf(\U/A)$ and $\mrX=\Ll({\mr a}/A)$.
  If we let $G_1$ be the set of automorphisms that fix a model $M\supseteq A$ and $G_n=G_1^n$, then $d({\mr a},{\mr b})$ concides with the dinstance in the Lascar graph.
\end{example} 

We are interested in sufficient conditions for $\mrX$ to have finite diameter.
The notions introduced in Section~\ref{strong_genericity} offer some hints.

\begin{proposition}\label{prop_wpers_finite_diameter}
  If $\mrX$ has a weakly persistent subset of finite diameter, then $\mrX$ itself has finite diamenter.
\end{proposition}

\begin{proof}
  Let $\mrC\subseteq\mrX$ be a weakly persistent set of diameter $n$.
  Let $H\subseteq G$ be finite such that $\cup\,H\,\mrC$ is persistent.
  We claim that also $\cup\,H\,\mrC$ has finite diameter.
  Let ${\mr a}\in\mrC$ be arbitrary.
  Let $m$ be larger than $d(h{\mr a}, k{\mr a})$ for all $h,k,\in H$.
  Now, let $h{\mr b}$ and $k{\mr c}$, for some $h,k,\in H$ and ${\mr b},{\mr c}\in\mrC$, be two arbitrary elements of $\cup\,H\,\mrC$.
  As $h\mrC$ and $k\mrC$ have the same diameter of $\mrC$, 

  \ceq{\hfill d(h{\mr b},\, k{\mr c})}{\le}{d(h{\mr b},\,h{\mr a})\ +\ d(h{\mr a},\, k{\mr a})\ +\ d(k{\mr a},\,k{\mr c})}

  \ceq{}{\le}{n+m+n.}

  This proves that $\cup\,H\,\mrC$ has finite diameter.
  Therefore, without loss of generality, we may assume that $\mrC$ itself is persistent.
  
  By the transitivity of the action, any two elements of $\mrX$ are of the form $h{\mr a}$, $k{\mr a}$ for some $h,k\in G$ and some ${\mr a}\in\mrC$.
  By percistency, there are ${\mr c}\in\mrC\cap h\mrC$ and ${\mr d}\in\mrC\cap k\mrC$.
  Then 

  \ceq{\hfill d(h{\mr a},\, k{\mr a})}{\le}{d(h{\mr a},\,{\mr c})\ +\ d({\mr c},\, {\mr d})\ +\ d({\mr d},\,k{\mr a})}

  \ceq{}{\le}{n+n+n.}

  Therefore the diameter of $\mrX$ does not exceed $3n$.
\end{proof}

\begin{theorem}
  Suppose that $\mrX$ and the sets ${\mr\X_n}=G_n{\mr a}$, for some ${\mr a}\in\mrX$, are type-definable.
  Then $\mrX$ has finite diameter.
\end{theorem}

\begin{proof}
  By Proposition~\ref{prop_wpers_finite_diameter}, it suffices to prove that ${\mr\X_n}$ is weakly persistent.
  Define

  \ceq{\hfill q({\mr x})}{=}{\{\theta({\mr x})\in L(\U)\ :\ \theta({\mr x})\textrm{ strongly generic}\}.}
  
  By Corollary~\ref{corol_q_w_pers}, with $L$ for $L_\phi$, it suffices to prove that for some $n$ the type $q({\mr x})$ is finitely satisfied in $\mr\X_n$.
  Suppose not.
  Let $\psi_n({\mr x})\in q$ be a formula that is not satisfied in $\mr\X_n$.
  The type $\{\psi_n({\mr x}):n\in\omega\}$ is finitely satisfied in $\mrX$.
  Then it has a realization in $\mrX$. 
  As this realization belongs to some ${\mr\X_n}$ this contradicts the definition of $\psi_n({\mr x})$. 
\end{proof}

% \ceq{\hfill p_n({\mr\X},{\mr\X})}{=}{\{\<{\mr a},{\mr b}\>\in{\mr\X^2}\ :\ {\mr a}\in G_n{\mr b}\}}

% Below we write ${\mr x}\in G_n{\mr y}$ for the type $p_n({\mr x},{\mr y})$ and ${\mr x}\in G\,{\mr y}$ for the disjunction of all these types (n.b.\@ an infinite disjunction of types need not be a type).

%%%%%%%%%%%%%%%%%%%%%%%%%%%%%%%%%%%
%%%%%%%%%%%%%%%%%%%%%%%%%%%%%%%%%%%
%%%%%%%%%%%%%%%%%%%%%%%%%%%%%%%%%%%
%%%%%%%%%%%%%%%%%%%%%%%%%%%%%%%%%%%
%%%%%%%%%%%%%%%%%%%%%%%%%%%%%%%%%%%
\section{A simpler landscape}

Under suitable assumptions (e.g.\@ the sability of $\phi({\mr x}\,;{\gr z})$) some of the notions introduced in this chapter coalesce and we are left with cleaner theory.
We prove the following theorem.

\begin{theorem}\label{thm_coalesce}
  (Assume~\ref{notation_GXphi})\ \  
  The following are equivalent
  \begin{itemize}
    \item[1.] persistent $\Delta$-definable sets are hereditarely persistent
    \item[2.] generic $\Delta$-definable sets are strongly generic
    \item[3.] generic $\Delta$-definable sets are closed under intersection
    \item[4.] weakly persisent $\Delta$-definable sets are persistent.
  \end{itemize}
\end{theorem}

\begin{proof}
  \ssf1$\IMP$\ssf2.
  It suffices to prove that generic sets are closed under intersection.
  Let $\mrC$ and $\mrD$ be generic $\Delta$-definable sets.
  Suppose for a contradiction that $\mrC\cap\mrD$ is not generic.
  By \ssf1 and Theorem~\ref{thm_generic_invariant2} there is an invariant global $\Delta$-type $p({\mr x})$ containing ${\mr x}\in\neg\mrC\cup\neg\mrD$.
  By completeness either $p({\mr x})\proves{\mr x}\in\neg\mrC$ or $p({\mr x})\proves{\mr x}\in\neg\mrD$.
  This is a contradiction because by Theorem~\ref{thm_generic_invariant} $p({\mr x})\proves{\mr x}\in\mrC$ and $p({\mr x})\proves{\mr x}\in\mrD$.

  \ssf2$\IFF$\ssf3$\IFF$\ssf4. Clear.

  \ssf4$\IMP$\ssf1.
  Let $q({\mr x})=\{\theta({\mr x})\in L_\phi(\U)\,:\,\theta({\mr x})\textrm{ generic}\}$.
  By \ssf2 this is the same type defined in Corollary~\ref{corol_q_w_pers}.
  Therefore, any completion of $q({\mr x})$ is, by \ssf4, persistent.
  Let $\mrD$ be a persistent $\Delta$-definable set.
  By Theorems~\ref{thm_generic_invariant} and~\ref{thm_generic_invariant2} it suffices to show that $\mrD$ is consistent with $q({\mr x})$.
  Suppose not, then $q({\mr x})\proves {\mr x}\in\neg\mrD$.
  Therefore, by \ssf3, $\neg\mrD$ is generic.
  This is a contradiction by Fact~\ref{fact_fip}.
\end{proof}


%%%%%%%%%%%%%%%%%%%%%%%%%%
%%%%%%%%%%%%%%%%%%%%%%%%%%
%%%%%%%%%%%%%%%%%%%%%%%%%%
%%%%%%%%%%%%%%%%%%%%%%%%%%
%%%%%%%%%%%%%%%%%%%%%%%%%%
\section{Definable groups}

In this section $G=\grZ$ is type-definable over some set $A$.
Also $\mrX\subseteq\U^{\mr x}$ is type-definable over $A$.
The group operations and the group action are definable.
Clearly, $\grZ$ also acts on itself by left multiplication.
We use the symbol $\,\cdot\,$ for both the group multiplication and the group action.

Let $\psi({\mr x}\,;y)\in L(A)$.
We write $\phi({\mr x}\,;{\gr z}\,;y)$ for the formula $\psi({\gr z^{-1}}\cdot{\mr x}\,;y)$.
In this section $\Delta$ contains the formulas $\phi({\mr x}\,;{\gr z}\,;a)$ where $a$ ranges over the realizations of a given $q(y)\in S(A)$.
Note that $\phi(\mrX\,;\grZ\,;a)$ is invariant under the action of $\grZ$ for every $a$, therefore Assumption~\ref{notation_GXphi} is satisfied.

Let ${\gr 1}$ be the identity of $\grZ$ which, for simplicity, we assume is a constant of $L$.

\begin{fact}
  There are some formulas $\gamma(y),\pi(y)\in L(A)$ such that, for every $a\in\U^y$

  \ceq{\hfill\gamma(a)}{\IFF}{\phi({\mr x}\,;{\gr 1}\,;a)\textrm{ is generic}}

  \ceq{\hfill\pi(a)}{\IFF}{\phi({\mr x}\,;{\gr 1}\,;a)\textrm{ is persistent}.}

  Similar claims holds for $\neg\phi({\mr x}\,;{\gr 1}\,;a)$.
\end{fact}

\begin{proof}
  By an easy argument of compactness, there is an $n$ such that if $\phi({\mr x}\,;{\gr 1}\,;a)$ is generic then it is also $n$-generic.
  Then

  \ceq{\hfill\gamma(y)}{=}{\E{\gr z_1},\dots,{\gr z_n}\ \A{\mr x}\bigvee_{i=1}^n\phi({\mr x}\,;{\gr z_i}\,;y)}

  The other claims follow easily.
\end{proof}

Note that $\phi(\mrX\,;{\gr g}\,;a)={\gr g}\cdot\phi(\mrX\,;{\gr 1}\,;a)$. 
Therefore $\phi({\mr x}\,;{\gr g}\,;a)$ is generic/persistent if and only if $\phi({\mr x}\,;{\gr 1}\,;a)$ is generic/persistent.
It follows that $\Aut(\U/A)$ maps generic/persistent sets of the form $\phi(\mrX\,;{\gr g}\,;a)$ to generic/persistent sets of the same form.
% The same is true for generic/persistent types $p({\mr x})\in S_\Delta(\grZ)$ in fact, up to equivalence, we may assume that they contain only formulas of the form $\phi({\mr x}\,;{\gr g}\,;a)$ and negation thereof.

\begin{fact}
  If $p({\mr x})\in S_\Delta(\grZ)$ is persistent (equivalently, $\grZ$-invariant) then it is invariant over $A$ -- i.e.\@ under the action of $\Aut(\U/A)$.
\end{fact}

\begin{proof}
  By Theorem~\ref{thm_generic_invariant} $p({\mr x})$ is  $\grZ$-invariant.
  Therefore $\phi({\mr x}\,;{\gr g}\,;a)\in p$ if and only if  $\phi({\mr x}\,;{\gr 1}\,;a)\in p$.
  As $\phi({\mr x}\,;{\gr 1}\,;a)\iff\phi({\mr x}\,;{\gr 1}\,;fa)$ for every $f\in\Aut(\U/A)$, invariance over $A$ follows.
\end{proof}

% The set $\mrD$ is $\Delta$-definable, namely it is definable by the formula $\phi({\mr x}\,;{\gr 1})$ where ${\gr 1}$ is the identity of $\mrG$.


% We define an immersion of 

% \ceq{\hfill\sigma\ :\ \Aut(\U/A)}{\to}{\grG}\\
% \ceq{\hfill f}{\mapsto}{f\gr1}

% \begin{fact}
%   Let $\mrD\subseteq\U^{\mr x}$ be definable over $A$.
%   Let $\phi({\mr x}\,;{\gr z})$ be the formula ${\mr x}\in{\gr z}\cdot\mrD$.
%   \begin{itemize}
%     \item[1.] $\phi({\mr x}\,;{\gr 1})$ is $\grG$-generic
%     \item[2.] $\mrD$ is generic over $A$.
%   \end{itemize}
% \end{fact}

\begin{assumption}
  We assume that the equivalent conditions of Theorem~\ref{thm_coalesce} hold for the current $G=\grZ$ and $\Delta$.
\end{assumption}

Under this assumption there is a unique maximal generic type $q({\mr x})\subseteq L_\Delta(\grZ)$.
The types $p({\mr x})\in S_\Delta(\grZ)$ that extend $q({\mr x})$ are persistent, hence invariant -- under the action of $\grZ$ and $\Aut(\U/A)$.

%%%%%%%%%%%%%%%%%%%%%%%%%%
%%%%%%%%%%%%%%%%%%%%%%%%%%
%%%%%%%%%%%%%%%%%%%%%%%%%%
%%%%%%%%%%%%%%%%%%%%%%%%%%
%%%%%%%%%%%%%%%%%%%%%%%%%%
\section{Notes and references}

  \begin{biblist}[]\normalsize
    \bib{CK}{article}{
   author={Chernikov, Artem},
   author={Kaplan, Itay},
   title={Forking and dividing in ${\rm NTP}_2$ theories},
   journal={J. Symbolic Logic},
   volume={77},
   date={2012},
  %  number={1},
   pages={1--20},
  %  issn={0022-4812},
  %  review={\MR{2951626}},
  %  doi={10.2178/jsl/1327068688},
}
  \end{biblist}

  