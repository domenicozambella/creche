% !TEX root = creche.tex
\chapter{Actions of groups}
\label{actions}

\def\medrel#1{\parbox[t]{5ex}{$\displaystyle\hfil #1$}}
\def\ceq#1#2#3{\parbox[t]{17ex}{$\displaystyle #1$}\medrel{#2}{$\displaystyle #3$}}

In this chapter, $L$ is a signature, $T$ is a complete theory without finite models, and $\U$ is a saturated model of inaccessible cardinality $\kappa$ strictly larger than $|L|$.
We use the same notation and make the same implicit assumptions as in Section~\ref{monster}.

%%%%%%%%%%%%%%%%%%%%%%%%%%%%%%%%%%%%%%%%%%%%
%%%%%%%%%%%%%%%%%%%%%%%%%%%%%%%%%%%%%%%%%%%%
%%%%%%%%%%%%%%%%%%%%%%%%%%%%%%%%%%%%%%%%%%%%
%%%%%%%%%%%%%%%%%%%%%%%%%%%%%%%%%%%%%%%%%%%%
\section{The dual perspective on invariance}\label{dual_perspective}

The notions of invariance introduceted in the previous chapters extend seemlessly to the action of a group $G$ on $\U^{\gr z}$.
Let $\grD\subseteq\U^{\gr z}$.
We say that $\grD$ is \emph{invariant\/} under the action of $G$, or \emph{$G$-invariant,} if it is fixed setwise by $G$.
That is, $g\grD=\grD$ for every $g\in G$.
Yet in other words, if

\ceq{\ssf{is1.}\hfill {\gr a}\in\grD}{\iff}{g{\gr a}\in\grD}\hfill for every ${\gr a}\in\U^{\gr z}$ and every $g\in G$.

A formula is invariant if the set it defines is invariant.
We say that $p({\mr x})\subseteq L(\U)$ is \emph{invariant\/} under the action of $G$, or \emph{$G$-invariant,} if for every formula $\phi({\mr x}\,;{\gr z})\in L$ 

\ceq{\ssf{it1.}\hfill\phi({\mr x}\,;{\gr a})\in p}{\IFF}{\phi({\mr x}\,;g{\gr a})\in p}\hfill for every ${\gr a}\in\U^{\gr z}$ and every $g\in G$.

It should be clear that invariant under the action of $\Aut(\U/A)$ coincides with invariant over $A$ and Lascar invariant over $A$ coincides with invariant under the action of $\Autf(\U/A)$.

Note that $p({\mr x})$ is invariant exactly when the sets $\gr\D_{p,\phi}\subseteq\U^{\gr z}$ are.
In this section we want to discuss invariance using sets $\mrD\subseteq\U^{mr x}$.
We need stronger assumptions on the action of $G$.

\begin{assumption}\label{notation_GXphi}
  Let $G$ be a group that acts on $\U^{\mr x}$ and on $\U^{\gr z}$.
  Below $\phi({\mr x}\,;{\gr z})\in L$ is a given formula such that $\phi(\U^{\mr x};\U^{\gr z})$ is invariant under the action of $G$.
  Finally, $\mrX\subseteq\U^{\mr x}$ is a nonempty $G$-invariant set.
\end{assumption}

An immediate consequence of the invariance of $\phi(\U^{\mr x};\U^{\gr z})$ is 
that any $G$-tranlate of a $\phi$-definable set is $\phi$-definable.
In particular, by reasoning as in Remark~\ref{rem_image_def_set} we obtain that for every $\phi$-formula $\theta({\mr x}\,;{\gr\bar z})$ and every $g\in G$

\ceq{\hfill g[\theta(\U^{\mr x};{\gr\bar b})]}{=}{\theta(\U^{\mr x};g{\gr\bar b}).}

Therefore $p({\mr x})\subseteq L_\phi(\U)$ is invariant if

\ceq{\hfill p({\mr x})\proves{\mr x}\in\mrD}{\IFF}{p({\mr x})\proves{\mr x}\in g\mrD}\hfill for every $\phi$-definable $\mrD\subseteq\U^{\mr x}$ and $g\in G$.

At the first reading the reader may assume that $G=\Aut(\U)$ and that $\mrX=\U^{\mr x}$.
Note that when $G$ acts by automorphisms $\phi(\U^{\mr x};\U^{\gr z})$ is invariant for all $\phi({\mr x}\,;{\gr z})\in L$.
Therefore when $G$ acts by automorphisms the results in this section holds when working with $L$ in place of $L_\phi$.

A set $\mrD\subseteq\U^{\mr x}$ is \emph{generic\/} in $\mrX$ under the action of $G$, \emph{$G$-generic\/} for short, if finitely many $G$-translates of $\mrD$ cover $\mrX$; we say \emph{$n$-$G$-generic\/} if $\le n$ translates suffices.
Dually, we say that $\mrD$ is \emph{persistent\/} in $\mrX$ under the action of $G$, \emph{$G$-persistent\/} for short, if the intersection of any finitely many $G$-translates of $\mrD$ has a solution in $\mrX$; we say \emph{$n$-$G$-persistent\/} when the request is limited to $\le n$ translates.
We may drop reference to $G$ and $\mrX$ when these are clear from the context.

The same properties may be attributed to formulas (as these are identified with the set they define).
When these properties are attributed to a type $p({\mr x})$, we understand that they hold for the conjunctions of formulas in $p({\mr x})$.
See Exercise~\ref{ex_persistent_types} for an alternative characterization when $p({\mr x})$ is small.

\noindent\llap{\textcolor{red}{\Large\warning}\kern1.5ex}\ignorespaces
The terminology above non-standard.
In~\cite{CK} the authors write \textit{quasi-non-dividing\/} for \textit{persistent}.
Their terminology has good motivations, but it would be a mouthful in our context.

\begin{example}
  If $p({\mr x})\subseteq L(\U)$ is finitely satisfiable in $A$ then $p({\mr x})$ is persistent under the action of $\Aut(\U/A)$.
  In fact, the same ${\mr a}\in A^{\mr x}$ that satisfies $\phi({\mr x})$ also satisfies every $\Aut(\U/A)$-translate of $\phi({\mr x})$.
\end{example}

In this chapter many proofs require some juggling with negations.

\begin{fact}\label{fact_fip}
  (Assume~\ref{notation_GXphi})\ \  
  The following are equivalent
  \begin{itemize}
    \item[1.] $\mrD$ is not $n$-generic in $\mrX$
    \item[2.] $\neg\mrD$ is $n$-persistent in $\mrX$.
  \end{itemize}
\end{fact}

\begin{proof}
  Immediate by spelling out the definitions\smallskip
  \begin{itemize}
    \item[1.] there are no $g_1,\dots,g_n\in G$ such that \smash{$\displaystyle\mrX\ \subseteq\ \bigcup_{i=1}^n g_i{\mr\D}$}
    \item[2.]  $\displaystyle\0\ \neq\ \mrX\,\cap\,\bigcap_{i=1}^n\neg g_i{\mr\D}$ for every $g_1,\dots,g_n\in G$.\qedhere
  \end{itemize} 
\end{proof}

\begin{theorem}\label{thm_generic_invariant}
  (Assume~\ref{notation_GXphi})\ \  
  Let $p({\mr x})\in S_\phi(\U)$ be finitely satisfiable in $\mrX$.
  Then the following are equivalent
  \begin{itemize}
    \item[1.] $p({\mr x})$ is invariant
    \item[2.] $p({\mr x})\proves{\mr x}\in\mrD$ for every generic $\phi$-definable set $\mrD$
    \item[5.] $p({\mr x})$ is persistent.
  \end{itemize}
\end{theorem}

\begin{proof}
  \ssf1$\IMP$\ssf2.
  Let $g_1{\mr\D},\dots,g_n{\mr\D}$ be translations of $\mrD$ that cover $\mrX$.
  Negate \ssf2.
  By completeness, $p({\mr x})\proves {\mr x}\notin\mrD$.
  Hence, from invariance we obtain

  \ceq{\hfill p({\mr x})}{\proves}{{\mr x}\ \notin\ \bigcup_{i=1}^ng_i\mrD.}

  This contradicts the finite satisfiability of $p({\mr x})$ in $\mrX$.
  
  \ssf2$\IMP$\ssf3.
  Let $\mrD$ be defined by a conjunction of formulas in $p({\mr x})$.
  If $\mrD$ is not persistent then, by Fact~\ref{fact_fip}, $\neg\mrD$ is generic. 
  By \ssf2, $p({\mr x})\proves{\mr x}\notin\mrD$, a contradiction.

  \ssf3$\IMP$\ssf1.
  % uppose $p({\mr x})$ is not $(n+2)$-persistent and let $n$ be minimal.
  % Let $g_1,\dots,g_{n+2}\in G$ and $\theta({\mr x})\,;{\gr\bar b}$ witness this.
  % Then 
  If $p({\mr x})$ is not invariant then, by completeness, $p({\mr x})\proves\phi({\mr x}\,;{\gr b})\wedge\neg\phi({\mr x}\,;g{\gr b})$ for some $g\in G$.
  Clearly $\phi({\mr x}\,;{\gr b})\wedge\neg\phi({\mr x}\,;g{\gr b})$ is not persistent as it is inconsistent with its $g$-translate.
\end{proof}

The theorem yields an immediate necessary condition for the existence of an invariant global $\phi$-type.

\begin{corollary}
  (Assume~\ref{notation_GXphi})\ \  
  If there exists an invariant global $\phi$-type then for every $\phi$-definable set $\mrD$
  \begin{itemize}
    \item[1.] $\mrD$ and $\neg\mrD$ cannot be both generic
    \item[2.] if $\mrD$ is generic than it is persistent.
  \end{itemize}
\end{corollary}

\begin{proof}
  By Fact~\ref{fact_fip}, \ssf1 and \ssf2 are equivalent; \ssf1 is an immediate consequenc of \ssf2 of Theorem~\ref{thm_generic_invariant}.
\end{proof}

The following theorem gives a necessary and sufficient condition for the  existence of global invariant $\phi$-type.
Ideally, we would like to prove that every persistent $\phi$-type extends to a global persitent type.
Unfortunately this is not true -- we need a stronger property.
A $\phi$-definable set $\mrD$ is \emph{hereditarely persistent\/} if every finite cover of $\mrD$ by $\phi$-definable sets contains a persistent set.
In~\cite{CK} this property is called \textit{quasi-non-forking,} but we persist in the unorthodoxy.
A type is hereditarely persistent if every conjunction of formulas in the type is hereditarely persistent.

\begin{theorem}\label{thm_generic_invariant2}
  (Assume~\ref{notation_GXphi})\ \  
  Let $q({\mr x})\subseteq L(\U)$.
  Then the following are equivalent 
  \begin{itemize}
    \item[1.] $q({\mr x})$ extends to an invariant type $p({\mr x})\in S_\phi(\U)$ finitely satisfiable in $\mrX$
    \item[2.] $q({\mr x})$ is hereditarely persistent.
  \end{itemize}
\end{theorem}

\begin{proof}
  \ssf1$\IMP$\ssf2.
  Let $\theta({\mr x})$ be a conjunction of formulas in $q({\mr x})$.
  Suppose ${\mr\C_1},\dots,{\mr\C_n}$ cover $\theta(\U^{\mr x})$ and pick $p({\mr x})$ as in \ssf1.
  By completeness, $p({\mr x})\proves {\mr x}\in{\mr\C_i}$ for some $i$.
  Then, by Theorem~\ref{thm_generic_invariant}, $\neg{\mr\C_i}$ is not generic.
  Therefore, by Fact~\ref{fact_fip}, ${\mr\C_i}$ is persistent.

  \ssf2$\IMP$\ssf1.
  Let $p({\mr x})$ be maximal among the $\phi$-types that contain $q({\mr x})$ and are such that $\theta(\U^{\mr x})$ is hereditarely persistent for every $\theta({\mr x})$ that is conjunction of formulas in $p({\mr x})$.
  We claim that $p$ is a complete $\phi$-type.
  Suppose for a contradiction that $\theta({\mr x}),\neg\theta({\mr x})\notin p$.
  By maximality there is some formula $\psi({\mr x})$, a conjunction of formulas in $p({\mr x})$ and some ${\mr\C_1},\dots,{\mr\C_n}$ that cover both $\psi(\U^{\mr x})\cap\theta(\U^{\mr x})$ and $\psi(\U^{\mr x})\smallsetminus\theta(\U^{\mr x})$ and such that no ${\mr\C_i}$ is persistent.
  As ${\mr\C_1},\dots,{\mr\C_n}$ cover $\psi(\U^{\mr x})$ this is a contradiction.
  It is only left to show that $p({\mr x})$ is finitely satisfiable in $\mrX$ and invariant.
  Finite satisfiability follows from persistency.
  From completeness and Theorem~\ref{thm_generic_invariant} we obtain invariance.
\end{proof}

\begin{exercise}
  Prove that if $p({\mr x})\in S(\U)$ is finitely satisfiable in every $M\supseteq A$ then it is persistent under the action of $\Autf(\U/A)$.
  Is the same true for incomplete types?
\end{exercise}

\begin{exercise}
  Let $p(x)\in S(\U)$ be persistent.
  Prove that the following are equivalent
  \begin{itemize}
    \item[1.] $p(x)$ is invariant and finitely satisfiable in $\X$
    \item[2.] $p(x)\proves x\in\D$ for every 2-generic definable set $\D$
    \item[3.] $p(x)$ is 2-persistent.
  \end{itemize}
\end{exercise}

\begin{exercise}
  Let $\D\subseteq\U^x$ be $\phi$-definable.
  Prove that the following are equivalent 
  \begin{itemize}
    \item[1.] there is an invariant type $p(x)\in S_\phi(\U)$ finitely satisfiable in $\X\cap\D$
    \item[2.] every finite cover of $\D$ by $\phi$-definable sets contains a 2-persistent set.
  \end{itemize}
\end{exercise}

\begin{exercise}\label{ex_gen_sat}
  Let $p(x)\subseteq L_\phi(\U)$ be generic.
  Prove that it is finitely satisfiable.
\end{exercise}

\begin{exercise}\label{ex_persistent_types}
  Prove that for every $p(x)\subseteq L_\phi(A)$ following are equivalent
  \begin{itemize}
    \item[1.] $p(x)$ is persistent
    \item[2.] $p(\U^x)$ is persistent.
  \end{itemize}
\end{exercise}

%%%%%%%%%%%%%%%%%%%%%%%%%%
%%%%%%%%%%%%%%%%%%%%%%%%%%
%%%%%%%%%%%%%%%%%%%%%%%%%%
%%%%%%%%%%%%%%%%%%%%%%%%%%
%%%%%%%%%%%%%%%%%%%%%%%%%%
\section{Strong genericity}\label{strong_genericity}

Unfortunatelly, genericy is not preserved under intersection.
To obtain closure under intersection, we need to push the concept to a higher level of complexity.
This is only required in Section~\ref{newelski}.

A set $\mrD\subseteq\U^{\mr x}$ is \emph{strongly generic\/} if the intersection of any finitely many translations of $\mrD$ is generic.
Dually, we say that $\mrD$ is \emph{weakly persistent\/} if the union of some finitely many translations of $\mrD$ is peristent.
Again, the same properties may be attributed to formulas and types.

\begin{lemma}\label{lem_strongly_generic}
  (Assume~\ref{notation_GXphi})\ \  
  Then the intersection of finitely many strongly generic sets is strongly generic.
\end{lemma}

\begin{proof}
  As the theorem does not mention definability, we may assume that all sets mentioned below are subsets of $\mrX$.
  Let ${\mr\D}$ and ${\mr\C}$ be strongly generic and let $K\subseteq G$ be an arbitrary finite set.
  It suffices to prove that 
  
  \ceq{\hfill\mrB}{=}{\bigcap_{k\in K}k\,({\mr\C}\cap{\mr\D})}

  is generic. 
  Clearly $\mrB={\mr\C'}\cap{\mr\D'}$, where
  
  \ceq{\hfill{\mr\C'}}{=}{\bigcap_{k\in K}k\,{\mr\C}}
  \ceq{\hfill{\rm and}\hfill{\mr\D'}}{=}{\bigcap_{k\in K}k\,{\mr\D}.}

  Note that ${\mr\C'}$ and ${\mr\D'}$ are both strongly generic.
  In particular \smash{$\displaystyle\mrX=\bigcup_{h\in H}h\,\mr\D'$} for some finite $H\subseteq G$.
  As, trivially

  \ceq{\hfill\bigcup_{h\in H}h\,\mrB}{=}{\bigcup_{h\in H}\Big[h\,{\mr\C'}\ \cap\ h\,{\mr\D'}\Big]}

  we conclude that
  
  \ceq{\hfill\bigcup_{h\in H}h\,\mrB}{\supseteq}{\bigcap_{h\in H}h\,{\mr\C'}}

  The set on the r.h.s.\@ is (strongly) generic.
  Therefore also the set on the l.h.s.\@ is generic.
  The genericity of $\mrB$ follows.
\end{proof}

\begin{corollary}\label{corol_str_gen}
  (Assume~\ref{notation_GXphi})\ \  
  Let $q({\mr x})=\{\theta({\mr x})\in L_\phi(\U)\,:\, \theta({\mr x})\textrm{ strongly generic}\}$.
  Then $q({\mr x})$ is finitely satisfiable in $\mrX$, strongly generic, and invariant.
\end{corollary}

\begin{proof}
  Strong genericity is an immediate consequence of Lemma~\ref{lem_strongly_generic}.
  Finite satisfiability is a consequence of genericity, see Exercise~\ref{ex_gen_sat}.
  As for invariance, note that any translate of a strongly generic formula is also strongly generic.
\end{proof}

\begin{corollary}\label{corol_q_w_pers}
  (Assume~\ref{notation_GXphi})\ \  
  Let $q({\mr x})$ be as in Corollary~\ref{corol_str_gen}.
  Let $p({\mr x})\subseteq L(\U)$ be such that $p({\mr x})\cup q({\mr x})$ is finitely satisfied in $\mrX$.
  Then $p({\mr x})$ is weakly persistent.
\end{corollary}

\begin{proof}
  Let $\theta({\mr x})\in p$.
  As $q({\mr x})$ is finitely satisfiable in $\theta(\U^{\mr x})$, we cannot have that $\neg\theta({\mr x})$ is strongly generic.
  Reasoning as for Fact~\ref{fact_fip} we obtain that for every set $\mrD$ the following are equivalent
  \begin{itemize}
    \item[1.] $\mrD$ is not strongly generic in $\mrX$
    \item[2.] $\neg\mrD$ is weakly persistent in $\mrX$.
  \end{itemize}
  By applying this equivalence with $\neg\theta(\U^{\mr x})$ for $\mrD$, we conclude that $\theta({\mr x})$ weakly persistent.
\end{proof}

Under suitable assumptions some of the notions introduced in this chapter may coincide.
We will encounter some of these assumptions in following chapter, for the time being we prove the following theorem.

\begin{theorem}
  (Assume~\ref{notation_GXphi})\ \  
  The following are equivalent
  \begin{itemize}
    \item[1.] persistent $\phi$-definable sets are hereditarely persistent
    \item[2.] generic $\phi$-definable sets are strongly generic
    \item[3.] weakly persisent $\phi$-definable sets are persistent.
  \end{itemize}
\end{theorem}

\begin{proof}
  \ssf1$\IMP$\ssf2.
  It suffices to prove that generic sets are closed under intersection.
  Let $\mrC$ and $\mrD$ be generic $\phi$-definable sets.
  Suppose for a contradiction that $\mrC\cap\mrD$ is not generic.
  By \ssf1 and Theorem~\ref{thm_generic_invariant2} there is an invariant global $\phi$-type $p({\mr x})$ containing ${\mr x}\in\neg\mrC\cup\neg\mrD$.
  By completeness either $p({\mr x})\proves{\mr x}\in\neg\mrC$ or $p({\mr x})\proves{\mr x}\in\neg\mrD$.
  This is a contradiction because by Theorem~\ref{thm_generic_invariant} $p({\mr x})\proves{\mr x}\in\mrC$ and $p({\mr x})\proves{\mr x}\in\mrD$.

  \ssf2$\IFF$\ssf3. Clear.

  \ssf3$\IMP$\ssf1.
  Let $q({\mr x})=\{\theta({\mr x})\in L_\phi(\U)\,:\,\theta({\mr x})\textrm{ generic}\}$.
  By \ssf2 this is the same type defined in Corollary~\ref{corol_q_w_pers}.
  Therefore, any completion of $q({\mr x})$ is, by \ssf3, persistent.
  Let $\mrD$ be a persistent $\phi$-definable set.
  By Theorems~\ref{thm_generic_invariant} and~\ref{thm_generic_invariant2} it suffices to show that $\mrD$ is consistent with $q({\mr x})$.
  Suppose not, then $q({\mr x})\proves {\mr x}\in\neg\mrD$.
  By \ssf2 generic sets are closed under conjunction, therefore $\neg\mrD$ is generic.
  This is a contradiction by Fact~\ref{fact_fip}.
\end{proof}

\begin{exercise}
  Show that generic sets need not be closed under intersection.
  Hint: try e.g.\@ with the cyclic order in Example~\ref{ex_cyclic_order}.
\end{exercise}

\begin{exercise}
  Prove that the following are equivalent
  \begin{itemize}
    \item[1.] every persistent $\phi$-formula is hereditarely persistent
    \item[2.] for every $\phi$-definable sets $\C$ and $\D$, if $\C$ is not generic then $\D\cup\C$ and $\neg\D\cup\C$ cannot be both generic.
  \end{itemize}
\end{exercise}


%%%%%%%%%%%%%%%%%%%%%%%%%%%%%%%%%%%
%%%%%%%%%%%%%%%%%%%%%%%%%%%%%%%%%%%
%%%%%%%%%%%%%%%%%%%%%%%%%%%%%%%%%%%
%%%%%%%%%%%%%%%%%%%%%%%%%%%%%%%%%%%
%%%%%%%%%%%%%%%%%%%%%%%%%%%%%%%%%%%
\section{The diameter of a Lascar type}\label{newelski}

The Lascar strong type $\Ll({\mr a}/A)$ is the union of a chain of type-definable sets of the form $\big\{{\mr x}\ :\ d_A({\mr a},{\mr x})\le n\big\}$.
In this section we prove that $\Ll({\mr a}/A)$ is type-definable if and only this chain is finite.
In other words, if the connected component of $a$ in the Lascar graph has finite diameter.
It is convenient to address the problem in more general terms.

Let $G$ be a normal subgroup of $\Aut(\U)$.
Assume $G$ is the union of a countable chain of sets $\<G_n:n\in\omega\>$ with the following properties
\begin{itemize}
  \item[1.] every $G_n$ is symmetric i.e.\@ it contains the unit and is closed under inverse
  \item[2.] every $G_n$ is conjugancy invariant i.e.\@ $g\,G_ng^{-1}=G_n$ for every $g\in G$
  \item[3.] $G_nG_m\subseteq  G_{n+m}$ for every $n,m\in\omega$.
\end{itemize}

Let $\mrX\subseteq\U^{\mr x}$ be a set on which $G$ acts transitively i.e., $G\,{\mr a}=\mrX$ for every ${\mr a}\in\mrX$.
We define a discrete metric on $\mrX$. 
For ${\mr a},{\mr b}\in\mrX$ let $d({\mr a},{\mr b})$ be the minimal $n$ such that ${\mr a}\in G_n{\mr b}$.
This defines a metric by \ssf1 and \ssf3.
By \ssf2, this metric in $G$-invariant.
The \emph{diameter\/} of a set $\mrC\subseteq\mrX$ is the supremum of $d({\mr a},{\mr b})$ for ${\mr a},{\mr b}\in\mrC$.

\begin{example}
  Let $G$ is $\Autf(\U/A)$ and $\mrX=\Ll({\mr a}/A)$.
  If we let $G_1$ be the set of automorphisms that fix a model $M\supseteq A$ and $G_n=G_1^n$, then $d({\mr a},{\mr b})$ concides with the dinstance in the Lascar graph.
\end{example} 

We are interested in sufficient conditions for $\mrX$ to have finite diameter.
The notions introduced in Section~\ref{strong_genericity} offer some hints.

\begin{proposition}\label{prop_wpers_finite_diameter}
  If $\mrX$ has a weakly persistent subset of finite diameter, then $\mrX$ itself has finite diamenter.
\end{proposition}

\begin{proof}
  Let $\mrC\subseteq\mrX$ be a weakly persistent set of diameter $n$.
  Let $H\subseteq G$ be finite such that 

  \ceq{\hfill H\mrC}{=}{\bigcup_{h\in H}h\,\mrC}

  is persistent.
  We claim that $H\mrC$ has finite diameter.
  Let ${\mr a}\in\mrC$ be arbitrary.
  Let $m$ be larger than $d(h{\mr a}, k{\mr a})$ for all $h,k,\in H$.
  Now, let $h{\mr b}$ and $k{\mr c}$, for some $h,k,\in H$ and ${\mr b},{\mr c}\in\mrC$, be two arbitrary elements of $H\mrC$.
  As $h\mrC$ and $k\mrC$ have the same diameter of $\mrC$, 

  \ceq{\hfill d(h{\mr b},\, k{\mr c})}{\le}{d(h{\mr b},\,h{\mr a})\ +\ d(h{\mr a},\, k{\mr a})\ +\ d(k{\mr a},\,k{\mr c})}

  \ceq{}{\le}{n+m+n.}

  This proves that $H\mrC$ has finite diameter.
  Therefore, without loss of generality, we may assume that $\mrC$ itself is persistent.
  
  By the transitivity of the action, any two elements of $\mrX$ are of the form $h{\mr a}$, $k{\mr a}$ for some $h,k\in G$ and some ${\mr a}\in\mrC$.
  By percistency, there are ${\mr c}\in\mrC\cap h\mrC$ and ${\mr d}\in\mrC\cap k\mrC$.
  Then 

  \ceq{\hfill d(h{\mr a},\, k{\mr a})}{\le}{d(h{\mr a},\,{\mr c})\ +\ d({\mr c},\, {\mr d})\ +\ d({\mr d},\,k{\mr a})}

  \ceq{}{\le}{n+n+n.}

  Therefore the diameter of $\mrX$ does not exceed $3n$.
\end{proof}

\begin{theorem}
  Suppose that $\mrX$ and the sets ${\mr\X_n}=G_n{\mr a}$, for some ${\mr a}\in\mrX$, are type-definable.
  Then $\mrX$ has finite diameter.
\end{theorem}

\begin{proof}
  By Proposition~\ref{prop_wpers_finite_diameter}, it suffices to prove that ${\mr\X_n}$ is weakly persistent.
  Define

  \ceq{\hfill q({\mr x})}{=}{\{\theta({\mr x})\in L(\U)\ :\ \theta({\mr x})\textrm{ strongly generic}\}.}
  
  By Corollary~\ref{corol_q_w_pers}, with $L$ for $L_\phi$, it suffices to prove that for some $n$ the type $q({\mr x})$ is finitely satisfied in $\mr\X_n$.
  Suppose not.
  Let $\psi_n({\mr x})\in q$ be a formula that is not satisfied in $\mr\X_n$.
  The type $\{\psi_n({\mr x}):n\in\omega\}$ is finitely satisfied in $\mrX$.
  Then it has a realization in $\mrX$. 
  As this realization belongs to some ${\mr\X_n}$ this contradicts the definition of $\psi_n({\mr x})$. 
\end{proof}

% \ceq{\hfill p_n({\mr\X},{\mr\X})}{=}{\{\<{\mr a},{\mr b}\>\in{\mr\X^2}\ :\ {\mr a}\in G_n{\mr b}\}}

% Below we write ${\mr x}\in G_n{\mr y}$ for the type $p_n({\mr x},{\mr y})$ and ${\mr x}\in G\,{\mr y}$ for the disjunction of all these types (n.b.\@ an infinite disjunction of types need not be a type).

%%%%%%%%%%%%%%%%%%%%%%%%%%
%%%%%%%%%%%%%%%%%%%%%%%%%%
%%%%%%%%%%%%%%%%%%%%%%%%%%
%%%%%%%%%%%%%%%%%%%%%%%%%%
%%%%%%%%%%%%%%%%%%%%%%%%%%
\section{Definable groups}

In this section $\grG\subseteq\U^{\gr z}$ is group that is type-definable over some set of parameters $A$.
This group acts on a set $\mrX\subseteq\U^{\mr x}$ which is also type-definable over $A$.
The group operations and the group action are definable.
Clearly, $\grG$ also acts on itself by left multiplication.
We use the symbol $\,\cdot\,$ for both the group multiplication and the group action.

Let $\mrD=\psi(\U^{\mr x}\,;a)$ for some $\psi({\mr x}\,;y)\in L$ and $a\in\U^y$.
We write $\phi({\mr x}\,;{\gr z})$ or ${\mr x}\in{\gr z}\cdot\mrD$ for the formula $\psi({\gr z^{-1}}\cdot{\mr x}\,;a)$.
Note that $\phi(\U^{\mr x}\,;\U^{\gr z})$ is invariant under the action of $\grG$.

% We define an immersion of 

% \ceq{\hfill\sigma\ :\ \Aut(\U/A)}{\to}{\grG}\\
% \ceq{\hfill f}{\mapsto}{f\gr1}

% \begin{fact}
%   Let $\mrD\subseteq\U^{\mr x}$ be definable over $A$.
%   Let $\phi({\mr x}\,;{\gr z})$ be the formula ${\mr x}\in{\gr z}\cdot\mrD$.
%   \begin{itemize}
%     \item[1.] $\phi({\mr x}\,;{\gr 1})$ is $\grG$-generic
%     \item[2.] $\mrD$ is generic over $A$.
%   \end{itemize}
% \end{fact}

%%%%%%%%%%%%%%%%%%%%%%%%%%
%%%%%%%%%%%%%%%%%%%%%%%%%%
%%%%%%%%%%%%%%%%%%%%%%%%%%
%%%%%%%%%%%%%%%%%%%%%%%%%%
%%%%%%%%%%%%%%%%%%%%%%%%%%
\section{Notes and references}

  \begin{biblist}[]\normalsize
    \bib{CK}{article}{
   author={Chernikov, Artem},
   author={Kaplan, Itay},
   title={Forking and dividing in ${\rm NTP}_2$ theories},
   journal={J. Symbolic Logic},
   volume={77},
   date={2012},
  %  number={1},
   pages={1--20},
  %  issn={0022-4812},
  %  review={\MR{2951626}},
  %  doi={10.2178/jsl/1327068688},
}
    
  \end{biblist}

  