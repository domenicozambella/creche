
\section{Uniform elimination of imaginaries}

\def\medrel#1{\parbox[t]{5ex}{$\displaystyle\hfil #1$}}
\def\ceq#1#2#3{\parbox[t]{35ex}{$\displaystyle #1$}\medrel{#2}{$\displaystyle #3$}}

Sometimes in the literature elimination of imaginaries is confused with uniform elimination of imaginaries, see e.g.~\cite{TZ}*{Definition 8.4.2}.
Theorem~\ref{thm_ei_unif} below shows that the difference is immaterial.
This section is more technical and could be skipped at a first reading.

Let us rephrase the definition of elimination of imaginaries: for every formula $\phi({\mr x}\,;u)\in L$ and every tuple $c\in\U^u$ there is a formula $\sigma({\mr x}\,;{\gr z})\in L$ such that

\ceq{\hfill\E^{=1} {\gr z}\;\A {\mr x}\ \bigg[\phi({\mr x}\,;c)}{\iff}{\sigma({\mr x}\,;{\gr z})\bigg].}

A priori, the formula $\sigma({\mr x}\,;{\gr z})$ may depend on $c$ in a very wild manner.
We say that $T$ has \emph{uniform elimination of imaginaries\/} if for every $\phi({\mr x}\,;u)$ there are a formula $\sigma({\mr x}\,;{\gr z})$ and a formula $\rho({\gr z})$ such that 

\ceq{\ssf{uei}\hfill\A u\;\E^{=1} {\gr z}\;\bigg[\rho({\gr z})\ \wedge\ \A{\mr x}\ \big[\phi({\mr x}\,;u)}{\iff}{\sigma({\mr x}\,;{\gr z})\big]\bigg].}

The role of the formula $\rho({\gr z})$ above is mysterious.
It is clarified by the following propositions.
In fact, uniform elimination of imaginaries is equivalent to a very natural property which in words says: every definable equivalence relation is the kernel of a definable function.
Recall that the kernel of the function $f$ is the relation $fa=fb$.

Uniform elimination of imaginaries is convenient when dealing with interpretations of structure inside other structures.
Let us consider a simple concrete example.
Suppose $\U$ is a group and let $\Hh$ be a definable normal subgroup of $\U$.
The elements of the quotient structure $\U/\Hh$ are equivalence classes of a definable equivalence relation.
If there is uniform elimination of imaginaries we can identify $\U/\Hh$ with an actual definable subset of $\G$ (the range of the function $f$ above).
Moreover, the group operation of $\G$ are definable functions.
As working in $\U/\Hh$ may be notationally cumbersome, $\G$ may offer a convenient alternative.

\begin{proposition}\label{prop_uei_standard}
The following are equivalent
\begin{itemize}
\item[1.] $T$ has uniform elimination of imaginaries
\end{itemize}
\begin{itemize}
\item[2.] for every $\phi({\mr x}\,;u)$ such that  $\A u\;\E{\mr x}\;\phi({\mr x}\,;u)$ there is a formula $\sigma({\mr x}\,;{\gr z})$ such that 
\end{itemize}

\ceq{\hfill\A u\;\E^{=1} {\gr z}\;\A{\mr x}\ \bigg[\phi({\mr x}\,;u)}{\iff}{\sigma({\mr x}\,;{\gr z})\bigg];}

\begin{itemize}
\item[3.] for every equivalence formula $\epsilon({\mr x}\,;{\mr u})\in L$ there is given by $\sigma({\mr x}\,;{\gr z})\in L$ such that
\end{itemize}


\ceq{\hfill\A{\mr u}\;\E^{=1} {\gr z}\;\A {\mr x}\ \bigg[\epsilon({\mr x}\,;{\mr u})}{\iff}{\sigma({\mr x}\,;{\gr z})\bigg].}

\end{proposition}

Note that the definable function $f{\mr u}={\gr z}$ mentioned above is defined by the formula 

\ceq{\hfill\theta({\mr u}\,;{\gr z})\medrel{=}\A {\mr x}\ \bigg[\,\epsilon({\mr x}\,;{\mr u})}{\iff}{\sigma({\mr x}\,;{\gr z})\bigg].}

\begin{proof}
\ssf{1}$\IMP$\ssf{2}.
If $\A u\;\E{\mr x}\;\phi({\mr x}\,;u)$ we can rewrite \ssf{uei} as 

\ceq{\hfill\A u\;\E^{=1} {\gr z}\;\A{\mr x}\ \bigg[\phi({\mr x}\,;u)}{\iff}{\rho({\gr z})\wedge\sigma({\mr x}\,;{\gr z})\bigg].}

\ssf{2}$\IMP$\ssf{3}.
Clear.

\ssf{3}$\IMP$\ssf{1}.
Apply \ssf{3} to the equivalence formula

\ceq{\hfill\epsilon(u\,;v)\medrel{=}\A{\mr x}\ \bigg[\phi({\mr x}\,;u)}{\iff}{\phi({\mr x}\,;v)\bigg]}

Let $\theta(u\,;{\gr z})$ be defined as above.
The reader may check that \ssf{uei} holds substituting for $\delta({\gr z})$ the formula $\E u\;\theta(u\,;{\gr z})$ and and for $\sigma({\mr x}\,;{\gr z})$ the formula $\E u\;\big[\phi({\mr x}\,;u)\wedge\theta(u\,;{\gr z})\big]$.
\end{proof}

By the following theorem, uniformity comes almost for free.

%%%%%%%%%%%%%%
\begin{theorem}\label{thm_ei_unif} The following are equivalent
\begin{itemize}
\item[1.] $T$ has uniform elimination of imaginaries
\item[2.] $\dcl\0$ contains at least two elements and $T$ has elimination of imaginaries.
\end{itemize}
\end{theorem}

%\def\ceq#1#2#3{\parbox[t]{35ex}{$\displaystyle #1$}\parbox[t]{5ex}{$\displaystyle\hfil #2$}{$\displaystyle #3$}}

\begin{proof}\ssf{1}$\IMP$\ssf{2}.
Let $\phi({\mr x},u)$ be the formula $u_1=u_2$.
From \ssf{1} we obtain a formula $\sigma({\mr x}\,;{\gr z})\in L$ such that 

\ceq{\hfill\A u_1,u_2\;\E^{=1} {\gr z}\;\bigg[\rho({\gr z})\ \wedge\ \A x\ \big[u_1=u_2}{\iff}{\sigma({\mr x}\,;{\gr z})\big]\bigg].}

Therefore the formulas $\E^{=1} {\gr z}\,\big[\rho({\gr z})\wedge\A {\mr x}\,\sigma({\mr x}\,;{\gr z})\big]$ and $\E^{=1} {\gr z}\,\big[\rho({\gr z})\wedge\A {\mr x}\neg\sigma({\mr x}\,;{\gr z})\big]$ are both true.
The witnesses of $\E^{=1} {\gr z}$ in these two formulas are two distinct elements of $\dcl\0$.

\ssf{2}$\IMP$\ssf{1}.
Assume \ssf{2} and fix a formula formula $\phi({\mr x}\,;u)$ such that $\phi({\mr x},a)$ is consistent for every $a\in\U^{|u|}$.
We prove \ssf{2} of Proposition~\ref{prop_uei_standard}.

Let $p(u)$ be the type that contains the formulas

\ceq{\hfill\neg\,\E^{=1}{\gr z}\,\A {\mr x}\ \bigg[\phi({\mr x}\,;u)}{\iff}{\sigma({\mr x};{\gr z})\bigg],}

where $\sigma({\mr x};{\gr z})$ ranges over all formulas in $L$.
By elimination of imaginaries $p(u)$ is not consistent.
Therefore, by compactness, there are some formulas $\sigma_i({\mr x},{\gr z})$ such that

\ceq{\sharp\hfill\A u\;\bigvee^n_{i=0}\E^{=1}{\gr z}\;\A {\mr x}\;\bigg[\phi({\mr x}\,;u)}{\iff}{\sigma_i({\mr x}\,;{\gr z})\bigg].}

To prove the theorem we need to move the disjunction in front of the $\sigma_i({\mr x}\,;{\gr z})$.

We can assume that if  $\sigma_i({\mr x}\,;{\gr b})\iff\sigma_i'({\mr x}\,;{\gr b'})$ for some $\<{\gr b},i\>\neq\<{\gr b'},i'\>$ then $\sigma_i({\mr x}\,;{\gr b})$ is inconsistent.
Otherwise we can substitute the formula $\sigma_i({\mr x}\,;{\gr z})$ with  


\hfil$\displaystyle\sigma_i({\mr x}\,;{\gr z}) \;\wedge\; \bigwedge_{j\le i}\neg\E{\gr y}\neq{\gr b}\,\A {\mr x}\,\bigg[\sigma_j({\mr x}\,;{\gr y})\iff\sigma_i({\mr x}\,;{\gr z})\bigg]$.


As $\phi({\mr x},a)$ is consistent for every $a$, the substitution does not break the validity of $\,\sharp$.

Fix some distinct $\0$-definable tuples $d_0,\dots, d_n$ of the same length (these are easy to obtain from two $\0$-definable elements).
We claim that from $\,\sharp\,$ it follows that

\ceq{\hfill\A u\;\E^{=1}{\gr z},y\;\A {\mr x}\;\Bigg[\vphantom{\bigvee^n_{i=0}}\phi({\mr x}\,;u)}{\iff}{\bigvee^n_{i=1}\bigg[\sigma_i({\mr x}\,;{\gr z})\;\wedge\; y=d_i\bigg]\Bigg].}

(The tuple ${\gr z},y$ plays the role of ${\gr z}$.)  We fix some $a$ and check that the formula below has a unique solution

\ceq{\flat\hfill\A {\mr x}\;\Bigg[\vphantom{\bigvee^n_{i=0}}\phi({\mr x}\,;a)}{\iff}{\bigvee^n_{i=1}\bigg[\sigma_i({\mr x}\,;{\gr z})\;\wedge\; y=d_i\bigg]\Bigg].}

Existence follows immediately from $\,\sharp$.
As for uniqueness, note that if ${\gr b},d_i$ and ${\gr b'},d_{i'}$ are two distinct solution of $\,\flat\,$ then $\sigma_i({\mr x}\,;{\gr b})\iff\sigma_{i'}({\mr x}\,;{\gr b'})$ for some $\<{\gr b},i\>\neq\<{\gr b'},i'\>$.
By what assumed on $\sigma_i({\mr x}\,;{\gr z})$, we obtain that $\phi({\mr x}\,;a)$ is inconsistent.
A contradiction which proves the theorem.
\end{proof}