% !TEX root = creche.tex
\chapter{Imaginaries}
\label{imaginary}

\def\medrel#1{\parbox[t]{6ex}{$\displaystyle\hfil #1$}}
\def\ceq#1#2#3{\parbox[t]{25ex}{$\displaystyle #1$}\medrel{#2}{$\displaystyle #3$}}


%\setcounter{page}{1}
\def\vl{\mr}

The description of first-order definability is simplified if we allow definable sets to be used as second-order parameters in formulas.
This leads to the theory of (elimination of) \textit{imaginaries}.
The technical reason that induced Shelah to introduce imaginaries will only be clear later, see Section~\ref{stationarity}, but the theory is of independent interest.

In this chapter we fix a signature $L$, a complete theory $T$ without finite models, and a saturated model $\U$ of inaccessible cardinality $\kappa$ strictly larger than $|L|$.
The notation and implicit assumptions are as in Section~\ref{monster}.

\section{Many-sorted structures}
\label{many-sorted}
\def\Ar{{\rm Ar}}

A many-sorted language consists of three disjoint sets.
Besides the usual $L_{\rm fun}$ and $L_{\rm rel}$, we have a set $L_{\rm srt}$  whose elements are called \emph{sorts}.
The language also includes a \emph{(many-sorted) arity function} that assigns to function and relation symbols $r$, $f$ a tuple of sorts of finite positive length which we call \emph{arity}.

A many-sorted structure $M$ consists of
\begin{itemize}
\item[1.] a set $M_s$, for each $s\in L_{\rm srt}$
\item[2.] a function $f^M:M_{s_1}\times\dots\times M_{s_n}\to M_{s_0}$, for each $f\in L_{\rm fun}$ of arity $\<s_0,\dots,s_n\>$
\item[3.] a relation $r^M\subseteq M_{s_0}\times\dots\times M_{s_n}$, for each $r\in L_{\rm rel}$ of arity $\<s_0,\dots,s_n\>$.
\end{itemize}

For every sort $s$ we fix a sufficiently large set of variables $V_s$.
Now we define terms and their respective sorts by induction.

All variables are terms of their respective sort.
If $t_1,\dots,t_n$ are terms of sorts $s_1,\dots,s_n$ and $f\in L_{\rm fun}$ is of arity $\<s_0,\dots,s_n\>$ then $f\,t_1,\dots,t_n$ is a term of sort $s_0$.


Formulas are defined as follows.
If $r\in L_{\rm rel}$ has arity $\<s_0,\dots,s_n\>$ then $r\,t_0,\dots,t_n$ is a formula.
Also, $t_1=t_2$ is formula for every pair of terms of equal sort.
All other formulas are constructed by induction using the propositional connectives $\neg$ and $\vee$ and the quantifier $\E x$ (or any other reasonable choice of logical connectives).

Truth of formulas is defined as for one-sorted languages, except that here we require that the witness of the quantifier $\E x$ belongs to $M_s$, where $s$ is the sort of the variable $x$.

Models of second-order logic are arguably the most widely used examples of many-sorted structures.
They may be described using a language with a sort $n$ for every $n\in\omega$.
The sort $0$ is used for the first-order elements; the sort $n>0$ is used for relations of arity $n$.
For every $n>0$ the language has a relation symbol $\in_n$ of arity $\<0^n,n\>$, where $0^n\,=\, 0\,\stackrel{n\ \rm times}{\dots\dots}\,0$.
%
%\ceq{\hfill 0^n}{=}{\underbrace{0,\dots,0}_{n\mbox{\scriptsize-times}}.} 
%
There are also arbitrarily many function and relation symbols of sort $\<0^n\>$ for any $n>0$.

%%%%%%%%%%%%%%%%%%%%%%%%%%%%%%%
%%%%%%%%%%%%%%%%%%%%%%%%%%%%%%%
%%%%%%%%%%%%%%%%%%%%%%%%%%%%%%%
\section{The eq-expansion}\label{immaginari}

\noindent\llap{\textcolor{red}{\Large\warning}\kern1.5ex}Warning: the structure $\U^\eq$ and the theory $T^\eq$ defined below do not coincide with the standard ones introduced by Shelah.
As the difference is merely cosmetic, introducing new notation would be overkill and we prefer to abuse the existing terminology.
In Section~\ref{imaginaries_long} below we compare our definition with the standard one.
%This will also explain the symbol used.

Given a language $L$, we define a many-sorted language \emph{$L^{\eq}$\/} which has a sort for each partitioned formula $\sigma({\mr x}\,;{\gr z})\in L$ and a sort \emph{$0$\/} which we call the \emph{home sort}.
(Partitioned formulas have been introduced in Definition~\ref{def_partitioned_fla}.)
For legibility, we pretend that all formulas $\sigma$ depend on the same variables. 
So we assume that ${\mr x}\,;{\gr z}$ are infinite tuples.
Hence, with the notation of the previous section $L_{\rm srt}=\{0\}\cup L_{{\mr x}\,;{\gr z}}$. 

The home sort is also called \emph{first-order sort,}
the other sorts are called \emph{second-order.}

Let \emph{$\U^{\eq}$\/} be the many-sorted structure that has $\U$ as domain of the home sort.
The domain for the sort $\sigma({\mr x}\,;{\gr z})$ contains 
the definable sets ${\vl\Aa}=\sigma(\U^{\mr x};{\gr b})$ as ${\gr b}$ ranges over $\U^{\gr z}$.

The language of $\U^\eq$ is denoted by $L^\eq$.
It contains all the relations and functions of the first order language $L$ with the same interpretation as in the one-sorted case.  
Moreover $L^\eq$ contains a relation symbol $\in_{\sigma({\mr x}\,;{\gr z})}$ for each sort $\sigma({\mr x}\,;{\gr z})$.
These relation symbols have arity \smash{$\big\<0^{|{\mr x_\sigma}|},\sigma({\mr x}\,;{\gr z})\big\>$}, where ${\mr x_\sigma}$ are the variables in ${\mr x}$ that actually occur in $\sigma$.
These relations are interpreted as set membership.
As there is no risk of ambiguity, in what follows we omit $\sigma$ from the subscripts.

We write \emph{$T^{\eq}$\/} for $\Th(\U^\eq)$.
As usual \emph{$L^\eq$\/} also denotes the set of formulas constructed in this language and, if $A\subseteq\U^\eq$, we write  \emph{$L^\eq(A)$\/} for the language and the set of formulas that use elements of $A$ as parameters.

\noindent\llap{\textcolor{red}{\Large\warning}\kern1.5ex}We write \emph{$L(A)$\/} for the set of formulas in $L^\eq(A)$ that contain no second-order variables, neither free nor quantified (when $A\subseteq\U^\eq$, it may contain second-order parameters).

We use the symbol ${\vl\X}$ to denote a generic second-order variable.

% the others are called \emph{imaginary elements}. So, imaginary elements are just definable sets (which below we denote by lower case letters $a,b,c,\dots$).

It is important to note right away that this expansion of $\U$ is a mild one: the definable subsets of the home sort of $\U^\eq$ are the same as those of $\U$.
In particular, iterating the expansion would not yield anything new.

\begin{proposition}\label{prop_eqmild}
Let ${\vl\bar\X}={\vl\X_1},\dots,{\vl\X_n}$ be a tuple of second order variables of sort $\sigma_i({\mr x}\,;{\gr z})$.
Then for every formula $\phi({\mr x}\,;{\vl\bar \X})\in L^\eq$ there is a formula $\phi'({\mr x}\,;{\gr\bar z})\in L$, where ${\gr\bar z}={\gr z_1},\dots,{\gr z_n}$ is a tuple variables of length $|{\gr z}|$, such that the following holds in $\U^\eq$

\ceq{\hfill \phi({\mr x}\,;{\vl\bar\Aa})}{\iff}{\phi'({\mr x}\,;{\gr\bar b})}

for every ${\vl\bar\Aa}={\vl\Aa_1},\dots,{\vl\Aa_n}$ and ${\gr\bar b}={\gr b_1},\dots,{\gr b_n}$ such that ${\vl\Aa_i}=\sigma(\U^{\mr x};{\gr b_i})$.

% For simplicity we are assuming that the tuples ${\gr z_i}$ do not have variables in common.
\end{proposition}

When $n=0$ the proposition asserts that $L_{{\mr x}}^\eq$ and $L_{{\mr x}}$ have the same expressive power.

\begin{proof}[Proof (sketch)]
By induction on syntax.
When $\phi$ is atomic, we set $\phi'=\phi$ unless $\phi$ is of the form ${\mr t}\in {\vl\X_i}$ for some tuple of terms ${\mr t}$ or it has the form ${\vl\X_i}={\vl\X_j}$.
In the first case $\phi'$ is the formula $\sigma_i({\mr t}\,;{\gr z_i})$.
In the second case it is the formula $\A {\mr x}\,\big[\sigma_i({\mr x}\,;{\gr z_i}) \iff\sigma_j({\mr x}\,;{\gr z_j})\big]$. 

The connectives stay unchanged except for the quantifiers $\E {\vl\X}$, where ${\vl\X}$ is a second-order variable, say of sort $\sigma({\mr x}\,;{\gr z})$.
These quantifiers are replaced by $\E{\gr z}$.
\end{proof}

Proposition~\ref{prop_eqmild} implies in particular that we can always replace $\E {\vl\X}$ by $\E {\gr z}$ if we substitute $\sigma({\mr t}\,;{\gr z})$ for ${\mr t}\in {\vl\X}$ in the quantified formula.

%Exercise~\ref{ex_feqthm_senza_eq} below should convince the reader that some trivial moves may be highly convenient.


\begin{remark}\label{rem_eqmild}
Proposition~\ref{prop_eqmild} should convince the reader that the move from $\U$ to $\U^\eq$ is \textit{almost\/} trivial.
For instance, it implies that for every $A\subseteq\U^\eq$, there exists a $B\subseteq\U$ such that $L(B)$ is at least as expressive as $L(A)$.
By this we mean that every formula in $L(A)$ is equivalent to some formula in $L(B)$.
The set $B\subseteq\U$ contains the parameters that define the definable sets in $A\subseteq\U^\eq$.
The point of $\U^\eq$ is that there might not be any $B\subseteq\U$ such that $L(B)$ is \textit{exactly\/} as expressive as $L^\eq(A)$.

For instance, suppose $L$ contains only a binary relation which is interpreted as an equivalence relation with infinitely many infinite classes.
Let ${\vl\Aa}$ be an equivalence class and let $A=\{{\vl\Aa}\}$.
Then for ${\vl\Aa}$ is definable in $L(B)$ if and only if $B\cap {\vl\Aa}\neq\0$.
But no element of ${\vl\Aa}$ is definable in $L(A)$.
\end{remark}

If $\V\preceq\U$ we write \emph{$\V^{\eq}$\/} for the substructure of $\U^\eq$ that has $\V$ as domain of the home sort and  the set of definable sets of the form $\sigma(\U^{\mr x};{\gr b})$ for some ${\gr b}\in \V^{\gr z}$ as domain of the sort $\sigma({\mr x}\,;{\gr z})$.
The following proposition claims that the elementary substructures of $\U^\eq$ are exactly those of the form $\V^\eq$ for some $\V\preceq\U$.

\begin{proposition}
The following are equivalent for every structure $\V^\dag$ of signature $L^\eq$\nobreak
\begin{itemize}
\item[1.] $\V^\dag\preceq\U^\eq$
\item[2.] $\V^\dag=\V^\eq$ for some $\V\preceq\U$.
\end{itemize}
\end{proposition}
\begin{proof}
Implication \ssf{2}$\IMP$\ssf{1} is a direct consequence of Proposition~\ref{prop_eqmild}.
We prove \ssf{1}$\IMP$\ssf{2}.
Let $\V$ be the domain of the home sort of $\V^\dag$.
It is clear that $\V\preceq\U$.
Let ${\vl\Aa}\in\V^\eq$ have sort $\sigma({\mr x}\,;{\gr z})$, say ${\vl\Aa}=\sigma(\U^{\mr x};{\gr b})$ for some ${\gr b}\in\V^{\gr z}$.
As $\E^{=1}{\vl\X}\, \A{\mr x}\,\big[{\mr x}\in{\vl\X}\iff\sigma({\mr x}\,;{\gr b})\big]$ holds in $\U^\eq$, by elementarity it holds in $\V^\dag$ and therefore ${\vl\Aa}\in\V^\dag$.
This proves $\V^\eq\subseteq\V^\dag$.
A similar argument proves the converse inclusion.
Given ${\vl\Aa}\in\V^\dag$ of sort $\sigma({\mr x}\,;{\gr z})$, the formula $\A{\mr x}\,\big[{\mr x}\in{\vl\Aa}\iff\sigma({\mr x}\,;{\gr b})\big]$ holds in $\V^\dag$ for some ${\gr b}$ in the home sort.
%
By elementarity, ${\vl\Aa}=\sigma(\U^{\mr x};{\gr b})$ for some ${\gr b}\in\V$.
\end{proof}

\begin{proposition}\label{prop_Ueq_saturated}
Let $A\subseteq\U^\eq$.
Then every type $p({\mr u}\,;{\vl\X})\subseteq L^\eq(A)$ that is finitely consistent in $\U^\eq$ is realized in $\U^\eq$.
That is, $\U^\eq$ is saturated.
\end{proposition}
\begin{proof}
By Remark~\ref{rem_eqmild}, there are some $B\subseteq\U$ and some $q({\mr u}\,;{\vl\X})\subseteq L^\eq(B)$ equivalent to $p({\mr u}\,;{\vl\X})$.
This already proves the proposition when ${\vl\X}$ is the empty tuple.
Otherwise, let $q'({\mr u}\,;{\gr z})$ be obtained by replacing every formula $\phi({\mr u}\,;{\vl\X})$ in $q({\mr u}\,;{\vl\X})$ with the formula $\phi'({\mr u}\,;{\gr z})$ given in Proposition~\ref{prop_eqmild}.
Then $q'({\mr u}\,;{\gr z})$ is finitely consistent in $\U$.
Assume for clarity of notation that ${\vl\X}$ is a single variable of sort $\sigma({\mr x}\,;{\gr z})$.
If ${\mr c}\,;{\gr b}\models q'({\mr u}\,;{\gr z})$, then  ${\mr c}\,;\sigma(\U^{\mr x};{\gr b})\models q({\mr u}\,;{\vl\X})$.
\end{proof}


Automorphisms of a many-sorted structure are defined in the obvious way: sorts are preserved and so are functions and relations.
Every automorphism $f:\U\imp\U$ extends to an automorphism $f:\U^\eq\imp\U^\eq$ as follows.
If ${\vl\Aa}=\sigma(\U^{\mr x};{\gr b})$ we define $f{\vl\Aa}=\sigma(\U^{\mr x};f{\gr b})=f\big[{\vl\Aa}\big]$, which clearly preserves the sort and the relation $\in$.
Clearly, this extension is unique.

The homogeneity of $\U^\eq$ follows by back-and-forth as in the one-sorted case.

\begin{proposition}\label{prop_Ueq_homogeneous}
Every elementary map $k:\U^\eq\to\U^\eq$ of cardinality $<\kappa$ extends to an automorphism of $\U^\eq$.
\end{proposition} 

%%%%%%%%%%%%%%%%%%%%%%%%%%%%%%%%%%%
%%%%%%%%%%%%%%%%%%%%%%%%%%%%%%%%%%%
%%%%%%%%%%%%%%%%%%%%%%%%%%%%%%%%%%%
%%%%%%%%%%%%%%%%%%%%%%%%%%%%%%%%%%%
\section{The eq-definable closure}
We may safely identify automorphism of $\U$ with automorphisms of $\U^\eq$.
Let $A\subseteq\U^\eq$ and let $a$ be a tuple of elements of $\U^{\eq}$.
We denote by \emph{$\Aut(\U/A)$\/} the set of automorphisms (of $\U^\eq$) that fix all elements of $A$.
The symbol \emph{$\orbit(a/A)$\/} denotes the \emph{orbit of $a$ over $A$}.
This has been defined in Section~\ref{homogeneous} and now we apply it to $\U^\eq$

\ceq{\hfill\emph{$\orbit(a/A)$}}{=}{\big\{fa\;:\; f\in\Aut(\U/A)\big\}.}

By homogeneity, $\O(a/A)=p(\U^\eq)$ where $p(v)=\tp(a/A)$.
When $\O(a/A)=\{a\}$ we say that $a$ is \emph{invariant over $A$\/} or \emph{$A$-invariant}, for short.


\begin{definition}\label{def_def}
Let $A\subseteq\U^\eq$ and $a\in\U^\eq$.
When $\phi(a)\wedge\E^{=1}v\,\phi(v)$ holds for some formula $\phi(v)\in L^\eq(A)$, we say that \emph{$a$ is definable over $A$}.
We write \emph{\rm dcl$^{\rm eq}(A)$\/} for the set of those $a\in\U^\eq$ that are definable over $A$.
We write \emph{\rm dcl$(A)$\/} for $\dcl^\eq(A)\cap\U$.
This is the natural generalization of the notion of definability introduced in Section~\ref{acl}.
\end{definition}

The definition above treats first- and second-order elements of $\U^\eq$ uniformly.
The following propositions proves that when $a\in\U^\eq$ is a definable set, the notion of definability coincides with the one usually applied to sets.

\begin{proposition}\label{prop_def_def_set}
Let $A\subseteq\U^\eq$ and let ${\vl\Aa}\in\U^\eq$ have sort $\sigma({\mr x}\,;{\gr z})$.
Then the following are equivalent
\begin{itemize}
\item[1.] ${\vl\Aa}\in\dcl^\eq(A)$
\item[2.] ${\vl\Aa}=\psi(\U^{\mr x})$ for some $\psi({\mr x})\in L(A)$.
\end{itemize}
\end{proposition}

\begin{proof}
Implication \ssf{2}$\IMP$\ssf{1} is clear because extensionality is implicit in the definition of $\U^\eq$.
We prove \ssf{1}$\IMP$\ssf{2}.
Let $\phi({\vl\X})\in L^\eq(A)$ be a formula ${\vl\Aa}$ is the unique solution of.
Then \ssf{2} holds with $\E {\vl\X}\,\big[{\mr x}\in {\vl\X}\wedge\phi({\vl\X})\big]$ for $\psi'({\mr x})$. This $\psi'({\mr x})$ is a formula in $L^\eq(A)$. Proposition~\ref{prop_eqmild} yields the required formula $\psi({\mr x})\in L(A)$.
\end{proof}

The saturation and homogeneity of $\U^\eq$ allows us to prove the following proposition with virtually the same proof as for Theorem~\ref{thm_dgalois}

\begin{theorem}\label{thm_Galois_def=def}
For any $A\subseteq\U^\eq$ and $a\in\U^\eq$ the following are equivalent
\begin{itemize}
\item[1.] $a$ is invariant over $A$
\item[2.] $a\in\dcl^\eq(A)$.
\end{itemize}
\end{theorem}

By Proposition~\ref{prop_def_def_set}, Theorem~\ref{thm_Galois_def=def} when applied to a definable set ${\vl\Aa}$ gives an alternative proof of Proposition~\ref{prop_inv_def}.

We conclude this section with a remark about the canonicity of the definitions of sets.
The formula $\psi({\mr x})$ in Proposition~\ref{prop_def_def_set} need not be the sort $\sigma({\mr x}\,;{\gr z})$.
For example, consider the theory of a binary equivalence relation $e({\mr x}\,;{\gr z})$ with two infinite classes, let ${\vl\Aa}$ be one of these classes and let $A\neq\0$ be such that $A\cap {\vl\Aa}=\0$.
Then ${\vl\Aa}$ is definable over $A$ though not by some formula of the form $e({\mr x}\,;{\gr b})$ for some ${\gr b}\in A$.
Things change if we replace $A$ with a model.

\begin{proposition}\label{prop_standard_def_set}
Let $M$ be a model and let ${\vl\Aa}$ be an element of sort $\sigma({\mr x}\,;{\gr z})$.
Then the following are equivalent
\begin{itemize}
\item[1.]  ${\vl\Aa}\in\dcl^\eq(M)$
\item[2.]  ${\vl\Aa}=\sigma(\U^{\mr x};{\gr b})$ for some ${\gr b}\in M^{\gr z}$.
\end{itemize}
In particular $M^\eq=\dcl^\eq(M)$.
\end{proposition}

\begin{proof}
Assume \ssf{1} and let $\psi({\mr x})\in L(M)$ be such that ${\vl\Aa}=\psi(\U^{\mr x})$.
Such a formula exists by Proposition~\ref{prop_def_def_set}.
Then $\E{\gr z}\,\A{\mr x}\,\big[\psi({\mr x})\iff\sigma({\mr x}\,;{\gr z})\big]$ holds in $\U$.
By elementarity it holds in $M$, therefore $\E{\gr z}$ has a witness in $M$.
This proves \ssf{1}$\IMP$\ssf{2}, the converse implication is obvious.
\end{proof}

%%%%%%%%%%%%%%%%%%%%%%%%%%%%%%%%%%%%%
%%%%%%%%%%%%%%%%%%%%%%%%%%%%%%%%%%%%%
%%%%%%%%%%%%%%%%%%%%%%%%%%%%%%%%%%%%%
%%%%%%%%%%%%%%%%%%%%%%%%%%%%%%%%%%%%%
%%%%%%%%%%%%%%%%%%%%%%%%%%%%%%%%%%%%%
%%%%%%%%%%%%%%%%%%%%%%%%%%%%%%%%%%%%%
\section{The eq-algebraic closure}

The following is the natural generalization of the notion introduced in Section~\ref{acl}.

\begin{definition}
  Let $A\subseteq\U^\eq$ and $a\in\U^\eq$.
  We say that \emph{$a$ is algebraic over $A$\/} if a formula of the form $\phi(a)\wedge\E^{=k}v\,\phi(v)$ holds for some $\phi(v)\in L^\eq(A)$ and some positive integer $k$.
  We write \emph{\rm acl$^{\rm eq}(A)$\/} for the set of those $a\in\U^\eq$ that are algebraic over $A$.
  We write \emph{\rm acl$(A)$\/} for $\acl^\eq(A)\cap\U$.
\end{definition}

% \begin{proposition}\label{acl123} For every $A\subseteq\U^\eq$ and every $a\in\U^{\eq}$
% \begin{itemize}
% \item[1.]  if $a\in\acl^\eq A$ then $a\in\acl^\eq B$ for some finite $B\subseteq A$;
% \item[2.]  if $B\subseteq A$ then $\acl^\eq B\subseteq \acl^\eq A$;
% \item[3.]  $A\subseteq \acl^\eq A$;
% \item[4.]  $\acl^\eq A=\acl^\eq\big(\acl^\eq A\big)$.
% \end{itemize} 
% \end{proposition}

The following proposition is proved with virtually the same proof as Theorem~\ref{thm_fmgalois}

\begin{theorem}\label{thm_Galois_alg=alg}
For every $A\subseteq\U^\eq$ and every $a\in\U^\eq$ the following are equivalent
\begin{itemize}
\item[1.] $\O(a/A)$ is finite
\item[2.] $a\in\acl^\eq(A)$
\item[3.] $a\in M^\eq$ for every model such that $A\subseteq M^\eq$.
\end{itemize}
\end{theorem}

It is worthwile to spell out the equivalence \ssf2$\IFF$\ssf3 of the theorem above when $a$ is a definable set ${\vl\Aa}$.
Namely, ${\vl\Aa}\in\acl^\eq(A)$ if and only if ${\vl\Aa}$ is definable over every model containing $A$.

We say \emph{finite equivalence relation\/} for an equivalence relation with finitely many classes.
A \emph{finite equivalence formula\/} or \emph{type\/} is a formula, respectively a type, that defines a finite equivalence relation.
Theorem~\ref{thm_set_alg=feq} belows proves that sets algebraic over $A$ are union of classes of a finite equivalence relations definable over $A$.


\begin{theorem}\label{thm_set_alg=feq}
Let $A\subseteq\U^\eq$ and let ${\vl\Aa}\in\U^\eq$ be an element of sort $\sigma({\mr x}\,;{\gr z})$.
Then the following are equivalent
\begin{itemize}
\item[1.] ${\vl\Aa}\in\acl^\eq(A)$
\item[2.] for some finite equivalence formula $\epsilon({\mr x}\,;{\mr y})\in L(A)$ and some ${\mr c_1},\dots,{\mr c_n}\in\U^{|{\mr y}|}$
\end{itemize}

\ceq{\hfill {\mr x}\in{\vl\Aa}}{\iff}{\bigvee^n_{i=1} \epsilon({\mr x}\,;{\mr c_i}).}

\end{theorem}
\begin{proof} \ssf{2}$\IMP$\ssf{1}\quad If  $\epsilon({\mr x}\,;{\mr y})$ has $m$ classes, then $\O(\Aa/A)$ contains at most $\displaystyle\binom{m}{n}$ sets.

\ssf{1}$\IMP$\ssf{2}\quad Let $\phi({\vl\X})\in L^\eq(A)$ be an algebraic formula that has ${\vl\Aa}$ among its solutions and define

\ceq{\hfill\epsilon({\mr x}\,;{\mr y})}{=}{\A {\vl\X}\,\bigg[\phi({\vl\X})\ \imp\ \big[{\mr x}\in{\vl\X}\iff{\mr y}\in{\vl\X}\,\big]\bigg]}

If $\phi({\vl\X})$ has $n$, solutions, then $\epsilon({\mr x}\,;{\mr y})$ has at most $2^n$ equivalence classes.
Clearly, ${\vl\Aa}$ is union of some these classes.
\end{proof}

\begin{definition}\label{def_Sh_strong_type}
We write \emph{${\mr a}\stackrel{\smash{\scalebox{.5}{\rm Sh}}}{\equiv}_A{\mr b}$} when $\epsilon({\mr a}\,;{\mr b})$ holds for every finite equivalence formula $\epsilon({\mr x}\,;{\mr y})\in L(A)$.
In words we say that ${\mr a}$ and ${\mr b}$ have the same \emph{Shelah strong-type\/} over $A$.
\end{definition}

By the following proposition, the Shelah strong type of ${\mr a}$ over $A$ is $\tp({\mr a}/\acl^\eq\!A)$.


\begin{proposition}\label{prop_Shelah_strong_types}
Let $A\subseteq\U^\eq$ and let ${\mr a},{\mr b}\in\U^{\mr x}$.
Then the following are equivalent\nobreak
\begin{itemize}
\item[1.]  ${\mr a}\ \equivSh_A\, {\mr b}$
\item[2.]  ${\mr a}\,\equiv_{\acl^\eq\! A}{\mr b}$.
\end{itemize} 
\end{proposition}
\begin{proof}
\ssf{2}$\IMP$\ssf{1}\quad Assume $\neg$\ssf{1} and let $\epsilon({\mr x}\,;{\mr y})\in L(A)$ be a finite equivalence formula such that $\neg\epsilon({\mr a}\,;{\mr b})$.
Let ${\vl\D}=\epsilon(\U^{\mr x};{\mr b})$, then  ${\mr b}\in{\vl\D}$ and ${\mr a}\notin{\vl\D}$.
As $\epsilon({\mr x}\,;{\mr y})$ is an $A$-invariant finite equivalence formula, ${\vl\D}\in\acl^\eq(A)$, and $\neg$\ssf{2} follows.


\ssf{1}$\IMP$\ssf{2}\quad Assume $\neg$\ssf{2} and let $\phi({\mr x})\in L\big(\acl^\eq(A)\big)$ be  such that $\phi({\mr a})\niff\phi({\mr b})$.
Let ${\vl\D}=\phi(\U^{\mr x})$, then ${\vl\D}\in\acl^\eq(A)$.
Therefore, by Proposition~\ref{thm_set_alg=feq}, the set ${\vl\D}$ is union of equivalence classes of some finite equivalence formula $\epsilon({\mr x}\,;{\mr y})\in L(A)$.
Then $\neg\epsilon({\mr a}\,;{\mr b})$ and $\neg$\ssf{1} follows.
\end{proof}

\noindent\llap{\textcolor{red}{\Large\warning}\kern1.5ex}We write \emph{$\S({\gr a}/A)$\/} for the intersection of all definable sets that contain $a$ and are algebraic over $A$.
By the proposition above $\S({\gr a}/A)=\big\{{\gr b}\,:\,{\gr b}\,\equivSh_A {\gr a}\big\}=\O({\gr a}/\acl^\eq A)$.


\begin{exercise}
Let $p({\mr x})\subseteq L(A)$ and let $\phi({\mr x}\,;{\mr y})\in L(A)$ be a formula that defines, when restricted to $p(\U^{\mr x})$, an equivalence relation with finitely many classes.
Prove that there is a finite equivalence relation definable over $A$ that coincides with $\phi({\mr x}\,;{\mr y})$ on $p(\U^{\mr x})$.
\end{exercise}

\begin{exercise}\label{ex_feqthm_senza_eq}
Let $A\subseteq\U$ and let ${\vl\Aa}$ be a definable set with finite orbit over $A$.
Without using the $\eq$-expansion, prove that ${\vl\Aa}$ is union of classes of a finite equivalence relation definable over $A$.
\end{exercise}

\begin{exercise}
Let $T$ be strongly minimal and let $\phi({\mr x}\,;{\gr z})\in L(A)$ with $|{\mr x}|=1$.
For arbitrary ${\gr b}\in\U^{\gr z}$, prove that if the orbit of $\phi(\U^{\mr x}; {\gr b})$ over $A$ is finite, then $\phi(\U^{\mr x}; {\gr b})$ is definable over $\acl A$.

Hint: you can use Theorem~\ref{thm_set_alg=feq}.
\end{exercise}


%%%%%%%%%%%%%%%%%%%%%%%%%%%%%%%%%%%%%%
%%%%%%%%%%%%%%%%%%%%%%%%%%%%%%%%%%%%%%
%%%%%%%%%%%%%%%%%%%%%%%%%%%%%%%%%%%%%%
%%%%%%%%%%%%%%%%%%%%%%%%%%%%%%%%%%%%%%
%%%%%%%%%%%%%%%%%%%%%%%%%%%%%%%%%%%%%%
\section{Elimination of imaginaries}\label{elimination_imaginaries}

\def\medrel#1{\parbox[t]{5ex}{$\displaystyle\hfil #1$}}
\def\ceq#1#2#3{\parbox[t]{30ex}{$\displaystyle #1$}\medrel{#2}{$\displaystyle #3$}}

For the time being, we agree that \textit{imaginary\/} is just another word for definable set.
Though this is not formally correct (cfr.\@ Section~\ref{imaginaries_long}), it is morally true and helps to understand the terminology.
The concept of elimination of imaginaries has been introduced by Poizat who also proved Theorem~\ref{elimimacf} below.
A theory has elimination of imaginaries if for every $A\subseteq\U^\eq$, there is a $B\subseteq\U$ such that $L(B)$ and $L(A)$ have the same expressive power (i.e.\@ they are the same up to to equivalence).


\begin{definition}\label{defelimanazioneimmaginari}
We say that \emph{$T$ has elimination of imaginaries\/} if for every definable set ${\vl\Aa}$ there is a formula $\phi({\mr x}\,;{\gr z})\in L$ such that\smallskip

\ceq{\ssf{ei}\hfill\E^{=1} {\gr z}\;\A {\mr x}\ \bigg[{\mr x}\in{\vl\Aa}}{\iff}{\phi({\mr x}\,;{\gr z})\bigg]}

We say that the witness of $\E^{=1}{\gr z}$ in the formula above is a \emph{canonical parameter\/} of ${\vl\Aa}$ or a \emph{canonical name\/} for ${\vl\Aa}$.
A set may have different canonical parameters for different formulas $\phi({\mr x}\,;{\gr z})$.

We say that $T$ has \emph{weak elimination of imaginaries\/} if\smallskip

\ceq{\ssf{wei}\hfill\E^{=k} {\gr z}\;\A {\mr x}\ \bigg[{\mr x}\in{\vl\Aa}}{\iff}{\phi({\mr x}\,;{\gr z})\bigg]}

for some positive integer $k$.
\end{definition}

In the formulas above we allow ${\gr z}$ to be the empty string.
In this case we read \ssf{ei} and \ssf{wei} omitting the quantifiers $\E^{=1}{\gr z}$, respectively $\E^{=k}{\gr z}$.
Therefore $\0$-definable sets have all (at least) the empty string as a canonical parameter.

To show that the notions above are well-defined properties of a theory one needs to check that they are independent of our choice of monster model.
We leave this to the reader as an exercise.

We say that two tuples $a$ and $b$ of elements of $\U^\eq$ are \emph{interdefinable\/} if $\dcl^\eq(a)=\dcl^\eq(b)$.
By Theorem~\ref{thm_Galois_def=def} this is equivalent to saying that $\Aut(\U/a)=\Aut(\U/b)$, that is, the automorphisms that fix $a$ fix also $b$, and vice versa.

\begin{theorem}\label{thm_elimimd}
The following are equivalent
\begin{itemize}
\item[1.] $T$ has weak elimination of imaginaries
\item[2.] every definable set is interdefinable with a finite set
\item[3.] every definable set ${\vl\Aa}\in\dcl^\eq\big(\acl\{{\vl\Aa}\}\big)$.
\end{itemize}
\end{theorem}

% N.B.\@ it is important to note the difference between $\acl{\vl\Aa}$ and $\acl\{{\vl\Aa}\}$. In the first, the parameter are the elements of ${\vl\Aa}$. In the latter, the set ${\vl\Aa}$ is the (second-order) parameter.

\begin{proof}\ssf{1}$\IMP$\ssf{2} Assume \ssf{1} and let ${\gr\B}$ be the set of solutions of the formula

\ceq{\hfill\A{\mr x}\ \bigg[{\mr x}\in{\vl\Aa}}{\iff}{\sigma({\mr x}\,;{\gr z})\bigg].}

Hence ${\gr\B}$ is finite and ${\gr\B}\in\dcl^\eq\{{\vl\Aa}\}$.
We also have ${\vl\Aa}\in\dcl^\eq\{{\gr\B}\}$ because ${\vl\Aa}$ is definable by the formula $\E{\gr z}\,\big[{\gr z}\in {\gr\B}\wedge \sigma({\mr x}\,;{\gr z})\big]$.
Therefore $\dcl^\eq\{{\vl\Aa}\}=\dcl^\eq\{{\gr\B}\}$.

\ssf{2}$\IMP$\ssf{3} Assume $\dcl^\eq\{{\vl\Aa}\}=\dcl^\eq\{{\gr\B}\}$ for some finite set ${\gr\B}$.
The elements of ${\gr\B}$, say ${\gr b_1},\dots,{\gr b_n}$, are the (finitely many) solutions of the formula ${\gr z}\in{\gr\B}$.
Therefore ${\gr b_1},\dots,{\gr b_n}\in\acl\{{\gr\B}\}=\acl\{{\vl\Aa}\}$.
Let $\phi({\vl\X}\,;{\gr\B})$ be a formula that has ${\vl\Aa}$ as unique solution.
Then $\E{\vl\Y} \big[{\vl\Y}=\big\{{\gr b_1},\dots,{\gr b_n}\big\}\wedge \phi({\vl\X}\,;{\vl\Y})\big]$ is the formula that proves \ssf{3}.

\ssf{3}$\IMP$\ssf{1} Assume \ssf{3}.
As \ssf{wei} holds trivially for all $\0$-definable sets, we may assume ${\vl\Aa}\neq\0$.
Let $\sigma({\mr x}\,;{\gr z})$ be such that

\ceq{\hfill\A{\mr x}\ \bigg[{\mr x}\in{\vl\Aa}}{\iff}{\sigma({\mr x}\,;{\gr b})\bigg].}

for some tuple ${\gr b}$ of elements of $\acl\{{\vl\Aa}\}$.
Fix some algebraic formula $\delta({\gr z}\,;{\vl\Aa})$ satisfied by  ${\gr b}$ and write $\psi({\gr z}\,;{\vl\X})$ for the formula

\ceq{\hfill \A{\mr x}\ \bigg[{\mr x}\in{\vl\X}}{\iff}{\sigma({\mr x}\,;{\gr z})\bigg]\ \wedge\ \delta({\gr z}\,;{\vl\X}).}

The formula $\,\sharp\,$ below is clearly satisfied by ${\gr b}$ therefore, if we can prove that it has finitely many solutions, \ssf{wei} follows from Proposition~\ref{prop_eqmild}

\ceq{\sharp\hfill\A{\mr x}\ \bigg[{\mr x}\in{\vl\Aa}}{\iff}{\sigma({\mr x}\,;{\gr z})\;\wedge\; \E{\vl\X}\,\psi({\gr z}\,;{\vl\X})\bigg].}

We check that any ${\gr c}$ that satisfies $\,\sharp\,$ also satisfies $\delta({\gr z}\,;{\vl\Aa})$.
As ${\vl\Aa}$ is non empty, $\psi({\gr c}\,;{\vl\Aa'})$ holds for some ${\vl\Aa'}$.
By $\,\sharp\,$ and the definition of $\psi({\gr z}\,;{\vl\X})$ we obtain ${\vl\Aa'}={\vl\Aa}$ and  $\delta({\gr c}\,;{\vl\Aa})$.
\end{proof}


There are a notions of elimination of imaginaries that are weaker than weak elimination.
For instance, we say that $T$ has \emph{geometric elimination of imaginaries\/} if for every $A\subseteq\U$

\ceq{\hfill {\vl\Aa} }{\in}{{\acl}^\eq\big(\acl\{{\vl\Aa}\}\big)}

This will not  be applied in these notes, but see Exercises~\ref{ex_est} and~\ref{ex_wei=est+gei}.

\begin{theorem}\label{elimimf} 
The following are equivalent
\begin{itemize}
\item[1.] $T$ has elimination of imaginaries
\item[2.] every definable set is interdefinable with a tuple of real elements
\item[3.] every definable set ${\vl\Aa}$ is definable over $\dcl\{{\vl\Aa}\}$.
\end{itemize}
\end{theorem}

\begin{proof}
Implications \ssf{1}$\IMP$\ssf{2}$\IMP$\ssf{3} are immediate.
Implication \ssf{3}$\IMP$\ssf{1} is identical to the homologous implication in Theorem~\ref{thm_elimimd}, just substitute algebraic with definable.
\end{proof}



\begin{exercise}
  Let $T$ have elimination of imaginaries and $\phi({\mr x}\,;{\gr z})\in L(A)$.
  For arbitrary ${\gr c}\in\U^{\gr z}$, prove that if the orbit of $\phi(\U^{\mr x}; {\gr c})$ over $A$ is finite, then $\phi(\U^{\mr x}; {\gr c})$ is definable over $\acl A$.
  \end{exercise}
  
  
  \begin{exercise}\label{ex_est}
    Prove that following are equivalent for every $A\subseteq\U$
    \begin{itemize}
    \item[1.]  $\acl^\eq A=\dcl^\eq\big(\acl A\big)$ for every $A\subseteq\U$
    \item[2.]  $\Aut(\U/\acl^\eq A)= \Aut(\U/\acl A)$
    \item[3.] ${\mr a}\,\equiv_{\acl A}{\mr b}\ \IFF\ {\mr a}\ \smash{\equivSh_A}\, {\mr b}$ \ \ for every $A\subseteq\U$ and  ${\mr a},{\mr b}\in\U^{<\omega}$.
    \end{itemize} 
  \end{exercise}
  
  
  \begin{exercise}\label{ex_wei=est+gei}
    Prove that the following are equivalent
    \begin{itemize}
    \item[1.]  $T$ has weak elimination of imaginaries
    \item[2.]  $T$ has geometric elimination of imaginaries and \ssf{1} of Exercise~\ref{ex_est} holds.
    \end{itemize} 
  \end{exercise}
  
  \begin{example}
    Prove that the following are equivalent
    \begin{itemize}
    \item[1.] $T$ has weak elimination of imaginaries
    \item[2.] for every ${\vl\Aa}\in \U^\eq$ there exists the least algebraically closed set $A\subseteq\U$ such that ${\vl\Aa}$ is definable over $A$.
    \end{itemize} 
  \end{example}

%%%%%%%%%%%%%%%%%%%%%%%%%%%%%%%%%%%
%%%%%%%%%%%%%%%%%%%%%%%%%%%%%%%%%%%
%%%%%%%%%%%%%%%%%%%%%%%%%%%%%%%%%%%
%%%%%%%%%%%%%%%%%%%%%%%%%%%%%%%%%%%
\section{Examples}

We now consider elimination of imaginaries in two concrete theories that have been introduced in the previous chapters: 
algebraically closed fields and the random graph.

\begin{lemma}\label{tvtestimmaginari}
The following is a sufficient condition for weak elimination of imaginaries

$\sharp$\qquad for every $A\subseteq\U^\eq$, every consistent $\phi({\gr z})\in L(A)$ has a solution in $\acl A$.
\end{lemma}

\begin{proof}
Let ${\vl\Aa}\in\U^\eq$ be a definable set of sort $\sigma({\mr x}\,;{\gr z})$.
Then $\A{\mr x}\ \big[{\mr x}\in{\vl\Aa}\iff\sigma({\mr x}\,;{\gr z})\big]$ is consistent and, by $\,\sharp\,$ it has a solution in $\acl\{{\vl\Aa}\}$.
Hence weak elimination follows from Theorem~\ref{thm_elimimd}.
\end{proof}

\begin{theorem}\label{elimimsm} 
Let $T$ be a complete, strongly minimal theory.
Then, if $\acl\0$ is infinite, $T$ has weak elimination of imaginaries.
\end{theorem}

\begin{proof}
%We prove condition $\,\sharp\,$ of lemma~\ref{tvtestimmaginari}. Let $A\subseteq\U^\eq$ and let $\phi({\gr z})\in L(A)$ be some consistent formula. Proceed by induction on the length of ${\gr z}$. If ${\gr z}$ is the empty tuple $\,\sharp\,$ holds trivially. Now, consider the formula $\phi({\gr z}\,;{\mr x})$  where $|{\mr x}|=1$. By induction hypothesis $\E{\mr x}\,\phi({\gr z}\,;{\mr x})$ has a solution ${\gr c}$ in $\acl(A)$. Now, $\phi({\gr c}\,;{\mr x})$ is either an algebraic formula, then it has a solution in $\acl\big(\acl(A)\big)=\acl(A)$ or it is co-algebraic, then it has a solution in $\acl(A)$ which is infinite by hypothesis.
If $\acl\0$ is infinite, $\acl A$ is a model for every $A$ (cfr.\@ Exercise~\ref{ex_infinite_acl}) so condition $\,\sharp\,$ of lemma~\ref{tvtestimmaginari} holds by elementarity and the theorem follows.
\end{proof}


\begin{theorem}\label{elimimacf} 
The theories $T^p_{\rm acf}$ have elimination of imaginaries.
\end{theorem}

\begin{proof}
By Theorem~\ref{elimimsm} we know that $T^p_{\rm acf}$ has weak elimination of imaginaries.
Therefore, by Theorem~\ref{thm_elimimd} it suffices to prove that every finite set ${\vl\Aa}$ is interdefinable with a tuple.
Let ${\vl\Aa}=\big\{{\gr a_1},\dots,{\gr a_n}\big\}$ where each ${\gr a_i}$ is a tuple ${\mr a_{i,1}},\dots,{\mr a_{i,m}}$ of elements of $\U$.
Given ${\vl\Aa}$ we define the term

\ceq{\hfill t_{\vl\Aa}({\mr x}\,;{\gr y})}{=}{\displaystyle\prod^n_{i=1}\bigg({\mr x}-\sum^m_{k=1}{\mr a_{i,k}}\; {\mr y_k}\bigg).}\hfill where ${\gr y}={\mr y_1},\dots,{\mr y_m}$


Note that (the interpretation of) the term $t_{\vl\Aa}({\mr x}\,;{\gr y})$ is independent on particular indexing of the set ${\vl\Aa}$.
So, any automorphism that fixes ${\vl\Aa}$, fixes the  $t_{\vl\Aa}({\mr x}\,;{\gr y})$.
Now rewrite $t_{\vl\Aa}({\mr x}\,;{\gr y})$ as a sum of monomials and let $c$ be the tuple of coefficients of these monomials.
The tuple $c$ uniquely determines $t_{\vl\Aa}({\mr x}\,;{\gr y})$ and vice versa.
Therefore every automorphism that fixes  ${\vl\Aa}$ fixes  $c$ and vice versa.
Hence  ${\vl\Aa}$ and  $c$ are interdefinable.
\end{proof}

The rest of this section is dedicated to the proof that the theory of the random graph has weak elimination of quantifiers.

\begin{lemma}\label{lem_suff_wei}
  The following is a sufficient condition for weak elimination of imaginaries

  $\natural$\qquad $\D\subseteq\U^{|x|}$ is definable both over $A$ and over $B$, then it is definable over $A\cap B$.
\end{lemma}

\begin{proof}
  Let $\D\subseteq\U^{|x|}$ be a definable set, and pick a finite set $A\subseteq\U$ of minimal size over which $\D$ is definable. 
  Now, let $f$ be an automorphism that fixes $\D$. Then $\D$ is definable over $A\cap f[A]$, so by minimality of $|A|$ we have $A= f[A]$. 
  Now let $a$ be an enumeration of $A$.
  By the above, $a$ has a finite orbit under $\Aut(\U/\{\D\})$.
  Hence $A\subseteq\acl\{\D\}$.
  Therefore $\D\in\dcl^\eq\big(\acl\{\D\}\big)$
  The lemma follows by applying Theorem~\ref{thm_elimimd}.
\end{proof}

\begin{lemma}\label{lem_type_amalgamation}
  Let $\U\models T_{\rm rg}$.
  Let $a$ and $b$ be tuples such that $\range a\cap A,\range b\cap B\subseteq A\cap B$.
  Assume also that $a\equiv_{A\cap B}b$.
  Then there is a tuple $c$ such that $a\equiv_A c\,\equiv_B b$.
\end{lemma}

\begin{proof}
  For simplicity assume that $\range a\cap A=\range b\cap B=\0$.
  The general case follows easily.

  By induction on the length of $a$ and $b$. 
  The claim is trivial if this length is $0$ so, assume they have length $n+1$ and that there are $c_0,\dots,c_{n-1}\notin A, B,\range a,\range b$ such that $a_{\restriction n}\equiv_A c_0,\dots,c_{n-1}\,\equiv_B b_{\restriction n}$.

  We assume that $\range a\cap\range b=\0$ otherwise we replace $b$ with any tuple $b'\equiv_Bb$ that is disjoint of $\range a$.

  Now, let $A'=A\cup\{a_0,\dots,a_{n-1}\}$ and $B'=B\cup\{b_0,\dots,b_{n-1}\}$.
  By the assumption above $A'\cap B'=A\cap B$.

  Let $c_n$ be a vertex the satisfies the following formulas

  \ceq{\ssf{1a}\hfill\bigwedge_{a'\in A'}\Big[r(x,a')}{\iff}{r(a_n,a')\Big]}

  \ceq{\ssf{1b}\hfill\bigwedge_{b'\in B'}\Big[r(x,b')}{\iff}{r(b_n,b')\Big]}

  \ceq{\ssf{2a}\hfill\bigwedge_{i<n}\Big[r(x,c_i)}{\iff}{r(a_n,a_i)\Big]}

  \ceq{\ssf{2b}\hfill\bigwedge_{i<n}\Big[r(x,c_i)}{\iff}{r(b_n,b_i)\Big]}

  Note that \ssf{1a} and \ssf{1b} are mutually consistent by the assumption $a\equiv_{A\cap B}b$.
  By the same assumption \ssf{2a} and \ssf{2b}.
  As $c_0,\dots,c_{n-1}\notin A, B,\range a,\range b$ the four formulas above are mutually consistent.

  Clearly, any $c_n$ not in $A, B,\range a,\range b$ proves the lemma.
\end{proof}


\begin{lemma}
  The theory $T_{\rm rg}$ has weak elimination of imaginaries.
\end{lemma}

\begin{proof}
  We prove ${\natural}$ of Lemma~\ref{lem_suff_wei}.
  Suppose not and pick $a,b\in\U^{|x|}$ such that $a\equiv_{A\cap B}b$ and $a\in\D\niff b\in\D$.

  Note that $\D$ is definable over $f[A]$ for every $f\in\Aut(\U/B)$ and the same holds swapping $A$ and $B$.
  Therefore we can replace $A$ and $B$ with suitable $f[A]$ and $g[B]$ that are disjoint of $\range a$ and $\range b$, rispectively (for simplicity we are assuming that $a$ and $b$ do not contain elements in $A\cap B$).
  Clearly $f[A]\cap g[B]=A\cap B$.

  By Lemma~\ref{lem_type_amalgamation}, there is $c$ such that $a\equiv_{f[A]} c\,\equiv_{g[B]} b$.
  This contradicts the fact that $\D$ is definable both over $f[A]$ and over $g[B]$.
\end{proof}


% \begin{exercise}
% prove that a theory has weak elimination of imaginaries if for every $A\subseteq\U^\eq$, there is a $B\subseteq\U$ such that $L\big(\acl(B)\big)$ and $L\big(acl(A)\big)$ have the same expressive power.
% \end{exercise}

\section{Imaginaries: the true story}\label{imaginaries_long}

The point of the expansion to $\U^\eq$ is to add a canonical parameter for each definable set.
In fact, in $\U^\eq$ every definable subset of $\U^{\gr z}$ is the canonical parameter of itself.
This allows us to deal with theories without elimination of imaginaries in the most straightforward way.

The expansion to $\U^\eq$ that was originally introduced by Shelah (and still used everywhere else) is slightly different from the one introduced here.
For a given set ${\vl\Aa}=\sigma(\U^{\mr x};{\gr b})$ Shelah considers the equivalence relation defined by the formula

\ceq{\hfill \epsilon({\gr z}\,;{\gr z'})}{=}{\A{\mr x}\,\bigg[\sigma({\mr x}\,;{\gr z})\medrel{\iff}\sigma({\mr x}\,;{\gr z'})\bigg].}

The equivalence class of ${\gr b}$ in the relation $\epsilon({\gr z}\,;{\gr z'})$ is what Shelah uses as canonical parameter of the set ${\vl\Aa}$.

Shelah's $\U^\eq$ has a sort for each $\0$-definable equivalence relation $\epsilon({\gr z}\,;{\gr z'})$.
The domain of the sort $\epsilon({\gr z}\,;{\gr z'})$ contains the classes of the equivalence relation defined by $\epsilon({\gr z}\,;{\gr z'})$.
These equivalence classes are called \emph{imaginaries}.
Shelah's $L^\eq$ contains functions that map tuples in the home sort to their equivalence class.

% Tortuous routes are somewhat typical of Shelah.
% This did not hinder him from becoming one of the most productive mathematician of his century.
% Lesser minds may find it useful to take a simplified route.
