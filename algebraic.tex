% !TEX root = creche.tex
\documentclass[creche.tex]{subfiles}
\begin{document}
\chapter{Some algebraic structures}
\label{algebraic}

\def\medrel#1{\parbox[b]{6ex}{\hfil$#1$}}
\def\ceq#1#2#3{\parbox[b]{25ex}{$\displaystyle #1$}\medrel{#2}$\displaystyle #3$}

The main result in this chapter is Corollary~\ref{coroll_Nullstellensatz}, the elimination of quantifiers in algebraically closed fields, which in algebra is called Chevalley's Theorem on constructible sets. From it we derive Hilbert's Nullstellensatz~\ref{coroll_Nullstellensatz}. Finally, we isolate the model theoretic properties of those types that correspond to the algebraic notions of prime and radical ideal of polynomials.

%At the cost of a few repetitions the first three sections of the chapter are not a prerequisite for the rest. 
The first sections of this chapter are not a pre-requisite for Sections~\ref{anelli}\kern1.5pt--\kern1pt\ref{Nullstellensatz}, at the cost of a few repetitions in the latter.

\begin{notation}\label{notation1}
Recall that when $A\subseteq M$ we denote by \emph{$\<A\>_M$\/} the substructure of $M$ generated by $A$. Then $\<A\>_M\subseteq N$ is equivalent to $N\models \Diag\,\<A\>_M$. The diagram of a structure has been defined at the end of Section~\hyperref[frammenti]{\ref*{types}.\ref*{frammenti}}.

In this chapter, whenever some $A\subseteq M\models T$ are fixed, by \textit{model\/} we mean superstructures of $\<A\>_M$ that models $T$. The notions of \textit{logical consequence}, \textit{consistency}, \textit{completeness}, etc.\@ are modified accordingly, and we write $\,\proves\,$ for $T\cup\Diag\<A\>_M\proves\,$. 

We say that a type $p({\mr x})$ is \emph{trivial\/} if $\,\proves p({\mr x})$.\QED
\end{notation}

%%%%%%%%%%%%%%%%%%%%%%%%%%%%%%%%%%
%%%%%%%%%%%%%%%%%%%%%%%%%%%%%%%%%%
%%%%%%%%%%%%%%%%%%%%%%%%%%%%%%%%%%
%%%%%%%%%%%%%%%%%%%%%%%%%%%%%%%%%%
%%%%%%%%%%%%%%%%%%%%%%%%%%%%%%%%%%
\section{Abelian groups}
\label{gruppi}
In this section the language $L$ is that of additive groups. The theory \emph{$T_{\rm ag}$} of \emph{abelian groups\/} is axiomatized by the universal closure of the following axioms
\begin{itemize}
\item[a1] $(x+y) +z\ =\ y+(x+z)$;
\item[a2] $x+(-x)\ =\ (-x)+x\ =\ 0$;
\item[a3] $x+0\ = \ 0+x\ = \ x$;
\item[a4] $x+y\ =\ y+x$.
\end{itemize}
Let $x$ be a tuple of variables of length $\alpha$, an ordinal. We write $L_{{\rm ter},x}$ for the set of terms $t({\mr x})$ with free variables among $x$. On this set we define the equivalence relation

\ceq{\hfill \emph{$t(x)\sim s(x)$}}%
{=}%
{T_{\rm ag}\proves t({\mr x})=s({\mr x}).}

We define the group operations on $L_{{\rm ter},x}/\mathord{\sim}$ in the obvious way. We denote by $\ZZ^{\oplus\alpha}$ the set of tuples of integers of length $\alpha$ that are almost always $0$. The group operations on $\ZZ^{\oplus\alpha}$ are defined coordinate-wise. The following immediate proposition implies in particular that  $L_{{\rm ter},x}/\mathord{\sim}$ is isomorphic to $\ZZ^{\oplus\alpha}$.

\begin{proposition}
\label{corol_formacanonicaterminimoduli}
Let $A\subseteq M\models T_{\rm ag}$. Then for every formula $\phi({\mr x})\in\atL(A)$ there are $n\in\ZZ^{\oplus\alpha}$ and $c\in\<A\>_M$ such that 

\ceq{\hfill\medrel{\proves}\phi({\mr x})}{\iff}{\sum_{i<\alpha}n_ix_i\ =\ c}.

where $n=\<n_i:i<\alpha\>$ and  $x=\<x_i:i<\alpha\>$.\QED

\end{proposition}

\begin{proof}
Up to equivalence over $T_{\rm ag}$ the formula $\phi({\mr x})$ has the form $s({\mr x})=t(a)$ for some parameter-free terms $s({\mr x})$ and $t(z)$. Over $\Diag\,\<A\>_M$, we can replace $t(a)$ with a single element of  $\<A\>_M$ and write $s({\mr x})$ as the linear combination shown above.
\end{proof}


\begin{definition} Let $M\models T_{\rm ag}$. For $A\subseteq M$ and  $c\in M$. We say that  \emph{$c$ is independent from $A$\/} if $\<A\>_M\cap\<c\>_M=\big\{0\big\}$. Otherwise we say that \emph{$c$ is dependent from $A$}. The \emph{rank\/} of $M$ is the least cardinality of a subset $A\subseteq M$ such that all elements in $M$ are dependent from $A$. We denote it by \emph{\rm\bf rank\,$M$}.\QED
\end{definition}  

Note that when $M$ is a vector space the condition $\<A\>_M\cap\<c\>_M=\big\{0\big\}$ is equivalent to saying that $c$ is not a linear combination of vectors in $A$ and $\rank M$ coincides with the dimension of $M$. In fact, what we do here for abelian groups could be easily generalized to $D\jj$modules, where $D$ is any integral domain, and in particular to vector spaces.

The following proposition gives a convenient syntactic characterization of independence.


\begin{proposition}
Let $A\subseteq M\models T_{\rm ag}$. Let ${\mr c}\in M$ be such that ${\mr c}^n\neq0$ for every $n\in\omega\sm\{0\}$ (i.e.\@ ${\mr c}$ is not a torsion element). Then the following are equivalent.
\begin{itemize}
 \item[1.] $\<A\>_M\cap\<{\mr c}\>_M=\big\{0\big\}$;
 \item[2.] $p({\mr x})=\attp_M({\mr c}/A)$ is trivial (see Notation~\ref{note_tipi_diagramma} and ~\ref{notation1}).
\end{itemize}
\end{proposition}
\begin{proof}
\ssf{1}$\IMP$\ssf{2}\quad By Proposition~\ref{corol_formacanonicaterminimoduli}, a non trivial formula in $p({\mr x})$ may be assumed to have the form $n{\mr x}=a$ for some $n$ and some $a$. If such a formula is satisfied by ${\mr c}$ then $a\in\<A\>_M\cap\<{\mr c}\>_M$. As ${\mr c}$ is not a torsion element, $n=0$ and the equation is trivial.

\ssf{2}$\IMP$\ssf{1}\quad If $\<A\>_M\cap\<{\mr c}\>_M\neq\big\{0\big\}$ then $n{\mr c}=a$ for some $a\in\<A\>_M\sm\{0\}$. Then ${\mr c}$ satisfies some non trivial equation $n{\mr x}=a$.
\end{proof}


\begin{remark}\label{oss_liberi_qf}
Let $k:M\to N$ be a partial embedding and let ${\gr a}$ be an enumeration of $\dom k$. We claim that  $k\cup\big\{\<{\mr b},{\mr c}\>\big\}:M\to N$ is a partial embedding for every ${\mr b}\in M$ and ${\mr c}\in N$ that are independent from ${\gr a}$, respectively $k{\gr a}$. In fact, it suffices to check that $M,{\mr b},{\gr a}\,\equiv_{\rm at}N, {\mr c},k{\gr a}$. Suppose $\phi({\mr x}\,;{\gr z})\in\atL$ \ is such that $M\models\phi({\mr b}\,;{\gr a})$. Then by independence $\phi({\mr x}\,;{\gr a})$ is trivial, i.e.

\ceq{\hfill T_{\rm ag} \cup\ \Diag\<{\gr a}\>_M}{\proves}{\phi({\mr x}\,;{\gr a}).}

As $\<{\gr a}\>_M$ and  $\<k{\gr a}\>_N$ are isomorphic structures 

\ceq{\hfill T_{\rm ag}\ \cup\ \Diag\<k{\gr a}\>_N}{\proves}{\phi({\mr x}\,;k{\gr a}).}

Therefore $N\models\phi({\mr c}\,;k{\gr a})$. The proof that $N\models\phi({\mr c}\,;k{\gr a})$ implies $M\models\phi({\mr b}\,;{\gr a})$ is similar.\QED 
\end{remark}



%%%%%%%%%%%%%%%%%%%%%%%%%%%%%%%%%%%
%%%%%%%%%%%%%%%%%%%%%%%%%%%%%%%%%%%
%%%%%%%%%%%%%%%%%%%%%%%%%%%%%%%%%%%
\section{Torsion-free abelian groups}

The theory of \emph{torsion-free abelian groups\/} extends $T_{\rm ag}$ with the following axioms for all positive integers $n$
\begin{itemize}
\item[st] $nx= 0\imp x=0$.
\end{itemize}
We denote this theory by \emph{$T_{\rm\bf tfag}$}. It is not difficult to see that in a torsion-free abelian group every equation of the form $\,nx = a\,$ has at most one solution.


\begin{proposition}
Let $M\models T_{\rm tfag}$ be uncountable. Then $\rank M=|M|$. 
\end{proposition}

\begin{proof}
Let $A\subseteq M$ have cardinality $<|M|$. We claim that $M$ contains some element that is independent from $A$. It suffices to show that the number of elements that are dependent from $A$ is $<|M|$. If $c\in M$ is dependent from $A$ then, by Proposition~\ref{prop_mst_tipi princ_comp}, it is a solution of some formula $\atL(A)$. As there is no torsion, such a formula has at most one solution. Therefore the number of elements that are dependent from $A$ is at most $|\atL(A)|$, that is $\max\big\{|A|,\omega\big\}$. If $M$ is uncountable the claim follows.
\end{proof}

\begin{proposition}\label{prop_mst_tipi princ_comp}
Let $A\subseteq M\models T_{\rm tfag}$.  Let $p({\mr x})=\atpmtp({\mr b}/A)$, where ${\mr b} \in M$. Then one of the following holds:   
\begin{itemize}
\item[1.] ${\mr b}$ is independent from $A$;
\item[2.] $M\models\phi({\mr b})$ for some $\phi({\mr x})\in\atL(A)$ such that $\ \proves\ \phi({\mr x})\imp p({\mr x})$.
\end{itemize}\end{proposition}

Note the similarity with Example~\ref{ex_infiniti_colori}, where the independent type is $q({\mr x})$ and the isolating formulas are the $r_i({\mr x})$. 

It is important to observe that the set $A$ above may be infinite. This is essential to obtain Corollary~\ref{corol_ModDivUltraOmog}, and it is one of the main differences between this example and the examples encountered in Chapter~\ref{relational}. % (see Exercise~\ref{ex_ol+grf_tipi_princ_comp}).
%L'insieme $A$ pu\`o essere preso infinito, questo permetter\`a di dimostrare il lemma~\ref{lemma_modulo_divisibile_ricco_nonnumer} senza restrizioni sulla cardinalit\`a e quindi di ottenere la categoricit\`a non numerabile, vedi corollario~\ref{thm_ModDivUltraOmog}. Non \`e cos\`i nell'esercizio~\ref{ex_ol+grf_tipi_princ_comp}. 

\begin{proof}
If ${\mr b}$ is dependent from $A$, then ${\mr b}$ satisfies a non trivial atomic formula $\phi({\mr x})$ which we claim is the formula required in \ssf{2}. It suffices to show that $\phi({\mr x})$ implies a complete $\atpmL(A)$-type. Clearly this type must be $p({\mr x})$. Let $\phi({\mr x})$ have the form $n{\mr x}=a$ for some $n\in\ZZ\sm\{0\}$ and ${\gr a}\in\<A\>_M\sm\{0\}$. We show that for every $m\in\ZZ$ and every ${\gr c}\in\<A\>_M$ one of the following holds
\begin{itemize}
\item[a.] $\proves\   n{\mr x}={\gr a}\imp m{\mr x}={\gr c}$;
\item[b.] $\proves\   n{\mr x}={\gr a}\imp m{\mr x}\neq {\gr c}$.
\end{itemize} 
Suppose not for a contradiction that neither \ssf{a} nor \ssf{b} holds and fix models $N_1$, $N_2$ and some $b_i\in N_i$ such that
\begin{itemize}
\item[a'.] $N_1\models n{\mr b_1}={\gr a}$ \ and \ $N_1\models m{\mr b_1}\neq {\gr c}$;
\item[b'.] $N_2\models n{\mr b_2}={\gr a}$ \ and \ $N_2\models m{\mr b_2} = {\gr c}$.
\end{itemize} 
From \ssf{b'} we obtain $N_2\models m{\gr a}=n{\gr c}$. As $N_1$ is torsion-free, from \ssf{a'} we obtain $N_1\models m{\gr a}\neq n{\gr c}$. But $m{\gr a}=n{\gr c}$ is a formula with parameters in $\<A\>_M$, so it should have the same truth value in all superstructures of $\<A\>_M$, a contradiction.
\end{proof}




%\begin{definition} Let $A\subseteq M\models T_{\rm tfag}$. We say that an element $c\in M$ is \emph{independent from $A$\/} if the type $p({\mr x})=\attp_M(c/A)$ is trivial, that is, $N\models \A x\,p({\mr x})$ for every $\<A\>_M\subseteq N\models T_{\rm tfag}$.\QED\end{definition}  


% 
% 
% \begin{proposition}\label{prop_independent_ag}
% Let $k:M\to N$ be a partial isomorphism between models of $T_{\rm tfag}$. Let $b\in M$ and $c\in N$ be independent from $\dom k$, respectively $\range k$. Then  $k\cup\big\{\<b,c\>\big\}:M\to N$ is partial isomorphism.
% \end{proposition}
% 
% \begin{proof}
% It suffices to prove that $k\cup\big\{\<b,c\>\big\}:M\to N$ preserves the truth of atomic formulas and then apply the same argument to $k^{-1}:N\to M$. Let $a$ be an enumeration of $\dom k$ and let $p(x,z)=\attp(b,a)$. As $p(x,a)$ is a trivial type, then $M'\models\A x\,p(x,a)$ for every model $\<\dom k\>_M\subseteq M'\models T_{\rm tfag}$. As $\<\dom k\>_M$ is isomorphic to $\<\range k\>_N$ then  $N\models\A x\,p(x,ka)$, in particular $N\models p(c,ka)$. By symmetry, also $b$ realize the atomic type of $c$ over $\range a$ and the proposition follows. 
% \end{proof}


% \begin{exercise}\label{ex_ol+grf_tipi_princ_comp}
% Let $T$ be one of the theories $T_{\rm lo}$ or $T_{\rm graph}$. Let $A\subseteq M\models T$ be finite. For $b\in M$ let $p({\mr x})=\atpmtp_M(b/A)$. Prove that $\phi({\mr x})\imp p({\mr x})$ for some $\phi({\mr x})$, conjunction of formulas in $p({\mr x})$. Find a counterexample with $A$ infinite.\QED
% \end{exercise}

%%%%%%%%%%%%%%%%%%%%%%%%%%%%%%%%%%
%%%%%%%%%%%%%%%%%%%%%%%%%%%%%%%%%%
%%%%%%%%%%%%%%%%%%%%%%%%%%%%%%%%%%
\section{Divisible abelian groups}

The theory of \emph{divisible abelian groups\/} extends $T_{\rm tfag}$ with the following axioms for all  integers $n\neq0$
\begin{itemize}
\item[div$_n$] $y\neq0\imp\E x\ nx=y$.
\end{itemize} 
We denote this theory by \emph{$T_{\rm\bf dag}$}.

\begin{proposition}\label{prop_md_cons_sodd}
Let $A\subseteq M\models T_{\rm tfag}$ and let $\phi({\mr x})\in\atL(A)$, where $|{\mr x}|=1$, be consistent. Then $N\models\E {\mr x} \,\phi({\mr x})$ for every model $N\models T_{\rm dag}$.
\end{proposition}

Note that in the proposition above  \textit{consistent\/} means satisfied in some $M'$ such that $\<A_M\>\subseteq M'\models T_{\rm tfag}$.

The claim in the proposition holds more generally for all $\phi({\mr x})\in L_{\rm qf}$ and also when ${\mr x}$ is a tuple of variables. This follows from Lemma~\ref{lemma_modulo_divisibile_ricco_nonnumer}, whose proof uses the proposition.

\begin{proof}
We can assume that $\phi({\mr x})$ has the form $n {\mr x}= {\gr a}$ for some $n\in\ZZ$ and some ${\gr a}\in\<A\>_M$. If $n=0$, then ${\gr a}=0$ since $\phi({\mr x})$ is consistent, and the claim is trivial. If $n\neq0$ then by consistency ${\gr a}\neq0$, hence a solution exist in $N$ by axiom \ssf{div$_n$}.
\end{proof}

\begin{exercise}
Prove a converse of Proposition~\ref{prop_md_cons_sodd}. Let $A\subseteq N\models T_{\rm ag}$ and let ${\mr x}$ be a single variable. Prove that if $N\models\E{\mr x} \,\phi({\mr x})$ for every consistent $\phi({\mr x})\in\atL(A)$, then $N\models T_{\rm dag}$.\QED
\end{exercise}

We are ready to prove that divisible abelian groups of infinite rank are $\omega$-rich. We shall be slightly more general than that.

\begin{lemma}\label{lemma_modulo_divisibile_ricco_nonnumer}
Let $k:M\imp N$ be a partial isomorphism of cardinality $<\lambda$, where $M\models T_{\rm tfag}$ and $N\models T_{\rm dag}$ is a model of rank $\ \ge\lambda$. Then for every ${\mr b}\in M$ there is ${\mr c}\in N$ such that $k\cup\big\{\<{\mr b}\,;{\mr c}\>\big\}:M\imp N$ is a partial isomorphism.
\end{lemma}

\begin{proof}
Let ${\gr a}$ be an enumeration of $\dom k$ and let $p({\mr x}\,;{\gr z})=\atpmtp({\mr b}\,;{\gr a})$.  The required ${\mr c}$ has to realize $p({\mr x}\,;k{\gr a})$. We consider two cases. If ${\mr b}$ is dependent from ${\gr a}$, then Proposition~\ref{prop_mst_tipi princ_comp} yields a formula $\phi({\mr x}\,;{\gr z})\in\atL$ such that 

\noindent\rlap{\ssf{i.}}\hspace{5ex} $\proves\ \phi({\mr x}\,;{\gr a})\imp p({\mr x}\,;{\gr a})$

\noindent\rlap{\ssf{ii.}}\hspace{5ex} $\phi({\mr x}\,;{\gr a})$\ \ is consistent.

By isomorphism,  \ssf{i} and \ssf{ii} hold with ${\gr a}$ replaced by $k{\gr a}$. Then by Proposition~\ref{prop_md_cons_sodd} the formula $\phi({\mr x}\,;k{\gr a})$ has a solution ${\mr c} \in N$.

The second case, which has no analogue in Lemma~\ref{lem_ordinericco}, is when ${\mr b}$ is independent from ${\gr a}$. Then by   Remark~\ref{oss_liberi_qf} we may choose ${\mr c}$ to be any element of $N$ independent from $k{\gr a}$. Such an element exists because $N$ has rank at least $\lambda$. 
\end{proof}

\begin{corollary}\label{corol_ModDivUltraOmog}
Every uncountable model of $T_{\rm dag}$ is rich in the category of models of $T_{\rm tfag}$ and partial isomorphisms. In particular $T_{\rm dag}$ is uncountably categorical, it is complete, and has quantifier elimination.
\end{corollary}
\begin{proof}
Any uncountable $N\models T_{\rm dag}$ has rank $|N|$, therefore it is rich by Lemma~\ref{lemma_modulo_divisibile_ricco_nonnumer}; categoricity and completeness follow. As for quantifier elimination, let $k:M\to N$ is a partial isomorphism between models of $T_{\rm dag}$. If $M\preceq M'$ and $M\preceq N'$ are elementary superstructures of uncountable cardinality then $k:M'\to N'$ is elementary by Theorem~\ref{thm_morphism_rich_elementary} and this suffices to conclude that  $k:M\to N$ is elementary. 
\end{proof}

\begin{exercise}
Prove that every model of $T_{\rm dag}$ is $\omega\jj$ultraomogeneous (indipendenlty of cardinality and rank).\QED
\end{exercise}

%%%%%%%%%%%%%%%%%%%%%%%%%%%%%%%%%%
%%%%%%%%%%%%%%%%%%%%%%%%%%%%%%%%%%
%%%%%%%%%%%%%%%%%%%%%%%%%%%%%%%%%%
%%%%%%%%%%%%%%%%%%%%%%%%%%%%%%%%%%
%%%%%%%%%%%%%%%%%%%%%%%%%%%%%%%%%%
\section{Commutative rings}
\label{anelli}

In this section $L$ is the \emph{language of (unital) rings}. It contains two constants $0$ and $1$ the unary operation $-$ and two binary operations $+$ and $\cdot$. The theory of rings contains the following axioms

\begin{itemize}
\item[a1-a4] as for abelian groups
\item[r1] $(x\mdot y)\mdot z\  =\ y\mdot(x\mdot z)$,
\item[r2] $1\mdot x\ =\ x\mdot 1\ =\ x$,
\item[r3] $(x+y)\mdot z\ =\ x\mdot z + y\mdot z$,
\item[r4] $z\mdot (x+y)\ =\ z\mdot x + z\mdot y$.
\end{itemize}

All the rings we consider are \emph{commutative\/} 

\begin{itemize}
\item[c] $x\mdot y\ =\ y\mdot x$.
\end{itemize}
 
We denote the theory of commutative rings by \emph{$T_{\rm cr}$}.

In what follows the theory $T_{\rm cr}\cup\Diag\<A\>_M$ is implicit in the sense of Notation~\ref{notation1}, so it is important to remember that $\Diag\<A\>_M$ is not trivial even when $A=\0$. In fact, $\Diag\<\0\>_M$ determines the characteristic of the models.

Let  $A\subseteq M\models T_{\rm cr}$ and let $x$ be a tuple of variables of length $\alpha$, an ordinal. We write $L_{{\rm ter},x}(A)$ for the set of terms $t({\mr x})$ with free variables among $x$ and parameters in $A$. On this set we define the equivalence relation

\ceq{\hfill \emph{$t(x)\sim s(x)$}}%
{=}%
{\proves \ t({\mr x})=s({\mr x}).}

On $L_{{\rm ter},x}(A)/\mathord{\sim}$ we define the ring operations in the obvious way so that $L_{{\rm ter},x}(A)/\mathord{\sim}$ is a commutative ring. We denote by $A[x]$ the set of polynomials with variables among $x$ and parameters in $\<A\>_M$. The ring operations on $A[x]$ are defined as usual. The following proposition (which is clear, but tedious to prove) implies in particular that $L_{{\rm ter},x}(A)/\mathord{\sim}$ is isomorphic to $A[x]$. For simplicity we state it only for $|x|=1$.

\begin{proposition}
\label{prop_formacanonicaterminiau}
Let  $A\subseteq M\models T_{\rm cr}$ and let $x$ be a single variable. Then for every formula $\phi({\mr x})\in \atL(A)$ there is a unique $n<\omega$ and a unique tuple $a=\<a_i:i\le n\>$ of elements of $\<A\>_M$ such that $a_n\neq0$ and 

\ceq{}{\proves}{\phi({\mr x})\ \iff\ \sum_{i\le n}a_ix^i\ =\ 0}.\QED
\end{proposition}

The integer $n$ in the proposition above is called the \emph{degree of $\phi(x)$}.

\begin{definition} Let $A\subseteq M\models T_{\rm cr}$. We say that an element $b\in M$ is \emph{transcendent over $A$\/} if the type $p({\mr x})=\attp_M(b/A)$ is trivial (see Notation~\ref{note_tipi_diagramma} and ~\ref{notation1}). Otherwise we say that $b$ is \emph{algebraic over $A$}. The \emph{transcendence degree\/} of $M$ is the least cardinality of a subset $A\subseteq M$ such that all elements of $M$ are algebraic over $A$.\QED
\end{definition}  

\begin{remark}\label{oss_liberi_cr}
Remark~\ref{oss_liberi_qf} holds here with `independent' replaced by `transcendent' and $T_{\rm ag}$ replaced by $T_{\rm cr}$.\QED
\end{remark}

\section{Integral domains}

Let $a\in M\models T$. We say that $a$ is a \emph{zero divisor} if $a\,b=0$ for some $b\in M\sm\{0\}$. An \emph{integral domain\/} is a commutative ring without zero divisors. The theorey of integral domains contains the axioms of commutative rings and the following
\begin{itemize}
\item[nt.] $0\neq 1$
\item[id.] $x\mdot y =0\ \imp\ x=0\ \vee\ y=0$.
\end{itemize}

We denote the theory  of integral domains by \emph{$T_{\rm id}$}.


\begin{proposition}
Let $M\models T_{\rm id}$ be uncountable. Then $M$ has transcendence degree $|M|$.
\end{proposition}

\begin{proof}
Every polynomial has finitely many solutions in an integral domain and there are $|L(A)|$ polynomials over $A$.
\end{proof}



\begin{proposition}\label{prop_di_tipi princ_comp}
Let $A\subseteq M\models T_{\rm id}$. For ${\mr b}\in M$ let $p({\mr x})=\atpmtp({\mr b}/A)$. Then one of the following holds  
\begin{itemize}
\item[1.] ${\mr b}$ is transcendental over $A$;
\item[2.] $M\models\phi({\mr b})$ for some $\phi({\mr x})\in\atL(A)$ such that
$\ \proves\ \phi({\mr x})\imp p({\mr x})$.
\end{itemize}\end{proposition}

Note the similarity with Example~\ref{ex_infiniti_colori}, where the transcendental type is $q({\mr x})$ and the isolating formulas are the $r_i({\mr x})$. 

As in Proposition~\ref{prop_mst_tipi princ_comp}, the set $A$ may be infinite. This is essential to obtain Corollary~\ref{corol_acfUltraOmog}.


\begin{proof}
Suppose ${\mr b}$ is not transcendental, i.e.\@ it satisfies a non trivial atomic formula. Let $\phi({\mr x})\in\atL(A)$ be a non trivial formula with minimal degree such that $\phi({\mr b})$.  We prove that $\phi({\mr x})$ implies a complete $\atpmL(A)$-type.  Clearly this type must be $p({\mr x})$. We prove that for any $\xi({\mr x})\in\atL(A)$ one of the following holds
\begin{itemize}
\item[1.] $\ \proves\ \phi({\mr x})\imp\phantom{\neg}\xi({\mr x})$
\item[2.] $\ \proves\ \phi({\mr x})\imp\neg\xi({\mr x})$.
\end{itemize} 
Let us write $a({\mr x})=0$ and $a'({\mr x})=0$ for the formulas $\phi({\mr x})$ and $\xi({\mr x})$, respectively. Let $n$ be the degree of the polynomial $a({\mr x})$. If $\<A\>_M$ is a field, choose a polynomial $d({\mr x})$ of maximal degree such that
\begin{itemize}
\item[a.] $d({\mr x})t({\mr x})=a({\mr x})$
\item[a'.] $d({\mr x})t'({\mr x})=a'({\mr x})$,
\end{itemize} 
for some polynomials $t({\mr x})$ and $t'({\mr x})$. 

If $\<A\>_M$ is not a field, polynomials $d({\mr x})$, $t({\mr x})$ and $t'({\mr x})$ as above exist with coefficients in the field of fractions of $\<A\>_M$. Then \ssf{a} and \ssf{a'} hold up to a factor in $\<A\>_M$ which we absorb in $a({\mr x})$ and $a'({\mr x})$. 

From \ssf{a} we get $d({\mr b})=0$ or $t({\mr b})=0$. In the first case, as $a({\mr x})$ has minimal degree, we conclude that $t({\mr x})$ is constant.  This implies that any zero of $a({\mr x})$ is also a zero of $a'({\mr x})$, that is, it implies \ssf{1}. 

Now suppose $t({\mr b})=0$. Then the minimality of the degree of $a({\mr x})$ implies that $d({\mr x})=d$, where $d$ is a nonzero constant. If $\<A\>_M$ is a field, apply B\'ezout's identity to obtain two polynomials $c({\mr x})$ and $c'({\mr x})$ such that $d=a({\mr x})c({\mr x})+a'({\mr x})c'({\mr x})$. Then $a({\mr x})$ and $a'({\mr x})$ have no common zeros, and \ssf{2} follows. If $\<A\>_M$ is not a field, we use  B\'ezout's identity in the field of fractions of $\<A\>_M$ and, for some $d'\in\<A\>_M\sm\big\{0\big\}$, obtain $d'd=a({\mr x})c({\mr x})+a'({\mr x})c'({\mr x})$. Then we reach the same conclusion.
\end{proof}




%%%%%%%%%%%%%%%%%%%%%%%%%%%%%%%%%%%%%%
%%%%%%%%%%%%%%%%%%%%%%%%%%%%%%%%%%%%%%
%%%%%%%%%%%%%%%%%%%%%%%%%%%%%%%%%%%%%%
%%%%%%%%%%%%%%%%%%%%%%%%%%%%%%%%%%%%%%
%%%%%%%%%%%%%%%%%%%%%%%%%%%%%%%%%%%%%%
%%%%%%%%%%%%%%%%%%%%%%%%%%%%%%%%%%%%%%
\section{Algebraically closed fields}

Let $a,b\in M\models T_{\rm cr}$. We say that $b$ is the \emph{inverse\/} of $a$ if $a\mdot b=1$. A field is a commutative ring where every non-zero element has an inverse. The \emph{theory of fields\/} contains $T_{\rm id}$ and the axiom
\begin{itemize}
\item[f.]$\E  y\; \big[x\neq0\ \imp\ x\mdot y =1\big]$.
\end{itemize}
Fields are structures in the signature of rings: the language contains no symbol for the multiplicative inverse. So, substructures of fields are merely integral domains.

The theory of \emph{algebraically closed field}, which we denote by \emph{$T_{\rm acf}$}, also contains the following axioms for every positive integer $n$

\begin{itemize}
\item[ac$_n$.] $\E x\ \big(x^{n} + z_{n-1}x^{n-1} + \dots + z_1 x + z_0\ \ =\ \ 0\big)$
\end{itemize}

This not a complete theory because it does not decide the characteristic of the field. For a prime $p$, we define the theory \emph{$T^p_{\rm acf}$}, which contains $T_{\rm acf}$ and the axiom
\begin{itemize}
\item[ch$_p$.]$1+\dots \mbox{($p$ times)}\dots +1=0$.
\end{itemize}
The theory \emph{$T^0_{\rm acf}$\/} contains the negation of \ssf{ch$_p$} for all $p$. Note that all models of $T^p_{\rm acf}$ have the same characteristic in the model theoretic sense defined in~\ref{def_characteristic}. In fact, any two fields $M$ and $N$ with the same (algebraic) characteristic have isomorphic prime subfield, that is, $\<\0\>_M \simeq \<\0\>_N$.

\begin{proposition}\label{prop_acf_cons_sodd}
Let $A\subseteq M\models T_{\rm id}$ and let $\phi({\mr x})\in\atL(A)$, where $|{\mr x}|=1$, be consistent. Then $N\models\E {\mr x} \,\phi({\mr x})$ for every model $N\models T_{\rm acf}$.
\end{proposition}

Note that in the proposition above \textit{consistent\/} means satisfied in some $M'$ such that $\<A_M\>\subseteq M'\models T_{\rm id}$.

The claim in the proposition holds more generally for all $\phi({\mr x})\in L_{\rm qf}$ when $x$ is a tuple of variables. This follows from Lemma~\ref{lem_acf_ricco_nonnumer} whose proof uses the proposition.

\begin{proof}
Up to equivalence $\phi({\mr x})$ has the form ${\gr a_n}{\mr x}^n + \dots + {\gr a_1} {\mr x} + {\gr a_0}=0$ for some ${\gr a_i}\in\<A\>_N$. Choose $n$ minimal. If $n=0$, then ${\gr a_0}=0$ since $\phi({\mr x})$ is consistent, and the claim is trivial. Otherwise ${\gr a_n}\neq0$ and the claim follows from \ssf{f} and \ssf{ac$_n$}. 
\end{proof}

\begin{lemma}\label{lem_acf_ricco_nonnumer}
Let $k:M\imp N$ be a partial isomorphism of cardinality $<\lambda$, where $M\models T_{\rm id}$ and $N\models T_{\rm acf}$ has transcendence degree $\ge\lambda$. Then for every ${\mr b}\in M$ there is ${\mr c}\in N$ such that $k\cup\big\{\<{\mr b},{\mr c}\>\big\}:M\imp N$ is a partial isomorphism.
\end{lemma}

The following is the proof of Lemma~\ref{lemma_modulo_divisibile_ricco_nonnumer} which we have pasted here for the sake of the lazy reader.

\begin{proof}
Let ${\gr a}$ be an enumeration of $\dom k$ and let $p({\mr x}\,;{\gr z})=\atpmtp_M({\mr c}\,;{\gr a})$. The required ${\mr c}$ has to realize $p({\mr x}\,;k{\gr a})$. We consider two cases. If ${\mr b}$ is algebraic over ${\gr a}$, then Proposition~\ref{prop_di_tipi princ_comp} yields a formula $\phi({\mr x}\,;{\gr z})\in\atL$ such that 

\noindent\rlap{\ssf{i.}}\hspace{5ex} $\phi({\mr x}\,;{\gr a})\imp p({\mr x}\,;{\gr a})$

\noindent\rlap{\ssf{ii.}}\hspace{5ex} $\phi({\mr x}\,;{\gr a})$\ \ is consistent.

By isomorphism \ssf{i} and \ssf{ii} hold with ${\gr a}$ replaced by $k{\gr a}$. Then by Proposition~\ref{prop_acf_cons_sodd} the formula $\phi({\mr x}\,;k{\gr a})$ has a solution in ${\mr c}\in N$.

The second case, which has no analogue in Lemma~\ref{lem_ordinericco}, is when ${\mr b}$ is transcendent over ${\gr a}$. Then by  Remark~\ref{oss_liberi_cr} we may choose ${\mr c}$ to be any element of $N$ transcendent over $k{\gr a}$. This exists because $N$ has transcendence degree $\ \ge\lambda$. 
\end{proof}

\begin{corollary}\label{corol_acfUltraOmog}
Every uncountable model of $T^p_{\rm acf}$ is rich in the category of integral domains of characteristic $p$ and partial isomorphisms. In particular $T^p_{\rm acf}$ is uncountably categorical, and it is complete. Moreover, $T_{\rm acf}$ has quantifier elimination.
\end{corollary}

The proof requires the same argument as Corollary~\ref{corol_ModDivUltraOmog}.% However here we have to fix the characteristic of the integral domain, otherwise the resulting category is not connected. 

\begin{proof}
Every uncountable $N\models T^p_{\rm acf}$ has transcendence degree $|N|$, therefore it is rich by Lemma~\ref{lem_acf_ricco_nonnumer}. Categoricity and completeness follow. 

For quantifier elimination, let $k:M\to N$ be a partial embedding between models of $T_{\rm acf}$. Then $M$ and $N$ have the same charactiristic, i.e.\@ they are model of $T^p_{\rm acf}$ . Let $M'$ and $N'$ be elementary superstructures of $M$ and $N$ respectively of sufficiently large cardinality. As $M'$ and $N'$ are rich, $k:M'\to N'$ is elementary by Theorem~\ref{thm_morphism_rich_elementary}. Hence 
 $k:M\to N$ is also elementary.
\end{proof}


\begin{exercise}
Prove that every model of $T_{\rm acf}$ is $\omega\jj$ultrahomogeneous (indipendently of cardinality and transcendence degree).\QED
\end{exercise}


%%%%%%%%%%%%%%%%%%%%%%%%%%%%%
%%%%%%%%%%%%%%%%%%%%%%%%%%%%%
%%%%%%%%%%%%%%%%%%%%%%%%%%%%%
%%%%%%%%%%%%%%%%%%%%%%%%%%%%%
%%%%%%%%%%%%%%%%%%%%%%%%%%%%%
%%%%%%%%%%%%%%%%%%%%%%%%%%%%%
\section{Hilbert's Nullstellensatz}
\label{Nullstellensatz}

\def\medrel#1{\parbox{5ex}{\hfil$\displaystyle #1$}}
\def\ceq#1#2#3{\parbox{20ex}{$\displaystyle #1$}\medrel{#2}$\displaystyle  #3$}

In this section we fix a tuple of variables \emph{$x$\/} and a subset \emph{$A$} of an integral domain. We denote by \emph{$\Delta(A)$\/} the set of formulas of the form $t(x)=0$ where $t(x)$ is a term with parameters in $A$. So $\Delta(A)\jj$types are (possibly infinite) systems of polynomial equations with coefficients in $\<A\>_M$. 

For convenience we define the closure of $p(x)$ under logical consequences as follows

\ceq{\hfill \ccl\, p(x)}{=}{\Big\{t(x)=0\ :\    p(x)\, \proves\, t(x){=}0\Big\}}

Remember that we always work under the assumptions made in Notation~\ref{notation1}. In particular, in this section we work over the theory $T_{\rm id}\cup\Diag\<A\>_M$. In general, closure under logical consequences is an elusive notion. Hence Propositions~\ref{prop_Nullstellensatz} and~\ref{prop_chiusura-radicale} are useful because they give a model theoretical, respectively algebraic, characterisation of $\ccl\, p(x)$.

\begin{proposition}\label{prop_Nullstellensatz}
Let $A\subseteq M\models T_{\rm id}$ and let $p(x)$ be a $\Delta(A)\jj$type. Fix some $N$ of sufficiently large cardinality such that $\<A\>_M\subseteq N\models T_{\rm acf}$. Then 

\ceq{\hfill\ccl\, p(x)}{=}{\Big\{t(x)=0\,:\,  N\models\A x\,\big[ p(x)\imp t(x){=}0\big]\Big\}.}

\end{proposition}

It suffices to choose $N$ such that $|A|<|N|$ and $|x|\le |N|$; note that $x$ has possibly infinite length. In Corollary~\ref{coroll_Nullstellensatz} below we will considerably strengthen this proposition for finite $x$.

\begin{proof} Only the inclusion $\supseteq$ requires a proof.  Fix some polynomial $t(x){=}0\notin\ccl\, p(x)$. Then $p(x)\wedge t(x)\neq0$ is consistent, so there is a model $M'$ such that $M'\models p(a)\wedge t(a)\neq0$ for some $a\in {M'}^{|x|}$. Then there is a partial isomorphism $h:M'\to N$ that extends $\id_A$ and is defined on $a$, provided $N$ is large enough to accommodate $a$. This implies that $p(x)\wedge t(x)\neq0$ has a solution in $N$. Hence $t(x)$ does not belong to the set on the r.h.s.  
\end{proof}

Recall that $A[x]$ denotes the ring of polynomials with variables in $x$ and coefficients in $\<A\>_M$. We identify $\Delta(A)$ nd $A[x]$ in the obvious way. Consequently, a $\Delta(A)\jj$type $p(x)$ is identified with a set of polynomials $p\subseteq A[x]$. For $p\subseteq A[x]$ we write \emph{${\rm\bf rad}\,p$} for the \emph{radical ideal generated by $p$}, that is, the intersection of all prime ideals containing $p$. When $p=\rad p$, we say that  \emph{$p$ is a radical ideal}. Recall from algebra that if $p\subseteq A[x]$ is an ideal then

\ceq{\ssf{\#}\hfill\rad p}{=}{\Big\{t(x)\  :\ t^n(x)\in p \textrm{ for some positive integer }n\Big\},}

which explains the name.



\begin{proposition}\label{prop_chiusura-radicale}
Let $A\subseteq M\models T_{\rm id}$ and let $p(x)$ be a $\Delta(A)\jj$type. Then $\ccl\,p(x)\simeq\rad p$.
\end{proposition}
The proposition holds with a similar proof for the broader class of rings without nilpotent elements. (Which does not come as a surprise.)
\begin{proof}
$(\supseteq)$\quad We claim that $\ccl\,p(x)$ is an ideal. In fact, for every pair of $L(A)$-terms $t(x)$ and $s(x)$

\ceq{\hfill t(x)=0}{\proves}{s(x)t(x) = 0}

\ceq{\hfill s(x)=t(x)=0}{\proves}{s(x)+t(x) = 0}

Moreover, as integral domains do not have nilpotent elements

\ceq{\hfill t^n(x)=0}{\proves}{t(x) = 0}

By \ssf{\#} above, $\ccl\,p(x)$ is a radical ideal which proves $\ccl\,p(x)\supseteq \rad p$.

$(\subseteq)$\quad We fix some $t(x)\notin\rad p$ and prove that $p(x)\wedge t(x)\neq0$ is consistent. Let $q$ be some prime ideal containing $p$ such that $t(x)\notin q$.  As $q$ is prime, the ring $A[x]/q$ is an integral domain. The polynomials that vanish in  $A[x]/q$ at $x+q$ are exactly those in $q$. Hence $A[x]/q$ witnesses the consistency of $q(x)\wedge t(x)\neq0$.
\end{proof}

When $x$ is a finite tuple of variables, we can extend the validity of Proposition~\ref{prop_Nullstellensatz} to the case $A=M=N$.

\begin{corollary}[ (Hilbert's Nullstellensatz)]\label{coroll_Nullstellensatz}
Let $N\models T_{\rm acf}$ and let $p(x)$, where $|x|<\omega$, be a $\Delta(N)\jj$type. Then

\ceq{\hfill\rad p\medrel{\simeq}\ccl\, p(x)}{=}{\Big\{t(x)\  :\  N\models\A x\,[ p(x)\imp t(x)=0]\Big\}.}

\end{corollary}



\begin{proof}
Let $N'$ be a large elementary extension of $N$. By Proposition~\ref{prop_Nullstellensatz}, the claim holds for $N'$. By Hilbert's Basis Theorem, the ideal generated by $p$ is finitely generated hence $p(x)$ is equivalent to a formula (cfr.\@ Exercise~\ref{ex_fingen_principal}). Therefore, by elementarity, the claim holds for $N$.
\end{proof}


Hilbert's Nullstellensatz comes in two variants. The one in Corollary~\ref{coroll_Nullstellensatz} is sometimes referred to as the \textit{strong\/}  Nullstellensatz. The weaker variant is stated in Exercise~\ref{ex_weakNullstellensatz}.



 %For instance, when we call a $\Delta(A)\jj$type an \textit{ideal}, we identify it with a subset of the ring of polynomials $A[x]$.

%Hilbert's Nullstellensatz comes in two variants. We already have (up to some small input from algebra) the so-called weak form. This says that in an algebraically closed field every system of equation, which is not blatantly inconsistent, has a solution.
% 
% \begin{theorem}[(Hilbert's Nullstellensatz - weak form)]
% Let $A\subseteq M\models T_{\rm id}$ and let $p({\mr x})$ be a $\Delta(A)\jj$type. If $x$ is a finite tuple. Then for every $\<A\>_M\subseteq N\models T_{\rm acf}$
% \begin{itemize}
% \item[1.] $p({\mr x})$ has no solution in $N$;
% \item[2.] for some polynomial $t_i({\mr x})\in p$ and some parameters $c_i\in\<A\>_M$
% 
% \ceq{\hfill \sum^n_{i=1}c_i t_i({\mr x})}{=}{1}\QED
% \end{itemize}
% \end{theorem}
% 
% The theorem holds also for tuples $x$ of length $\le|N|$ if we require $|A|<|N|$. The proof is similar, with no need to apply Hilbert's Basis Theorem.
% 
% 
% \begin{proof}
% (\ssf{2}$\IMP$\ssf{1}) Trivial.
% 
% (\ssf{1}$\IMP$\ssf{2}) Negate \ssf{2} then we can extend $p$ to a maximal ideal $q$. Then $A[x]/q$ is field that extends $\<A\>_M$ and $q({\mr x})$ has a solution in $A[x]/q$. The identity map $\id_A:A[x]/q\to N$ is a partial immersion and, if $A|<|N|$ it can be extend to one which is defined on the solution of $q({\mr x})$ and \ssf{2} follows. In general, we need to apply   Hilbert's Basis Theorem which says that $p$ is finitely generated. In other words, $p({\mr x})$ is a principal $\Delta(A)\jj$type, see Exercise~\ref{ex_fingen_principal}. If $p({\mr x})$ is principal, we can replace $A$ with some finite subset. 
% \end{proof}


We conclude this section by showing that the notions of primeness for types and ideals coincide (if we restrict to types closed under logical consequences).

\begin{proposition}\label{prop_tipi_e_ideali_primi}
Let $A\subseteq M\models T_{\rm id}$ and let $p(x)$ be a $\Delta(A)\jj$type closed under logical consequences. Then the following are equivalent
\begin{itemize}
\item[1.] $p(x)$ is a prime $\Delta(A)\jj$type;
\item[2.] $p$ is a prime ideal.
\end{itemize}
\end{proposition}

\begin{proof}
\ssf{1}$\IMP$\ssf{2} \ Assume \ssf{1} and suppose that the polynomial $t(x)\cdot s(x)$ belongs to $p$. Clearly, over $T_{\rm id} \cup\Diag\<A\>_M$  we have $\ \proves\ t(x)\cdot s(x)=0\imp  t(x)=0\vee s(x)=0$. As $p(x)$ is a prime $\Delta(A)\jj$type, $p(x)\proves t(x)=0$ or $p(x)\proves s(x)=0$. By Proposition~\ref{prop_chiusura-radicale}, the type $p(x)$ is closed under logical consequences, therefore $t(x)\in p$ or $s(x)\in p$.

\ssf{2}$\IMP$\ssf{1} \ Assume $p$ is a prime ideal and for some $t_i(x)=0\in\Delta(A)$

\ceq{\hfill p(x)}{\proves}{\bigvee^n_{i=1} t_i(x)=0.}

Then

\ceq{\hfill p(x)}{\proves}{\prod^n_{i=1} t_i(x)\ =\ 0}

Since $p(x)$ is closed under logical consequences, and $p$ is a prime ideal, $t_i(x)\in p$ for some $i$. Hence $p(x)$ contains the equation $t_i(x)=0$. By Corollary~\ref{coroll_test_primalita} this suffices to prove that $p(x)$ is a prime $\Delta(A)\jj$type.
\end{proof}


%says that if the ideal generated by $p$ is proper (i.e.\@ does not contain $1$), then $p(x)$ has a solution in $N$. This is a consequence of Corollary~\ref{coroll_Nullstellensatz}. in fact, if $p(x)$ is inconsistent in $N$ the set at the r.h.s.\@ of the equality above is the whole of $A[x]$.


\begin{exercise}\label{ex_weakNullstellensatz}Let $N\models T_{\rm acf}$ and let $p(x)$ be a $\Delta(N)\jj$type where $|x|<\omega$. Prove that the following are equivalent
\begin{itemize}
\item[1.] $p$ is a proper ideal;
\item[2.] $p(x)$ has a solution in $N$.\QED
\end{itemize}
\end{exercise}


\begin{exercise}\label{ex_fingen_principal}
Let $A\subseteq M\models T_{\rm id}$ and let $p(x)$ be a $\Delta(A)\jj$type. Prove that the following are equivalent
\begin{itemize}
\item[1.] $p(x)$ is a principal $\Delta(A)\jj$type;
\item[2.] the ideal generated by $p$ is finitely generated.\QED
\end{itemize}
\end{exercise}


\end{document}
