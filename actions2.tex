% !TEX root = creche.tex
\chapter{Five notions of largeness}
\label{actions}

\def\medrel#1{\parbox[t]{5ex}{$\displaystyle\hfil #1$}}
\def\ceq#1#2#3{\parbox[t]{17ex}{$\displaystyle #1$}\medrel{#2}{$\displaystyle #3$}}

\noindent\llap{\textcolor{red}{\Large\warning}\kern1.5ex}\ignorespaces
Chapter under revision.

In this chapter, $L$ is a signature, $T$ is a complete theory without finite models, and $\U$ is a saturated model of inaccessible cardinality $\kappa$ strictly larger than $|L|$.
We use the same notation and make the same implicit assumptions as in Section~\ref{monster}.

Throughout this chapter \emph{$\grZ$\/} $\subseteq\U^{\gr z}$ is an arbitrary set (unless otherwise is explicitely stated).

%%%%%%%%%%%%%%%%%%%%%%%%%%%%%%%%%%%%%%%%%%%%
%%%%%%%%%%%%%%%%%%%%%%%%%%%%%%%%%%%%%%%%%%%%
%%%%%%%%%%%%%%%%%%%%%%%%%%%%%%%%%%%%%%%%%%%%
%%%%%%%%%%%%%%%%%%%%%%%%%%%%%%%%%%%%%%%%%%%%
\section{The dual perspective on invariance}\label{dual_perspective}
% \def\BDelta{\LDelta\kern-.4ex_{{\rm qf}}}

We set the stage in great (admitedly, somewhat overblown) generality.

Let \emph{$G$\/} be a semigroup that acts on the subsets of $\U^{\mr x}$, or on some Boolean algebra of subsets of $\U^{\mr x}$.
We write ${\cdot}$ for such action and for the operation of $G$.
We assume that the action of $G$ commutes with Boolean connectives.

Thoughout this chapter \emph{${\mr\pi}$\/} denotes a proper filter on the Boolean algebra mentioned above.
This filter will mostly be left implicit.
% As much of the following makes sense when ${\mr\pi}$ is a filter generated by some type-definable set $\mrX\subseteq\U^{\mr x}$, we encourage the reader to assume this on a first reading.

We say that ${\mr\Aa_1},\dots,{\mr\Aa_n}\subseteq\U^{\mr x}$ are consistent with ${\mr\pi}$ to mean that ${\mr\pi}\cup\{{\mr\Aa_1},\dots,{\mr\Aa_n}\}$ generates a proper filter.
We say that ${\mr\Aa_1},\dots,{\mr\Aa_n}$ cover ${\mr\pi}$ as synonymous of saying that the union of ${\mr\Aa_1},\dots,{\mr\Aa_n}$ is in ${\mr\pi}$.

We say that a set $\mrA\subseteq\U^{\mr x}$ is \emph{syndetic\/} under the action of $G$ , or \emph{$G$-syndetic\/} for short, if finitely many $G$-translates of $\mrA$ cover ${\mr\pi}$; we say \emph{$n$-$G$-syndetic\/} if $\le n$ translates suffices.
Dually, we say that $\mrA$ is \emph{thick\/} under the action of $G$, or \emph{$G$-thick\/} for short, if any finitely many $G$-translates of $\mrA$ is consistent with $\pi$; we say \emph{$n$-$G$-thick\/} when the request is limited to $\le n$ translates.

\noindent\llap{\textcolor{red}{\Large\warning}\kern1.5ex}\ignorespaces
The terminology above is taken from topological dynamics.
In other contexts (see eg.~\cite{Newelski09}) syndetic sets are called \textit{generic.}
% In~\cite{CK} the authors write \textit{quasi-non-dividing\/} for \textit{thick\/} under the action of $\Aut(\U/A)$.
% Their terminology has good motivations, but it would be a mouthful if adapted to our context.

When more than one filter ${\mr\pi}$ is at play, we say that $\mrA$ is $G$-syndetic (respectively, $G$-thick) \emph{relative\/} to ${\mr\pi}$.

\begin{notation}
  When $C\subseteq G$ we write \emph{$C{\cdot}\mrA$\/} for $\{h{\cdot}\mrA : h\in C\}$.
\end{notation}

Therefore, $\mrA$ is $G$-syndetic if there is a finite $C\subseteq G$ such that $\cup\,C{\cdot}\mrA$ is in ${\mr\pi}$; and $\mrA$ is $G$-thick if for every finite $C\subseteq G$ the set $\cap\,C{\cdot}\mrA$ is consistent with ${\mr\pi}$.

In this chapter many proofs require some juggling with negations as epitomized by the following fact.

\begin{fact}\label{fact_fip}
  The following are equivalent
  \begin{itemize}
    \item[1.] $\mrA$ is not $G$-syndetic
    \item[2.] $\neg\mrA$ is $G$-thick (throughout the chapter \emph{$\neg\mrA$\/} denotes the complement in $\U^{\mr x}$).
  \end{itemize}\smallskip
\end{fact}

\begin{proof}
  Spelling out the definitions, \ssf1 and \ssf2 are, respectively, equivalent to
  \begin{itemize}
    \item[1$'$.] there is no finite $C\subseteq G$ such that $\cup\, C{\cdot}\mrA$ covers ${\mr\pi}$.
    \item[2$'$.] $\cap\, C{\cdot}\neg\mrA$ consistent with ${\mr\pi}$ for every finite $C\subseteq G$.
  \end{itemize} 
  Then the equivalence is evident.
\end{proof}

The following dual pair of corollaries are linked to important properties discussed in Section~\ref{tame_landscape}.

\begin{corollary}
  The following are equivalent
  \begin{itemize}
    \item [1.] every $G$-syndedic set is $G$-thick
    \item [2.] $\mrA$ and $\neg\mrA$ are not both $G$-syndetic.
  \end{itemize}
\end{corollary}

\begin{corollary}
  The following are equivalent
  \begin{itemize}
    \item [1.] every $G$-thick set is $G$-syndetic
    \item [2.] $\mrA$ and $\neg\mrA$ are not both $G$-thick.
  \end{itemize}
\end{corollary}

We now focus on a particular algebra of subsets of $\U^{\mr x}$ and a particular type of action.

Let $\Delta\subseteq L(\U)$ be a small set of formulas of the form $\theta({\mr x}\,;{\gr\bar z})$ where ${\gr\bar z}=\<{\gr z_1},\dots,{\gr z_n}\>$ is a tuple of variables of the same sort as ${\gr z}$.
We assume $\Delta$ is closed under Boolean combinations.
We say that a set $\mrD$ is \emph{$\Delta(\grZ)$-definable\/} if it is of the form $\theta(\U^{\mr x}\,;{\gr\bar c})$ for some $\theta({\mr x}\,;{\gr\bar z})\in\Delta$ and ${\gr\bar c}\in{\gr\Z^n}$.
In this chapter the symbol $\mrD$ and $\mrC$ always denote $\Delta(\grZ)$-definable sets even when not explicitly stated.
We write ${\mr x}\in\mrD$ for the formula defining $\mrD$.

Now assume that $G$ acts on $\grZ$.
Then $G$ acts also on the family of $\Delta(\grZ)$-definable sets as follows.
When $g\in G$ then $g{\cdot}\mrD$ is the set $\theta(\U^{\mr x}\,;g{\cdot}{\gr\bar c})$.

We say that a set is \emph{$\Delta(\grZ)$-type-definable\/} if it is the intersection of a small family of $\Delta(\grZ)$-definable sets.
The group $G$ acts also on $\Delta(\grZ)$-type-definable sets in the obvious way.

In the following we work within the algebra of sets of $\Delta(\grZ)$-definable sets and ${\mr\pi}$ is the filter naturally associated to some finitely consistent $\Delta(\grZ)$-type (not necessarily a small one).
Only in a few occasions we need work with the algebra generated by the $\Delta(\grZ)$-type-definable sets~--~the move to this larger algebra is unproblematic and never explicitly mentioned.

For any $p({\mr x})\in L(\U)$, we write $p({\mr x})\proves\theta({\mr x})$ to mean that there is a conjunction $\psi({\mr x})$ of formulas in $p({\mr x})$ such that $\psi({\mr x})\imp\theta({\mr x})$ defines a set in ${\mr\pi}$.
Note that, again, the notation does not display the dependence on ${\mr\pi}$.

% \begin{remark}\label{rem_relative}
% Let $q({\mr x})\subseteq \Delta(\grZ)$.
% We say that $\mrA$ is $G$-syndetic \emph{relative\/} to $q({\mr x})$ if $q({\mr x})$ entails $\cup\,F{\cdot}\mrA$ for some finite $F\subseteq G$.
% Dually, $\mrA$ is $G$-thick \emph{relative\/} to $q({\mr x})$ if $G{\cdot}\mrA$ is finitely consistent with $q({\mr x})$.
% The notions of \textit{wide,\/} \textit{strong syndetic,\/} and \textit{weakly thick,\/}  which we indroduce below relitivize likewise.
% All results below easily relativize to any type $q({\mr x})$.
% We entrust the generalization to the reader (this is only required in Fact~\ref{fact_stationarity}).
% \end{remark}

We write \emph{$\Sigma_G$\/} for the collection of $\Delta(\grZ)$-definable $G$-syndetic sets and $\Sigma_G({\mr x})$ for the corresponding type, i.e.

\ceq{\hfill\emph{$\Sigma_G({\mr x})$}}{=}{\big\{\theta({\mr x})\in\Delta(\grZ)\ :\ \theta(\mrX)\textrm{ is }G\textrm{-syndetic}\big\}.}

Note that when $\Sigma_G({\mr x})\cup\{{\mr x}\in\mrD\}$ is finitely consistent with ${\mr\pi}$, then $\neg\mrD$ is not $G$-syndetic.
Then $\mrD$ is $G$-thick.
We prove that the converse implication holds if we replace ${\mr x}\in\mrD$ with complete type.

We write $S_{\Delta,\pi}(\grZ)$ for the set of $\Delta(\grZ)$-types that are maximal among those that are consistent with ${\mr\pi}$.

\begin{theorem}\label{thm_syndetic_invariant}
  For every $p({\mr x})\in S_{\Delta,\pi}(\grZ)$ the following are equivalent
  \begin{itemize}
    \item[1.] $p({\mr x})$ is $G$-invariant
    \item[2.] $p({\mr x})\proves\Sigma_G({\mr x})$
    %\item[2$'$.] $p({\mr x})\proves\Sigma_{G}({\mr x})$ and is finitely consistent
    \item[3.] $p({\mr x})$ is $G$-thick.
    %\item[3$'$.] $p({\mr x})$ is $G$-thick and is finitely consistent.
  \end{itemize}
\end{theorem}

\begin{proof}
  \ssf1$\IMP$\ssf2.
  Let $\mrD$ be $G$-syndetic.
  Pick $C\subseteq G$ be finite such that $C{\cdot}\mrD$ covers ${\mr\pi}$.
  Then $p({\mr x})$ is finitely consistent with ${\mr x}\in\cup\,C{\cdot}\mrD$.
  Therefore, by maximality, $p({\mr x})\proves {\mr x}\in h{\cdot}\mrD$ for some $h\in C$.
  Finally, by invariance, $p({\mr x})\proves{\mr x}\in\mrD$.
  
  \ssf2$\IMP$\ssf3.
  If conjunctions of formulas in $p({\mr x})$ are consistent with $\Sigma_G({\mr x})$, thickness follows from the above remark.

  \ssf3$\IMP$\ssf1.
  % uppose $p({\mr x})$ is not $(n+2)$-thick and let $n$ be minimal.
  % Let $g_1,\dots,g_{n+2}\in G$ and $\theta({\mr x})\,;{\gr\bar b}$ witness this.
  % Then 
  Negate \ssf1.
  Let $\mrD$ be a $G$-syndetic set.
  Let $g\in G$ such that $p({\mr x})\proves{\mr x}\in\mrD$ and $p({\mr x})\notproves g{\cdot}\mrD$.
  By maximality $p({\mr x})\proves{\mr x}\in(\mrD\cap\neg g{\cdot}\mrD)$.
  Clearly $\mrD\cap \neg g{\cdot}\mrD$ is not $2$-$G$-thick as it is inconsistent with its $g$-translate.
%  \ssf1$\IMP$\ssf2$'{\IFF}$\ssf3$'$. Apply what proved above with $\mrX$ for $\mrY$.
\end{proof}

% \begin{corollary}\label{corol_gammaG_invaqriancer}
%   The following are equivalent for every $ \Delta(\grZ)$-definable set $\mrD$
%   \begin{itemize}
%     \item [1.] $\Sigma_G({\mr x})\proves{\mr x}\in\mrD$
%     \item [2.] $p({\mr x})\proves{\mr x}\in\mrD$ for every $G$-thick $p({\mr x})\in S_{\Delta,\pi}(\grZ)$.
%   \end{itemize}
% \end{corollary}

% \begin{proof}
%   \ssf1$\IMP$\ssf2.
%   This is an immediate consequence of Theorem~\ref{thm_syndetic_invariant}.

%   \ssf2$\IMP$\ssf1.
%   Suppose $\Sigma_G({\mr x})\not\proves{\mr x}\in\mrD$.
%   Then there is a type $p({\mr x})\in S_{\Delta,\pi}(\grZ)$ finitely consistent with $\Sigma_G({\mr x})\cup\{{\mr x}\notin\mrD\}$.
%   By Corollary~\ref{corol_q_pers} $p({\mr x})$ is $G$-thick.
%   Then $\neg\ssf2$.
% \end{proof}

 The following immediate consequence is worth highlighting

\begin{corollary}\label{corol_def_mu}
  If a type a $G$-thick $p({\mr x})\in S_{\Delta,\pi}(\grZ)$ exists, then for any $\mrD$
  \begin{itemize}
    \item[1.] $\Sigma_G({\mr x})$ is finitely consistent with ${\mr\pi}$.
    \item[2.] there are $G$-thick types in $S_{\Delta,\pi}(\grZ)$.\smallskip
  \end{itemize}
\end{corollary}


The following theorem gives a necessary and sufficient condition for the  existence of global $G$-invariant $\Delta(\grZ)$-type.
Ideally, we would like that every $G$-thick $\Delta(\grZ)$-type extends to a global $G$-thick type.
Unfortunately this is not true in general (it is a strong assumption which is discussed in Section~\ref{tame_landscape}).

A set $\mrD$ is \emph{$G$-wide\/} if every finite cover of $\mrD$ by $\Delta(\grZ)$-definable sets contains a set that is $G$-thick.
A type is $G$-wide if every conjunction of formulas in the type is $G$-wide.

% In~\cite{CK} a similar property is called \textit{quasi-non-forking.}
% Our use of the term \textit{wide\/} is consistent with~\cite{Hr}, though we apply it to a narrow context.

\begin{theorem}\label{thm_syndetic_invariant2}
  For every $\Delta(\grZ)$-definable set $\mrD$ the following are equivalent 
  \begin{itemize}
    \item[1.] $\Sigma_G({\mr x})\cup\{{\mr x}\in\mrD\}$ is finitely consistent
    \item[2.] $p({\mr x})\proves {\mr x}\in\mrD$ for some $G$-thick type $p({\mr x})\in S_{\Delta,\pi}(\grZ)$
    \item[3.] $\mrD$ is $G$-wide.\smallskip
  \end{itemize}
\end{theorem}

\begin{proof}
  \ssf1$\IMP$\ssf2.
  Pick any $p({\mr x})\in S_{\Delta,\pi}(\grZ)$ that extends $\Sigma_G({\mr x})\cup\{{\mr x}\in\mrD\}$.
  Then $p({\mr x})$ is $G$-thick by Theorem~\ref{thm_syndetic_invariant}.

  \ssf2$\IMP$\ssf1.
  By Theorem~\ref{thm_syndetic_invariant}.

  \ssf2$\IMP$\ssf3.
  Let ${\mr\C_1},\dots,{\mr\C_n}$ cover $\mrD$.
  Pick $p({\mr x})$ as in \ssf2.
  By completeness, $p({\mr x})\proves {\mr x}\in{\mr\C_i}$ for some $i$.
  Therefore, ${\mr\C_i}$ is $G$-thick.

  \ssf3$\IMP$\ssf2.
  Let $p({\mr x})$ be maximal among the $\Delta(\grZ)$-types containing $\mrD$ that are $G$-wide.
  We claim that $p({\mr x})$ is a complete, $G$-thick $\Delta(\grZ)$-type.
  We prove completeness.
  Suppose for a contradiction that $\theta({\mr x}),\neg\theta({\mr x})\notin p$.
  By maximality there is some formula $\psi({\mr x})$, a conjunction of formulas in $p({\mr x})$, and some $\Delta(\grZ)$-definable sets ${\mr\C_1},\dots,{\mr\C_n}$ that cover both $\psi(\mrX)\cap\theta(\mrX)$ and $\psi(\mrX)\smallsetminus\theta(\mrX)$ and such that no ${\mr\C_i}$ is $G$-thick.
  As ${\mr\C_1},\dots,{\mr\C_n}$ cover $\psi(\mrX)$ this is a contradiction.
  It is only left to show that $p({\mr x})$ is $G$-thick.
  This follows from completeness and Theorem~\ref{thm_syndetic_invariant}.
\end{proof}

% The following easy consequence of the theorem will be used below.

% \begin{corollary}\label{corol_intersectionGwide}
%   Let $\mrD$ be $G$-wide.
%   Then $\mrD\cap g{\cdot}\mrD$ is $G$-wide for every $g\in G$.
% \end{corollary}

% \begin{proof}
%   Let $p({\mr x})\in S_{\Delta,\pi}(\grZ)$ be a $G$-thick type such that $p({\mr x})\proves{\mr x}\in\mrD$.
%   By $G$-invariance $p({\mr x})\proves{\mr x}\in g{\cdot}\mrD$.
% \end{proof}

% Let $q({\mr x})\subseteq \Delta(\grZ)$.
% We say that $q({\mr x})$ is \emph{$G$-prime\/} if for every $ \Delta(\grZ)$-definable set $\mrD$ and every $g_1,g_2\in G$ if $q({\mr x})\proves{\mr x}\in(g_1{\cdot}\mrD\cup g_2{\cdot}\mrD)$ then $q({\mr x})\proves{\mr x}\in g_i{\cdot}\mrD$ for some $i$.
% The following fact is immediate.

% \begin{fact}
%   The following are equivalent for every $ \Delta(\grZ)$-type $q({\mr x})$
%   \begin{itemize}
%     \item [1.] $q({\mr x})$ is $G$-prime and $G$-invariant
%     \item [2.] if $\mrD$ is $ \Delta(\grZ)$-definable, $g\in G$, and $q({\mr x})\proves{\mr x}\in(\mrD\cup g{\cdot}\mrD)$, then $q({\mr x})\proves{\mr x}\in\mrD$.
%   \end{itemize}
% \end{fact}

% \begin{proposition}
%   For every type $q({\mr x})\subseteq \Delta(\grZ)$ the following are equivalent
%   \begin{itemize}
%     \item [1.] $q({\mr x})\proves\Sigma_H({\mr x})$
%     \item [2.] $q({\mr x})$ is $G$-prime and $G$-invariant.
%   \end{itemize}
% \end{proposition}

% \begin{proof}
%   \ssf1$\IMP$\ssf2.
%   Suppose $q({\mr x})\proves{\mr x}\in(\mrD\cup g{\cdot}\mrD)$.
%   Let $p({\mr x})\in S_{\Delta,\pi}(\grZ)$ extend $q({\mr x})$.
%   By completeness $p({\mr x})\proves{\mr x}\in\mrD$ or $p({\mr x})\proves{\mr x}\in g{\cdot}\mrD$.
%   As $p({\mr x})$ is $G$-invariant by Theorem~\ref{thm_syndetic_invariant}, it follows that
%   $p({\mr x})\proves{\mr x}\in\mrD$.
%   Finally, as $p({\mr x})$ is arbitrary, $q({\mr x})\proves{\mr x}\in\mrD$. 

%   \ssf2$\IMP$\ssf1.
%   As $\mrD$ is $G$-syndetic and $q({\mr x})$ is $G$-prime, $q({\mr x})\proves{\mr x}\in g{\cdot}\mrD$ for some $g\in G$.
%   Finally $q({\mr x})\proves{\mr x}\in\mrD$ by invariance.
% \end{proof}


\begin{exercise}\label{fact_fip2}
  The following characterization of thick sets is also useful (and is sometimes taken as the definition).
  The following are equivalent
  \begin{itemize}
    \item[1.] $\Aa$ is $G$-thick
    \item[2.] $\Aa\cap\B\neq\varnothing$ for every set $\B$ that is $G$-syndetic.
  \end{itemize}
\end{exercise}

% \begin{proof}
%   \ssf1$\IMP$\ssf2.
%   If $\mrA$ is $G$-thick and $\mrA\cap\mrB=\varnothing$ then $\mrA\subseteq\neg\mrB$.
%   Then $\neg\mrB$ is also $G$-thick which, by Fact~\ref{fact_fip} implies that $\mrB$ is not $G$-syndetic.

%   \ssf2$\IMP$\ssf1.
%   If $\mrA$ is not $G$-thick then $\mrA$ is $G$-syndetic then $\neg$\ssf2.
% \end{proof}

\begin{exercise}
  Let $\X=\Z=\U$.
  Prove that if $p(x)\in S_{\Delta,\pi}(\U)$ is finitely satisfiable in every $M\supseteq A$ then it is thick under the action of $G=\Autf(\U/A)$.
  Is the same true for incomplete types?
\end{exercise}

\begin{exercise}
  Prove that the following are equivalent for every $\D$
  \begin{itemize}
    \item[1.] $\D$ is $G$-thick
    \item[2.] there is a type $p(x)\in S_{\Delta,\pi}(\Z)$ containing $\{x\in g{\cdot}\D\;:\;g\in G\}$.
  \end{itemize}
\end{exercise}

\begin{exercise}
  Let $p(x)\in S_{\Delta,\pi}(\Z)$.
  Prove that the following are equivalent
  \begin{itemize}
    \item[1.] $p(x)$ is $G$-invariant
    \item[2.] $p(x)\proves x\in\D$ for every 2-$G$-syndetic definable set $\D$
    \item[3.] $p(x)$ is 2-$G$-thick.
  \end{itemize}
\end{exercise}

\begin{exercise}
  Prove that the following are equivalent 
  \begin{itemize}
    \item[1.] there is a $G$-invariant type $p(x)\in S_{\Delta,\pi}(\Z)$ such that $p(x)\proves x\in\D$.
    \item[2.] every finite cover of $\D$ by $\BDelta(\Z)$-definable sets contains a 2-$G$-thick set.
  \end{itemize}
\end{exercise}

% \begin{exercise}\label{ex_gen_sat}
%   Let $p(x)\subseteq \BDelta(\Z)$ be $G$-syndetic.
%   Prove that it is finitely satisfiable in $\X$.
% \end{exercise}

\begin{exercise}\label{ex_thick_types}
  Let $\Y$ be a $\Delta(\Z)$-type-definable set.
  Prove the following are equivalent
  \begin{itemize}
    \item[1.] $\Y$ is $G$-thick
    \item[2.] $\Y$ is definable by a $G$-thick type.
  \end{itemize}
\end{exercise}

\begin{exercise}
  Give an example of a thick set that is not wide.
  Hint: find inspiration in Example~\ref{ex_cyclic_order}.
\end{exercise}

\begin{exercise}\label{ex_syndetic_type_vs_formulas}
  Let $\Y\subseteq\X$ be a $G$-syndetic $\Delta(\Z)$-type-definable set.
  Prove that $\Y$ is definable by an $\Delta(\Z)$-type containing only $G$-syndetic formulas.
\end{exercise}

% \begin{exercise}\label{ex_Hprime}
%   Prove that if $q(x)$ is $G$-prime and $q(x)\proves x\in \cup\,G{\cdot}\D$ for some finite $G\subseteq G$, then $q(x)\proves x\in\D$.
% \end{exercise}

% \begin{exercize}
%   Let $\mrD$ be a $ \Delta(\grZ)$-definable set.
%   Prove that the following are equivalent
%   \begin{itemize}
%     \item [1.] $\mrD$ is $G$-wide;
%     \item [2.] every finite cover of $\mrD$ by $\pmDelta({\grZ})$-definable sets contains a $G$-thick set.
%   \end{itemize}
% \end{exercize}M7FJT0RB61M

%%%%%%%%%%%%%%%%%%%%%%%%%%
%%%%%%%%%%%%%%%%%%%%%%%%%%
%%%%%%%%%%%%%%%%%%%%%%%%%%
%%%%%%%%%%%%%%%%%%%%%%%%%%
%%%%%%%%%%%%%%%%%%%%%%%%%%
\section{Notable subgroups}\label{G0}
\def\medrel#1{\parbox[t]{5ex}{$\displaystyle\hfil #1$}}
\def\ceq#1#2#3{\parbox[t]{12ex}{$\displaystyle #1$}\medrel{#2}{$\displaystyle #3$}}

Unfortunately, syndeticity is not preserved under intersection.
In particular $\Sigma_G({\mr x})$ is not a $G$-syndetic type, and it may even be inconsistent.
Then the following notion is relevant.
Recall that an underlying proper filter ${\mr\pi}$ is implicit in all what follows.

\begin{definition}\label{def_Q}\ 

  \ceq{\hfill\emph{$Q_G$\/}}{=}{\big\{q({\mr x})\subseteq\Sigma_G({\mr x})\;:\ q({\mr x})\textrm{ maximally }G\textrm{-syndetic}\big\}.}\smallskip

  In other words, the types in $Q_G$ are maximal among the subtypes of $\Sigma_G({\mr x})$ that are closed under conjunction.
\end{definition}

It is easy to see that $Q_G$ is closed under the action of $G$.
We write $\Stab(q)$ for the stabilizer of $q({\mr x})\subseteq\Delta(\grZ)$ in $G$, that is, the subgroup $\{g\in G:g{\cdot}q({\mr x})=q({\mr x})\}$.
We write $\Stab(\mrD)$ with a similar meaning.
Finally we define

\ceq{\hfill\emph{$G^1$}}{=}{\Stab(Q_G)\medrel{=}\bigcap_{q\in Q_G}\Stab(q).}

It is easy to verify that $G^1\trianglelefteq G$.

\begin{proposition}\label{prop_StabQ}\ \smallskip

  \ceq{\hfill G^1}{=}{\big\{g\in G\ :\ \mrD\cap g{\cdot}\mrD\in\Sigma_G\textrm{ for every }\mrD\in\Sigma_G\big\}.}\smallskip
\end{proposition}

\begin{proof}
  $\subseteq$.
  Pick any $k\in G^1$ and $\mrD\in\Sigma_G$.
  Let $q({\mr x})\in Q_G$ be a type containing ${\mr x}\in\mrD$.
  From the $G^1$-invariance of $q({\mr x})$ we obtain that $q({\mr x})\proves {\mr x}\in k{\cdot}\mrD$.
  Then $q({\mr x})\proves {\mr x}\in \mrD\cap k{\cdot}\mrD$, hence $\mrD\cap k{\cdot}\mrD$ is $G$-syndetic.

  $\supseteq$.
  Pick any $g\notin G^1$.
  Then $q({\mr x})\neq g{\cdot}q({\mr x})$ for some $q({\mr x})\in Q_G$.
  Let $\phi({\mr x})\in q$ such that $q({\mr x})\not\proves g{\cdot}\phi({\mr x})$.
  By maximality, $\psi({\mr x})\wedge g{\cdot}\phi({\mr x})$ is not $G$-syndetic for some $\psi({\mr x})\in q$.
  As $q({\mr x})$ is closed under conjunction, we can assume $\phi({\mr x})=\psi({\mr x})$, then $g$ does not belong to the set on the r.h.s.
\end{proof}

\begin{theorem}\label{thm_gammaK}
  Any finite conjunction of formulas in $\Sigma_{G^1}({\mr x})$ is $G$-syndetic.
  In particular $\Sigma_{G^1}({\mr x})$ is finitely consistent.
\end{theorem}

\begin{proof}
  \def\medrel#1{\parbox[t]{5ex}{$\displaystyle\hfil #1$}}
  \def\ceq#1#2#3{\parbox[t]{23ex}{$\displaystyle #1$}\medrel{#2}{$\displaystyle #3$}}
  Notice that from Proposition~\ref{prop_StabQ} it easily follows that for every $\mrD\in\Sigma_G$ and every finite $F\subseteq G^1$ the set $\cap\,F{\cdot}\mrD$ is $G$-syndetic.
   
  Let ${\mr\D_1},\dots,{\mr\D_n}\in\Sigma_{G^1}({\mr x})$.
  Assume inductively that ${\mr\D_1}\cap\dots\cap{\mr\D_{n-1}}$ is $G$-syndetic.
  Let $F\subseteq G^1$ be such that $\cup\,F{\cdot}{\mr\D_n}$ covers ${\mr\pi}$.
  Then
  
  \ceq{\hfill\cup\,F{\cdot}[{\mr\D_1}\cap\dots\cap{\mr\D_n}]}
  {\supseteq}{\cap\, F{\cdot}[{\mr\D_1}\cap\dots\cap{\mr\D_{n-1}}]\ \cap\ \cup\,F{\cdot}{\mr\D_n}}

  \ceq{}{=}{\cap\, F{\cdot}[{\mr\D_1}\cap\dots\cap{\mr\D_{n-1}}].}

  This last set is $G$-syndetic by the inductive hypothesis and what remarked above.
  The $G$-syndeticity of ${\mr\D_1}\cap\dots\cap{\mr\D_n}$ follows.
\end{proof}

Unfortunately, we are unable to conclude that the intersection of $G^1$-syndetic sets is $G^1$-syndetic.
% To obtain a subgroup with this property, we need to iterate the above construction.
% We define $G^{n+1}=\Stab(Q_{G^n})$ and let $G^\omega$ be the intersection of all $G^n$. 
% It is easy to see that $\Sigma_{G^\omega}$ is closed under conjunction.
% More on the properties of $G^\omega$ in Section~\ref{tame_landscape}.
% However, it is important to emphasize that, in general, $G^\omega$ could be trivial.

From Theorems~\ref{thm_syndetic_invariant2} and~\ref{thm_gammaK} it follows that $G^1$-syndetic sets are $G^1$-wide.
But we can do better.
% In particular, every $q({\mr x})\in Q_G$ has an extension to a $G^1$-invariant $p({\mr x})\in S_{\Delta,\pi}(\grZ)$.
% If this extension is unique we say that $q({\mr x})$ is stationary.
First, we remark a useful consequence of normality.

\begin{remark}\label{rem_invariance_normalsubg}
% \newlength{\ceqlength}
% \settowidth{\ceqlength}{p(x) is H-invariant\ }
% \def\medrel#1{\parbox[t]{5ex}{$\displaystyle\hfil #1$}}
% \def\ceq#1#2#3{\parbox[t]{\ceqlength}{$\displaystyle #1$}\medrel{#2}{$\displaystyle #3$}}
\def\medrel#1{\parbox[t]{5ex}{$\displaystyle\hfil #1$}}
\def\ceq#1#2#3{\parbox[t]{20ex}{$\displaystyle #1$}\medrel{#2}{$\displaystyle #3$}}
  Assume $H\trianglelefteq G$.
  For every $\mrD$ and every $g\in G$ \smallskip
  
  \ceq{\hfill\mrD\textrm{ is }H\textrm{-foo}}{\IFF}{g{\cdot}\mrD\textrm{ is }H\textrm{-foo},} \smallskip
  
  where \textit{foo\/} can be replaced by \textit{syndetic,} \textit{invariant,} \textit{thick,} \textit{wide.}
  In particular, the type $\Sigma_H({\mr x})$ is $G$-invariant.
  % It also follows that for any $p({\mr x})\in S_{\Delta,\pi}(\grZ)$
  % \begin{itemize}
  %   \item [3.] \ceq{\hfill p({\mr x})\textrm{ is }H\textrm{-invariant}}{\IFF}{g\cdot p({\mr x})\textrm{ is }H\textrm{-invariant}.}
  % \end{itemize}
\end{remark}

Recall that when $\Sigma_H({\mr x})$ is finitely consistent with ${\mr\pi}$ then $H$-syndetic sets are $H$-wide, see Theorem~\ref{thm_syndetic_invariant2}.
As it happens, under the assumption of normality, this can be strengthened as follows.

\begin{proposition}\label{prop_Gsyndetic_Hthick1}
  Assume $H\trianglelefteq G$ and that $\Sigma_H({\mr x})$ is finitely consistent with ${\mr\pi}$.
  Then every $G$-syndetic $\mrD$ is $H$-wide.
  In particular all types in $Q_G$ are $G^1$-wide.
\end{proposition}

\begin{proof}
  Let $p({\mr x})\in S_{\Delta,\pi}(\grZ)$ be finitely consistent with $\Sigma_H({\mr x})$.
  As $\mrD$ is $G$-syndetic, by completeness $p({\mr x})\proves{\mr x}\in g{\cdot}\mrD$ for some $g\in G$.
  As $p({\mr x})$ is $H$-thick, by Remark~\ref{rem_invariance_normalsubg} also $g{\cdot}\mrD$ is $H$-thick.
  Then ${\mr x}\in g{\cdot}\mrD$ is finitely consistent with $\Sigma_H({\mr x})$ and the proposition follows from Theorem~\ref{thm_syndetic_invariant2}.
\end{proof}

\begin{comment}

It would help to have a more syntactic description of $G^1$.
The group $G^0$ defined below is a candidate. %~--~we will see that under the assumption of stability it indeed coincides with $G^1$. 
We revisit the notions introduced in Section~\ref{eq_algebraic} with the only difference that we apply them to $ \Delta(\grZ)$, not to the full language.%$ L(\mrX,\grZ)$.

\begin{definition}\label{def_G0}
  Write \emph{$\Phi^0$\/} for the set of $ \Delta(\grZ)$-formulas/sets that have finite $G$-orbit and define%\vskip-1ex

  \ceq{\hfill\emph{$G^0$}}{=}{\Stab(\Phi)}\medrel{=}$\displaystyle\bigcap\big\{\Stab(\mrD) :\  \mrD\in\Phi^0\}$

  and%\vskip-.5ex

  \ceq{\hfill E^0}{=}{\big\{\epsilon_\D({\mr x},{\mr y})\ :\  \mrD\in\Phi^0\big\},}%\vskip-0.5ex
  
  where%\vskip-0.5ex

  \ceq{\hfill\epsilon_\D({\mr x},{\mr y})}{=}{\bigwedge_{g\in G}\big[{\mr x}\in g{\cdot}\mrD\iff{\mr y}\in g{\cdot}\mrD\big].}

  Let $P^0$ be the set of types of the form $\big\{\epsilon({\mr x}\,;{\mr a})\ :\ \epsilon({\mr x}\,;{\mr y})\in E^0\big\}$ for ${\mr a}\in\mrX$.
\end{definition}

Note that, when $G=\Aut(\U/A)$, then $G^0$ is the group of $\LDelta$-automorphisms of $\UDelta$ that fix $\LDelta\mbox-\acl^\eq\varnothing$.% and $E$ contains all the the $\varnothing$-definable finite equivalence relations.

\begin{remark}\label{lem_trans_action}
  Equivalence classes of relations in $E^0$ are in $\Phi^0$ and every set in $\Phi^0$ is union of such equivalence classes.
  Then $G^0$ is the stabilizer of $P^0$.\smallskip

  Assume $G$ acts transitively.
  Then the sets in $\Phi^0$ are $G$-syndetic.
  Every $p({\mr x})\in P^0$ extends to some $q({\mr x})\in Q_G$.
  Then $G^1\le G^0$.
\end{remark}

% \def\DLascar{L\kern-.6ex\raisebox{1.1ex}{\tiny$\Delta$}\kern-.1ex ascar}

Ideally, we would like to have $G^1=G^0$.
This happens when the types in $P^0$ have a unique extension to a type in $Q_G$.
We will prove that this is the case under the assumption of stability.

\end{comment}

A set $\mrY\subseteq\U^{\mr x}$ is \emph{$G$-minimal\/} if for every $g\in G$ either $g{\cdot}\mrY=\mrY$ or $\mrY\cap g{\cdot}\mrY=\varnothing$.

\begin{definition}\label{def_G00}
  Let $\Phi^0$ be the collection of $\Delta(\grZ)$-definable, $G$-minimal sets with a finite orbits that covers ${\mr\pi}$.
  Let $\Phi^{00}$ be the collection of $\Delta(\grZ)$-type-definable, $G$-minimal sets with a small $G$-orbit that covers ${\mr\pi}$
  Define\smallskip

  \ceq{\hfill\emph{$G^0$}}{=}{\Stab(\Phi^0)}\medrel{=}$\displaystyle\bigcap\big\{\Stab(\mrD)\ :\  \mrD\in\Phi^0\}$

  and

  \ceq{\hfill\emph{$G^{00}$}}{=}{\Stab(\Phi^{00})}\medrel{=}$\displaystyle\bigcap\big\{\Stab(\mrY)\ :\  \mrY\in\Phi^{00}\}$.

\end{definition}

\begin{proposition}
  Assume that ${\mr\pi}$ is generated by $\Delta(\grZ)$-definable sets.
  Then 
  
  \ceq{\hfill G^1}{\leq}{G^{00}}\medrel{\leq}$G^0$.
\end{proposition}

\begin{proof}
  The inclusion $G^{00}\leq G^0$ is trivial.
  Let $\mrY\in\Phi^{00}$.
  If $\mrD$ is any $\Delta(\grZ)$-definable set containing $\mrY$, a small number of translations of $\mrD$ cover ${\mr\pi}$.
  Therefore, by compactness, $\mrD$ is $G$-syndetic.
  Then the type defining $\mrY$ is $G$-syndetic.
  Then there is some $q({\mr x})\in Q_G$ entailing $\mrY$.
  By $G^1$-invariance $q({\mr x})$ entails also $k{\cdot}\mrY$ for any $k\in G^1$.
  As $\mrY$ is minimal, $k$ is in the stabilizer of $\mrY$.
\end{proof}

%%%%%%%%%%%%%%%%%%%%%%%%%%%
%%%%%%%%%%%%%%%%%%%%%%%%%%%
%%%%%%%%%%%%%%%%%%%%%%%%%%%
%%%%%%%%%%%%%%%%%%%%%%%%%%%
%%%%%%%%%%%%%%%%%%%%%%%%%%%
\section{Strong syndeticity}\label{strong_syndeticity}

A set $\mrD$ is \emph{strongly $G$-syndetic\/} if for every finite $F\subseteq G$ the set $\cap\,F{\cdot}\mrD$ is $G$-syndetic (recall that $F{\cdot}\mrD$ stands for $\{h{\cdot}\mrD : h\in F\}$).
Dually, we say that $\mrD$ is \emph{weakly $G$-thick\/} if for some finite $F\subseteq G$ the set $\cup\,F{\cdot}\mrD$ is thick.
Again, the same properties is attributed to the corresponding formulas and to types if every conjunction of formulas in the type has the property.

\noindent\llap{\textcolor{red}{\Large\warning}\kern1.5ex}\ignorespaces
In topological dynamic, strong syndedic sets are called \textit{thickly syndetic\/} and weak thickness is called \textit{piecewise syndetic.}
Newelski in~\cite{Newelski09} says \textit{weak generic\/} for weakly thick.
These terminologies defy my intuition.

\begin{lemma}\label{lem_strongly_syndetic}
  The intersection of two strongly $G$-syndetic sets is strongly $G$-syndetic.
\end{lemma}

\begin{proof}
  Let ${\mr\D}$ and ${\mr\C}$ be strongly $G$-syndetic and let $C\subseteq G$ be an arbitrary finite set.
  It suffices to prove that $\mrB=\cap\, C{\cdot}({\mr\C}\cap{\mr\D})$ is $G$-syndetic. 
  Clearly $\mrB={\mr\C'}\cap{\mr\D'}$, where ${\mr\C'}=\cap\, C{\cdot}{\mr\C}$ and ${\mr\D'}=\cap\, C{\cdot}{\mr\D}$.
  Note that ${\mr\C'}$ and ${\mr\D'}$ are both strongly $G$-syndetic.
  In particular $\cup\,F{\cdot}\mr\D'$ covers ${\mr\pi}$ for some finite $F\subseteq G$.
  Note that
  
  \ceq{\hfill\cup\,F{\cdot}\mrB}{=}{\cup\,F{\cdot}\big[{\mr\C'}\ \cap\ {\mr\D'}\big]}
  
  \ceq{}{\supseteq}{ \big(\cap\, F{\cdot}{\mr\C'}\big)\ \cap\ \big(\cup\,F{\cdot}{\mr\D'}\big)}
  
  \ceq{}{=}{\cap\, F{\cdot}{\mr\C'}}
  
  As ${\mr\C'}$ is strongly $G$-syndetic, $\cap\, F{\cdot}{\mr\C'}$ is $G$-syndetic.
  Therefore $\cup\,F{\cdot}\mrB$ is also $G$-syndetic.
  The $G$-syndeticity of $\mrB$ follows.
\end{proof}

Define the following type

\ceq{\hfill\emph{${\mathstrut}^{\rm s}\kern-.1ex\Sigma_G({\mr x})$}}{=}{\{\theta({\mr x})\in  \Delta(\grZ)\ :\ \theta({\mr x})\textrm{ is strongly }G\textrm{-syndetic}\}.}\smallskip

Note that Theorem~\ref{thm_gammaK} shows that $\Sigma_{G^1}({\mr x})\subseteq{\mathstrut}^{\rm s}\kern-.1ex\Sigma_G({\mr x})$.

\begin{corollary}\label{corol_str_gen}
  Then ${\mathstrut}^{\rm s}\kern-.1ex\Sigma_G({\mr x})$ is finitely consistent, strongly $G$-syndetic, and $G$-invariant.\smallskip

  Moreover, ${\mathstrut}^{\rm s}\kern-.1ex\Sigma_G({\mr x})$ is the maximal $G$-syndetic $G$-invariant type and\medskip

  \ceq{\hfill{\mathstrut}^{\rm s}\kern-.1ex\Sigma_G({\mr x})}{=}{\bigcap_{q\in Q_G}q({\mr x})}.
\end{corollary}

\begin{proof}
  Strong $G$-syndeticity is an immediate consequence of Lemma~\ref{lem_strongly_syndetic}.
  Finite consistency is a consequence of syndeticity.
  Finally, $G$-invariance is clear because any translate of a strongly $G$-syndetic formula is also strongly $G$-syndetic.
\end{proof}

The following should be compared with Theorem~\ref{thm_syndetic_invariant2}.

\begin{corollary}\label{corol_q_w_pers}  
  For every $\Delta(\grZ)$-definable set $\mrD$ the following are equivalent
  \begin{itemize}
    \item [1.] ${\mathstrut}^{\rm s}\kern-.1ex\Sigma_G({\mr x})\cup\{{\mr x}\in\mrD\}$ is finitely consistent
    \item [2.] $\mrD$ is weakly $G$-thick.
  \end{itemize}
\end{corollary}

\begin{proof}
  \ssf1$\IMP$\ssf2.
  If ${\mathstrut}^{\rm s}\kern-.1ex\Sigma_G({\mr x})\cup\{{\mr x}\in\mrD\}$ is finitely consistent, then $\neg\mrD$ is strongly $G$-syndetic.
  From Fact~\ref{fact_fip}, we obtain that $\neg\mrD$ not being strongly $G$-syndetic is equivalent to $\mrD$ being weakly $G$-thick.

  \ssf2$\IMP$\ssf1.
  Suppose ${\mathstrut}^{\rm s}\kern-.1ex\Sigma_G({\mr x})\proves{\mr x}\notin\mrD$.
  Then $\neg\mrD$ is strongly $G$-syndetic.
  From Fact~\ref{fact_fip}, $\mrD$ is not weakly $G$-thick.
\end{proof}

The following theorem asserts that the property of weak thickness is partition regular.

\begin{theorem}\label{thm_wt_partreg}
  If $\mrC\cup\mrB$ is weakly $G$-thick then $\mrB$ or $\mrC$ is weakly $G$-thick.
\end{theorem}

\begin{proof}
  As ${\mathstrut}^{\rm s}\kern-.1ex\Sigma_G({\mr x})$ is closed under conjunction.
  If ${\mr x}\in \mrC\cup\mrB$ is finitely consistent with ${\mathstrut}^{\rm s}\kern-.1ex\Sigma_G({\mr x})$ then so is one of the two sets.
\end{proof}

% Assume that every $\mrD\in \Delta(\grZ)$ has a small $G$-orbit.
% A set $\mrD$ with small $G$-orbit is $G$-thick if and only if $\cap\,G{\cdot}\mrD$ is nonempty.
% Note that the latter set is always $G$-invariant.

% \begin{proposition}
%   Assume that every $\mrD\in \Delta(\grZ)$ has a small $G$-orbit.
%   Then the following are equivalent for every $\mrD\in \Delta(\grZ)$
%   \begin{itemize}
%     \item [1.] $\mrD$ is weakly $G$-thick
%     \item [2.]$\mrD$ is syndetic relative to some $G$-invariant $ \Delta(\grZ)$-type-definable set $\mrC$.
%   \end{itemize}
% \end{proposition}

% \begin{proof}
%   \ssf1$\IMP$\ssf2. Let $F\subseteq G$ be finite and such that $\cup\,F{\cdot}\mrD$ is $G$-thick.
%   Then \ssf2 holds if we take $\cap\,G{\cdot}(\cup\,F{\cdot}\mrD)$ as $\mrC$.

%   \ssf2$\IMP$\ssf1. Let $F\subseteq G$ be finite and such that $\mrC\subseteq\cup\,F{\cdot}\mrD$.
%   As $\mrC$ is, trivially, $G$-thick, $\mrD$ is weakly $G$-thick.
% \end{proof}

% Note that 
% %from Corollary~\ref{corol_str_gen} and~\ref{corol_q_w_pers} 
% the $G$-translate of a weakly $G$-thick type is weakly $G$-thick.

\begin{exercise}
  Prove that the following are equivalent
  \begin{itemize}
    \item[1.] $\D$ is weakly $G$-thick
    \item[2.] $\D=\C\cap\B$ for some $G$-syndetic set $\B$ and some $G$-thick set $\C$
    \item [3.] there is a non $G$-syndetic set $\C$ such that $\D\cup\C$ is $G$-syndetic.
  \end{itemize}
\end{exercise}

\begin{exercise}
  Prove that $G$-syndetic sets are weakly $G$-thick.
\end{exercise}

% \begin{question}
%   Let $H\trianglelefteq G$.
%   Does strongly $G$-syndetic implies strongly $H$-syndetic or vice versa?
%   Does strongly $G$-syndetic implies $G^1$-syndetic?
% \end{question}

  
%%%%%%%%%%%%%%%%%%%%%%%%%%%%%%%%%%%
%%%%%%%%%%%%%%%%%%%%%%%%%%%%%%%%%%%
%%%%%%%%%%%%%%%%%%%%%%%%%%%%%%%%%%%
%%%%%%%%%%%%%%%%%%%%%%%%%%%%%%%%%%%
%%%%%%%%%%%%%%%%%%%%%%%%%%%%%%%%%%%
\begin{comment}
\section{The diameter of a Lascar type}\label{newelski}

As an application we prove an interesting property of the Lascar types.
Recall that $\Ll({\mr a}/A)$, the Lascar strong type of ${\mr a}\in\U^{\mr x}$, is the union of a chain of type-definable sets of the form $\big\{{\mr x}\ :\ d_A({\mr a},{\mr x})\le n\big\}$.
In this section we prove that $\Ll({\mr a}/A)$ is type-definable (if and) only this chain is finite.
In other words, only if the connected component of ${\mr a}$ in the Lascar graph has finite diameter.

It is convenient to address the problem in more general terms.
We assume that $H\le\Aut(\U/A)$ acts transitively on $\mrX$.
Let \emph{$C$\/} $\subseteq H$ be a set of generators of $H$ that is
\begin{itemize}
  \item[1.] symmetric i.e.\@ it contains the unit and is closed under inverse
  \item[2.] conjugacy invariant i.e.\@ $g{\cdot}C{\cdot}g^{-1}=C$ for every $g\in H$
\end{itemize}

We define a discrete metric on $\mrX$.
For ${\mr a},{\mr b}\in\mrX$ let \emph{$d_C({\mr a},{\mr b})$\/} be the minimal $n$ such that ${\mr a}\in C^n{\mr b}$.
This defines a metric which is $H$-invariant by \ssf2.
The \emph{diameter\/} of a set $\mrC\subseteq\mrX$ is the supremum of $d_C({\mr a},{\mr b})$ for ${\mr a},{\mr b}\in\mrC$.

We are interested in sufficient conditions for $\mrX$ to have finite diameter.
The notions introduced in Section~\ref{strong_syndeticity} offer some hint.

\begin{proposition}\label{prop_wpers_finite_diameter}
  Let $H$ act transitively on $\mrX$.
  Then, if $\mrX$ contains a weakly thick subset of finite diameter, $\mrX$ itself has finite diameter.
\end{proposition}

\begin{proof}
  Let $\mrC\subseteq\mrX$ be a set of finite diameter, say $n$.
  Let $F\subseteq H$ be finite.
  We claim that also $\cup\,F{\cdot}\mrC$ has finite diameter.
  In fact, pick any ${\mr a}\in\mrC$.
  Then $\mrC$ is contained in a ball of radius $n$ centered in ${\mr a}$. Let $m$ be the maximum of $d_C(h{\mr a}, k{\mr a})$ for $h,k,\in F$.
  Clearly, the diameter of $\cup\,F{\cdot}\mrC$ is at most $2n+m$.
  This proves the claim.

  By the claim, if there is a weakly thick subset of finite diameter, there is also a (plain) thick subset $\mrC\subseteq\mrX$ of finite diameter, say $n$.
  
  By the transitivity of the action, any two elements of $\mrX$ are of the form $h{\mr a}$, $k{\mr a}$ for some $h,k\in H$ and some ${\mr a}\in\mrC$.
  By thickness, there are ${\mr c}\in\mrC\cap h\mrC$ and ${\mr d}\in\mrC\cap k\mrC$.
  Then 

  \ceq{\hfill d_C(h{\mr a},\, k{\mr a})}{\le}{d_C(h{\mr a},\,{\mr c})\ +\ d_C({\mr c},\, {\mr d})\ +\ d_C({\mr d},\,k{\mr a})}

  \ceq{}{\le}{n+n+n.}

  Therefore the diameter of $\mrX$ does not exceed $3n$.
\end{proof}

\begin{theorem}\label{thm_newelski}
  Suppose that $\mrX$ and the sets ${\mr\X_n}=C^n{\mr a}$, for some ${\mr a}\in\mrX$, are type-definable.
  Then $\mrX$ has finite diameter.
\end{theorem}

\begin{proof}
  By Proposition~\ref{prop_wpers_finite_diameter}, it suffices to prove that ${\mr\X_n}$ is weakly thick.
  By Corollary~\ref{corol_q_w_pers} it suffices to show that for some $n$ the type ${\mathstrut}^{\rm s}\kern-.1ex\Sigma_G({\mr x})$ is finitely satisfiable in ${\mr\X_n}$.
  Suppose not.
  Let $\psi_n({\mr x})\in{}^{\rm s}\kern-.2ex\Sigma_G$ be a formula that is not satisfied in $\mr\X_n$.
  Then the type $p({\mr x})=\{\psi_n({\mr x}):n\in\omega\}$ is finitely consistent.
  From the type-definablity of $\mrX$ it follows that $p({\mr x})$ has a realization in $\mrX$.
  As this realization belongs to some ${\mr\X_n}$ we contradict the definition of $\psi_n({\mr x})$.
\end{proof}

% \ceq{\hfill p_n({\mr\X},{\mr\X})}{=}{\{\<{\mr a},{\mr b}\>\in{\mr\X^2}\ :\ {\mr a}\in G^1^n{\mr b}\}}

% Below we write ${\mr x}\in C^n{\mr y}$ for the type $p_n({\mr x},{\mr y})$ and ${\mr x}\in G\,{\mr y}$ for the disjunction of all these types (n.b.\@ an infinite disjunction of types need not be a type).

\begin{example}\label{ex_newelski}
  Let $\Delta=L_{{\mr x}\,{\gr z}}(A)$, where $|{\gr z}|=\omega$.
  Let $\mrX=\Ll({\mr a}/A)$ and $\grZ=\U^{\gr z}$.
  Assume that $\mrX$ is type definable.
  Let $G=\Aut(\U)$.
  Let $C\subseteq G$ be the set of automorphisms that fix a model containing $A$.
  Then the group $H$ generated by $C$ is $\Autf(\U/A)$ and $H{\cdot}{\mr a}=\mrX$.
  
  Then $d_C({\mr a},{\mr b})$ coincides with the distance in the Lascar graph.
  As shown in Proposition~\ref{prop_Lascar_distance_type_def} the sets $C^n{\cdot}{\mr a}=\{{\mr x}:d_C({\mr x},{\mr a})\le n\}$ are type definable.
  Then from Theorem~\ref{thm_newelski} it follows that $\Ll({\mr a}/A)$ is type definable (if and) only if it has finite diameter.
\end{example} 

\end{comment}
%%%%%%%%%%%%%%%%%%%%%%%%%%%%%%%%%%%
%%%%%%%%%%%%%%%%%%%%%%%%%%%%%%%%%%%
%%%%%%%%%%%%%%%%%%%%%%%%%%%%%%%%%%%
%%%%%%%%%%%%%%%%%%%%%%%%%%%%%%%%%%%
%%%%%%%%%%%%%%%%%%%%%%%%%%%%%%%%%%%
\section{A tamer landscape}\label{tame_landscape}

Under suitable assumptions some notions introduced in this chapter coalesce, and we are left with a tamer landscape.
We will see an example in Theorem~\ref{thm_thick_finsat}.

\begin{theorem}\label{thm_coalesce}
  The following are equivalent (\textit{set\/} is a short for $\Delta(\grZ)$-\textit{definable set})\nobreak
  \begin{itemize}
    \item[1.] $G$-thick sets are $G$-wide
    \item[2.] $G$-syndetic sets are closed under intersection 
    \item[3.] $G$-syndetic sets are strongly $G$-syndetic
    \item[4.] weakly $G$-thick sets are $G$-thick.
  \end{itemize}
\end{theorem}

\begin{proof}
  Clearly \ssf2$\IFF$\ssf3$\IFF$\ssf4.

  \ssf1$\IMP$\ssf2.
  Let $\mrC$ and $\mrD$ be $G$-syndetic sets.
  Suppose for a contradiction that $\mrC\cap\mrD$ is not $G$-syndetic.
  Then $\neg(\mrC\cap\mrD)$ is $G$-thick.
  By \ssf1 and Theorem~\ref{thm_syndetic_invariant2} there is a $G$-invariant type $p({\mr x})\in S_{\Delta,\pi}(\grZ)$ such that $p({\mr x})\proves{\mr x}\notin\mrC\cap\mrD$.
  By completeness either $p({\mr x})\proves{\mr x}\notin\mrC$ or $p({\mr x})\proves{\mr x}\notin\mrD$.
  This is a contradiction because by Theorem~\ref{thm_syndetic_invariant} $p({\mr x})\proves{\mr x}\in\mrC$ and $p({\mr x})\proves{\mr x}\in\mrD$.

  % \ssf2$\IMP$\ssf1.
  % By \ssf2, $\Sigma_G({\mr x})$ is consistent.
  % Then $\mrC$ is $G$-syndetic if and only if $\Sigma_G({\mr x})\proves{\mr x}\in\mrC$.
  % If $\mrD$ is $G$-thick, then $\neg\mrD$ is not $G$-syndetic.
  % Then $\Sigma_G({\mr x})\notproves{\mr x}\notin\mrD$ by what remarked above.
  % Then $\mrD$ is $G$-wide by Theorem~\ref{thm_syndetic_invariant2}.

  \ssf4$\IMP$\ssf1. By Theorem~\ref{thm_wt_partreg}
\end{proof}

It is convenient to prove one last characterization of the above phenomenon.
This is used in the next section. 

\begin{proposition}\label{prop_coalese}\ 
  The equivalent conditions in Theorem~\ref{thm_coalesce} are also equivalent to\smallskip

  \ceq{\ssf5.\hfill\Sigma_G({\mr x})}{=}{\bigcap\Big\{p({\mr x})\in S_{\Delta,\pi}(\grZ)\ :\ p({\mr x})\text{ is }G\text{-invariant}\Big\}.}
\end{proposition}

Note that $\subseteq$ is always true, therefore \ssf5 amounts to claiming that the type on the r.h.s.\@ is $G$-syndetic.
\begin{proof} 
  Assume \ssf1 of Theorem~\ref{thm_coalesce}.
  It suffices to prove $\supseteq$, because the converse inclusion is always true. 
  Let $\mrD$ be non $G$-syndetic.
  Then $\neg\mrD$ is $G$-thick and therefore $G$-wide.
  Then some $G$-invariant global type $p({\mr x})$ entails $\neg\mrD$.
  Therefore ${\mr x}\in\mrD$ does not belong to the type on the r.h.s.
  
  Vice versa, note that the type on the r.h.s.\@ is closed under conjunction.
  Then \ssf2 of Theorem~\ref{thm_coalesce} immediately follows from \ssf5.
\end{proof}

We say that ${\mr\pi}$ is \emph{$G$-stationary\/} if there is a unique $G$-in\-var\-iant $p({\mr x})\in S_{\Delta,\pi}(\grZ)$.

% The notions of relative syndeticity/thickness are defined in Remark~\ref{rem_relative}.
% It is easy to verify that the above theorem relativize to any type $q({\mr x})$.

The conditions in Theorem~\ref{thm_coalesce} together with stationarity, produce the ultimate simplification.

\begin{fact}\label{fact_stationarity}
  Let ${\mr\pi}$ be $G$-stationary. 
  Assume that the equivalent conditions in Theorem~\ref{thm_coalesce} hold.
  Then the following are equivalent
  \begin{itemize}
    \item [1.] $\mrD$ is $G$-syndetic
    \item [2.] $\mrD$ is $G$-wide/thick.
  \end{itemize}
\end{fact}

\begin{proof}
  1$\Rightarrow$2.
  As $\Sigma_G({\mr x})$ is consistent by assumption, $G$-syndetic sets are $G$-thick.
  (Stationarity is not required in this direction.)

  2$\Rightarrow$1.
  If $\mrD$ was not $G$-syndetic then $\neg\mrD$ would be $G$-thick and therefore, by the assumption, $G$-wide.
  This contradicts $G$-stationarity.
\end{proof}

%%%%%%%%%%%%%%%%%%%%%%
%%%%%%%%%%%%%%%%%%%%%%
%%%%%%%%%%%%%%%%%%%%%%
%%%%%%%%%%%%%%%%%%%%%%
%%%%%%%%%%%%%%%%%%%%%%
\section{Examples: the random graph}

In this section $\U$ is a saturated model of the of theory of the random graph.
Let $\grZ=\U$ and ${\mr\pi}=\{\U\}$.
Let $G={\rm Aut}(\U)$ acts naturally on $\grZ$.
We prove that the equivalent conditions in Theorem~\ref{thm_coalesce} hold.

There are two $G$-invariant global types $p_i({\mr x})$.
The first says that $x$ is adjacent to all vertices, the second says that $x$ is adjacent to none.
The type on the r.h.s.\@ of (5) in Proposition~\ref{prop_coalese} is equivalent to the disjunction of $p_1({\mr x})$ and $p_2({\mr x})$.
This is in turn equivalent to the type containing the formulas

\ceq{\hfill\varphi({\mr x},a)}{=}{r({\mr x},a_1)\leftrightarrow r({\mr x},a_2)\leftrightarrow\cdots\cdots\leftrightarrow r({\mr x},a_n)}

for every tuple of vertices $a=\langle a_1,\dots,a_n\rangle$.
Therefore we only need to prove that these formulas are $G$-syndetic.

\begin{theorem}
  The formula $\varphi_n({\mr x},a)$ is $G$-syndetic for all $n\ge 2$ and all $a=\langle a_1,\ldots,a_n\rangle$.
\end{theorem}

\begin{proof}
  Let $b^1,\ldots,b^n$ be tuples of length $n$ with disjoint ranges that have the same type as $a$ (over $\varnothing$). Let $C$ be the set of tuples $c=\langle c_1,\ldots,c_n\rangle$ such that $c_i\in{\rm range}(b^i)$.
  Finally when we pick the $b^i$, we do so in such a way that every tuple $c\in C$ has the same type as $a$.

  We claim that the following disjunction is a tautology

  \hfil$\displaystyle\bigvee_{c\in C}\varphi({\mr x},c)\ \vee\ \bigvee_{i=1}^n \varphi({\mr x},b^i)$

  Fix $x$.
  If $\varphi({\mr x},b^i)$ holds for some $i$, then we are done.
  Otherwise, for every $i$ we have some $k$ such that $r({\mr x},b^i_k)$ holds (as well as some $h$ such that $r({\mr x},b^i_h)$ does not).
  Let $c_i=b^i_k$ for some $k$ such that $r({\mr x},b^i_k)$ holds.
  Then $\varphi({\mr x},c)$ holds.
\end{proof}

%%%%%%%%%%%%%%%%%%%%%%%%%%
%%%%%%%%%%%%%%%%%%%%%%%%%%
%%%%%%%%%%%%%%%%%%%%%%%%%%
%%%%%%%%%%%%%%%%%%%%%%%%%%
%%%%%%%%%%%%%%%%%%%%%%%%%%
\section{Definable groups}\label{definablegroups}

\def\medrel#1{\parbox[t]{5ex}{$\displaystyle\hfil #1$}}
\def\ceq#1#2#3{\parbox[t]{16ex}{$\displaystyle #1$}\medrel{#2}{$\displaystyle #3$}}

\noindent\llap{\textcolor{red}{\Large\warning}\kern1.5ex}\ignorespaces
Section under major revision.
% \clearpage\setcounter{page}{1}
% \def\mr{}
% \def\gr{}

% \fancyhead{}
% \fancyfoot{}

In this section we assume that $\grZ$ is a group and that the operation on $\grZ$, which will be denoted by $\cdot$, is definable.



Let $M\preceq\U$ be a model.
We write $Z$ for 
$\grZ\cap M^{\gr z}$ respectively.

If $\mrD\subseteq\mrX$ we write \emph{${\rm cl}_Z(\mrD)$\/} for the closure of $\mrD$ in the topology generated by the subsets of $\mrX$ that are $\Delta(Z)$-definable.

When ${\gr g}\in\grZ$ and ${\mr a}\in\mrX$ we write ${\gr g}\cnonfork_M{\mr a}$ if for every $\varphi({\mr x})\in L(M)$ such that $\varphi({\gr g}{\cdot}{\mr a})$ we have that $\varphi(Z{\cdot}{\mr a})\neq\varnothing$.
When $\grC\subseteq\grZ$ and $\mrD\subseteq\mrX$ we define

\ceq{\hfill\emph{$\grC\diamond\mrD$}}{=}{\Big\{{\gr g}{\cdot}{\mr a}\ :\ {\gr g}\in\grC,\ {\mr a}\in\mrD,\ {\gr g}\cnonfork_M {\mr a}\Big\}}

% If $G\le{\rm Aut}(\UDelta)$ and $a\in\mrX$, we write $G{\cdot}a$ for the orbit of $a$ under the action of $G$.
When $\mrD$ is the orbit of ${\mr a}\in\mrX$ under the action of ${\rm Aut}(\U/M)\le\grZ$, we write $\grC\diamond {\mr a}$ for $\grC\diamond\mrD$.

Explicitely,

\ceq{\ssf1.\hfill{\rm cl}_M(\mrD)}{=}{\bigcap\Big\{\varphi(\mrX)\ :\ \varphi({\mr x})\in L(M)\text{ such that }\mrD\subseteq\varphi(\mrX)\Big\}}

\ceq{\ssf2.}{=}{\Big\{{\mr b}\ :\ \varphi(\mrD)\neq\varnothing\text{ for every }\varphi({\mr x})\in L(M)\text{ that is satisfied by }{\mr b}\Big\}}


% \begin{fact}
%   For every $\mrD\subseteq\mrX$\smallskip

%   \ceq{\hfill{\rm cl}_M(\cup\,Z{\cdot}\mrD)}{=}{\grZ\diamond\mrD}
% \end{fact}

% \begin{proof}
%   $\supseteq$. 
%   Let $b\in\grZ\diamond\mrD$, say $b=g{\cdot}a$ for some $g\in\grZ$ and $a\in\mrD$ such that $g\cnonfork_M a$.
%   Let $\varphi(x)\in L(M)$ be such that $Z{\cdot}\mrD\subseteq\varphi(\mrX)$.
%   Then $g\cnonfork_M a$ implies $\varphi(g{\cdot}a)$.
%   This proves that $b\in{\rm cl}_M(\cup\,Z{\cdot}\mrD)$.
  
%   $\subseteq$. 
%   Let $b\in{\rm cl}_M(\cup\,Z{\cdot}\mrD)$ be given and define $p(x)=\LDelta\mbox{-tp}(b/M)$.
%   It suffices to prove that for every $a\in\mrD$ the type $z\cnonfork_M a$ is consistent with $p(z{\cdot}a)$.
%   In fact, if $g\cnonfork_M a$, then $g{\cdot}a\in\grZ\diamond a$.
%   And, as $\grZ\diamond a$ is $M$-invariant, if $p(g{\cdot}a)$ then $b\in\grZ\diamond a$.
%   Now, to prove concistency, assume for a contradiction that $z\cnonfork_M a\rightarrow\neg\varphi(z{\cdot}a)$ for some $\varphi(x)\in p$.
%   As any $m\in Z$ realizes $z\cnonfork_M a$, no element of $Z{\cdot}a$ satisfies $\varphi(x)$.
%   This is a contradiction by \ssf2 above.
% \end{proof}


\begin{fact}
  For every ${\mr a}\in\mrX$\smallskip

  \ceq{\hfill{\rm cl}_M(Z{\cdot}{\mr a})}{=}{\grZ\diamond {\mr a}}
\end{fact}

\begin{proof}
  $\supseteq$. 
  Let ${\mr b}\in\grZ\diamond {\mr a}$, say ${\mr b}={\gr g}{\cdot}{\mr a}$ for some ${\gr g}\cnonfork_M {\mr a}$.
  Let $\varphi({\mr x})\in L(M)$ be such that $Z{\cdot}{\mr a}\subseteq\varphi(\mrX)$.
  Then ${\gr g}\cnonfork_M {\mr a}$ implies $\varphi({\gr g}{\cdot}{\mr a})$.
  This proves that ${\mr b}\in{\rm cl}_M(Z{\cdot}{\mr a})$.
  
  $\subseteq$. 
  Let ${\mr b}\in{\rm cl}_M(Z{\cdot}{\mr a})$ be given and define $p({\mr x})=\tp({\mr b}/M)$.
  It suffices to prove that ${\gr z}\cnonfork_M{\mr a}$ is consistent with $p({\gr z}{\cdot}{\mr a})$.
  In fact, if ${\gr g}\cnonfork_M{\mr a}$, then ${\gr g}{\cdot}{\mr a}\in\grZ\diamond {\mr a}$.
  And, as $\grZ\diamond {\mr a}$ is $M$-invariant, if $p({\gr g}{\cdot}{\mr a})$ then ${\mr b}\in\grZ\diamond{\mr a}$.
  Now, to prove concistency, assume for a contradiction that ${\gr z}\cnonfork_M {\mr a}\rightarrow\neg\varphi({\gr z}{\cdot}{\mr a})$ for some $\varphi({\mr x})\in p$.
  As any $c\in Z$ realizes ${\gr z}\cnonfork_M {\mr a}$, no element of $Z{\cdot}{\mr a}$ satisfies $\varphi({\mr x})$.
  This is a contradiction by \ssf2 above.
\end{proof}

% A subset $A\subseteq\mrX$ is dense if $\psi(A)\neq\varnothing$ for every $\psi(x)\in\Psi$.

Let $\mrD\subseteq\mrX$ be an $M$-definable set.
As $Z$ is small, $\mrD$ is $Z$-thick if and only if $\cap\,Z{\cdot}\mrD$ is consistent.

\begin{fact}\label{fact_thick_circ}
  Let $\mrD\subseteq\mrX$ be definable over $M$.
  Then the following are equivalent
  \begin{itemize}
    \item [1.] $\mrD$ is $Z$-thick
    \item [2.] $\grZ\diamond {\mr a}\subseteq\mrD$ for some ${\mr a}\in\mrD$.
  \end{itemize}
\end{fact}

\begin{proof}
  1$\Rightarrow$2. \ 
  Pick ${\mr a}\in\cap\,Z{\cdot}\mrD$.
  Suppose for a contradiction that ${\gr g}{\cdot}{\mr a}\notin\mrD$ for some ${\gr g}\in\grZ$ such that ${\gr g}\cnonfork_M {\mr a}$.
  Then $c{\cdot}{\mr a}\notin\mrD$ for some $c\in Z$.
  This contradicts the choice of ${\mr a}$.  

  2$\Rightarrow$1. \ 
  It suffices to note that $c\cdot\grZ\diamond {\mr a}=\grZ\diamond {\mr a}$ for every $c\in Z$.
\end{proof}

The following is the dual version of the above fact.

\begin{fact}
  Let $\mrD\subseteq\mrX$ be definable over $M$.
  Then the following are equivalent
  \begin{itemize}
    \item [1.] $\mrD$ is $Z$-syndetic
    \item [2.] $(\grZ\diamond {\mr a})\cap\mrD\neq\varnothing$ for every ${\mr a}\in\mrD$.
  \end{itemize}
\end{fact}

We say that $\mrC\subseteq\mrX$ is $\grZ{\diamond}$-invariant if $\grZ\diamond\mrC\subseteq\mrC$.
% Note that, when $\grZ=\mrX$, this says that $\mrC$ is a left ideal.
The following fact is easy.

\begin{fact}
  Every ($M$-invariant?) minimal $\grZ{\diamond}$-invariant subset of $\mrX$ is of the form $\grZ\diamond {\mr a}$ for some ${\mr a}\in\mrX$ so, in particular, it is type-definable over $M$.
  % We also have that $\grZ\diamond a=\M\diamond a$ for $\M$ any left ideal of $\grZ$.
\end{fact}

We say that ${\mr a}\in\mrX$ is \emph{minimal\/} if $\grZ\diamond {\mr a}$ is a minimal $\grZ{\diamond}$-invariant subset of $\mrX$.

\begin{theorem}
  Let $\mrD\subseteq\mrX$ be definable over $M$.
  Then the following are equivalent
  \begin{itemize}
    \item [1.] $\mrD$ is weakly $Z$-thick
    \item [2.] $\grZ\diamond {\mr a}\subseteq\mrD$ for some minimal ${\mr a}\in\mrD$.
  \end{itemize}
\end{theorem}

\begin{proof}
  2$\Rightarrow$1. \ 
  Let ${\mr b}\in\grZ\diamond {\mr a}$.
  Then ${\mr b}\in\grZ\diamond\mrD$.
  Therefore ${\mr b}\in c{\cdot}\mrD$ for some $c\in Z$.
  This proves that $\grZ\diamond {\mr a}\subseteq\cup\,Z{\cdot}\mrD$.
  By compactness $\grZ\diamond {\mr a}\subseteq\cup\,C{\cdot}\mrD$ for some finite $C\subseteq Z$.
  By Fact~\ref{fact_thick_circ}, $\mrD$ is weakly $Z$-thick.

  1$\Rightarrow$2. \ 
  Let $C\subseteq Z$ be a finite set such that $\cup\,C{\cdot}\mrD$ is $Z$-thick.
  By Fact~\ref{fact_thick_circ}, there is ${\mr a}\in\cup\,C{\cdot}\mrD$ such that $\grZ\diamond {\mr a}\subseteq\cup\,C{\cdot}\mrD$.
  Let ${\mr a'}\in\grZ\diamond {\mr a}$ be minimal.
  As ${\mr a'}\in c{\cdot}\mrD$ for some $c\in C$, we conclude that $c^{-1}\!\cdot {\mr a'}\in\mrD$.
  Then $c^{-1}\!\cdot {\mr a'}$ is the minimal element required by \ssf2.
\end{proof}

\begin{comment}

is a group that acts on $\mrX$.
The group operations and the group action are assumed definable.
We use the symbol $\,\cdot\,$ for both the group multiplication and the group action.
In this section $\Delta$ contains formulas $\phi({\mr x}\,;{\gr z})$ of the form  $\psi({\gr z^{-1}}\!\cdot{\mr x})$ for $\psi({\mr x})\in\Psi$.
The sets $\phi(\mrX\,;\grZ)$ are $\grZ$-invariant.
We write ${\gr 1}$ for the identity of $\grZ$.
Note that $\phi(\mrX\,;{\gr g})={\gr g}\cdot\phi(\mrX\,;{\gr 1})$.

It is worth noticing that automorphisms of $\UDelta$ need not preserve the group operations nor the group action.

To each ${\gr h}\in\grZ$ we associate the $\LDelta$-automorphism $\<{\mr a}\,;{\gr g}\>\mapsto\<{\gr h}{\cdot}{\mr a}\,;{\gr h}{\cdot}{\gr g}\>$.
Therefore $\grZ$ is, up to isomorphism, a subgroup of $\Aut(\U/A)$.
For any ${\gr g}\in\grZ$, the orbit of $\phi(\mrX\,;{\gr g})$ under the action of $\grZ$ is $\{\phi(\mrX\,;{\gr h}) : {\gr h}\in\grZ\}$.
Therefore it coincides with the orbit under the action of $\Aut(\U/A)$.
% The following fact follows immediately.

% \begin{fact}
%   For every  $ \Delta(\grZ)$-type-definable set $\mrD$ the following are 
%   \begin{itemize}
%     \item [1.] the $G$-orbit of $\mrD$ is bounded
%     \item [2.] the $\grZ$-orbit of $\mrD$ is bounded.
%   \end{itemize}
% \end{fact}


% In the applications $H$ will be either $\Autf(\UDelta)$ or $\Aut(\UDelta/M)$.

% \def\medrel#1{\parbox[t]{5ex}{$\displaystyle\hfil #1$}}
% \def\ceq#1#2#3{\parbox[t]{6ex}{$\displaystyle #1$}\medrel{#2}{$\displaystyle #3$}}

% \begin{proposition}\label{prop_Gsyndetic_thick}
%   Let $\mrD$ be a $ \Delta(\grZ)$-definable set.
%   Assume that $\Sigma_H({\mr x})$ is consistent.
%   Then \ssf1$\IMP$\ssf2 holds, where
%   \begin{itemize}
%     \item [1.] $\mrD$ is $\grZ$-syndetic
%     \item [2.] ${\gr g}{\cdot}\mrD$ is $H$-wide for every ${\gr g}\in\grZ$.
%   \end{itemize}
% \end{proposition}

% We will revisit this proposition under the assumption of stability and with $\Autf(\U^\Delta)$ for $H$~--~then the consistency of $\Sigma_H({\mr x})$ is guaranteed, and also the converse implication holds.
% See Theorem~\ref{thm_Gsyndetic_thick}.\vspace*{-0.5\baselineskip}
% %
% \begin{proof}
%   Let ${\gr g}$ be given.
%   If $\mrD$ is $\grZ$-syndetic, then so is ${\gr g}{\cdot}\mrD$.
%   Then ${\gr g}{\cdot}\mrD$ is, a fortiori, $G$-syndetic.
%   Therefore \ssf2 follows from Proposition~\ref{prop_Gsyndetic_Hthick}.
% \end{proof}

We also consider the action an arbitrary subgroup $H\le\Aut(\U/A)$.
We write \emph{$({\gr g})_H$\/} for the $H$-orbit of ${\gr g}$, that is, the set $\{f({\gr g})\ :\ f\in H\}$.

\begin{proposition}\label{prop_wideHcojugate}
  Let $\theta({\mr x}\,;{\gr z_1},\dots,{\gr z_n})$ be a Boolean combination of formulas $\phi_i({\mr x}\,;{\gr z_i})$ for some $\phi_i({\mr x}\,;{\gr z})\in\Delta$.
  Then for every ${\gr h_i}\in({\gr g_i})_H$ the following are equivalent
  \begin{itemize}
    \item [1.] $\theta({\mr x}\,;{\gr g_1},\dots,{\gr g_n})$ is $H$-wide
    \item [2.] $\theta({\mr x}\,;{\gr h_1},\dots,{\gr h_n})$ is $H$-wide.
  \end{itemize}
\end{proposition}

\begin{proof}
  Without loss of generality we can assume that only conjunctions occur in $\theta({\mr x}\,;{\gr g_1},\dots,{\gr g_n})$.
  Let ${\mr\C_i}=\phi_i(\mrX\,;{\gr 1})$.
  Then \ssf1 says that $\mrC={\gr g_1}{\cdot}{\mr\C_1}\cap\dots\cap{\gr g_n}{\cdot}{\mr\C_n}$ is $H$-wide.
  Let $f_i\in H$ be such that ${\gr h_i}=f_i({\gr g_i})$.
  Then, by Corollary~\ref{corol_intersectionGwide} also the intersection of the sets $f_i[\mrC]$ is $H$-wide.
  A fortiori the intersection of the sets $f_i[{\gr g_i}{\cdot}{\mr\C_i}]$ is $H$-wide.
  As $f_i[{\gr g_i}{\cdot}{\mr\C_i}]=f_i({\gr g_i}){\cdot}{\mr\C_i}$ the equivalence follows.
\end{proof}

Note that applying the theorem to  $H=\grZ$ we obtain that if $\theta({\mr x}\,;{\gr 1},\dots,{\gr 1})$ is $\grZ$-wide then $\theta({\mr x}\,;{\gr g_1},\dots,{\gr g_n})$ is $\grZ$-wide for every ${\gr g_1},\dots,{\gr g_n}\in\grZ$.
Below we prove a generalization of this. 

If $\grA\subseteq\grZ$, we write \emph{$\<\grA\>$} for the subgroup generated by $\grA$.

\begin{proposition}\label{prop_stabilizer1}
  \def\medrel#1{\parbox[t]{5ex}{$\displaystyle\hfil #1$}}
  \def\ceq#1#2#3{\parbox[t]{15ex}{$\displaystyle #1$}\medrel{#2}{$\displaystyle #3$}}
  Assume $H{\cdot}\grZ=\grZ{\cdot}H$.
   Let $\theta({\mr x}\,;{\gr z_1},\dots,{\gr z_n})$ be a Boolean combination of formulas $\phi_i({\mr x}\,;{\gr z_i})$ for some $\phi_i({\mr x}\,;{\gr z})\in\Delta$.
  Let ${\gr g}\in\grZ$ be arbitrary.
  Assume that $\theta({\mr x}\,;{\gr 1},\dots,{\gr 1})$ is $H$-wide.
  Then $\theta({\mr x}\,;{\gr h_1},\dots,{\gr h_n})$ is $H$-wide for every 
  
  \ceq{\hfill{\gr h_1},\dots,{\gr h_n}}{\in}{\Big\<\bigcup_{{\gr g}\in\grZ}({\gr g})_H^{-1}\!\cdot({\gr g})_H\Big\>.}
\end{proposition}

\begin{proof}
  We proceed by induction on the number of factors of the form ${\gr a^{-1}}\!\cdot{\gr b}$, for some ${\gr a},{\gr b}\in({\gr g})_H$, that occur in ${\gr h_1},\dots,{\gr h_n}$.
  Without loss of generality we can assume that only conjunctions occur in $\theta({\mr x}\,;{\gr g_1},\dots,{\gr g_n})$.
  Let ${\mr\C_i}=\phi_i(\mrX\,;{\gr 1})$.
  Assume inductively that ${\gr h_1}{\cdot}{\mr\C_1}\cap\dots\cap{\gr h_n}{\cdot}{\mr\C_n}$ is $H$-wide.
  Pick two arbitrary ${\gr a},{\gr b}\in({\gr g})_H$.
  By Remark~\ref{rem_invariance_normalsubg} 
  
  \hspace*{7ex}${\gr a}\cdot{\mr\C_1}\ \cap\ {\gr a}\cdot{\gr h_1^{-1}}\!\cdot{\gr h_2}\cdot{\mr\C_2}\ \cap\ \dots\dots\ \cap\ {\gr a}\cdot{\gr h_1^{-1}}\!\cdot{\gr h_n}\cdot{\mr\C_n}$ is $H$-wide.
  
  By Proposition~\ref{prop_wideHcojugate}, in this intersection we can replace ${\gr a}\cdot{\mr\C_1}$ by ${\gr b}\cdot{\mr\C_1}$.
  Then finally 

  \hspace*{7ex}${\gr h_1}\cdot{\gr a^{-1}}\!\cdot{\gr b}\cdot{\mr\C_1}\ \cap\ {\gr h_2}\cdot{\mr\C_2}\ \cap\ \dots\dots\ \cap\ {\gr h_n}\cdot{\mr\C_n}$ is $H$-wide.
\end{proof}

% In Section~\ref{stable_groups} we will give an interesting characterization of this group under the assumption of stability.

\begin{comment}
\section{Tentative}

In this section we work over a given $M\preceq\U^\Delta$.

Let $\mrD\subseteq\mrX$ be definable by some $p({\mr x})\subseteq\BDelta(M)$.
We say that $\mrD$ has 

A formula $\psi({\mr x})\in L$ is \emph{symmetric\/} if $\psi({\gr g}\cdot{\mr x})\iff\psi({\gr g^{-1}}\!\cdot{\mr x})$ holds for every ${\gr g}\in\grZ$.
We say that $\Psi\subseteq L_{\mr x}$ is symmetric if every formula in $\Psi$ is symmetric.
When $\Psi$ is symmetric, the map $\iota:\<{\mr a}\,;{\gr g}\>\mapsto\<{\mr a}\,;{\gr g^{-1}}\>$ is an automorphism of $\U^\Delta$.
As $H$ is normal and $\iota$ is an involution, $\iota\,H=H{\cdot}\iota$ and $H\cup\iota\,H$ is a subgroup of $G$.
In particular, if $f\in H$ and ${\gr g}\in\grZ$ then $f({\gr g^{-1}})=f'({\gr g})^{\gr -1}$ for some $f'\in H$.
We write \emph{$G^1$\/} for $H\cup\iota\,H$.

When $\Psi$ is symmetric, we can strengthen Fact~\ref{prop_wideHcojugate} as follows.

\begin{fact}\label{prop_wideHcojugate_symm}
  Let $\mrD=\phi(\mrX\,;{\gr 1})$ for some $\phi({\mr x}\,;{\gr z})\in\pmDelta$.
  For $i=1,2$ let ${\gr h_i}\in({\gr g_i})_{G^1}$. 
  Then \ssf1 and \ssf2 of Fact~\ref{prop_wideHcojugate} are equivalent
\end{fact}

\begin{proof}
  Let $f_i\in G^1$ be such that ${\gr h_i}\in f_i{\gr g_i})$.
  By Remark~\ref{rem_invariance_normalsubg} it suffices to consider the case $f_1\in H$ and $f_2\in\iota\,H$.
  Then $f_2({\gr g_2})=f_3({\gr g_2^{-1}})$.
  By Fact~\ref{prop_wideHcojugate}, \ssf2 is equivalent to claiming that ${\gr g_1}{\cdot}\mrD\ \cap\ {\gr g_2^{-1}}{\cdot}\mrD$ is $H$-wide. 
  By symmetry ${\gr g_2^{-1}}{\cdot}\mrD={\gr g_2}{\cdot}\mrD$, so the fact follows.
\end{proof}


For any $\grW\subseteq\grZ$, we write $\<\grW\>$ for the subgroup generated by $\grW$.

\begin{proposition}
  Assume $\Psi$ is symmetric.
  Let $\mrD=\phi(\mrX\,;{\gr 1})$ for some $\phi({\mr x}\,;{\gr z})\in\pmDelta$.
  Let ${\gr g}$ be such that $\mrD\cap{\gr g}{\cdot}\mrD$ is $H$-wide.
  Then $\mrD\cap{\gr g'}{\cdot}\mrD$ is $H$-wide for every ${\gr g'}\in\<\,({\gr g})_{G^1}^2\>$.
\end{proposition}

\begin{proof}
  Assume inductively that $\mrD\cap{\gr g'}{\cdot}\mrD$ is $H$-wide, where ${\gr g'}\in\<\,({\gr g})_{G^1}^2\>$.
  Pick two arbitrary ${\gr a},{\gr b}\in({\gr g})_{G^1}$.
  We claim that $\mrD\cap{\gr a}{\cdot}{\gr b}{\cdot}{\gr g'}{\cdot}\mrD$ is $H$-wide.
  From the induction hypothesis it follows that ${\gr b}{\cdot}\mrD\cap{\gr b}{\cdot}{\gr g'}{\cdot}\mrD$ is $H$-wide.
  By Fact~\ref{prop_wideHcojugate_symm}, ${\gr a^{-1}}{\cdot}\mrD\cap{\gr b}{\cdot}{\gr g'}{\cdot}\mrD$ is $H$-wide and the claim follows.
\end{proof}
\end{comment}


%%%%%%%%%%%%%%%%%%%%%%%%%%
%%%%%%%%%%%%%%%%%%%%%%%%%%
%%%%%%%%%%%%%%%%%%%%%%%%%%
%%%%%%%%%%%%%%%%%%%%%%%%%%
%%%%%%%%%%%%%%%%%%%%%%%%%%
\section{Notes and references}

In Example~\ref{ex_newelski} we prove a theorem of Newelski's~\cite{Newelski}.
The original proof is rather long and complex.
A simplified proof (also due, reportedly, to Newelski) appears in~\cite{Pelaez}*{Section 3.3} and~\cite{Casanovas}*{Chapter 9}.
The proof here is a streamlined and generalized version of the latter~--~inspired by~\cite{Z16}.

\begin{biblist}[]\normalsize
\bib{Casanovas}{book}{
  author={Casanovas, Enrique},
  title={Simple theories and hyperimaginaries},
  series={Lecture Notes in Logic},
  volume={39},
  publisher={Cambridge
  University Press},
  date={2011},
  % pages={xiv+169},
  % isbn={978-0-521-11955-9},
  % review={\MR{2814891}},
  % doi={10.1017/CBO9781139003728},
}\smallskip
% \bib{CK}{article}{
%   author={Chernikov, Artem},
%   author={Kaplan, Itay},
%   title={Forking and dividing in ${\rm NTP}_2$ theories},
%   journal={J. Symbolic Logic},
% %  volume={77},
%   date={2012},
% %  number={1},
% %  pages={1--20},
% %  issn={0022-4812},
% %  review={\MR{2951626}},
% %  doi={10.2178/jsl/1327068688},
% }\smallskip
\bib{Newelski09}{article}{
  author={Newelski, Ludomir},
  title={Topological dynamics of definable group actions},
  journal={J. Symbolic Logic},
  date={2009},
}\smallskip
\bib{Hr}{article}{
  label={Hr},
  author={Hrushovski, Ehud},
  title={Stable group theory and approximate subgroups},
  journal={J. Amer. Math. Soc.},
  volume={25},
  date={2012},
  number={1},
  pages={189--243},
  % issn={0894-0347},
  % doi={10.1090/S0894-0347-2011-00708-S},
}\smallskip
\bib{Newelski}{article}{
  author={Newelski, Ludomir},
  title={The diameter of a Lascar strong type},
  journal={Fund. Math.},
%  volume={176},
  date={2003},
%  number={2},
%  pages={157--170},
%  issn={0016-2736},
%  review={\MR{1971306}},
%  doi={10.4064/fm176-2-4},
}\smallskip
\bib{Pelaez}{book}{
  author={Pel\'aez, Rodrigo},
  title={\href{http://www.ub.edu/modeltheory/documentos/ThesisRPP.pdf}{About the Lascar group}},
  series={PhD Thesis},
  publisher={Universitat de Barcelona, Departament de L\'ogica, Hist\'oria i Filosofia de la Ci\'encia},
  date={2008},
  }\smallskip
\bib{Z16}{article}{
  author={Zambella, Domenico},
  title={On the diameter of Lascar strong types after Ludomir Newelski},
  conference={
  title={A tribute to Albert Visser},
  },
  book={
  % series={Tributes},
  % volume={30},
  publisher={Coll. Publ., [London]},
  },
  date={2016},
  status={\href{https://arxiv.org/abs/1605.00218}{arXiv:1605.00218}},
  %  pages={231--236},
  %  review={\MR{3559880}},
}\smallskip
\end{biblist}