\documentclass[creche.tex]{subfiles}
\begin{document}
\chapter*{Preface}
\addcontentsline{toc}{chapter}{Preface}
\label{praface}

These are some old (but regularly updated) notes of a course which I gave for a few years in Amsterdam, and for many more in Turin. They are incomplete - a few more chapters will be added soon. Find a \textit{nightly build\/} version in 

\url{https://github.com/domenicozambella/creche}


Below I list a few peculiarities.

\begin{itemize}
\item Fraïssé limits are presented in a slightly general setting which accommodates more examples (and I'll add a few more). This is used e.g. to discuss saturation.
\item Quantifier elimination for ACF and Hilbert's Nullstellensatz are presented with more details than usual (which may annoy some experts but hopefully help some students).
\item The proof of the Omitting Types Theorem uses a model theoretic construction (different from the standard syntactic proof).
\item Imaginaries and the eq-expansion are introduced from the (equivalent) dual perspective: the canonical name of a definable set is the set itself.
\item Lascar and Kim-Pillay strong types are introduced in a slightly unconventional way. I am grateful to Krzysztof Krupiński for helping me squaring the circle with KP-types.
\item Ramsey Theorem is derived from the existence of coheir sequences. Admittedly, this not the shortest proof, but it may be interesting. I plan to add a proof of Hindman's theorem along the same lines.
\end{itemize}
\end{document}


