% !TEX root = creche.tex
\documentclass[creche.tex]{subfiles}
\begin{document}
\chapter*{Preface}
\addcontentsline{toc}{chapter}{Preface}



\begin{minipage}{0.5\textwidth}\sl
This book was written to answer one question ``Does a recursion theorist dare to write a book on model theory?''

\hfill Gerald E.\@ Sacks

\hfill Saturated Model Theory (1972)
\vspace*{10ex}
\end{minipage}
\hfill
\begin{minipage}{0.45\textwidth}\sl
\includegraphics[width=.99\textwidth]{elephant_playing_piano.jpg}
\end{minipage}
\medskip

\label{praface}
% \vspace*{-8ex}
% \hfill\includegraphics[width=.4\textwidth]{elephant_playing_piano.jpg}

% \vspace*{-22ex}
% \begin{minipage}{0.58\textwidth}
% These are the notes of a course that I have given for a few years in Amsterdam and many more in Turin. A few chapters are missing and will be added soon. I keep the most recent version in
% 
% \smallskip
% \href{github.com/domenicozambella/creche}{https://github.com/domenicozambella/creche}
% \smallskip
% 
% \noindent\llap{\textcolor{red}{\Large\danger}\kern1.5ex}A warning sign is put along the text when the notation is nonstandard. Occasionally, the whole exposition is substantially nonstandard.
% Below a few examples.
% \end{minipage}
%
%\vskip5ex

These are the notes of a course that I have given for a few years in Amsterdam and many more in Turin.
Since then they have grown up considerably and other chapters will be added soon.
Find the most recent version in 

\hfil\href{https://github.com/domenicozambella/creche}{\tt https://github.com/domenicozambella/creche}


\noindent\llap{\textcolor{red}{\Large\danger}\kern1.5ex}A warning sign in the margin indicates that the notation is nonstandard.
Occasionally, the whole exposition is substantially nonstandard.
Below are a few examples.



\begin{itemize}
\item \hyperref[rich]{Fraïssé limits} are presented in a general setting that accommodates a large variety of examples (and I'll add a few more).
This setting is used, for example, to discuss saturation.
\item \hyperref[algebraic]{Quantifier elimination for ACF} and Hilbert's Nullstellensatz are presented in more detail than is usual.
(This may annoy some readers, but I hope it will help others.)
\item The proof of the \hyperref[countable]{Omitting Types Theorem} uses a model theoretic construction.
\item \hyperref[imaginary]{Imaginaries and the eq-expansion} are introduced from the (equivalent) dual perspective that the canonical name of a definable set is the set itself.
\item \hyperref[Ramsey]{Ramsey's Theorem} is derived from the existence of coheir sequences.
This is not the shortest proof, but it should be interesting.
\item Along the same lines we sketch a proof of \hyperref[Hindman]{Hindman's Theorem} which based on coheir sequences and uniqueness of coheir extensions. In the same chapter we also prove the \hyperref[HJ]{Hales-Jewett Theorem}.
\item \hyperref[invariantL]{Lascar and Kim-Pillay types} are introduced in a slightly unconventional way.
\item \hyperref[newelski]{Newelski's Theorem\/} on the diameter of Lascar types is proved in an elementary self-contained way.
\item  \hyperref[external]{Stability and NIP\/} are introduced very briefly.
We only discuss the properties of externally definable sets, which we identify with \textit{approximable sets}.
\end{itemize}
\end{document}
