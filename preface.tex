% !TEX root = creche.tex
\chapter*{Preface}
\addcontentsline{toc}{chapter}{Preface}
\vskip-3ex
\hfill\begin{minipage}{0.5\textwidth}\sl
This book was written to answer one question ``Does a recursion theorist dare to write a book on model theory?''

\hfill Gerald E.\@ Sacks

\hfill Saturated Model Theory (1972)
\vspace*{8ex}
\end{minipage}
\hfill
\begin{minipage}{0.45\textwidth}\sl
\includegraphics[width=.99\textwidth]{elephant_playing_piano.jpg}
\end{minipage}
\medskip

\label{praface}


These are the notes of a course that I have given for a few years in Amsterdam and many more in Turin.
Since then they have grown and other chapters will be added soon.
Find the most recent version in 

\hfil\href{https://github.com/domenicozambella/creche}{\tt https://github.com/domenicozambella/creche}


\noindent\llap{\textcolor{red}{\Large\danger}\kern1.5ex}A warning sign in the margin indicates that the notation is nonstandard.
Occasionally, the whole exposition is substantially nonstandard.
Below are a few examples.

\begin{itemize}
\item \hyperref[rich]{Fraïssé limits} are presented in a general setting that accommodates a large variety of examples (and I'll add a few more).
This setting is used, for example, to discuss saturation.
\item \hyperref[algebraic]{Quantifier elimination for ACF} and Hilbert's Nullstellensatz are presented in more detail than is usual.
(This may annoy some readers, but I hope it will help others.)
\item The proof of the \hyperref[countable]{Omitting Types Theorem} uses a model theoretic construction which highlights the analogy with the Kuratowski-Ulam Theorem.
\item \hyperref[imaginary]{Imaginaries and the eq-expansion} are introduced from the (equivalent) dual perspective that the canonical name of a definable set is the set itself.
\item \hyperref[Ramsey]{Ramsey's Theorem} is derived from the existence of coheir sequences.
This is not the shortest proof, but it is an instructive application of coheirs.
\item Along the same lines we prove the theorems of \hyperref[Hindman]{Hindman} and of \hyperref[HJ]{Hales-Jewett}. This is an instructive application of the uniqueness of coheir extensions.
\item \hyperref[invariantL]{Lascar and Kim-Pillay types} are introduced in a slightly unconventional way.
\item \hyperref[newelski]{Newelski's Theorem\/} on the diameter of Lascar types is proved in an elementary self-contained way.
\item  \hyperref[external]{Stability and NIP\/} are introduced very briefly.
We only discuss the properties of externally definable sets, which we identify with \textit{approximable sets}.
%\item \noindent\emph{Syntax highlighting\/} looks strange in a text of math. There are good reasons: it is difficult to be {\gr coherent}; it is easy to be {\mr eye-o}{\gr ffend}{\mr ing}. These notes may contain a few examples. 
\end{itemize}

\bigskip\bigskip
\hfill Torino, June 2019
