% !TEX root = creche.tex
\chapter{Ordered algebraic structures}
\label{oalgebraic}


Work in progress

%%%%%%%%%%%%%%%%%%%%%%%%%%%%%%%%%%
%%%%%%%%%%%%%%%%%%%%%%%%%%%%%%%%%%
%%%%%%%%%%%%%%%%%%%%%%%%%%%%%%%%%%
\section{Ordered divisible abelian groups}

\def\medrel#1{\parbox{5ex}{\hfil$\displaystyle #1$}}
\def\ceq#1#2#3{\parbox{20ex}{$\displaystyle #1$}\medrel{#2}$\displaystyle  #3$}

We work in language of ordered additive groups $L=\{0,-,+, <\}$.
%
The theory of \emph{ordered abelian groups,} denoted by \emph{$T_{\rm oag}$,} contains $T_{\rm lo}\cup T_{\rm ag}$ and the axiom

\begin{itemize}
  \item[oa.] $x<y\ \imp\ x+z<y+z$.
\end{itemize}

It is not difficult to verify that for every positive integer $n$

\ceq{\#\hfill T_{\rm oag}}{\proves}{x < y \ \iff\ n\,x<n\,y.}

In particular, every model of $T_{\rm oag}$ is a torsion-free group and its order has no endpoints.

We write \emph{$T_{\rm odag}$\/} for the theory of \emph{ordered divisible abelian groups\/} which is obtained by adding to $T_{\rm oag}$ the axiom of divisibility, \ssf{div} of Section~\ref{dag}.
%
Note that

\ceq{\hfill T_{\rm oag}}{\proves}{x < y\ \wedge\ 2z=x+y\ \ \imp\ \ x<z<y.}

Hence $T_{\rm odag}$ is a dense linear order.


We base our study of $T_{\rm odag}$ on what we know about models of $T_{\rm dag}$ and $T_{\rm dlo}$.
%
For this reason we split $L_{\rm at}$ into equations and inequalities.
%
We use the notation \emph{$L_{\rm at,=}$\/} and \emph{$L_{\rm at,<}$\/} respectively.

When $p(x)\subseteq\atL(A)$ we write $p_{=}(x)$ and $p_{<}(x)$ for the intersection of $p(x)$ with $L_{\rm at,=}$ and $L_{\rm at,<}$ respectively.
Note that, by the linearity of the order, the types $\attp(b/A)$ and $\atpmtp(b/A)$ are equivalent.

If $p_{=}(x)$ is trivial, as defined in Notation~\ref{notation1}, then we say that $p(x)$ is \emph{equationally trivial.}
In this case $\proves p(x)\iff p_<(x)$.

For any $A\subseteq M\models T_{\rm oag}$ and ${\mr c}\in M$, we say that ${\mr c}$ is \emph{independent\/} from $A$ if $\attp_M({\mr c}/A)$ is equationally trivial.
Remark~\ref{oss_liberi_qf} takes the following form

\begin{remark}\label{rk_oag_ind}
  Let $k:M\to N$ be a partial embedding and let ${\gr a}$ be an enumeration of $\dom k$.
  %
  We claim that $k\cup\big\{\<{\mr b},{\mr c}\>\big\}:M\to N$ is a partial embedding for every ${\mr b}\in M$ and ${\mr c}\in N$ that are independent from ${\gr a}$, respectively $k{\gr a}$ and such that $\attp_{M,<}({\mr b}\,;{\gr a})=\attp_{N,<}({\mr c}\,;k{\gr a})$. 
\end{remark}

Dependent elements behave pretty much as those in models of $T_{\rm tfag}$.
%
In fact Proposition~\ref{prop_tfag_tipi_princ_comp} holds also here. 
The main difference is that now it is no longer true that all independent elements have the same type.

\begin{proposition}\label{prop_oag_tipi_princ_comp}
  Let $A\subseteq M\models T_{\rm oag}$.
  %
  Let $p({\mr x})=\attp({\mr b}/A)$, where ${\mr b} \in M$.
  %
  Then one of the following holds   
  \begin{itemize}
  \item[1.] ${\mr b}$ is independent from $A$;
  \item[2.] $M\models\phi({\mr b})$ for some $\phi({\mr x})\in L_{\rm at}(A)$ such that $\ \proves\ \phi({\mr x})\imp p({\mr x})$.
  \end{itemize}
\end{proposition}

\begin{proof}
  If we choose the equation $\phi({\mr x})$ as in the proof of Proposition~\ref{prop_tfag_tipi_princ_comp} we obtain $\proves\phi({\mr x})\imp p_{=}({\mr x})$ with the same proof.
  %
  We only need to show that $\proves\phi({\mr x})\imp p_{<}({\mr x})$.
  %
  We repeat the argument of the proof of Proposition~\ref{prop_tfag_tipi_princ_comp} with $<$ for $=$.
  %
  We need to prove that for every $m\in\ZZ$ and every ${\gr c}\in\<A\>_M$ one of the following holds
  \begin{itemize}
  \item[a.] $\proves\   n{\mr x}={\gr a}\imp{\gr c}<m{\mr x}$;
  \item[b.] $\proves\   n{\mr x}={\gr a}\imp m{\mr x}<{\gr c}$;
  \item[c.] $\proves\   n{\mr x}={\gr a}\imp m{\mr x}={\gr c}$.
  \end{itemize}
  Suppose for a contradiction that neither \ssf{a} nor \ssf{b} holds and fix models $N_1$, $N_2$ and some $b_i\in N_i$ such that
  \begin{itemize}
  \item[a$'$.] $N_1\models n{\mr b_1}={\gr a}$ \ and \ $N_1\models {\gr c}\not<m{\mr b_1}$;
  \item[b$'$.] $N_2\models n{\mr b_2}={\gr a}$ \ and \ $N_2\models m{\mr b_2}\not< {\gr c}$;
  \item[b$'$.] $N_3\models n{\mr b_3}={\gr a}$ \ and \ $N_3\models m{\mr b_3}\not< {\gr c}$.
  %
  \end{itemize}
  From \ssf{a$'$} we infer that $N_1\models n{\gr c}\not<m{\gr a}$.
  %
  From \ssf{b$'$} we infer $N_2\models m{\gr a}\not<n{\gr c}$.
  %
  From \ssf{c$'$} we infer $N_3\models m{\gr a}\neq{\gr c}$. 
  But this three formulas also hold in $\<A\>_M$.
  A contradiction.
\end{proof}

As Remark~\ref{rk_oag_ind} suggests, infinite rank alone is not sufficient to guarantee richness.
%
The following property will be required.
\begin{definition}
  We say that $M\models T_{\rm oag}$ is \emph{$\omega_1$-large\/} if for every countable $A\subseteq M$, every finitely consistent $p({\mr x})\subseteq L_{\rm at, <}(A)$ is realized in $M$.
\end{definition}

%
The term \textit{large\/} is unimaginative, but it is provisional and will not occur elsewhere.
%
It will be clear in Section~\ref{saturation_section} that largeness captures the notion of $\omega$-saturation for models of $T_{\rm oag}$. The proof of the following proposition is left to the reader as an exercise.

% \begin{example}
%   The additive groups $\QQ$ and $\RR$, with the usual order, are natural models of $T_{\rm odag}$.
%   % 
%   It is easy to see that $\QQ$ has rank $1$ while $\RR$ has infinite rank for cardinality reasons.

%   In contrast to $T_{\rm dag}$, infinite rank does not fully capture the notion of $\omega$-richness among models of $T_{\rm odag}$ as the following example shows.

%   Consider the type $p(x)\subseteq L$ that contains the formulas $x>n\,1$ for all $n$.
%   %
%   This type is clearly finitely consistent in $\RR$ and, by compactness, it is realized in some elementary extension of $\RR$, but it is not realized in $\RR$.
%   %
%   This elementary extension shows that $\RR$ is not rich.
% \end{example}

% The example above motivates the following definition.



\begin{proposition}\label{prop_odag_large_exists}
  Every $M \models T_{\rm odag}$ has a large elementary extension. 
\end{proposition}

As it happens, large implies infinite rank.

\begin{proposition}\label{prop_odag_ind_dense}
  Let $M\models T_{\rm odag}$ be $\omega_1$-large.
  Then for every countable $A\subseteq M$ there is a dense set of elements that are independent of $A$.
  In particular $M$ has infinite rank.
\end{proposition}

\begin{proof}
  Let $a<b$ be arbitrary elements of $M$.
  It suffices to construct a finitely consistent type $p(x)\subseteq L_{\rm at, <}(A,a,b)$ containing $a<x<b$ that has only solutions independent of $A$.
  Dependent elements are solution of a non trivial equation with parameters in $A$.
  As there are countably many of such equations and each has a unique solution, we can enumerate $\<a_i:i\in\omega\>$ all elements dependent from $A$.
  By density, the type $\{a_i\neq x\}$ is finitely consistent with  $a<x<b$.
  We obtain $p(x)$ by replacing $a_i\neq x$ with either $x<a_i$ or $a_i<x$ according to which one preserve finite consistency.
\end{proof}

We are ready to prove that $\omega_1$-large models of $T_{\rm odag}$ are $\omega_1$-rich.

\begin{lemma}\label{lem_odag_rich}
  Let $N\models T_{\rm odag}$ be $\omega_1$-large.
  %
  Then for every $M\models T_{\rm oag}$, every ${\mr b}\in M$ and every countable partial embedding $k:M\imp N$ there is a ${\mr c}\in N$ such that $k\cup\big\{\<{\mr b}\,;{\mr c}\>\big\}:M\imp N$ is a partial isomorphism.
  
  (That is, $N$ is $\omega_1$-rich in the category of models of $T_{\rm oag}$ with partial embeddings as morphisms.)
\end{lemma}

\begin{proof}
  Let ${\gr a}$ be an enumeration of $\dom k$ and let $p({\mr x}\,;{\gr z})=\atpmtp({\mr b}\,;{\gr a})$.
  %
  The required ${\mr c}$ has to realize $p({\mr x}\,;k{\gr a})$.
  %
  We consider two cases.
  %  
  If ${\mr b}$ is dependent from ${\gr a}$, then Proposition~\ref{prop_oag_tipi_princ_comp} yields a formula $\phi({\mr x}\,;{\gr z})\in L_{\rm at,=}$ such that 

  \noindent\rlap{\ssf{i.}}\hspace{5ex} $\proves\ \phi({\mr x}\,;{\gr a})\imp p({\mr x}\,;{\gr a})$;

  \noindent\rlap{\ssf{ii.}}\hspace{5ex} $\phi({\mr x} \,;{\gr a})$ is consistent\hfill (the underlying theory is $T_{\rm oag}\cup\<{\gr a}\>_M$).

  By isomorphism, \ssf{i} and \ssf{ii} hold with ${\gr a}$ replaced by $k{\gr a}$.
  %
  As remarked at the beginning of this section $M\models T_{\rm tfag}$ and $N\models T_{\rm dag}$.
  %
  As  $\phi({\mr x}\,;{\gr z})\in L_{\rm at, =}$, we can apply Proposition~\ref{prop_md_cons_sodd} and infer that the formula $\phi({\mr x}\,;k{\gr a})$ has a solution ${\mr c} \in N$.

  The second case is when ${\mr b}$ is independent from ${\gr a}$.
  %
  Then by Remark~\ref{rk_oag_ind} it suffices to find some ${\mr c}\in N$ that is independent from $k{\gr a}$ and realizes $p_<({\mr x}\,;k{\gr a})$.
  As $p_<({\mr x}\,;k{\gr a})$ is finitely consistent, the $\omega_1$-largeness of $N$ and  Proposition~\ref{prop_odag_ind_dense} yields the required ${\mr c}$.
\end{proof}

Unsurprisingly the lemma above yields no categoricity result.
%
However quantifier elimination follows in the usual way.

\begin{corollary}
  The theory $T_{\rm odag}$ has elimination of quantifiers.
\end{corollary}
  
\begin{proof}
  Let $k:M\to N$ be a partial embedding between models of $T_{\rm odag}$.
  %
  If we prove that $k$ is elementary, elimination of quantifiers follows by Theorem~\ref{thm_tutteleimmersionisonoelementari_QE}.
  
  The argument is the usual one.
  %
  We can assume that $k$ is finite and $M$, $N$ countable.
  %
  Let $M'$ and $N'$ be the large countable elementary superstructures of $M$, respectively $N$ given by Proposition~\ref{prop_odag_large_exists}.
  %
  As $M'$ and $N'$ are rich, $k:M'\to N'$ is elementary by Theorem~\ref{thm_morphism_rich_elementary}.
  %
  Hence $k:M\to N$ is also elementary.
\end{proof}

%%%%%%%%%%%%%%%%%%%%%%%%%%%%%%%%%%
%%%%%%%%%%%%%%%%%%%%%%%%%%%%%%%%%%
%%%%%%%%%%%%%%%%%%%%%%%%%%%%%%%%%%
\section{Presburger arithmetic}

T.b.c.

%%%%%%%%%%%%%%%%%%%%%%%%%%%%%%%%%%
%%%%%%%%%%%%%%%%%%%%%%%%%%%%%%%%%%
%%%%%%%%%%%%%%%%%%%%%%%%%%%%%%%%%%
\section{Real closed fields}

We work in the language of ordered rings $L=\{0,\,1,-,+,\,\cdot,\,<\}$.
%
The theory of \emph{ordered integral domains,} $T_{\rm oid}$ contains $T_{\rm oid}\cup T_{\rm oag}$ and the axioms

\begin{itemize}
  \item[1.]  $0<1$
  \item[op.] $0<z\wedge x<y\ \imp\ z\,x<z\,y$.
\end{itemize}

It follows immediately that ordered integral domains have characteristic $0$.

As in the previous section, we split $L_{\rm at}$ into equations $L_{\rm at,=}$ and inequalities $L_{\rm at,<}$.
Note that, by the linearity of the order, the types $\attp(b/A)$ and $\atpmtp(b/A)$ are equivalent.

The type $p(x)$ is \emph{equationally trivial\/} if $p_{=}(x)$ is trivial, as defined in Notation~\ref{notation1}. 
In this case $\proves p(x)\iff p_<(x)$.
If $\attp({\mr b}/A)$ equationally trivial, we say that ${\rm b}$ is \emph{transcendental\/} over $A$.
%
Otherwise we say that it is \emph{algebraic\/} over $A$.

\begin{proposition}\label{prop_di_tipi princ_comp}
  Let $A\subseteq M\models T_{\rm oid}$.
  %
  For ${\mr b}\in M$ let $p({\mr x})=\attp({\mr b}/A)$.
  %
  Then one of the following holds 
  \begin{itemize}
  \item[1.] ${\mr b}$ is transcendental over $A$;
  \item[2.] $M\models\phi({\mr b})$ for some $\phi({\mr x})\in L_{\rm at}(A)$ such that $\ \proves\phi({\mr x})\imp p({\mr x})$.
  \end{itemize}
\end{proposition}

\begin{proof}
  Negate \ssf1 and let $\phi({\mr x})\in\atL(A)$ be as in the proof of Proposition~\ref{prop_di_tipi_princ_comp}.
  Then $\ \proves\phi({\mr x})\imp p_{=}({\mr x})$.
  %
  Let ${\mr a},{\mr c}\in\<A\>_M$ be such that ${\mr a}<{\mr b}<{\mr c}$ is the unique element in $\<A\>_M$ satisfying ${\mr a}<{\mr x}<{\mr c}$ and $\phi({\mr x})$.
  We prove that for every polynomial $t({\mr x})$ with parameters in $\<A\>_M$ such that $\proves\ \phi({\mr x})\imp t({\mr x})\neq0$ one of the following holds
  \begin{itemize}
  \item[a.] $\proves\   {\mr a}<{\mr x}<{\mr c}\ \wedge\ \phi({\mr x})\imp t({\mr x})>0$;
  \item[b.] $\proves\   {\mr a}<{\mr x}<{\mr c}\ \wedge\ \phi({\mr x})\imp t({\mr x})<0$.
  \end{itemize}
  
  Suppose not for a contradiction. 
  %
  Then there are two models $\<A\>_M\subseteq N_i\models T_{\rm oid}$, some ${\mr b_i}\in N_i$ such that $N_i\models{\mr a}<{\mr b_i}<{\mr c}$
  \begin{itemize}
  \item[a$'$.] $N_1\models \phi({\mr b_1})$ \ and \ $N_1\models t({\mr b_1}) > 0$;
  \item[b$'$.] $N_2\models \phi({\mr b_2})$ \ and \ $N_2\models t({\mr b_2}) < 0$.
  \end{itemize}
\end{proof}