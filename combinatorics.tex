% !TEX root = creche.tex
\documentclass[creche.tex]{subfiles}
\begin{document}
\chapter{Some combinatorics}
\label{combinatorics}

\def\vc{{\footnotesize VC}}
\def\nip{{\footnotesize NIP}}


\def\medrel#1{\parbox[t]{6ex}{$\displaystyle\hfil #1$}}
\def\ceq#1#2#3{\parbox[t]{25ex}{$\displaystyle #1$}\medrel{#2}{$\displaystyle #3$}}

\definecolor{brown}{RGB}{150, 50, 10}
\definecolor{green}{RGB}{10,120, 20}
\def\mr{\color{brown}}
\def\gr{\color{green}}

In this chapter we expose some results in combinatorics that are relevant to model theory. Here we only deal with set-systems i.e., pairs that consist of a set $\Omega$ and a collection of subsets $\Delta\subseteq \P(\Omega)$. There is no other structure, and we not even mention first-order formulas. So the chapter may be read independently from the rest of the notes. Eventually the results presented here will be applied to describe externally definable sets in stable and \nip{} theories. Precisely, in Section~\ref{stability} we deal implicitly with stable formulas and in the other section with \nip{} formulas.


\section{A general setting}\label{theme}

Let $\Omega$ be any set and $\Delta\subseteq \P(\Omega)$ be two fixed non-empty sets. The pair $\<\Omega,\Delta\>$ is called a \emph{set-system} and sometimes may be denoted simply by $\Delta$. In some literature set-systems go under the names of \textit{hypergraphs}, \textit{range spaces}, \textit{concept classes}, and possibly more. Here we occasionally call \emph{edges} the elements of $\Delta$, typically we use symbol $\phi$ for edges.

For $B\subseteq \Omega$, we call the set $\phi\cap B$ the \emph{trace of $\phi$ on $B$}. The set traces on $B$ of the elements of $\Delta$ is called the \emph{trace on $B$ of the set-system $\Delta$\/} and is denoted by

\ceq{\hfill\emph{$\Delta\mathord\restriction B$}}{\deq}{\big\{\phi\cap B\ :\ \phi\in\Delta\big\}.} 

A \emph{sub-set-system of $\<\Omega,\Delta\>$\/} is a pair $\<B,D\>$ where $B\subseteq\Omega$ and $D\subseteq\Delta\mathord\restriction B$.  Sub-set-systems of the form $\<B,\Delta\mathord\restriction B\>$ are also known as a \emph{full sub-set-systems}. %Interesting subsystems are those of the form $\big\<B,\ \Delta\cap\P(B)\big\>$ for some $B\subseteq\Omega$.

Let $\le_I$ be an order relation on a set $I$. We say that \emph{$\Delta$ realizes $\le_I$\/} if there is a full sub-set-system $\<B,D\>$ such that $\le_I$ is isomorphic to $\subseteq_D$, where the latter is the usual subset relation on the elements of $D$. We repeat this definition with a more explicit notation as it is necessary below. We say that $\Delta$ realizes $\le_I$ if there is a sequence $\<\phi_i:i\in I\>$ of elements of $\Delta$ such that for some $B\subseteq\Omega$ 

\ceq{\hfill \phi_{i}\cap B\ \subseteq\ \phi_{j}\cap B}{\IFF}{i\le_Ij}

If $\Delta$ does not realize $\le_I$ we say that  \emph{$\Delta$ omits $\le_I$}. Mathematicians appreciate simplicity, so we are mainly interested in set-systems that omits complicated orders.


Typically $\Delta$ is not closed under Boolean operations and it may be interesting to know how the complexity of the set-systems increase when closing $\Delta$ under union, intersections, etc. Here is the notation that we use. We write \emph{$\wedge\Delta$\/} for the set whose elements are of the form $\phi_1\cap \phi_2$ for some $\phi_i\in\Delta$. We write \emph{$\{\wedge\}\Delta$\/} for the closure of $\Delta$ under intersection. The sets \emph{$\vee\Delta$\/} and  \emph{$\{\vee\}\Delta$\/} are defined in a similar way. Note that $\{\vee\}\{\wedge\}\Delta$ coincides with \emph{$\{\vee\wedge\}\Delta$}, the closure of $\Delta$ under union and intersection. Also, we write \emph{$\neg\Delta$\/} for set whose elements are \emph{$\neg\phi$}, that is $\Omega\sm\phi$, for $\phi\in\Delta$.  You should not confuse $\neg\Delta$ with \emph{$\{\neg\}\Delta$\/} which is the closure of $\Delta$ under complementation, $\Delta\cup\neg\Delta$.


We are also interested in the subsets of $\Omega$ that are close to $\Delta$ in a sense that we know make precise. We say that $D\subseteq\Omega$ is \emph{approximable by $\Delta$\/} if for every finite $B\subseteq\Omega$ there is some $\phi\in\Delta$ such that $B\cap\phi=B\cap D$. We say that $D\subseteq\Omega$ is \emph{approximable from below\/} in $\Delta$ if we can further require that $\phi\subseteq D$. If instead we require $D\subseteq\phi$ then we say that $D$ is \emph{approximable from above}. Clearly $D$ is approximable from above if and only if $\neg D=\Omega\sm D$ is approximable from below in the set-system $\neg\Delta$. 

It may be useful to rephrase the definition above in topological terms. Endow $\P(\Omega)$ with the topology that has as basic open sets those of the form $\big\{D\subseteq \Omega\ :\ F\subseteq D\subseteq C\big\}$ where $F$ is finite and $C$ co-finite. This is also a base of closed sets. Up to identifying $\P(\Omega)$ with $2^\Omega$, this is the product topology on $2^\Omega$ which is a compact, zero-dimensional topology. Let \emph{$\bar\Delta$\/} denote the closure $\Delta$ in this topology. Then $D$ is approximable in $\Delta$ if and only if $D\in\bar\Delta$.

For every $a\in\Omega$ define $\Delta_a=\big\{\phi\in\Delta: a\in\phi\big\}$. Let $\Omega^*=\Delta$ and $\Delta^*=\big\{\Delta_a: a\in\Omega\big\}$. We say that \emph{$\<\Omega^*,\Delta^*\>$ is the dual system\/} of $\<\Omega,\Delta\>$. The operation of duality is more natural is represented in the following way. We now write $\Delta$ for a pure set. To every relation $R\subseteq\Omega\times\Delta$ we associate a set-system. The edges are $\big\{x: xRa\big\}$ for $a\in\Delta$. We say that $R$ is extensional if $\big\{x: xRa\big\}\,=\,\big\{x: xRb\big\}$ only if $a=b$. If $R$ is not extensional we can factor by the obvious equivalence relation and obtain an extensional relation with the same associated set-system. So, suppose that $R$ is extensional. It is easy to check that the set-system associated to $R^{-1}\subseteq\Delta\times\Omega$ is dual to the one associated to $R$.

%The set-systems that are interesting for model theory are obtained from some partitioned formula $\phi({\mr x\,};{\gr z})\in L$ defining $\Omega=\U^{|{\mr x}|}$ and $\Delta=\{\phi({\mr a\,};{\gr \U}): {\mr a}\in\U^{|{\mr x}|}\}$. All set-systems may be assumed to have this form or to be obtained quotienting a set-system of this form by some equivalence relation. We write $\phi({\mr x\,};{\gr z})^*$ for the formula $\phi({\mr x\,};{\gr z})$ partitioned with the roles of ${\mr x}$ and ${\gr z}$ inverted. Assume for simplicity (to avoid quotienting) that there is a one-to-one correspondence between $\Delta$ and $\U^{|z|}$. Then $\Omega^*=\U^{|{\gr z}|}$ and $\Delta^*=\{\phi({\mr\U\,};{\gr b}): {\gr b}\in\U^{|{\gr z}|}\}$.


\section{The combinatorial bare bones of stability}\label{stability}

We write \emph{$\le_\omega$\/} for the usual order on $\omega$ and \emph{$\le^{\scriptscriptstyle -1}_\omega$\/} for the reverse order. For $k<\omega$ we write \emph{$\le_k$\/} for the natural order on the set $k=\big\{0,\dots,k-1\big\}$. In this section we prove that if $\Delta$ omits $\le_k$ for some $k<\omega$ then $\bar\Delta\subseteq\{\vee\wedge\}\Delta$.

\begin{lemma}\label{lem_ramsey} Suppose $\Delta$ omits $\le_\omega$ then 
\begin{itemize}
\item[1.] $\wedge\Delta$ omits $\le_\omega$.
\item[2.] $\vee\Delta$ omits $\le_\omega$.
\end{itemize}
And the same holds with $\le^{\scriptscriptstyle -1}_\omega$ for $\le_\omega$.
\end{lemma}
\begin{proof}
It suffices to prove claim \ssf{1} both for $\le_\omega$ and $\le^{\scriptscriptstyle -1}_\omega$.  In fact, $\Delta$ omits $\le_\omega$ if and only if $\neg\Delta$ omits $\le^{\scriptscriptstyle -1}_\omega$.

Suppose $\wedge\Delta$ realizes $\le_\omega$. That is, for some $B$ and some $\<\phi_i, \psi_i\in\Delta\ :\ i<\omega\>$
\begin{itemize}
\item[a.] $(\phi_i\cap\psi_i)\cap B\ \subseteq\ (\phi_j\cap\psi_j)\cap B\ \ \IFF\ \ i\le j$
\end{itemize}

We can assume that $\phi_0\cap\psi_0$ is non empty, otherwise we shift the sequence by $1$. Choose  $\big\{a_i:i<\omega\big\}\subseteq B$ recursively: $a_0\in \phi_0\cap\psi_0$ and $a_{i+1}\in  (\phi_{i+1}\cap\psi_{i+1})\smallsetminus(\phi_i\cap\psi_i)$. We obtain 

\begin{itemize}
\item[b.] $a_i\ \in\ \phi_j\cap\psi_j\ \ \IFF\ \ i\le j$
\end{itemize}

Hence for every pair $j<i<\omega$ at least one the following obtains

\begin{itemize}
\item[c.] $a_i\notin\phi_j$\qquad or\qquad $a_i\notin\psi_j$
\end{itemize}
By Ramsey's Theorem there is and infinite $I\subseteq\omega$ such that $i,j\in I$ satisfies the same (say, the first) inequality in \ssf{c} whenever $j<i$. Then 

\begin{itemize}
\item[d.] $a_i\in\phi_j\ \ \IFF\ \ i\le j$\qquad for every $i,j\in I$
\end{itemize}
So, if we define $C=\big\{a_i:i\in I\big\}$ we may conclude
\begin{itemize}
\item[]$\phi_i\cap C\ \subseteq\ \phi_j\cap C\ \ \IFF\ \ i\le j$\qquad for every $i,j\in I$.
\end{itemize}
As $I$ is infinite, this suffices to show that $\Delta$ realizes $\le_\omega$.

As for $\le^{\scriptscriptstyle -1}_\omega$ the proof is identical but for the choice of $\big\{a_i:i<\omega\big\}\subseteq B$. In this case we pick some elements  $a_i\in  (\phi_i\cap\psi_i)\smallsetminus(\phi_{i+1}\cap\psi_{i+1})$.
\end{proof}

A finitary version of Lemma~\ref{lem_ramsey} can be proved along the same idea.


\begin{proposition}\label{lem_ramsey2} For every $n<\omega$ there is a $k<\omega$ such that for every $\Delta$ that omits $\le_k$
\begin{itemize}
\item[1.] $\wedge\Delta$ omits $\le_n$;
\item[2.] $\vee\Delta$ omits $\le_n$.
\end{itemize}
\end{proposition}
\begin{proof}
In this finite version of the lemma \ssf{1} and \ssf{2} are equivalent. If we repeat the proof of Lemma~\ref{lem_ramsey} with $k$ for $\omega$, we need that every function $f:[k]^2\to2$ is constant in $[I]^2$ for some $I\subseteq k$ of cardinality $n$. Then the finite Ramsey theorem yields the required $k$.
\end{proof}




% 
% 
% \begin{lemma}\label{lem_fin_apprx_below} If $\Delta$ omits $\le^{\scriptscriptstyle -1}_\omega$ then every approximable set is approximable in $\{\wedge\}\Delta$ from above.
% \end{lemma}
% \begin{proof}
% \def\ceq#1#2#3#4{\parbox[t]{20ex}{$\displaystyle #1$}\parbox{5ex}{$\displaystyle\hfil #2$}{$\displaystyle #3$}\parbox{5ex}{$\displaystyle\hfil #4$}}
% 
% Let $D\subseteq\Omega$ be approximable. It suffices to show that for every finite $C\subseteq\Omega$ there is a $\phi\in\{\wedge\}\Delta$ such that $C\subseteq\phi\subseteq D$. Let $B_0=C$ and define recursively $C\subseteq\phi_i\in\Delta$ and a chain of finite sets $B_i\subseteq\Omega$. As $D$ is approximable, there is a $\phi_i\in\Delta$ such that 
% 
% \ceq{\hfill \phi_i\cap B_i}{=}{D\cap B_i.}{}
% 
% If $\phi_0\cap\dots\cap\phi_i\subseteq D$ terminate the construction: we have as required 
% 
% \ceq{\hfill C}{\subseteq}{\phi_0\cap\dots\cap\phi_i}{\subseteq}$D$
% 
% Otherwise pick some arbitrary $a_i\in \phi_0\cap\dots\cap\phi_i\sm D$ and let $B_{i+1}=B_i\cup\{a_i\}$. Let
% 
% \ceq{\hfill B}{=}{\bigcup_{i\in\omega}B_i}{}
% 
% and notice that
% 
% \ceq{\hfill \phi_i\cap B}{\subseteq}{\phi_j\cap B}{}$\IFF\qquad j\le i$.
% 
% As $\Delta$ omits $\le^{\scriptscriptstyle -1}_\omega$, the construction terminates at some finite stage.
% \end{proof}
% 
% As $D$ is approximable from above in $\Delta$ if and only if $\neg D$ is approximable from below in $\neg\Delta$, Lemma~\ref{lem_fin_apprx_below} is essentially equivalent to the following.
% 
% \begin{lemma} If $\Delta$ omits $\le_\omega$ then every approximable set is approximable in $\{\vee\}\Delta$ from above.\QED
% \end{lemma}



\begin{theorem}\label{thm_stable_approx} If $\Delta$ omits $\le_k$ then 
\begin{itemize}
 \item[1.] every set that is approximable in $\Delta$ is approximable from below in $\wedge\!\!^k\Delta$;
 \item[2.] every set that is approximable in $\Delta$ is approximable from above in $\vee\!^k\Delta$;
 \item[3.] every set approximable in $\Delta$ belongs to $\vee\!^h\wedge\!\!^k\Delta$ for some $h<\omega$.
\end{itemize}
\end{theorem}
\begin{proof}\quad
\def\ceq#1#2#3#4{\parbox[t]{20ex}{$\displaystyle #1$}\parbox{5ex}{$\displaystyle\hfil #2$}{$\displaystyle #3$}\parbox{5ex}{$\displaystyle\hfil #4$}}
\begin{itemize}
\item[1.]Let $D\subseteq\Omega$ be approximable in $\Delta$. We need to show that for every finite $C\subseteq D$ there is a $\phi\in\wedge^k\Delta$ such that $C\subseteq\phi\subseteq D$. Let $B_0=C$ and define recursively $C\subseteq\phi_i\in\Delta$ and a chain of finite sets $B_i\subseteq\Omega$. As $D$ is approximable, there is a $\phi_i\in\Delta$ such that 

\ceq{\hfill \phi_i\cap B_i}{=}{D\cap B_i.}{}

If $\phi_0\cap\dots\cap\phi_i\subseteq D$ terminate the construction: we have as required 

\ceq{\hfill C}{\subseteq}{\phi_0\cap\dots\cap\phi_i}{\subseteq}$D$

Otherwise pick some arbitrary $a_i\in \phi_0\cap\dots\cap\phi_i\sm D$ and let $B_{i+1}=B_i\cup\{a_i\}$. Let

\ceq{\hfill B}{=}{\bigcup_{i\in\omega}B_i}{}

and notice that

\ceq{\hfill \phi_i\cap B}{\subseteq}{\phi_j\cap B}{}$\IFF\qquad j\le i$.

As $\Delta$ omits $\le_k$, the construction terminates at some stage $<k$.
\item[2.] Easy, similar to \ssf{2}.
\item[3.] Let $D\subseteq\Omega$ be approximable in $\Delta$. Then by \ssf{1} it is approximable from below in $\wedge\!\!^k\!\Delta$. So, by Lemma~\ref{lem_ramsey2}, it omits $\le_h$ for some $h<\omega$.  Therefore, by claim \ssf{2}, it belongs to $\vee\!^h\wedge\!\!^k\Delta$.\qedhere
\end{itemize}
\vskip-1\baselineskip
\end{proof}

% 
% 
% \begin{theorem} For every $k$ there is an $n$ such that for every $\Delta$ omits $\le_k$ 
% \begin{itemize}
%  \item[1.] every approximable set is approximable in $\wedge\!\!^k\Delta$ from below;
%  \item[2.] every approximable set belongs to $\{\vee\wedge\}\Delta$;
% \end{itemize}
% Suppose $D$ is approximable. Then by \ssf{1} it is approximable from below in in $\{\wedge^\!\!k\}\Delta$. By Lemma~\ref{lem_ramsey} $\wedge\!\!^k\!\Delta$ omits $\le$ hence and \ssf{2}, every approximable sets is approximable both from above and from below in $\{\vee\wedge\}\Delta$. Hence, by claim \ssf{3} it belongs to $\{\vee\wedge\}\Delta$.
% \end{theorem}



%%%%%%%%%%%%%%%%%%%%%%%
%%%%%%%%%%%%%%%%%%%%%%%
%%%%%%%%%%%%%%%%%%%%%%%
%%%%%%%%%%%%%%%%%%%%%%%
%%%%%%%%%%%%%%%%%%%%%%%
\section{Vapnik-Chervonenkis dimension}\label{vc_dimension}


In the following sections we deal with set-systems of finite Vapnik-Cher\-vo\-nen\-kis dimension.


\begin{definition}\label{def_VCdim}
The \emph{Vapnik-Cher\-vo\-nen\-kis dimension of $\Delta$}, abbreviated by \emph{\vc-dimension}, is the largest cardinality of a finite set $B\subseteq\Omega$ such that $\Delta\mathord\restriction B=\P B$. If such a maximum does not exist we say that $\Delta$ has infinite \vc-dimension. In lengthier words, the \vc-dimension is the largest $k$ such that there are $\<a_i\in\Omega :i<k\>$ and $\<\phi_J:J\subseteq k\>$ such that


\ceq{\hfill a_i\in\phi_J}{\IFF}{i\in J}\qquad  for every $i<k$ and every $J\subseteq k$.
\end{definition}


The dimension is $0$ when $\Delta$ contains only one element. Set-systems with dimension is $1$ may contain a collection of disjoint sets or chain of sets. The set of intervals of reals has dimension $2$, the set of squares in $\RR^2$ has dimension $3$.

The following lemma has been proved independently by Shelah, Sauer, and Vapnik-Cher\-vo\-nen\-kis around 1970. Shelah was working in model theory while Sauer and Vapnik-Chervonenkis were in statistics. 

\begin{proposition}[(Sauer's Lemma)]\label{lem_Sauer}
If $\Delta$ has finite \vc-dimension $k$ then \smash{$\displaystyle \left|\Delta\mathord\restriction B \right|\ \le\ \bigsum^{k}_{i=0} \binom{n}{i}$} for every finite set $B\subseteq\Omega$ of cardinality $n> k$.
\end{proposition}

\begin{proof}
We proceed by induction on $k$. If $\Delta$ has null \vc-dimension, both sides of the inequality are $1$. Now assume the lemma is true for $k-1$, we claim that for every $n> k$

\ceq{\hfill\left|\Delta\mathord\restriction B \right|}{\le}{\bigsum^{k}_{i=0} \binom{n}{i}}.

We proceed by induction on $n=|B|$. If $n=k$ the claim says $|\Delta\mathord\restriction B|\le 2^n$ which is clearly true. Fix some $a\in B$ and define

\ceq{\hfill \Gamma}{=}{\Big\{C\subseteq B\sm a\ :\ C,C\cup\{a\}\in\Delta\mathord\restriction B \Big\}.}

where we use $B\sm a$ as a shorthand for $B\sm\{a\}$. We may assume that $\Gamma$ is non empty otherwise the claim is obvious. Observe that

\ceq{\hfill |\Delta\mathord\restriction B|}{=}{|\Delta\mathord\restriction (B\sm a)|\ +\ |\Gamma|.}

Given that $\Delta$ has \vc-dimension $k$, it is clear that $\Gamma$ has \vc-dimension at most $k-1$. We apply the induction hypothesis on $n$ to $B\sm a$ and the induction hypothesis on $k$ to the set-system $\Gamma$. Note that $\Gamma=\Gamma\restriction(B\sm a)$. We obtain

\ceq{\hfill |\Delta\mathord\restriction B|}{\le}{\bigsum^{k}_{i=0} \binom{n-1}{i}\ \  +\ \  \bigsum^{k-1}_{i=0}  \binom{n-1}{i}}

\ceq{}{=}{\binom{n-1}{0}\ \ +\ \ \bigsum^{k}_{i=1}\ \ \binom{n-1}{i}\ +\  \binom{n-1}{i-1}}

\ceq{}{=}{\binom{n-1}{0}\ \ +\ \ \bigsum^{k}_{i=1}\ \ \binom{n}{i}}

\ceq{}{\le}{\bigsum^{k}_{i=0} \binom{n}{i}}

which proves the claim, and with it, the lemma.
\end{proof}
The following is called the \emph{shatter function\/}

\ceq{\hfill S(n,\Delta)}{\deq}{\max\Big\{|\Delta\mathord\restriction B|\ \ :\ \ B\subseteq\Omega,\ |B|=n\Big\}}

Next corollary states an important dichotomy. It says that the shatter function grows exponentially unless the \vc-dimension is finite. In this case the growth is only polynomial.


\begin{corollary}\label{coroll_Sauer}
For every set $\Delta$ one of the following obtains
\begin{itemize}
 \item[1.] the \vc-dimension is infinite and $S(n,\Delta)=2^n$ for every positive integer $n$;
 \item[2.] the \vc-dimension is $k$ and $S(n,\Delta)\ \le\ n^k$ for every integer $n> k$.
\end{itemize}
\end{corollary}
\begin{proof}
If \vc-dimension is infinite claim \ssf{1} is obvious. So suppose $\Delta$ has \vc-dimension is $k$ and observe that when $k>1$, Sauer's lemma

\ceq{\hfill S(n,\Delta)}{\le}{\bigsum^k_{i=0}\binom{n}{i}}\medrel{\le}$\displaystyle\bigsum^k_{i=0}\frac{n^i}{i!}$\medrel{\le}$n^k$.

When $k=0$ the claim is obvious. When $k=1$ we obtain the bound $n+1$. We can get rid of the extra $1$ with a minor (otherwise uninteresting) improvement of Sauer's lemma.
\end{proof}

% \begin{corollary}
% For every set $\Delta$ one of the following obtains
% \begin{itemize}
%  \item[1.] the \vc-dimension is infinite and $S(n,\Delta)=2^n$ for every integer $n\ge0$;
%  \item[2.] the \vc-dimension is $k<\omega$ and $S(n,\Delta)\ \le\ C_kn^k$ for every integer $n\ge k$.
%  \item[2.] the \vc-dimension is $k=0,1$ and $S(n,\Delta)\ =\ 1+n^k$ for every integer $n\ge 1$.
% \end{itemize}
% The constant $C_k$ is $e/k$, where $e$ is the base of the natural logarithm.
% \end{corollary}
% \begin{proof}
% If \vc-dimension is infinite the first claim is obvious. So assume $\Delta$ has finite \vc-dimension is $k$. Then
% 
% \ceq{\hfill \bigsum^{k}_{i=0}\binom{n}{i}}{\le}{\bigsum^{k}_{i=0}\binom{n}{i}\Bigg(\frac{n}{k}\Bigg)^{k-i}}
% 
% \ceq{}{\le}{\Bigg(\frac{n}{k}\Bigg)^k\bigsum^{n}_{i=0}\binom{n}{i}\Bigg(\frac{k}{n}\Bigg)^i}
% 
% \ceq{}{\le}{\Bigg(\frac{n}{k}\Bigg)^k\Bigg(1+\frac{k}{n}\Bigg)^{n}}
% 
% \ceq{}{\le}{\Bigg(\frac{n}{k}\Bigg)^ke^k}
% 
% For the last inequality is clear as $\ln(1+x)\le x$.
% \end{proof}

The dimension of the dual set-system $\<\Omega\!^*,\Delta\!^*\>$ is called \emph{\vc$^*\!$-dimension of $\Delta$}. Unraveling the definition we obtain that the \vc$^*\!$-dimension is the largest $k$ for which there is no tuple $\phi_0,\dots,\phi_{k-1}\in\Delta$ and $\<\phi_I: I\subseteq k\>$ such that

\ceq{\hfill a_I\in\phi_j}{\IFF}{j\in I}\qquad for every $I\subseteq k$ and every $j\in k$.


\begin{proposition}\label{prop_vc*}
If  $\Delta$ has \vc-dimension $<k$ then its \vc$^*\!$-dimension is $<2^k$.
\end{proposition}

\begin{proof}
Assume that the \vc$^*\!$-dimension of $\Delta$ is $\ge 2^k$. Then, replacing  $k$ with $\P(k)$ in the definition above we obtain two sequences $\<\phi_I: I\subseteq k\>$ and $\<a_{\mr \J}: {\mr \J}\subseteq \P(k)\>$ such that

\ceq{\natural\hfill a_{\mr\J}\in\phi_I}{\IFF}{I\in{\mr \J}}\qquad for every $I\subseteq k$ and every ${\mr \J}\subseteq \P k$.

For $j\in k$ define ${\mr \J_j}=\{I\subseteq k: {\mr j}\in I\}$. As a particular case of $\natural$ we obtain

\ceq{\hfill a_{\mr \J_j}\in\phi_I}{\IFF}{I\in{\mr \J_j}}\qquad for every ${\mr j}\in k$.

which is equivalent to

\ceq{\hfill a_{\mr \J_j}\in\phi_I}{\IFF}{{\mr j}\in I}\qquad for every $I\subseteq k$ and every ${\mr j}\in k$.

Now, defining $a_{\mr j}=a_{\mr \J_j}$ we obtain $\sharp$ of Definition~\ref{def_VCdim}, hence $\Delta$ has \vc-di\-mens\-ion $\ge k$.
\end{proof}

\begin{exercise}
Prove that if $\Delta$ has finite \vc-dimension then also $\wedge\Delta$ has finite \vc-dimension.  Hint: apply Corollary~\ref{coroll_Sauer}.
\end{exercise}

 
%%%%%%%%%%%%%%%%%%%%%%%
%%%%%%%%%%%%%%%%%%%%%%%
%%%%%%%%%%%%%%%%%%%%%%%
%%%%%%%%%%%%%%%%%%%%%%%
%%%%%%%%%%%%%%%%%%%%%%%
\section{Epsilon-approximations}\label{epsilon_approximations}

\def\Av{\mathbin{\textrm{Av}}}
\def\disc{\mathbin{\textrm{disc}}}
\def\ceq#1#2#3{\parbox[t]{35ex}{$\displaystyle #1$}\parbox{5ex}{$\displaystyle\hfil #2$}{$\displaystyle #3$}}

Let $\mu$ be some probability measure on $\Omega$ that makes all sets in $\Delta$ measurable. We say that a finite set $B$ is an \emph{$\epsilon$-approximation of $\mu$\/} or \emph{$\epsilon\jj$sample\/} if for every $\phi\in\Delta$

\ceq{\star\hfill\left|\mu(\phi) - \frac{\big|B\cap\phi\big|}{|B|}\right|}{\le}{\epsilon}

The uniform probability measures on a finite set $\Omega$ admits $\epsilon\jj$approximations for arbitrarily small $\epsilon$. In fact, $\Omega$ itself is a $0\jj$approximation. This is not true if the measure is non-uniform (see Exercise~\ref{ex_counterexample}). This is unfortunate for the intended applications. So we introduce a weaker notion of approximation.

A \emph{finite multi-set\/} is a map $B:\Omega\to\omega$ that has finitely many non-zero values. The \emph{cardinality of $B$\/} is defined as

\ceq{\hfill\emph{$|B|$}}{=}{\sum_{a\in\Omega} B(a)}

When $B$ is a multi-set and $\phi\in\Delta$ we read \emph{$B\cap\phi$\/} as the product of $B$ with the characteristic function of $\phi$. We say that $B:\Omega\to\omega$ is a \emph{multi-set $\epsilon\jj$approximation\/} of $\mu$ if inequality $\star$ holds with the reading adapted to multi-sets. It is clear that every measure on a finite set admits multi-set $\epsilon\jj$approximations for arbitrarily small $\epsilon$ or, when $\mu$ is rational-valued, even a $0\jj$approximation. 

% It is useful to introduce an even weaker notion. We say that a finite set $B$ is an \emph{$\epsilon$-net for $\mu$\/} if for every $\phi\in\Delta$
% 
% \ceq{\hfill\mu(\phi)>\epsilon}{\IMP}{B\cap\phi\neq\0}
% 
% Clearly, an $\epsilon\jj$approximation is in particular an $\epsilon\jj$net. More generally, if $B:\Omega\to\omega$ is a multi-set $\epsilon\jj$approximation then $\big\{a\in\Omega\ :\ B(a)\neq0\big\}$ is an $\epsilon\jj$net.

For the moment we work with (plain) approximations. Given $\epsilon$ we are interested in the least $n$ such that some $\epsilon\jj$approximations of cardinality $n$ exist. The idea is to start with a large approximation and reduce size at the cost of slightly enlarging $\epsilon$. We now introduce a powerful technique for achieving this.

Let $B\subseteq\Omega$ have finite cardinality $n$. We call a total map $c:B\to\{-1,+1\}$ a \emph{coloring of $B$}. For any $\phi\in\Delta\cup\{\Omega\}$ we define

\ceq{\hfill c_\phi}{\deq}{\sum_{a\in B\cap \phi} c(a).}

The \emph{(relative) discrepancy of $\Delta\mathord\restriction B$\/} is 

\ceq{\hfill \delta_B}{\deq}{\min_{c:B\to\{\pm 1\}}\ \ \max_{\phi\in\Delta\cup\{\Omega\}}\ \frac{|c_\phi|}{n}}

%\ceq{\hfill \disc(n,\Delta)}{=}{\max_{|B|=n}\ \disc(\Delta\mathord\restriction B)}
The next lemma demonstrate that discrepancy can be applied to reduce the size of an approximation. 

\begin{lemma}\label{lem_aprossimazionediapprossimazione}
Let $B\subseteq\Omega$ be an $\epsilon$-approximation of $\mu$ of cardinality $n$ with discrepancy $\delta_B$. Then there is an $(\epsilon+2\delta_B)\jj$approximation $B^+\subseteq B$ of cardinality $\le n/2$.
\end{lemma}

\begin{proof}
Fix $c:B\to\{\pm1\}$ such that $|c_\phi|\le n\delta_B$ per ogni $\phi\in\Delta\cup\{\Omega\}$ and define $B^\pm=c^{-1}[\pm1]$ and  $n^\pm=|B^\pm|$. Then 

\ceq{\hfill c_\phi}{=}{|B^+\cap\phi|-|B^-\cap\phi|.}

In particular $c_B=n^+-n^-$. We may assume that $n^+\le n/2$, otherwise swap $+1$ and $-1$, then $c_B<0$ and we have 

\ceq{\ssf{1.}\hfill\frac{|B\cap\phi|}{n}}{=}{\frac{2|B^+\cap\phi|}{n} - \frac{c_\phi}{n}}

\ceq{}{\le}{\frac{|B^+\cap\phi|}{n^+} + \delta_B}

We also have 

\ceq{\ssf{2.}\hfill\frac{|B\cap\phi|}{n}}{=}{\frac{2|B^+\cap\phi|-c_\phi}{2n^+}\cdot\left(1-\displaystyle\frac{c_B}{2n^+}\right)^{-1}}

\ceq{}{\ge}{\left(\frac{|B^+\cap\phi|}{n^+}-\delta_B\right)\big(1+\delta_B\big)^{-1}}

\ceq{}{\ge}{\left(\frac{|B^+\cap\phi|}{n^+}-\delta_B\right)\big(1-\delta_B\big)}

\ceq{}{\ge}{\frac{|B^+\cap\phi|}{n^+} - 2\delta_B}

Combining \ssf{1} and \ssf{2} we obtain 

\ceq{\hfill\left|\frac{|B\cap\phi|}{n}\ -\ \frac{|B^+\cap\phi|}{n^+}\right|}{\le}{2\delta_B}

hence

\ceq{\hfill\left|\mu(\phi)\ -\ \frac{|B^+\cap\phi|}{n^+}\right|}{\le}{\epsilon + 2\delta_B}

as claimed by the lemma.
\end{proof}

The lemma above tells that approximations with small discrepancy are useful, but as yet we have no clue as to finding one. It is immediate that if $\Delta\mathord\restriction B=\P(B)$ then $\delta_B$ is large (i.e.\@ close to $1/2$). We are going to prove that when $\Delta\mathord\restriction B$ is small in comparison with $B$, then the discrepancy of $B$ is not too large. We use a probabilistic argument to prove this bound, hence a brief digression into probability theory. The following inequality is a classical tool in this context.

\begin{lemma}[(Chernoff bound)]\label{Chernoff}
For $i=1,\dots,n$ let $X_i$ be independent identically distributed random variables such that $\Pr(X_i=\pm1)=1/2$. Then for every $\delta\in\RR^+$

\ceq{\hfill \Pr\bigg(\sum^n_{i=1}X_i>n\delta\bigg)}{\le}{\exp(-\frac{n}{2}\delta^2)}
\end{lemma}
\begin{proof}
Let $\displaystyle Y=\sum^n_{i=1}X_i$ and let $t\in\RR^+$ be arbitrary. Then

\ceq{\sharp\hfill \Pr(Y>n\delta)}{=}{ \Pr\big(e^{tY}>e^{tn\delta}\big)}

\ceq{~}{\le}{\frac{{\rm E}\big(e^{tY}\big)}{e^{tn\delta}}}

In fact, the equality follows because the exponential is an increasing function and the inequality is Markov's inequality, which says that $a\Pr(X>a)\le{\rm E}(X)$ for every $a$ and is immediate to verify. Now observe that

\ceq{\hfill {\rm E}\big(e^{tX_i}\big)}{=}{\frac12e^t+\frac12e^{-t}}

\ceq{~}{=}{\sum^\infty_{i=0}\frac{t^i}{i!} + \sum^\infty_{i=0}\frac{(-t)^{i}}{i!}}

\ceq{~}{=}{\sum^\infty_{i=0}\frac{t^{2i}}{(2i)!}}

\ceq{~}{\le}{\sum^\infty_{i=0}\frac{(t^2/2)^i}{i!}}

\ceq{~}{=}{e^{t^2/2}}

From this, by independence we have 

\ceq{\hfill {\rm E}\big(e^{tY}\big)}{=}{\prod^n_{i=1}e^{tX_i}}$\medrel{=}e^{nt^2/2}$


Applying this to $\sharp$ inequality we obtain $\Pr(Y>n\delta)\le e^{nt^2/2-tn\delta}$ and finally the Chernoff inequality is obtained substituting $\delta$ for $t$. 
\end{proof}

\begin{lemma}\label{lem_discrepanzarandom} Let $B\subseteq\Omega$ and suppose that $n\le|B|$ and $|\Delta \mathord\restriction B|\le m$ is finite. Then

\ceq{\hfill \delta_B}{\le}{\frac{1}{n}\sqrt{2n\ln(3m)\strut}}
\end{lemma}

% Then it admits $\epsilon\jj$approximations of size at most $\natural$ for every $0<\epsilon<1$.



% \begin{proof}
% Let $n'$ be the largest integer such that $2n'\le n$ Fix a tuple $\<a_i:i<2n'\>$ of distinct elements of $B$. Let $C$ be the set of coloring such that $c(a_i)+c(a_{i+n'})=0$ for every $i\le n'$. Imagine $C$ as a probability space: each $c$ is obtained by flipping independently $n'$ times a fair coin. To each $\phi$ associate a sequence of identically distributed independent random variables $X^\phi_i$, for $i\le n'$, defining $X^\phi_i(c)=c(a_i)$. Then apply Lemma~\ref{Chernoff} to obtain
% 
% \ceq{\hfill \Pr(c_\phi\ge n'\delta)}{\le}{e^{-n'\delta^2/2}}
% 
% By symmetry
% 
% \ceq{\hfill \Pr(|c_\phi|\ge n'\delta)}{\le}{2e^{-n'\delta^2/2}}
% 
% Substitute $\sqrt{(2/n')\ln(em)}$ for $\delta$, where $e$ is the base of the natural logarithm
% 
% \ceq{\hfill \Pr\big(|c_\phi|\ge\sqrt{2n'\ln(em)}\big)}{\le}{\frac{2}{em}}
% 
% As $|\Delta\mathord\restriction B|\le m$, for at least one $\phi$ it must be $\Pr\big(|c_\phi|\big)\ge\sqrt{2n'\ln(em)}\le2/e$. In particular, for such a $\phi$ there exist some $c$ such that $|c_\phi|<\sqrt{2n'\ln(em)}$. Hence 
% 
% \ceq{\hfill\delta_B}{\le}{\frac1n|c_\phi|}
% 
% \ceq{~}{\le}{\frac1n\sqrt{2n'\ln(em)}}
% 
% \ceq{~}{\le}{\frac1n\sqrt{n\ln(em)}}
% 
% which proves the require bound.
% \end{proof}



\begin{proof}
Fix a tuple $\<a_i:i<n\>$ of distinct elements of $B$. Let $C$ be the set of colorings of $B$. Imagine $C$ as a probability space where each $c:B\to\{\pm1\}$ is obtained by flipping independently $n$ times a fair coin. For each $i<n$ define the random variables $X_i(c)=c(a_i)$. Then apply Lemma~\ref{Chernoff} to obtain 

\ceq{\hfill \Pr\big(c_\phi\ge n_\phi\delta\big)}{\le}{e^{- n_\phi\delta^2/2}}\hfill where $n_\phi=|\phi|$.

By symmetry

\ceq{\hfill \Pr\big(|c_\phi|\ge n_\phi\delta\big)}{\le}{2e^{-n_\phi\delta^2/2}}

Substitute $\sqrt{(2/n_\phi)\ln(3m)}$ for $\delta$

\ceq{\hfill \Pr\Big(|c_\phi|\ge\sqrt{2n_\phi\ln(3m)}\,\Big)}{\le}{\frac{2}{3m}}


Hence

\ceq{\hfill \Pr\Big(|c_\phi|\ge\sqrt{2n\ln(3m)\strut}\,\Big)}{\le}{\frac{2}{3m}}

As $|\Delta\mathord\restriction B|\le m$, 

\ceq{\hfill\Pr\Big(\E\phi\in\Delta\quad |c_\phi|\ge\sqrt{2n\ln(3m)\strut}\,\Big)}{\le}{\frac{2}{3}}\medrel{<}$1$

Therefore

\ceq{\hfill\Pr\Big(\A\phi\in\Delta\quad |c_\phi|<\sqrt{2n\ln(3m)\strut}\,\Big)}{>}{0}

In particular, for such a $\phi$ there exist some $c$ such that $|c_\phi|<\sqrt{2n\ln(3m)\strut}$ for all $\phi\in\Delta$. So, finally 

\ceq{\hfill\delta_B}{\le}{\displaystyle\frac1n\sqrt{2n\ln(3m)\strut}}

which proves the require bound.
\end{proof}







\begin{theorem}\label{thm_epsilon_approx}
Fix some positive $\epsilon<1$. Let $\Delta$ have \vc-dimension $1<k<\omega$. Let $\mu$ be some probability measure that admits an $\epsilon/2$-approximation (of arbitrary size). Then there is an $\epsilon$-approximation of cardinality 

\ceq{\natural}{\le}{2^8\frac{k}{\epsilon^2}\ln\frac{1}{\epsilon^2}.}

\end{theorem}

Note that the bound claimed by the theorem only depends on the dimension of $\Delta$ and it is independent of $\mu$.

\begin{proof}
For $i=0,\dots,h$ we construct a decreasing chain $B_i$ of $\epsilon_i$-approximations. By assumption we can require $B_0$ is an $\epsilon_0=\epsilon/2$-approximation. We denote by $n_i$ and $\delta_i$ the cardinality, respectively the discrepancy, of $B_i$. Without loss of generality we can assume that $n_0$ is a power of two, so by lemma~\ref{lem_aprossimazionediapprossimazione}, we can require that $\epsilon_{i+1}\le\epsilon_i+2\delta_i$ and $n_i=2^{-i}n_0$. Then

\ceq{\hfill \epsilon_h}{=}{\frac\epsilon2\ +\ 2\sum^h_{i=1}\delta_i}

The construction stops at the least $h$ such that $\epsilon<\epsilon_h+2\delta_h$. So we have ``only'' have to prove that (independently of $n_0$) this $h$ satisfies

\ceq{\natural\natural\hfill 2^{-h}n_0}{\le}{2^8\frac{k}{\epsilon^2}\ln\frac{1}{\epsilon^2}.}

We may rewrite the condition $\epsilon<\epsilon_h+2\delta_h$ as

\ceq{\hfill\epsilon}{<}{4\sum^{h+1}_{i=1}\delta_i}

To get rid of some annoying square roots that will appear soon, we substitute the latter inequality with

\ceq{\hfill\epsilon^2}{<}{2^5\sum^{h+1}_{i=1}\delta^2_i}

% This suffices because (for any numbers $\delta_i$)
% 
% \ceq{\hfill\Big(\sum^{h+1}_{i=1}\delta_i\Big)^2}{<}{2\sum^{h+1}_{i=1}\delta^2_i}

We do not know $\delta_i$ but Lemma~\ref{lem_discrepanzarandom}, together with Sauer's Lemma~\ref{lem_Sauer}, gives an upper bound

\ceq{\hfill\delta^2_i}{\le}{\frac{2}{n_i}\ln n_i^{k}}

\ceq{}{=}{\frac{2^{i+1}}{n_0}\Big(k\ln 2^{-i}+ k\ln n_0\Big)}

\ceq{}{\le}{\frac{2^{i+1}}{n_0}k\ln n_0}

Then we obtain

\ceq{\hfill\sum^{h+1}_{i=1}\delta^2_i }{\le}{2^{h+2}k\ \frac{\ln n_0}{n_0}}

The theorem is complete as any $h$ that satisfies the inequality

\ceq{\hfill \epsilon^2}{<}{2^{h+7}k\ \frac{\ln n_0}{n_0}}

verifies $\natural\natural$ (and in particular the $h$ at which the constructions stops). The verification of is immediate, it suffices to substitute for $\epsilon^2$ the r.h.s.\@ of the inequality above and recall that we can assume that $n_0$ is sufficiently large.
\end{proof}

%We need to deal with arbitrary measures on finite sets $\Omega$. Then the uniform measure on $\Omega$ admits a $0\jj$approximation, namely $\Omega$ itself. Then it admits $\epsilon\jj$approximations of size at most $\natural$ for every $0<\epsilon<1$.

% The following proposition is immediate.
% 
% \begin{proposition}\label{prop_0_multiapprox}
% Let $\Delta$ be a finite set-system then every probability measure $\mu$ admits a multi-set  $\epsilon\jj$approximation and every positive $\epsilon$. (Even a $0\jj$approximation, if $\mu$ is rational valued.)\QED
% \end{proposition}

\begin{corollary}\label{coroll_epsilon_multiapprox}Let $\Delta$ be a finite set-system of \vc-dimension $1<k<\omega$. Then for every positive $\epsilon<1$, every probability measure $\mu$ admits a multi-set $\epsilon$-approximation of cardinality bounded by $\natural$ of Theorem~\ref{thm_epsilon_approx}.
\end{corollary}
\begin{proof}
Without loss of generality we can assume that $\mu$ is rational valued. Then there is a uniform probability measure $\mu'$ on some finite set $\Omega'$ and a surjection $f:\Omega'\to\Omega$ such that $\mu(f^{-1}\phi)=\mu(\phi)$. By Theorem~\ref{thm_epsilon_approx} $\mu'$ admits an $\epsilon\jj$approximation $B'$ with cardinality by $\natural$, for every positive $\epsilon<1$. We know define a multi-set $B$ such that $B(a)=|B'\cap f^{-1}|$ for every $a\in\Omega$. As $|B'|=|B|$ and $|B'\cap f^{-1}\phi|=|B\cap\phi|$, this is the required multi-set $\epsilon$-approximation.
\end{proof}

The following corollary is used in Theorem~\ref{thm_qq} below. Though it is sufficient for our application, stronger bound are known (essentially, we can replace $\epsilon$ for $\epsilon^2$). 


\begin{corollary}\label{coroll_epsilon_net}
Let $\Delta$ be a finite set-system of \vc-dimension $1<k<\omega$. Then for every positive $\epsilon<1$, every probability measure $\mu$ admits a multi-set $\epsilon$-net of cardinality bounded by $\natural$ of Theorem~\ref{thm_epsilon_approx}.
\end{corollary}

\begin{exercise}\label{ex_counterexample}
Suppose that $\{a\},\{b\}\in\Delta$ for some $a,b\in\Omega$. Show that, if $\mu$ and $\epsilon$ are such that $0<\epsilon<\mu(a)$ and $2\epsilon<\mu(b)-\mu(a)$, then $\mu$ admits no $\epsilon\jj$approximation.\QED
\end{exercise}



%%%%%%%%%%%%%%%%%%%%%%%%%%%
%%%%%%%%%%%%%%%%%%%%%%%%%%%
%%%%%%%%%%%%%%%%%%%%%%%%%%%
%%%%%%%%%%%%%%%%%%%%%%%%%%%
%%%%%%%%%%%%%%%%%%%%%%%%%%%
\section{A piercing problem}\label{qq}

After Pierre Simon, after Jiri Matousek, after Noga Alon and Daniel Kleitman.

Let $q\le p<\omega$ we say that $\Delta$ has the $(p,q)\jj$property if out of every $p$ sets in $\Delta$ some $q$ have non empty intersection. It has the dual $(p,q)\jj$property if for every $p$ point in $\Omega$ some $q$ are belong to the same set in $\Delta$. 

The following a is a particular case of a theorem of  Matousek (based on a proof of Noga Alon and Daniel Kleitman). The proof by Pierre Simon is simpler. [I will add details, references, and much more.]

\begin{theorem}\label{thm_qq}
Let $\Omega$ be finite and let $\Delta$ have \vc-dimension $k<\omega$. There are some integers $q$ and $h$ that depends only on $k$ such that, if every $B\subseteq\Omega$ of cardinality $q$ is a subset of some $\phi\in\Delta$, then $\Omega$ is covered by some $X\subseteq\Delta$ of cardinality $h$.
\end{theorem}

This theorem is often stated in the dual form which explains why it is sometimes called a piercing (or Helly-type) theorem: there are some integers $q$ and $h$ that depends only on $k$ such that, if every $q$ sets in $\Delta$ have non-empty intersection, then there is a $B\subseteq\Omega$ of cardinality $h$ that intersects every $\phi\in\Delta$. See Exercise~\ref{ex_dual_qqthm}.

% \begin{proof}
% Let $\<a_i:i<m\>$ and $\<\phi_j:j<n\>$ enumerate $\Omega$ and $\Delta$ without repetitions. Recall that the dual system is $\Omega^*=\Delta$ and $\Delta\!^{*}\!=\big\{\Delta_i: i<m\big\}$, where $\Delta_i=\big\{\phi\in\Delta: a_i\in\phi\big\}$. By Proposition~\ref{prop_vc*} the dual system has \vc-dimension $2^k$. Choose some $\epsilon=1/2$. From Theorem~\ref{thm_epsilon_approx} we obtain an $h$ such that every probability measure $\mu^*\!$ on \ $\Delta$ has an $\epsilon\jj$ap\-prox\-i\-ma\-tion of cardinality $h$. As $\epsilon$ is fixed, this $h$ only depends on $k$. Then there is a muti-set $X:\Delta\to\omega$ of cardinality $h$ such that
% 
% \parbox{10ex}{$\sharp$}
% $\displaystyle\left|\mu\!^*\!(\Delta_i)\ -\ \frac{|X\cap\Delta_i|}{h}\right|\medrel{\le}\epsilon.$
% 
% Recall how we read notation with multi-sets
% 
% \parbox{10ex}{~}
% $\displaystyle|X\cap\Delta_i|\medrel{=}\sum_{a_i\in\phi} X(\phi)$
% 
% Now, suppose that the probability measure on $\Delta$ is such that $\mu\!^*\!(\Delta_i)>\epsilon$ for all $i<m$, then the set $Y=\big\{\phi\in\Delta\ :\ X(\phi)\neq\0\big\}$ intersects every $\phi\in\Delta$. Moreover $|Y|\le h$ as required.
% 
% For $i<m$ and $j<n$ let $P_{i,j}=1$ if $a_i\in\phi_j$ and $0$ otherwise. Suppose we can find some $x_j\in\RR$ such that for every $i<m$
% 
% \parbox{10ex}{$\flat$}
% $\displaystyle\sum_{j<n}\big(P_{i,j}-\epsilon\big)x_j\medrel{>}0$
% 
% Then we can assume that all $x_j$ are positive. In fact,
% 
% \parbox{10ex}{~}
% $\displaystyle\sum_{j<n}P_{i,j}\medrel{\ge}1$
% 
% (because every $a_i$ belongs to some $\phi_j$) so we can translate translate solutions of any positive quantity. We can also rescale solutions so, setting $\mu\!^*(\phi_j)\deq x_j$, we obtain a well-defined probability measure
% 
% \parbox{10ex}{~}
% $\displaystyle\mu\!^*(\Delta_i)\medrel{=}\sum_{j<n}P_{i,j}\,x_j\medrel{>}\epsilon$
% 
% We only have to prove that equation $\flat$ has a solution. It suffices to show that clause \ssf{2} of Farkas' Lemma cannot obtain, that is, for every $\lambda_i\in\RR_+$ there are some $x_j\in\RR_+$ such that
% 
% \parbox{10ex}{$\flat\flat$}
% $\displaystyle\sum_{i<m}\lambda_i\sum_{j<n}\big(P_{i,j}-\epsilon\big)x_j\medrel{>}0$.
% 
% As we can assume the $\lambda_i$ add to $1$, setting $\mu(a_i)\deq\lambda_i$, we obtain a probability measure on $\Omega$. Note that for every $j$ we have
% 
% \parbox{10ex}{~}
% $\displaystyle\mu(\phi_j)\medrel{=}\sum_{i<m}\lambda_i P_{i,j}.$
% 
% By Theorem~\ref{thm_epsilon_approx}, once again, there is a $q$ such that for every probability measure $\mu$ on $\Omega$ there is a multi-set $B:\Omega\to\omega$ of cardinality $q$ such that
% 
% \parbox{10ex}{$\sharp\sharp$}
% $\displaystyle\left|\mu(\phi_j)\ -\ \frac{\big|B\cap\phi_j\big|}{q}\right|\medrel{\le}\epsilon.$
% 
% As $\epsilon$ is fixed, $q$ only depends on $k$. By assumption, there is a $\check\jmath$ such that $\big\{a:B(a)\neq0\big\}\subseteq\phi_{\check\jmath}$ hence $|B\cap\phi_{\check\jmath}|=q$. Therefore $\mu(\phi_{\check\jmath})>1-\epsilon$. Let $\check x_j$ be $1$ for $j=\check\jmath$ and $0$ otherwise. We claim that the tuple $\<x_j:j<n\>$ is a solution of $\flat\flat$. In fact
% 
% \parbox{10ex}{~}
% $\displaystyle\sum_{i<m}\lambda_i\sum_{j<n}\big(P_{i,j}-\epsilon\big)\check x_j\medrel{=}{\sum_{i<m}\lambda_i\big(P_{i,\check\jmath}-\epsilon\big)}\medrel{=}\mu(\phi_{\check\jmath})-\epsilon\medrel{\ge}1-\epsilon\medrel{>}0$.
% 
% This concludes the proof.
% \end{proof}



\begin{proof}
Let $\<a_i:i<m\>$ and $\<\phi_j:j<n\>$ enumerate $\Omega$ and $\Delta$ without repetitions. Recall that the dual system is $\Omega\!^*=\Delta$ and $\Delta\!^{*}\!=\big\{\Delta_i: i<m\big\}$, where $\Delta_i=\big\{\phi\in\Delta: a_i\in\phi\big\}$. By Proposition~\ref{prop_vc*} the dual system has \vc-dimension $2^k$. Choose some $\epsilon=1/2$. From Corollary~\ref{coroll_epsilon_net} we obtain an $h$ such that every probability measure $\mu\!^*$ on \ $\Delta$ admits an $\epsilon\jj$net of cardinality $h$. As $\epsilon$ is fixed, this $h$ only depends on $k$. Restating this explicitly there is a set $X\subseteq\Omega\!^*$ of cardinality $h$ such that

\parbox{10ex}{$\sharp$}
$\displaystyle\left|\mu\!^*\!(\Delta_i)\right|>\epsilon\medrel{\IMP}X\cap\Delta_i\neq\0$

So, if the probability measure on $\Delta$ is such that $\mu\!^*\!(\Delta_i)>\epsilon$ for all $i<m$, then $X\subseteq\Delta$ cover $\Omega$ as required by the theorem. So we only need to find such a probability measure.

For $i<m$ and $j<n$ let $P_{i,j}=1$ if $a_i\in\phi_j$ and $0$ otherwise. Suppose we can find some $x_j\in\RR$ such that for every $i<m$

\parbox{10ex}{$\flat$}
$\displaystyle\sum_{j<n}\big(P_{i,j}-\epsilon\big)x_j\medrel{>}0$

Then we can assume that all $x_j$ are positive. In fact,

\parbox{10ex}{~}
$\displaystyle\sum_{j<n}P_{i,j}\medrel{\ge}1$

(because every $a_i$ belongs to some $\phi_j$) so we can translate solutions of any positive quantity. We can also rescale solutions so, setting $\mu\!^*(\phi_j)\deq x_j$, we obtain a well-defined probability measure

\parbox{10ex}{~}
$\displaystyle\mu\!^*(\Delta_i)\medrel{=}\sum_{j<n}P_{i,j}\,x_j\medrel{>}\epsilon$

We only have to prove that equation $\flat$ has a solution. It suffices to show that clause \ssf{2} of Farkas' Lemma cannot obtain, that is, for every $\lambda_i\in\RR_+$ there are some $x_j\in\RR_+$ such that

\parbox{10ex}{$\flat\flat$}
$\displaystyle\sum_{i<m}\lambda_i\sum_{j<n}\big(P_{i,j}-\epsilon\big)x_j\medrel{>}0$.

As we can assume the $\lambda_i$ add to $1$, setting $\mu(a_i)\deq\lambda_i$, we obtain a probability measure on $\Omega$. Note that for every $j$ we have

\parbox{10ex}{~}
$\displaystyle\mu(\phi_j)\medrel{=}\sum_{i<m}\lambda_i P_{i,j}.$

Apply Corollary~\ref{coroll_epsilon_net} once again to the set-system $\<\Omega,\neg\Delta\>$. There is a $q$ such that for every probability measure $\mu$ on $\Omega$ there is a set $B\subseteq\Omega$ of cardinality $q$ such that

\parbox{10ex}{$\sharp\sharp$}
$\displaystyle\mu(\neg\phi_j)>\epsilon\medrel{\IMP}B\not\subseteq\phi_j$

As $\epsilon$ is fixed, $q$ only depends on $k$. By assumption, there is a $\check\jmath$ such that $B\subseteq\phi_{\check\jmath}$ and therefore $\mu(\phi_{\check\jmath})\ge1-\epsilon$. Let $\check x_j$ be $1$ for $j=\check\jmath$ and $0$ otherwise. We claim that the tuple $\<x_j:j<n\>$ is a solution of $\flat\flat$. In fact

\parbox{10ex}{~}
$\displaystyle\sum_{i<m}\lambda_i\sum_{j<n}\big(P_{i,j}-\epsilon\big)\check x_j\medrel{=}{\sum_{i<m}\lambda_i\big(P_{i,\check\jmath}-\epsilon\big)}\medrel{=}\mu(\phi_{\check\jmath})-\epsilon\medrel{\ge}1-\epsilon\medrel{>}0$.
This concludes the proof.
\end{proof}

The following is a classical result. The proof may be found in any introductory text of convex analysis or linear programming. There are many (sometimes non equivalent) ways to state it, we recommend  Terence Tao's exposition which may be found in his mathematical blog (we have reversed strict and weak inequalities, but the proof is identical).

\begin{proposition}[(Farkas' Lemma)]
For $i<m$ let $P_i:\RR^n\to\RR$ be affine linear functions. Then the following are equivalent\nobreak  
\begin{itemize}
\item[1.] $\displaystyle\bigwedge_{i<m}P_i(x)> 0$ for some $x\in\RR^n$;
\item[2.] there are no $\lambda_i\in\RR_+$ such that \smash{$\displaystyle\sum_{i<m}\lambda_iP_i(x)< 0$} for every $x\in\RR^n$.\QED
\end{itemize}
\end{proposition}

% \begin{proof}
% \def\myC{\raisebox{-.5ex}{\Bigg[}}
% \def\myJ{\raisebox{-.5ex}{\Bigg]}}
% 
% By induction on $n$. It will help induction 
% 
% \begin{itemize}
% \item[1.] $\displaystyle\bigvee_{j<m}\myC\bigwedge_{i<m}P_{i,j}(x)> 0\myJ$ for some $x\in\RR^n$;
% \item[2.] for every $j<m$ there are some $\lambda_i\in\RR_+$ such that \smash{$\displaystyle\sum_{i<m}\lambda_iP_{i,j}(x)\le 0$} for every $x\in\RR^n$.
% \end{itemize}
% 
% Let $|x|=n$ and $|y|=1$. Up to rescaling we can assume that $P_{i,j}(x,y)$ have the form 
% 
% \parbox[t]{10ex}{\hfil$\displaystyle\bigvee_{j<m}$}
% $\displaystyle\myC\;\bigwedge_{i\in I_1}P_{i,j}(x)+y>0\medrel{\wedge}\bigwedge_{i\in I_2}y-P_{i,j}(x)>0\medrel{\wedge}\bigwedge_{i\in I_3}P_{i,j}(x)>0\myJ$
% 
% It is easy to see that these equations can ve solved if we can first solve in the following equations in $x$   
% 
% \parbox[t]{10ex}{\hfil$\displaystyle\bigvee_{j<m}$}
% $\displaystyle\min_{i\in I_1} P_{i,j}(x)\ \ge\ \max_{i\in I_2}P_{i,j}(x)\wedge\myC\,\bigwedge_{i\in I_3}P_{i,j}(x)>0\myJ$
% 
% 
% The former can be written 
% 
% \end{proof}

\begin{exercise}\label{ex_dual_qqthm}
Let $k$, $q$ and $h$ be some integers as in Theorem~\ref{thm_qq} and let $\Delta$ be a set-system with dual \vc-dimension $k$. Suppose that any $q$ sets in $\Delta$ have non-empty intersection and prove that there is some set $B\subseteq\Omega$ of cardinality $h$ that intersects every $\phi\in\Delta$.\QED
\end{exercise}




\section{Appendix: Farkas' lemma}\label{appendix}

Below, an affine version of Farkas' Lemma tailored to our purposes.

\begin{proposition}
Fix some $v_1,\dots,v_n,u\in\QQ^k$ and let $r_1,\dots,r_n,s\in\QQ$. Let 

\hfil $X(r_1,\dots,r_n)=\{x\in\QQ^k\ :\ r_i\le v_i\cdot x, \textrm{ for every }i=1,\dots,n\}$. 

Then the following are equivalent
\begin{itemize}
\item[1.] $s\le u\cdot x$ for all $x\in X(r_1,\dots,r_n)$;
\item[2.] there exist $q_1,\dots,q_n\in\QQ^+$ such that \smash{$\displaystyle\sum^n_{i=1} q_iv_i=u$}.
\end{itemize}
\end{proposition}
\begin{proof}
Implication \ssf{2}$\IMP$\ssf{1} is immediate. To prove \ssf{1}$\IMP$\ssf{2} assume \ssf{1}. We claim that
\begin{itemize}
\item[3.] $0\le u\cdot x$ for all $x\in X(0,\dots,0)$.
\end{itemize}
Suppose not and let $x\in X(0,\dots,0)$ be such that $u\cdot x<0$. Let $y\in X(r_1,\dots,r_n)$. Then $y+ax\in X(r_1\dots,r_n)$ for every $a\in\QQ^+$. From \ssf{1} we obtain $s\le u\cdot (y+ax)$ which, for $a$ is sufficiently large, is a contradiction. 

From \ssf{3} it follows that if $x\cdot v_i=0$ for $i=1,\dots,n$ then $0= u\cdot x$. We can assume without loss of generality that $v_1,\dots,v_n$ are linearly independent. Then by linear algebra \smash{$\sum q_iv_i=u$} for some $q_i\in\QQ$. Now fix $i$ and verify that $q_i$ is non-negative. Let $x_i$ such that $x_i\cdot v_j=1$, if $i=j$, and $0$ otherwise. Then $0\le x_i\cdot u=q_i$.
\end{proof}



 
 
\end{document}