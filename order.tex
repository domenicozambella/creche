\documentclass[creche.tex]{subfiles}
\begin{document}
\chapter{Some ordered structures}
\label{saturation}
 
\def\medrel#1{\parbox[t]{6ex}{$\displaystyle\hfil #1$}}
\def\ceq#1#2#3{\parbox{25ex}{$\displaystyle #1$}\medrel{#2}$\displaystyle  #3$}

%%%%%%%%%%%%%%%%%%%%%%%%%%%%%
%%%%%%%%%%%%%%%%%%%%%%%%%%%%%
%%%%%%%%%%%%%%%%%%%%%%%%%%%%%
%%%%%%%%%%%%%%%%%%%%%%%%%%%%%
\section{Farkas' lemma}
In this section $L$ is the language of vector spaces over $\RR$ augmented with the binary relation $<$. Let \emph{$\RR_+$\/} be the set of positive reals. We write \emph{$T_0$\/} for the theory that contains by the axioms of vector spaces and the following

\begin{itemize}
\item[lin.] $x<y\ \vee\ x=y\ \vee\  y<x$\hfill linearity;
\item[tr.] $x<y\ \imp\ x+z < y+z$\hfill translation invariance;
\item[ph.] $x<y\ \imp\ r x < r y$\quad for all $r\in\RR_+$\hfill positive homogeneity.
\end{itemize}
For the remaining of the section we work in he category of models of $T_0$ with partial isomorphisms as morphisms. 

The theory \emph{$T_1$\/} also contains the density axiom 

\begin{itemize}
\item[nt.]$\E x,y\ x\neq y$\hfill non triviality;
\item[d.] $x<y\ \imp \E z\ x<z<y$\hfill density.
\end{itemize}

Clearly $\RR$, with the obvious interpretation of the language, is a model of $T_1$. Below we prove that $\RR$ is rich and that every saturated model of $T_1$ is rich.

\begin{theorem}
The model $\RR$ is rich and, if $|\RR|\le\lambda$, every $\lambda$-saturated model of $T_1$ is $\lambda$-rich.
\end{theorem}



The proof of the theorem is given below we first discuss some consequences. From~\ref{ricchezza_saturazione_EQ} we obtain the following.

\begin{corollary}
The theory $T_1$ has quantifier elimination and is complete.
\end{corollary}

Let $M$ be structure, $A\subseteq M$ and $x=\<x_i:i<\alpha\>$ a tuple of variables. The $L$-terms have the form 

\ceq{\hfill t(x)}{=}{\sum_{i<\alpha} r_ix_i}

for some tuple $r=\<r_i:i<\alpha\>\in\RR^{\oplus\alpha}$. Therefore the $L(A)$-terms have the form 

\ceq{\hfill t(x)}{=}{\sum_{i<\alpha} r_ix_i+a}

for some $a\in\<A\>_M$. 

Let $A\subseteq M\models T_0$. We say that a type $p(x)\subseteq L(A)$ is \emph{(algebraically) trivial\/} if 

\ceq{\hfill T_0\cup\Diag\<A\>_M}{\proves}{p(x) \iff B<x<C}

for some $B,C\subseteq\<A\>_M$. Let $c\in M$ and let $A\subseteq M$. We say that $c\in M$ is \emph{(algebraically) independent\/} from $A$ if $\attp(c/A)$ is trivial.  

Let $t(x)=t_1(x),\dots,t_n(x)$ be a finite tuple of $L(A)$-terms. The \emph{positive cone\/} generated by $t(x)$ is the set terms 

\def\cone{{\rm cone_+}}
\ceq{\hfill\emph{\rm\bf cone$_+t(x)$}}{=}{\Bigg\{\sum^n_{i=1}r_it_i(x)\ :\ r_i\in\RR_+\Bigg\}}

\begin{proposition}
Let $\phi(x)\in L_{\rm at}(A)$ have the form

$\displaystyle\sum^n_{i=1}t_i(x)<0 \sum^m_{i=n+1}t_i(x)$
\end{proposition}



In this section we fix a tuple of variables \emph{$x$\/} and a subset \emph{$A$} of an integral domain. We denote by \emph{$\Delta(A)$\/} the set of formulas of the form $0<t(x)$ where $t(x)$ is a term with parameters in $A$. So $\Delta(A)\jj$types are (possibly infinite) systems of linear inequalities with coefficients in $\<A\>_M$. 

For convenience we define the closure of $p(x)$ under logical consequences as follows

\ceq{\hfill \ccl\, p(x)}{=}{\Big\{t(x)=0\ :\    p(x)\, \proves\, t(x){=}0\Big\}}


\begin{proposition}[ (Fakas' Lemma)]
Let $p(x)$ be a $\Delta(A)$-type then $0\notin \cone t(x)$ 
\end{proposition}



\end{document}


