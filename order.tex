% !TEX root = creche.tex
\documentclass[creche.tex]{subfiles}
\begin{document}
\chapter{Some ordered structures}
\label{saturation}
 
\def\medrel#1{\parbox[t]{5ex}{$\displaystyle\hfil #1$}}
\def\ceq#1#2#3{\parbox{15ex}{$\displaystyle #1$}\medrel{#2}$\displaystyle  #3$}





\begin{notation}\label{notation1}
Recall that when $A\subseteq M$ we denote by \emph{$\<A\>_M$\/} the substructure of $M$ generated by $A$. Then $\<A\>_M\subseteq N$ is equivalent to $N\models \Diag\,\<A\>_M$. The diagram of a structure has been defined at the end of Section~\hyperref[frammenti]{\ref{types}.\ref{frammenti}}.

In this chapter, whenever some $A\subseteq M\models T$ are fixed, by \textit{model\/} we mean superstructures of $\<A\>_M$ that models $T$. The notions of \textit{logical consequence}, \textit{consistency}, \textit{completeness}, etc.\@ are modified accordingly, i.e.\@ they are understood to be over $T\cup\Diag\<A\>_M$.\QED
\end{notation}



%%%%%%%%%%%%%%%%%%%%%%%%%%%%%
%%%%%%%%%%%%%%%%%%%%%%%%%%%%%
%%%%%%%%%%%%%%%%%%%%%%%%%%%%%
%%%%%%%%%%%%%%%%%%%%%%%%%%%%%
\section{Farkas' lemma}
In this section $L$ is the language of vector spaces over $\RR$ augmented with the binary relation $<$. Let \emph{$\RR_+$\/} be the set of positive (i.e.\@ $>0$) reals. We write \emph{$T_0$\/} for the theory that contains by the axioms of vector spaces and the following

\begin{itemize}
\item[lin.] $x<y\ \vee\ x=y\ \vee\  y<x$\hfill linearity;
\item[tr.] $x<y\ \imp\ x+z < y+z$\hfill translation invariance;
\item[ph.] $x<y\ \imp\ r x < r y$\quad for all $r\in\RR_+$\hfill positive homogeneity.
\end{itemize}
For the remaining of the section we work in he category of models of $T_0$ with partial isomorphisms as morphisms. 

The theory \emph{$T_1$\/} also contains the density axiom 

\begin{itemize}
\item[nt.]$\E x,y\ x\neq y$\hfill non triviality;
\item[d.] $x<y\ \imp \E z\ x<z<y$\hfill density.
\end{itemize}

Clearly $\RR$, with the obvious interpretation of the language, is a model of $T_1$. Below we prove that $\RR$ is rich and that every saturated model of $T_1$ is rich.

\begin{theorem}
The model $\RR$ is rich and, if $|\RR|\le\lambda$, every $\lambda$-saturated model of $T_1$ is $\lambda$-rich.
\end{theorem}



The proof of the theorem is given below we first discuss some consequences. From Theorem~\ref{thm_ricchezza_saturazione_QE} we obtain the following.

\begin{corollary}
The theory $T_1$ has quantifier elimination and is complete.
\end{corollary}

Let $M$ be structure, $A\subseteq M$ and $x=\<x_i:i<\alpha\>$ a tuple of variables. The $L$-terms have the form 

\ceq{\hfill t(x)}{=}{\sum_{i<\alpha} r_ix_i}

for some tuple $r=\<r_i:i<\alpha\>\in\RR^{\oplus\alpha}$. Therefore the $L(A)$-terms have the form 

\ceq{\hfill t(x)}{=}{\sum_{i<\alpha} r_ix_i+a}

for some $a\in\<A\>_M$. 

Let $A\subseteq M\models T_0$. We say that a type $p(x)\subseteq L(A)$ is \emph{(algebraically) trivial\/} if 

\ceq{\hfill T_0\cup\Diag\<A\>_M}{\proves}{p(x) \iff B<x<C}

for some $B,C\subseteq\<A\>_M$. Let $c\in M$ and let $A\subseteq M$. We say that $c\in M$ is \emph{(algebraically) independent\/} from $A$ if $\attp(c/A)$ is trivial.  


In this section we fix a tuple of variables \emph{$x$\/} and a subset \emph{$A$} of an integral domain. We denote by \emph{$\Delta(A)$\/} the set of formulas of the form $0<t(x)$ where $t(x)$ is a term with parameters in $A$. So $\Delta(A)\jj$types are (possibly infinite) systems of linear inequalities with coefficients in $\<A\>_M$. 

For convenience we define the closure of $p(x)$ under logical consequences as follows

\ceq{\hfill \ccl\, p(x)}{=}{\Big\{t(x){>}0\ :\    p(x)\, \proves\, t(x){>}0\Big\}}

The \emph{convex cone closure\/} of $p(x)$ is the type 

\def\xcl{{\rm xcl}}
\ceq{\hfill\emph{\rm\bf xcl\,$p(x)$}}{=}{\bigg\{\sum^n_{i=1}r_it_i(x) >0\ :\ n\in\ZZ_+,\ \ r_i\in\RR_+,\ \textrm { and }\ t_i(x)\in p\bigg\}}

\begin{proposition}\label{prop_Nullstellensatz}
Let $A\subseteq M\models T_0$ and let $p(x)$ be a $\Delta(A)\jj$type. Fix some $N$ of sufficiently large cardinality such that $\<A\>_M\subseteq N\models T_1$. Then 

\ceq{\hfill\ccl\, p(x)}{=}{\Big\{t(x){>}0\,:\,  N\models\A x\,\big[ p(x)\imp t(x){>}0\big]\Big\}.}

\end{proposition}


\begin{proof} Only the inclusion $\supseteq$ requires a proof.  Fix some formula $t(x){>}0\notin\ccl\, p(x)$. Then $p(x)\wedge t(x)\le0$ is consistent, then $M'\models p(a)\wedge t(a)\leq0$ for some model $M'$ and some $a\in {M'}^{|x|}$. Then there is a partial isomorphism $h:M'\to N$ that extends $\id_A$ and is defined on $a$, provided $N$ is large enough to accommodate $a$. This implies that $p(x)\wedge t(x)\leq0$ has a solution in $N$. Hence $t(x)>0$ does not belong to the set on the r.h.s.  
\end{proof}

\begin{proposition}
Let $p(x)$ be a $\Delta(A)$-type then $\ccl\, p(x)=\xcl\,p(x)$
\end{proposition}


\begin{proof} Only the inclusion $\subseteq$ requires a proof. We fix some $t(x)\notin\xcl p$ and prove that $p(x)\wedge t(x)\leq0$ is consistent. Let $q$ be some prime ideal containing $p$ such that $t(x)\notin q$.  As $q$ is prime, the ring $A[x]/q$ is an integral domain. The polynomials that vanish in  $A[x]/q$ at $x+q$ are exactly those in $q$. Hence $A[x]/q$ witnesses the consistency of $q(x)\wedge t(x)\neq0$.
\end{proof}





\begin{proposition}[ (Fakas' Lemma)]
Let $p(x)$ be a $\Delta(A)$-type then 
\end{proposition}



\end{document}


