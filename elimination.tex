% !TEX root = creche.tex
\chapter{Preservation theorems}
\label{eliminazione}
\def\ceq#1#2#3{\noindent\smash{\parbox[t]{15ex}{$\displaystyle #1$}}\parbox{6ex}{\hfil $#2$}{$\displaystyle #3$}}


A few results dating from the 1950s describe the relationship between syntactic and semantic properties of first-order formulas. These theorems states that a certain syntactic class of formulas contains, up to logical equivalence, all first-order formulas preserved under a certain algebraic relationship between structures (the existence of a particular kind of homomorphism). Criteria for quantifier-elimination follow from these theorems.

\section{Lyndon-Robinson Lemma}
\label{TeoremidiPreservazione}

In this section $T$ is a consistent theory without finite models and $\Delta$ is a set of formulas closed under renaming of variables. At a first reading the reader is encouraged to assume that $\Delta$ is $L_{\rm at}$. 

If $\ssf{C}\subseteq\{\A,\E,\neg,\vee,\wedge\}$ is a set of connectives, we write \emph{$\ssf{C}\Delta$} for the closure of $\Delta$ with respect to all connectives in $\ssf{C}$. We write \emph{$\neg\Delta$} for the set containing the negation of the formulas in $\Delta$. Warning: do not confuse $\neg\Delta$ with $\{\neg\}\Delta$ and be aware that $\ssf{C}_1\ssf{C}_2\Delta$ does not always coincide with $\big(\ssf{C}_1\cup\ssf{C}_2)\Delta$.





% Diremo che $\Delta$ \`e un \emph{insieme di eliminazione\/} per $T$ se ogni formula $\phi(x)$, dove $x$ \`e un'arbitraria tupla di variabili che varia con $\phi(x)$, \`e equivalente modulo $T$ ad una formula in $\Delta$. Detto esplicitamente, se for every formula $\phi(x)$ esiste una formula $\psi(x)$ in $\Delta$ tale che
% 
% \hfil$\displaystyle T\proves\ \phi(x)\iff\psi(x)$.
% 
% Per esempio $\Delta$ pu\`o essere l'insieme delle formule senza quantificatori, l'insieme delle formule esistenziali, o delle combinazioni booleane di queste. Quando $\Delta$ \`e l'insieme delle formule senza quantificatori 



Recall that \emph{$\Delta$-morphism\/} is a map $k:M\imp N$ that preserves the truth of formulas in $\Delta$. It is immediate that $\Delta$-morphism are automatically $\{\mathord\wedge\mathord\vee\}\Delta$-morphisms. As $\Delta$ is closed under renaming of variables, $\{\E\}\Delta$-morphism are $\{\E\mathord\wedge\mathord\vee\}\Delta$-morphisms and similarly $\{\A\}\Delta$-morphism are $\{\A\mathord\wedge\mathord\vee\}\Delta$-morphisms.

Below we use the following proposition without further reference.

\begin{proposition}\label{prop_pokj}
Fix $M\models T$ and $a\in M^{|x|}$. Let $q(x)=\Deltatp_M(a)$. Then for every $\phi(x)\in L$ the following are equivalent
\begin{itemize}
\item[1.] $N\models\phi(ka)$ for every $k:M\imp N\models T$ that is a $\Delta$-morphism defined in $a$;
%\item[1'.] $N\models\phi(b)$ for every $N\models T$ ed ogni tupla $b$ in $N$ tale che $M,a\Rrightarrow_\Delta N,b$.
\item[2.] $T\ \proves\  q(x)\imp\phi(x)$.
\end{itemize}
\end{proposition}
\begin{proof}\ssf{2}$\IMP$\ssf{1}\quad Immediate.  

\ssf{1}$\IMP$\ssf{2}\quad Negate \ssf{2}, then there are $N\models T$ and $b\in N^{|x|}$ such that $q(b)\wedge\neg\phi(b)$. Therefore the map $k:M\to N$, where $k=\{\<a,b\>\}$, contradicts \ssf{1}.
\end{proof}

The following is sometimes referred to as the Lyndon-Robinson Lemma.

\begin{lemma}\label{lem_qfdefinability}
For every $\phi(x)\in L$ the following are equivalent
\begin{itemize}
\item[1.] $\phi(x)$ is equivalent over $T$ to a formula in $\{\mathord\wedge\mathord\vee\}\Delta$;
\item[2.] $\phi(x)$ is preserved by $\Delta$-morphisms between models of $T$.
%\item[2'.] $M\models\phi(a)\IMP N\models\phi(b)$ for every $M,a\Rrightarrow_\Delta N,b$ con $M$ ed $N$ modelli di $T$.
\end{itemize}
\end{lemma}
\begin{proof} \ssf{1}$\IMP$\ssf{2}\quad Immediate.

\ssf{2}$\IMP$\ssf{1}\quad  We claim that from \ssf{2} it follows that


\ceq{\#\hfill T}{\proves}{\phi(x)\ \ \iff\ \ \bigvee\big\{\,p(x)\subseteq\Delta\ :\ T\proves p(x)\imp\phi(x)\,\big\}}

Direction $\pmi$ is clear. To verify $\imp$, fix $M\models T$ and let $a\in M^{|x|}$ be such that $M\models\phi(a)$.  From \ssf{2} it follows that $\phi(x)$ satisfy \ssf{1} of Proposition~\ref{prop_pokj}. Therefore $T\proves q(x)\imp\phi(x)$ for $q(x)=\Deltatp(a)$. Hence $q(x)$ is one of the types the occur in the disjuction in $\#$ which is satisfied by $a$.

From $\#$ and compactness we obtain

\ceq{\hfill T}{\proves}{\phi(x)\ \ \iff\ \ \bigvee\big\{\,\psi(x)\in\{\wedge\}\Delta\ :\ T\proves \psi(x)\imp\phi(x)\,\big\}}

Applying compactness again, we replace the infinite disjunction above with a finite one and prove \ssf{2}.
\end{proof}

% 
% \begin{corollary}\label{prop_po}
% Fissiamo un modello $M\models T$ ed una tupla $a$ di elementi di $M$. Sia $p(x)=\E\Deltatp_M(a)$. Per ogni formula pura $\phi(x)$ le seguenti affermazioni sono equivalenti:
% \begin{itemize}
% \item[1.] $N\models\phi(ha)$ for every modello $N\models T$ e for every $\Delta$-morphism totale $h:M\imp N$;
% \item[2.] $T\ \cup\ p(x)\proves\phi(x)$.
% \end{itemize}
% \end{corollary}
% \begin{proof} 
% L'implicazione \ssf{2}$\IMP$\ssf{1} \`e immediata per  la proposizione~\ref{presesis}. Dimostriamo \ssf{1}$\IMP$\ssf{2}. Se  $\neg\ssf{2}$ allora esiste un $N\models T$ ed una tupla $b$ tale che  $N\models p(b)\wedge\neg\phi(b)$. Possiamo assumere $N$ sia $\lambda$ satturo per $\lambda$ sufficientemente grande. La mappa $k:M\to N$ con $k=\{\<a,b\>\}$ \`e un $\E\Delta$-morphism e quindi si estende ad un $\E\Delta$-morphism totale $k:M\imp N$ che prova $\neg\ssf{1}$.
% \end{proof}

\begin{proposition}\label{prop_EDelta_estensione}
Let $N$ be $\lambda$-saturated and let $k:M\imp N$ be a $\Delta$-morphism of cardinality $<\lambda$. Then the following are equivalent
\begin{itemize}
\item[1.] $k:M\imp N$ is a $\{\E\}\Delta$-morphism;
\item[2.] for every $b\in M$ there is a $\{\E\}\Delta$-morphism $h:M\imp N$ defined in $b$ that extends $k$;
\item[3.] for every $b\in M^{<\omega}$ there is a $\Delta$-morphism $h:M\imp N$ defined in $b$ that extends $k$.
\end{itemize}
\end{proposition}

\begin{proof}
\ssf{1}$\IMP$\ssf{2}\quad Let $a$ enumerate $\dom k$. Define $p(x,y)=\{\E\}\Deltatp_M (a,b)$. By \ssf{1}, $p(ka,y)$ is finitely consistent in $N$. By saturation there is a $b'\in N$ that realizes $p(ka,y)$. Therefore, $h:M\to N$ where $h=k\cup\{\<b,b'\>\}$, witnesses \ssf{2}.

\ssf{2}$\IMP$\ssf{3}\quad Iterate $|b|$-times the extension in \ssf{2}. 

\ssf{3}$\IMP$\ssf{1}\quad Up to logical equivalence, all formulas in $\{\E\}\Delta$ are of the form  $\E y \phi (x,y)$ for some tuple of variables $y$ and some $\phi(x,y)$ in $\{\wedge\}\Delta$. Assume $M\models\E y\,\phi(a,y)$ and let $b$ be such that $M\models\phi(a,b)$. By \ssf{3}, we can extend $k$ to some $\Delta$-morphism $h:M\to N$ defined in $b$. Then $N\models\phi(ha,hb)$ and therefore $N\models\E y\,\phi(ka,y)$.
\end{proof}

Iterating the lemma above we obtain the following.

\begin{corollary}\label{corol_EDelta_estensione}
Let $N$ be $\lambda$-saturated and let $|M|\le\lambda$. Let $k:M\imp N$ be a $\Delta$-morphism of cardinality $<\lambda$. Then the following are equivalent
\begin{itemize}
\item[1.] $k:M\imp N$ is a $\{\E\}\Delta$-morphism;
\item[2.] $k:M\imp N$ extends to an $\{\E\}\Delta$-embedding;
\item[3.] $k:M\imp N$ extends to an $\Delta$-embedding.\QED
\end{itemize}
\end{corollary}

The following theorem is often paraphrased as follows: a formula is existential if and only if (its truth) is preserved under extensions of structures.

\begin{theorem}
For every $\phi(x)\in L$ the following are equivalent
\begin{itemize}
\item[1.] $\phi(x)$ is equivalent over $T$ to a formula in $\{\E{\wedge}{\vee}\}\Delta$;
\item[2.] $\phi(x)$ is preserved by $\Delta$-embedding between models of $T$.
\end{itemize}
\end{theorem}
\begin{proof}
\ssf{1}$\IMP$\ssf{2}\quad Immediate. 

\ssf{2}$\IMP$\ssf{1}\quad Negate \ssf{1}. By Lemma~\ref{lem_qfdefinability} there is a $\{\E\}\Delta$-morphism $k:M\to N$ between models of $T$ that does not preserve $\phi(x)$. We can assume that $N$ is $\lambda$-saturated for some sufficiently large $\lambda$. By Corollary~\ref{corol_EDelta_estensione} there is a $\Delta$-embedding $h:M\to N$ that extends $k$ and contradicts \ssf{2}.
\end{proof}

A dual version of the results above is obtained replacing embeddings by epimorphisms (surjective homomorphisms) and $\{\E\}$ by $\{\A\}$. If $\Delta$ contains the formula $x=y$ and is closed under negation, then $k:M\imp N$ is a $\Delta$-morphism if and only is $k^{-1}:N\imp M$ is a $\Delta$-morphism. In this case the dual version follows from what proved above. Without these assumptions the results need a similar but independent proof.

\begin{proposition}\label{prop_ADelta_estensione}
Let $M$ be $\lambda$-saturated and let $k:M\imp N$ be a $\Delta$-morphism of cardinality $<\lambda$. Then the following are equivalent
\begin{itemize}
\item[1.] $k:M\imp N$ is a $\{\A\}\Delta$-morphism;
\item[2.] for every $c\in N$ some $\{\A\}\Delta$-morphism $h:M\imp N$ extends $k$ and $c\in\range h$;
\item[3.] for every $c\in N^{<\omega}$ some $\Delta$-morphism $h:M\imp N$ extends $k$ and $c\in(\range h)^{<\omega}$.
\end{itemize}
\end{proposition}

\begin{proof} Left as an exercise for the reader. To prove implication \ssf{1}$\IMP$\ssf{2} define $p(x,y)=\neg\{\A\}\Deltatp_N (ka,c)$, where $a$ is a tuple that enumerates $\dom k$. From \ssf{1} we obtain that $p(a,y)$ is finitely consistent in $M$. Then we proceed as in the proof of Proposition~\ref{prop_EDelta_estensione}.
\end{proof}

\begin{corollary}\label{corol_ADelta_estensione}
Let $M$ be $\lambda$-saturated and let $|N|\le\lambda$. Let $k:M\imp N$ be a $\Delta$-morphism of cardinality $<\lambda$. Then the following are equivalent
\begin{itemize}
\item[1.] $k:M\imp N$ is a $\{\A\}\Delta$-morphism;
\item[2.] $k:M\imp N$ extends to an $\{\A\}\Delta$-epimorphism;
\item[3.] $k:M\imp N$ extends to an $\Delta$-epimorphism.\QED
\end{itemize}
\end{corollary}

Finally we obtain the following.

\begin{theorem}
The following are equivalent
\begin{itemize}
\item[1.] $\phi(x)$ is equivalent to a formula in $\{\A\}\Delta$;
\item[2.] every $\Delta$-epimorphism between models of $T$ preserves $\phi(x)$.\QED
\end{itemize}
\end{theorem}


%%%%%%%%%%%%%%%%%%%%%%%%%%%%%%%%%%%%%%%%%%%%%%%%%%
%%%%%%%%%%%%%%%%%%%%%%%%%%%%%%%%%%%%%%%%%%%%%%%%%%
%%%%%%%%%%%%%%%%%%%%%%%%%%%%%%%%%%%%%%%%%%%%%%%%%%
%%%%%%%%%%%%%%%%%%%%%%%%%%%%%%%%%%%%%%%%%%%%%%%%%%
%%%%%%%%%%%%%%%%%%%%%%%%%%%%%%%%%%%%%%%%%%%%%%%%%%
\section{Quantifier elimination by back-and-forth}
\label{eliminazionequantificatoricriterio}

We say that \emph{$T$ admits\/} (or \emph{has\/}) \emph{positive $\Delta$-elimination of quantifiers\/} if for every formula $\phi(x)$ in $\{\A,\E,\mathord\wedge,\mathord\vee\}\Delta$ there is a formula $\psi(x)$ in $\{\mathord\wedge,\mathord\vee\}\Delta$ such that

\ceq{\hfill T}{\proves}{\phi(x)\iff\psi(x).}

When $\Delta$ is closed under negation the attribute \textit{positive\/} becomes irrelevant and will be omitted. When $\Delta$ is $\atpmL$ or $\qfL$, we simply say that $T$ admits elimination of quantifiers. This is by far the most common case.

Quantifier elimination is often used to prove that a theory is complete because it reduces it to something much simpler to prove. The following is an immediate consequence of the definition above with $x$ replaced by the empty tuple.

\begin{remark}
If $T$ has elimination of quantifiers then the following are equivalent 
\begin{itemize}
\item[1.] $T$ decides all quantifier free sentences;
\item[2.] $T$ is complete.
\end{itemize}
Hence a theory with quantifier elimination is complete if it decides the characteristic, see Definition~\ref{rmk_characteristic}).\QED
\end{remark}

% Scriveremo  ${\vee}\hskip-.33ex\Delta$ e ${\wedge}\hskip-.2ex\Delta$ per l'insieme delle formule che sono disgiunzione, rispettivamente congiunzione, di formule in $\Delta$ e scriveremo $\E\Delta$ per le formule della forma $\E y\,\phi(x,y)$ dove $y$ \`e una tupla di variabili e $\phi(x,y)$ \`e in $\Delta$. Se vogliamo limitarci a tuple $y$ di lunghezza $1$ scriveremo $\E_1\Delta$. Infine $\pmDelta$ \`e l'insieme che contiene le formule in $\Delta$ e le loro negazioni. Quindi, a meno di equivalenza logica, le formule in $\veewedge\pmDelta$ sono le combinazioni booleane di formule in $\Delta$ e le formule in $\veewedge\Delta$ le combinazioni booleane positive.

The following is a consequence of Lemma~\ref{lem_qfdefinability}.

\begin{corollary}\label{criterioeq1}
The following are equivalent
\begin{itemize}
\item[1.] $T$ has $\Delta$-elimination of quantifiers;
\item[2.] every $\Delta$-morphism between models of $T$ is both a $\{\E\}\Delta$ and a $\{\A\}\Delta$-morphism.
\end{itemize}
\end{corollary}
\begin{proof}
\ssf{1}$\IMP$\ssf{2}\quad Immediate.

\ssf{2}$\IMP$\ssf{1}\quad We prove by induction of syntax that $\Delta$-morphism preserve the truth of all formulas in $\{\E\,\A\mathord\wedge\mathord\vee\}\Delta$, this suffices by Lemma~\ref{lem_qfdefinability}. Induction for the connectives $\vee$ and $\wedge$ is trivial.  So assume as induction hypothesis that the truth of  $\phi(x,y)$ is preserved.  By Lemma~\ref{lem_qfdefinability} $\phi(x,y)$ is equivalent to a formula in $\{\mathord\wedge\mathord\vee\}\Delta$, hence by \ssf{2} the truth of $\E y\, \phi (x,y)$ and $\A y \,\phi (x,y)$ is preserved.
\end{proof}

Condition \ssf{2} of the corollary above may be difficult to verify directly. The following corollary of Proposition~\ref{prop_EDelta_estensione} and~\ref{prop_ADelta_estensione} gives a back-and-forth condition with is easier to verify.

\begin{corollary}\label{criterioeq2}
Let $|L|\le\lambda$. The following are equivalent
\begin{itemize}
\item[1.] $T$ has $\Delta$-elimination of quantifiers;
\item[2.] for every finite $\Delta$-morphism $k:M\imp N$ between $\lambda$-saturated models of $T$:
\begin{itemize}                                                                                               \item[a.] for every $b\in M$ some $\Delta$-morphism $h:M\imp N$ extends $k$ and $b\in\dom h$.                                                                                \item[b.] for every $c\,\in\,N$ some $\Delta$-morphism $h:M\imp N$ extends $k$ and $c\in\range h$.\noindent\nolinebreak[4]\hfill\rlap{\ \ $\Box$}                                                                   \end{itemize}
\end{itemize}
\end{corollary}

Note that when $\Delta$ contains the formula $x=y$ and is closed under negation, then  $k:M\imp N$ is a $\Delta$-morphism if and only if $k^{-1}:N\imp M$ is a $\Delta$-morphism. In this case \ssf{a} and \ssf{b} are equivalent.

\begin{exercise}
Let $L$ be the language of strict orders. Let $T$ be axiomatized by $T_{\rm lo}$ (see Chapter~\ref{relational}) and the following pair of axioms
\begin{itemize}
\item[dis$\uparrow$.] $\E z\ \big[x<z\ \wedge\ \neg\E y\ x<y<z\big]$;
\item[dis$\downarrow$.] $\E z\ \big[ z<x\ \wedge\ \neg\E y\ z<y<x\big]$.
\end{itemize}
Note that $T$ is essentially the same theory as in Exercise~\ref{ex_QQxZZ_saturo} but in a different language. Let $\Delta$ be the set of formulas that contains (all alphabetic variants of) the formulas $x=y$, $x<y$, $\E^{=n}y\ (x\mathord<y\mathord<z)$ for all positive integers $n$. Prove that $T$ has $\Delta$-elimination of quantifiers.\QED
\end{exercise}


\begin{exercise}
Let $T$ be a complete theory without finite models in a language that consists only of unary predicates. Prove that $T$ has elimination of quantifiers.\QED
\end{exercise}

\begin{comment}


\section{La model-completezza}

Per ragioni storiche introduciamo la seguente nozione. 

Diremo che $T$ \`e $\Delta$-modello completa se ogni $\Delta$-morphism $h:M\to N$ totale tra modelli di $T$ \`e un $\{\A\E\mathord\wedge\!\mathord\vee\}\Delta$-morphism. 

Diremo che $T$ \`e $\Delta$-modello completa se ogni $\Delta$-morphism $h:M\to N$ suriettivo tra modelli di $T$ \`e un $\{\A\E\mathord\wedge\!\mathord\vee\}\Delta$-morphism. 




 (la terminologia \`e bizzarra Abraham Robinson.)


%%%%%%%%%%%%%%%%%%%%%
%%%%%%%%%%%%%%%%%%%%
%%%%%%%%%%%%%%%%%%%%%%%
%%%%%%%%%%%%%%%%%%%%%%%
\section{Un esempio}

Il seguente esempio ha il solo scopo di applicare in un caso molto semplice quando visto nel paragrafo~\ref{eliminazionequantificatoricriterio}. Vale un risultato pi\`u forte e pi\`u generale di quanto affermato dal teorema~\ref{eliminazionearitmeticaadditiva}. Questo infatti \`e un semplice corollario del teorema di Bauer-Monk sull'eliminazione dei quantificatori per i moduli (che qui non dimostreremo).

Il linguaggio, che qui denoteremo con $L_{\rm ga\small+1}$, estende quello dei gruppi additivi con la costante $1$. 

Per ogni intero $n$ scriveremo $\delta_n(x)$ per $\E y\ ny=x$. La formula $\delta_0(x)$ verr\`a letta come $x=0$.  Per il tutto questo paragrafo $\Delta$ denota l'insieme che contiene tutte le formule $\delta_n(x)$, con $n$ non negativo, ed \`e chiuso per sostituzione di variabili con termini puri. Osserviamo che per $\delta_0(x)$ \`e equivalente alla formula $x=0$ quindi $\Delta$ contiene a meno di equivalenza tutte le formule atomiche. Definiamo ora la teoria $T_{\rm 1}$. Questa contiene gli assiomi dei gruppi additivi e
\begin{itemize}
\item[1.] $\neg\delta_n(m)$ for every coppia di interi non negativi $m,n$ tali che $n\nmid m$;
\item[2.] $\displaystyle\A x\ \bigvee^{n-1}_{r=0}\delta_{n}(x+r)$ \ \ for every coppia di interi $m>0$ ed $n>1$.
\end{itemize}

Tra gli assiomi in \ssf{1} ci sono le formule $\neg\delta_0(n)$ for every $n$ positivo, i modelli di $T_1$ sono gruppi senza torsione e contengono tutti una copia di $\ZZ$.

\begin{lemma}
Sia $\Delta$ come sopra. La teoria $T_{\rm 1}$ \`e completa per gli enunciati in $\Delta$.
\end{lemma}

\begin{proof}
Gli enunciati in $\Delta$ hanno la forma $\delta_n(m)$. Il lemma dice che for every $M\models T_{\rm 1}$

\hfil $M\models\delta_n(m)\ \ \IFF\ \ \Z\models\delta_n(m)$.

la direzione $\PMI$ \`e ovvia perch\'e $\Z\subseteq M$. La direzione opposta \`e garantita dallo schema di assiomi \ssf{1}.
\end{proof}

\begin{theorem}\label{eliminazionearitmeticaadditiva} 
La teoria $T_1$ ammette $\Delta$-eliminazione dei quantificatori.
\end{theorem}

\begin{proof}
Fissiamo $M$ ed $N$ due modelli $\omega$-saturi di $T_{\rm 1}$ e sia $k:M\to N$  e sia $b$ un elemento di $M$. Verifichiamo che vale la condizione \ssf{2a} del corollario~\ref{criterioeq2}. Sia $a$ una enumerazione di $\dom k$ e poniamo $p(z,x)=\Deltatp_M(a,b)$, vogliamo mostrare che $p(c,x)$ \`e realizzato in $N$.

Prima di procedere osserviamo che dagli assiomi \ssf{1} e \ssf{2} segue 

\ssf{3.}\hfil $\displaystyle\neg\delta_n(x+r)\ \iff\ \bigvee^{n-1}_{r\neq s=0}\delta_n(x+s)$

Le formule in $p(z,x)$ hanno la forma $\delta_n\big(mx+t(z)\big)$ per un intero $m$ ed un termine $t(z)$. Modulo 

$T_1\cup\Th_\Delta(M/a)\proves q(x)\iff p(a,x)$


$T_1\cup\Th_\Delta(N/ka)\proves q(x)\iff p(ka,x)$


La seguente formula vale in tutti i modelli di $T_{\rm 1}$, segue semplicemente dagli assiomi dei gruppi abeliani,

\hfil$\delta_n(t(z)-r)\ \imp\ \A x\,\Big[\delta_{n}\big(mx+t(z)\big)\ \iff\ \delta_{n}\big(mx+r\big)\Big]$.


Per \ssf{2.} sappiamo che for every termine $t(z)$ esiste un intero $0\le r<n$ tale che $M\models\delta_{n}(t(a)+r)$. 




Quindi nelle formule che occorrono in $p(z,x)$ possiamo sostituire i termini $t(z)$ con le opportune costanti $r$ e ottenere un tipo $q(x)$ che in $M$ \`e equivalente $p(a,x)$ ed in $N$ \`e equivalente a $p(c,x)$. 

Per verificare che $p(c,x)$ \`e finitamente consistente in $N$ \`e sufficiente verificare che  $q(x)$ \`e finitamente consistente. Mostriamo che tutte le congiunzione di formule in $q(x)$ sono consistenti in $\Z$ e quindi, poich\'e sono formule esistenziali anche in $N$. Supponiamo per assurdo che esista una congiunzione di formule in $q(x)$ che non \`e soddisfatta in $\Z$, quindi che in $\Z$ valga 

\hfil$\displaystyle\neg\E x\ \bigwedge_{i\in I}\delta_{n_i}(m_ix+r_i)$.

Semplicemente per de Morgan questa formula \`e equivalente a

\hfil$\displaystyle\A x\ \bigvee_{i\in I}\neg\delta_{n_i}(m_ix+r_i)$

Quindi, sostituendo le negazioni con disgiunzioni come in \ssf{4}, ci riduciamo ad una formula della forma

\hfil$\displaystyle\A x\ \bigvee_{j\in J}\delta_{n_j}(m_jx+r_j)$

Questa, se vera in $\Z$, \`e un'istanza dello schema di assiomi \ssf{2}, quindi \`e vera in $M$. Contraddizione.
\end{proof}


\end{comment}


%%%%%%%%%%%%%%%%%%%%%
%%%%%%%%%%%%%%%%%%%%
%%%%%%%%%%%%%%%%%%%%%%%
%%%%%%%%%%%%%%%%%%%%%%%
%\section{Esempi: L'aritmetica di Pre\ss burger}


 
%\end{comment}
