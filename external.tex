% !TEX root = creche.tex
\documentclass[creche.tex]{subfiles}
\begin{document}
\chapter{Externally definable sets}
\label{external}

\def\medrel#1{\parbox[t]{6ex}{$\displaystyle\hfil #1$}}
\def\ceq#1#2#3{\parbox{25ex}{$\displaystyle #1$}\medrel{#2}$\displaystyle  #3$}



In this chapter we fix a signature $L$, a complete theory $T$ without finite models, and a saturated model $\U$ of inaccessible cardinality $\kappa>|L|$.
The notation and implicit assumptions are as in Section~\hyperref[monster]{\ref*{saturation}.\ref*{monster}}.



%%%%%%%%%%%%%%%%%%%%%%%%%%%%%
%%%%%%%%%%%%%%%%%%%%%%%%%%%%%
%%%%%%%%%%%%%%%%%%%%%%%%%%%%%
%%%%%%%%%%%%%%%%%%%%%%%%%%%%%
\section{Approximable sets}
\label{approximable}

\def\ceq#1#2#3{\parbox{25ex}{$\displaystyle #1$}\medrel{#2}$\displaystyle  #3$}

Let $\grC,\grD\subseteq\U^{|{\gr z}|}$.
The set $\grD\cap A^{|{\gr z}|}$ is called the \emph{trace\/} of $\grD$ over $A$.
We write $\grC=_A\grD$ if  $\grC$ and $\grD$ have the same trace on $A$.

Let $p({\mr x})\subseteq L(\U)$ be a consistent type.
Recall from Section~\hyperref[invariant_sets]{\ref*{invariant}.\ref*{invariant_sets}} that for every formula $\phi({\mr x}\,;{\gr z})\in L$ we define

\ceq{\hfill\emph{$\gr\D_{p,\phi}$}}{=}{\Big\{{\gr a}\in\U^{|{\gr z}|}\ :\ \phi({\mr x}\,;{\gr a})\in p\Big\}.}

We say that $\grD$ is \emph{externally definable\/} if it is of the form ${\gr\D_{p,\phi}}$ for a type $p(x)$ in $S(\U)$ or $S_\phi(\U)$.
We say that $\grD$ is externally definable \emph{by $p({\mr x})$ and $\phi({\mr x}\,;{\gr z})$}.

Equivalently, a set $\grD$ is externally definable if it is the trace over $\U$ of a set which is definable in some elementary extension of $\U$.
More precisely, $\grD$ is the trace on $\U$ of a set of the form $\phi({\mr ^{*\kern-.3ex}b}\,;{\gr ^{*\kern-.2ex}\U})$ where $ ^{*\kern-.2ex}\U$ is elementary extension of $\U$ and ${\mr  ^{*\kern-.3ex}b}\in {}^{*\kern-.2ex}\U^{|{\mr x}|}$.
The latter interpretation explains the terminology.

\noindent\llap{\textcolor{red}{\Large\danger}\kern1.5ex}We prefer to deal with external definability in a different, though equivalent, way.
This is not the most common approach.

\begin{definition}\label{def_approx}
We say that $\grD$ is \emph{approximated\/} by the formula $\phi({\mr x}\,;{\gr z})$ if for every finite $B$ there is a tuple ${\mr a}\in\U^{|{\mr x}|}$ such that $\phi({\mr a}\,;{\gr\U})=_B\grD$.
We call $\phi({\mr x}\,;{\gr z})$ the \emph{sort} of $\grD$.
If in addition $\phi({\mr a}\,;{\gr\U})\subseteq\grD$, we say that  $\grD$ is \emph{approximated from below}.
Equivalently, we say that  $\grD$ is approximated from below if for every finite ${\gr B}\subseteq\grD$ there is a tuple ${\mr a}\in\U^{|{\mr x}|}$ such that  ${\gr B}\subseteq\phi({\mr a}\,;{\gr\U})\subseteq\grD$.
The dual notion of \emph{approximation from above\/} is defined as expected (and coincides with $\neg\grD$ being approximated by $\neg\phi({\mr x}\,;{\gr z})$ from below).\QED
\end{definition} 


The following proposition is clear by compactness.

\begin{proposition}\label{prop_approx=external}
   For every $\grD$ the following are equivalent
   \begin{itemize}
   \item[1.] $\grD$ is approximated by $\phi({\mr x}\,;{\gr z})$;
   \item[2.] $\grD$ is externally definable by $\phi({\mr x}\,;{\gr z})$.\QED
   \end{itemize}
\end{proposition}

The rest of this section is only required in Chapter~\ref{vc}.

Approximability from below is an adaptation to our context of the notion of \textit{having an honest definition} in \cite{CS}.

\begin{definition}\label{def_defble_type}
We say that the global type $p\in S_{{\mr x}}(\U)$ is \emph{honestly definable\/} if for every $\phi({\mr x}\,;{\gr z})\in L$ the set ${\gr\D_{p,\phi}}$ is approximated from below (by some formula).
We say that $p$ is \emph{definable\/} if the sets ${\gr\D_{p,\phi}}$ are all definable (over $\U$).
Note that the terminology is misleading: honestly definable is weaker than definable.
\end{definition}

\begin{example}
Every definable set is trivially approximable.
Sets may be approximable by different formulas.
For instance, if $T=T_{\rm dlo}$, then $\D=\{z\in\U:a\le z\le b\}$ is approximable both from below and from above by the formula $x_1<z<x_2$ though it is not definable by this formula.

Now, let $T=T_{\rm rg}$.
Then every $\D\subseteq\U$ is approximable and, when $\D$ has small infinite cardinality, it is approximable from above but not from below.\QED
\end{example}

In Definition~\ref{def_approx}, the sort $\phi({\mr x}\,;{\gr z})$ is fixed (otherwise any set would be approximable) but this requirement of uniformity may be dropped if we allow $B$ to have larger cardinality.

\begin{proposition}\label{lem_approx_nonunif}
For every $\grD$ the following are equivalent
\begin{itemize}
\item[1.] $\grD$ is approximable;
\item[2.] for every $C\subseteq\U$ of cardinality $\le|T|$ there is $\psi({\gr z})\in L(\U)$ such that $\psi({\gr\U})=_C\grD$.
\end{itemize}
\end{proposition}

\begin{proof}
To prove \ssf{2}$\IMP$\ssf{1} assume \ssf{2} and negate \ssf{1} for a contradiction.
For each formula $\psi({\mr x}\,;{\gr z})\in L$ choose a finite set $B$ such that $\psi({\mr b}\,;{\gr\U})\neq_B\grD$ for every ${\mr b}\in\U^{|{\mr x}|}$.
Let $C$ be the union of all these finite sets.
Clearly $|C|\le|T|$.
By \ssf{2} there are a formula $\phi({\mr x}\,;{\gr z})$ and a tuple ${\mr c}$ such that $\phi({\mr c}\,;{\gr\U})=_C\grD$, contradicting the definition of $C$.
\end{proof}



\begin{remark}\label{prop_approx_el_eq}
If $\grD\subseteq\U^{|{\gr z}|}$ is approximated by $\phi({\mr x}\,;{\gr z})$ then so is any $\grC$ such that $\grC\equiv\grD$, see Section~\hyperref[expansions]{\ref*{invariantL}.\ref*{expansions}} for the notation.
In fact, if the set $\grD$ is approximable by $\phi({\mr x}\,;{\gr z})$ then for every $n$

\hfil$\displaystyle\A {\gr z_1},\dots,{\gr z_n}\;\E {\mr x}\ \bigwedge^n_{i=1}\big[\phi({\mr x}\,;{\gr z_i})\ \iff\ {\gr z_i}\in{\gr \D}\big]$.


So the same holds for any $\grC\equiv\grD$.
A similar remark apply to approximability from below and from above (e.g.\@ for approximability from below, add the conjunct $\A {\gr z}\,\big[\phi({\mr x}\,;{\gr z})\imp {\gr z}\in\grD\big]$ to the formula above).\QED
\end{remark}

From the following easy observation of Chernikov and Simon~\cite{CS} we obtain an interesting (and misterious) quantifier elimination result originally due to Shelah, see Corollary~\ref{corol_sh_exp_qe} below.

% \begin{proposition}
% Let $\C\subseteq\U^{|{\gr z},w|}$ be approximated from below by the formula $\phi({\mr x}\,;{\gr z},w)$. Then $\grD=\big\{{\gr z}:\E w\ \big({\gr z}\,w\in\C\big)\big\}$ is approximated from below by the formula $\E w\,\phi({\mr x}\,;{\gr z}\,w)$.
% \end{proposition}
% 
% \begin{proof}
% Let $B\subseteq\U$ be finite. 
% We want ${\mr a}\in\U^{|{\mr x}|}$ such that 
% 
% \ceq{\ssf{a.}\hfill \E w\ \big({\gr b}\,w\in\C\big)}{\iff}{\E w\,\phi({\mr a}\,;{\gr b}\,w)}\hfill for every ${\gr b}\in B^{|{\gr z}|}$
% 
% \ceq{\ssf{b.}\hfill \A {\gr z}\;\Big[\E w\,\phi({\mr a}\,;{\gr z}\,w)}{\imp}{\E w\ \big({\gr z}\,w\in\C\big)\Big]}
% 
% Let $C\subseteq\U$ be a finite set such that 
% 
% \ceq{\ssf{c.}\hfill \E w\in C^{|w|}\ \big({\gr b}\,w\in\C\big)}{\iff}{\E w\ \big({\gr b}\,w\in\C\big)}\hfill for every ${\gr b}\in B^{|{\gr z}|}$
% 
% As $\C$ is approximable from below, there is an ${\mr a}$ such that
% 
% \ceq{\ssf{a'.}\hfill {\gr b}\,c\in\C}{\iff}{\phi({\mr a}\,;{\gr b}\,c)}\hfill for every ${\gr b}\,c\in \big(B\cup C\big)^{|{\gr z}\,w|}$
% 
% \ceq{\ssf{b'.}\hfill \A {\gr z}\,w\ \Big[\phi({\mr a}\,;{\gr z}\,w)}{\imp}{{\gr z}\,w\in\C\Big]}
% 
% We obtain \ssf{b} from \ssf{b'} simply by logic. 
% Implication $\imp$ in \ssf{a} follows from \ssf{a'} and \ssf{c}. 
% Implication $\pmi$ follows from \ssf{b}.
% \end{proof}

\begin{proposition}\label{prop_sh_exp_qe}
Let $\C\subseteq\U^{|y\,{\gr z}|}$ be approximated from below by the formula $\phi({\mr x}\,;y\,{\gr z})$.
Then $\grD=\big\{{\gr z}:\E y\ \big(y\,{\gr z}\in\C\big)\big\}$ is approximated from below by the formula $\E y\,\phi({\mr x}\,;y\,{\gr z})$.
\end{proposition}

\begin{proof}
\def\vl{\gr}
Let $B\subseteq\U$ be finite. 
We want ${\mr a}\in\U^{|{\mr x}|}$ such that 

\ceq{\ssf{a.}\hfill \E y\ \big(y\,{\vl b}\in\C\big)}{\iff}{\E y\,\phi({\mr a}\,;y\,{\vl b})}\hfill for every ${\vl b}\in B^{|{\vl z}|}$

\ceq{\ssf{b.}\hfill \A {\vl z}\;\Big[\E y\,\phi({\mr a}\,;y\,{\vl z})}{\imp}{\E y\ \big(y\,{\vl z}\in\C\big)\Big]}

Let $D\subseteq\U$ be a finite set such that 

\ceq{\ssf{c.}\hfill \E y\in D^{|y|}\ \big(y\,{\vl b}\in\C\big)}{\iff}{\E y\ \big(y\,{\vl b}\in\C\big)}\hfill for every ${\vl b}\in B^{|{\vl z}|}$

As $\C$ is approximable from below, there is an ${\mr a}$ such that

\ceq{\ssf{a'.}\hfill d\,{\vl b}\in\C}{\iff}{\phi({\mr a}\,;d\,{\vl b})}\hfill for every $d\,{\vl b}\in \big(D\cup B\big)^{|y\,{\vl z}|}$

\ceq{\ssf{b'.}\hfill \A y\,{\vl z}\ \Big[\phi({\mr a}\,;y\,{\vl z})}{\imp}{y\,{\vl z}\in\C\Big]}

We obtain \ssf{b} from \ssf{b'} simply by logic. 
Implication $\imp$ in \ssf{a} follows from \ssf{a'} and \ssf{c}. 
Implication $\pmi$ follows from \ssf{b}.
\end{proof}



\begin{corollary}
If $p\in S_{{\mr x}}(\U)$ is honestly definable then the family of sets externally definable by $p$ is closed under quantifiers and Boolean combinations. 
\end{corollary}

\begin{proof}
The sets externally definable by $p({\mr x})$ are always closed under Boolean operations. By the proposition above, they are closed under quantifiers. 
\end{proof}

% Let ${\gr z}=\<z_i:i<\lambda\>$ and let $\grD\subseteq\U^{|{\gr z}|}$. 
% For $I\subseteq\lambda$ we write $\E z_{\restriction I}\,\grD$ for the projection of $\grD$ to the $\lambda\sm I$ coordinates, i.e.\@ the set $\{b_{\restriction\lambda\sm I}: {\gr b}\in\grD\}$. 
% Expand the language with a symbol for $\E z_{\restriction I}\,\grD$ for every finite $I\subseteq\lambda$. 
% Then the theory of $\U$ in the expanded language has positive $\Delta$-quantifier elimination, for $\Delta=L\cup\big\{\E z_{\restriction I}\,\grD: I\subseteq\lambda\textrm{ finite}\big\}$, see Section~\hyperref[eliminazionequantificatoricriterio]{\ref*{elimination}.\ref*{eliminazionequantificatoricriterio}}.

%%%%%%%%%%%%%%%%%%%%%%%%%%%%%%%%%%%%%%%%%%%%%%%%%%%%%%%%%%%%%%%%%%%%%%%%%%%%%%%%%%%%%%%
\section{Ladders and definability}


%Here we write \emph{$\phi({\mr x}\,;{\gr z})^*$\/} for the partitioned formula where the order of the two tuples is inverted (i.e.\@ it is the opposite of what displayed).

Let $\phi({\mr x}\,;{\gr z})\in L(\U)$ be a partitioned formula (these have been introduced in Definition~\ref{def_partitioned_fla}).
We say that $\<{\mr a_i}\,;{\gr b_i} : i<\alpha\>$ is a \emph{ladder sequence\/} for $\phi({\mr x}\,;{\gr z})$ if 

\ceq{\#\hfill i< j}{\IFF}{\phantom{\neg}\phi({\mr a_i}\,;{\gr b_j})}\hfill for all $0\le i,j<\alpha$

We say that the formula $\phi({\mr x}\,;{\gr z})$ is \emph{stable\/} if for some finite $n$ all ladders have length at most $n$. Otherwise we say it is \emph{unstable} or that it has the \emph{order property}.

Note that if a formulas admits ladder sequences of unbounded finite length, then it admits an infinite one.

The following easy exercise shows that stability is sort of chain condition.

\begin{exercise}
   Prove that the following are equivalent
   \begin{itemize}
   \item[1.]  $\phi({\mr x}\,;{\gr z})$ has a ladder sequence of length $n$;
   \item[2.] there a set ${\gr B}$ such that $\phi({\mr a_0}\,;{\gr B})\subset\dots\subset\phi({\mr a_n}\,;{\gr B})$ for some $\<{\mr a_i} : i<n\>$.\QED
   \end{itemize}
\end{exercise}


%It is clear that $\phi({\mr x}\,;{\gr z})$ is stable if and only if $\phi({\mr x}\,;{\gr z})^*$ is stable.

%Suppose $\phi({\mr x}\,;{\gr z})$ is unstable and let $\<{\mr a_i}\,;{\gr b_i} : i<\omega\>$ be a ladder sequence. By padding redundant variables, we can read $\phi({\mr x}\,;{\gr z})$ as a formula $\phi({\mr x}\,{\gr z}\,;{\mr y}\,{\gr w})$. Then $\phi({\mr x}\,{\gr z}\,;{\mr y}\,{\gr w})$ defines the order of the sequence $\<{\mr a_i}\,{\gr b_i} : i<\omega\>$. For this reason it is also common to say that $\phi({\mr x}\,;{\gr z})$ has the \emph{order property}, this simply means that it is unstable. 

The following theorem claims what is arguably one of the most important properties of stable formulas: any set externally definable by a stable formula is definable (by a related formula).

\begin{theorem}\label{thm_def_stable_formula}
Any $\grD\subseteq\U^{|{\gr z}|}$ approximated by a stable formula is definable.
More precisely, if $\phi({\mr x}\,;{\gr z})$ is a stable formula that approximates $\grD$ then there are ${\mr a_{1,1}},\dots,{\mr a_{n,m}}\in\U^{|{\mr x}|}$ such that 

\ceq{\hfill {\gr z}\in\grD}{\iff}{\bigvee^n_{i=1}\bigwedge^m_{j=1}\phi({\mr a_{i,j}}\,;{\gr z})}
\end{theorem}

\begin{proof}
The theorem follows immediately from the the three lemmas below.
\end{proof}

Below, in Theorem~\ref{thm_def_stable_formula2}, we proves the converse of teh theorem above:  if every set approxiamted by $\phi({\mr x}\,;{\gr z})$ is definable then  $\phi({\mr x}\,;{\gr z})$ is stable.

\begin{remark}\label{rem_sability_no_compactness}
The conclusion of Theorem~\ref{thm_def_stable_formula} is often stated in the following  apparently more general form:
for every $\Aa\subseteq\U$ there are ${\mr a_{1,1}},\dots,{\mr a_{n,m}}\in \Aa^{|{\mr x}|}$ such that for all ${\gr b}\in{\gr\Aa}^{|{\gr z}|}$

\ceq{\hfill {\gr b}\in\grD}{\iff}{\bigvee^n_{i=1}\bigwedge^m_{j=1}\phi({\mr a_{i,j}}\,;{\gr b}).}

The proof is exactly the same. In fact, elementarity and saturation are never used. In a sense, no model theory is used, either -- just finite combinatorics -- unlike in the proof of Theorem~\ref{thm_def_stable_formula2} where compactness is essential.\QED
\end{remark}

\begin{lemma}
If $\grD$ is approximated from below by a stable formula $\phi({\mr x}\,;{\gr z})$ then

\ceq{\hfill {\gr z}\in\grD}{\iff}{\bigvee^n_{i=0}\phi({\mr a_i}\,;{\gr z})}

for some ${\mr a_0},\dots,{\mr a_n}\in\U^{|{\mr x}|}$. 
\end{lemma}

\begin{proof}
The elements ${\mr a_0},\dots,{\mr a_n}$ are defined recursively together with some auxiliary elements ${\gr b_0},\dots,{\gr b_{n-1}}\in\grD$.

Suppose ${\gr b_0},\dots,{\gr b_{n-1}}$ have been defined (this assumption is empty if $n=0$).
We first define ${\mr a_n}$, then ${\gr b_n}$. 
Choose ${\mr a_n}\in\U^{|{\mr x}|}$ such that ${\gr b_0},\dots,{\gr b_{n-1}}\in\phi({\mr a_n}\,;{\gr\U})\subseteq\grD$.
This is possible because $\grD$ is approximated from below.
Now, if possible, choose ${\gr b_n}$ such that

\ceq{\hfill{\gr b_n}}{\in}{\grD\sm\bigcup^n_{i=0}\phi({\mr a_i}\,;{\gr\U})}.

Then $\<{\mr a_i}\,;{\gr b_i} : i\le n\>$ is a ladder sequence. 
By stability, for some $n$, the tuple ${\gr b_n}$ does not exist.
This yields the required ${\mr a_0},\dots,{\mr a_n}$.
\end{proof}


\begin{lemma}
If $\grD$ is approximated by a stable formula $\phi({\mr x}\,;{\gr z})$.
Then, for some $m$, the formula 

\ceq{\hfill\psi({\mr x_0},\dots,{\mr x_m}\,;{\gr z})}{=}{\bigwedge^m_{j=0}\phi({\mr x_j}\,;{\gr z})}

approximates $\grD$ from below.
\end{lemma}

\begin{proof}
Let $m$ be such that there is no ladder sequence for $\phi({\mr x}\,;{\gr z})$ of length greater then $m$.
Let ${\gr C}\subseteq\grD$ be finite.
We prove that there are some ${\mr a_0},\dots,{\mr a_m}$ such that ${\gr C}\subseteq\psi({\mr a_0},\dots,{\mr a_m}\,;{\gr\U})\subseteq\grD$.
As in the proof above, we define by recursion a ladder sequence for $\phi({\mr x}\,;{\gr z})$.
Suppose that ${\mr a_0},\dots,{\mr a_{n-1}}$ and ${\gr b_0},\dots,{\gr b_{n-1}}\notin\grD$ have been defined.
We first define ${\mr a_n}$, then ${\gr b_n}$. 
Choose ${\mr a_n}\in\U^{|{\mr x}|}$ such that 

\hfil${\gr C}\ \subseteq\ \phi({\mr a_n}\,;{\gr\U})\ \subseteq\ \U^{|{\gr z}|}\sm\{{\gr b_0},\dots,{\gr b_{n-1}}\}$.

This ${\mr a_n}$ exists, because $\grD$ is approximated by $\phi({\mr x}\,;{\gr z})$.
(Apply Definition~\ref{def_approx} with any $B$ such that ${\gr C}\cup\{{\gr b_0},\dots,{\gr b_{n-1}}\}\subseteq B^{|{\gr z}|}$.)
Then, if possible, let ${\gr b_n}$ such that

\ceq{\hfill{\gr b_n}}{\in}{\bigcap^n_{i=0}\phi({\mr a_i},{\gr\U})\sm\grD}

This procedure has to stop at some $n\le m$.
Hence the required parameters are ${\mr a_1},\dots,{\mr a_n}={\mr a_{n+1}}=\dots={\mr a_{m}}$.
\end{proof}

\begin{lemma}\label{lem_stable3}
If $\phi({\mr x}\,;{\gr z})$ is a stable formula then for every $m$ the formula $\psi({\mr x_0},\dots,{\mr x_m}\,;{\gr z})$ defined above is stable.
\end{lemma}

\begin{proof}
It suffices to prove that if $\phi_1({\mr x_1}\,;{\gr z})\wedge\phi_2({\mr x_2}\,;{\gr z})$ is unstable then one of the formulas $\phi_n({\mr x_i}\,;{\gr z})$ is unstable. For simplicity, we use that instability implies the existence of an infinite ladder (this uses compactness, apparently contradicting Remark~\ref{rem_sability_no_compactness}). We leave to the reader to adapt the argument so that compactness is not required.

Let ${\mr a^1_i},{\mr a^2_i}\in\U^{|{\mr x}|}$ and ${\gr b_i}\in\U^{|{\gr z}|}$ be such that 

\ceq{\hfill i\le j}{\IFF}{\phantom{\neg}\phi_1({\mr a^1_i}\,;{\gr b_j})\wedge\phi_2({\mr a^2_i}\,;{\gr b_j})}\hfill for all $i,j<\omega$

For $n=1,2$ let $H_n\subseteq{\omega\choose 2}$ contain those pairs $j<i$ such that $\neg\phi_n({\mr a^n_i}\,;{\gr b_j})$. 
By the equivalence above $H_1\cup H_2={\omega\choose 2}$. 
By the Ramsey Theorem there is an infinite set $H$ such that ${H\choose 2}\subseteq H_n$ for at least one of $n=1,2$. Suppose $H_1$ for definiteness. So, we obtain an infinite sequence $a^1_i$, $b_i$ such that

\ceq{\hfill j<i}{\IFF}{\neg\phi_1({\mr a^1_i}\,;{\gr b_j})}\hfill for all $i,j<\omega$

hence $\phi_1({\mr x_1}\,;{\gr z})$ is unstable.
\end{proof}

This last lemma concludes the proof of Theorem~\ref{thm_def_stable_formula}. 


\begin{theorem}\label{thm_def_stable_formula2}
  The following are equivalent
  \begin{itemize}
    \item[1.] $\phi({\mr x}\,;{\gr z})$ is stable;
    \item[2.] every subset of $\U^{|{\gr z}|}$ that is externally definable by $\phi({\mr x}\,;{\gr z})$ is definable;
    \item[3.] there are $\le\kappa$ subsets of $\U^{|{\gr z}|}$ that are externally definable by $\phi({\mr x}\,;{\gr z})$;
    \item[4.] there are $<2^\kappa$ subsets of $\U^{|{\gr z}|}$ that are externally definable by $\phi({\mr x}\,;{\gr z})$.
  \end{itemize}
\end{theorem}

\begin{proof}
\ssf{1}$\IMP$\ssf{2} is clear by Proposition~\ref{prop_approx=external} and Theorem~\ref{thm_def_stable_formula}.

\ssf{2}$\IMP$\ssf{3}$\IMP$\ssf{4} are obvious.

\ssf{4}$\IMP$\ssf{1} is proved by contraposition.
Suppose that $\phi({\mr x}\,;{\gr z})$ is not stable.
By compactness there is a ladder sequence  $\<{\mr a_i}\,;{\gr b_i} : i\in I\>$ where $I,<_I$ a dense linear order of cardinality $\kappa$ with $2^\kappa$ cuts, where by \textit{cut\/} we mean a subset $c\subseteq I$ that is closed downward.
For every such $c\subseteq I$ we pick a global type

\ceq{\hfill p_c({\mr x} ) }{\supseteq}{\big\{\phi({\mr x}\,;{\gr b_i})\iff i\in c\ :\ i\in I\big\}.}

Clearly sets ${\gr\D_{p_c,\,\phi} }$ are all distinct.
\end{proof}

% \begin{exercise}\label{ex_harrington}
% \def\grB{{\gr{\EuScript B}}}
% \def\mrA{{\mr{\EuScript A}}}
% The following is a version of Harrington's mysterious 
% Lemma (cfr.~\cite[Lemma 8.3.4]{TZ}).
% Let $\phi({\mr x}\,;{\gr z})\in L$ be a stable formula and suppose $\grB\subseteq\U^{|{\gr z}|}$ and $\mrA\subseteq\U^{|{\mr x}|}$ are approximated by $\phi({\mr x}\,;{\gr z})$ and  $\phi({\mr x}\,;{\gr z})^*$, respectively.
% Then at least one of the conditions \ssf{1} and \ssf{2} below occurs

% \begin{itemize}
% \item[1.]
% \ssf{a.}\kern2ex$\phi({\mr x}\,;{\gr z})^{\phantom{*}}\wedge\ {\mr x}\in{\mrA}$ approximates $\grB$ and\\
% \ssf{b.}\kern2ex$\phi({\mr x}\,;{\gr z})^*\wedge\ {\gr z}\in{\grB}$ approximates $\mrA$.

% \item[2.]
% \ssf{a.}\kern2ex$\phi({\mr x}\,;{\gr z})^{\phantom{*}}\wedge\ {\mr x}\notin{\mrA}$ approximates $\grB$ and\\
% \ssf{b.}\kern2ex$\phi({\mr x}\,;{\gr z})^*\wedge\ {\gr z}\notin{\grB}$ approximates $\mrA$.

% \end{itemize}
% Hint: Note that at least one of \ssf{1a} or \ssf{2a} occurs.
% Hence it suffices to prove that \ssf{1a}$\IMP$\ssf{1b} and \ssf{2a}$\IMP$\ssf{2b}. The two implications are essentially equivalent.
% \begin{proof}
% Note that at least one of \ssf{1a} or \ssf{2a} occurs.
% Hence it suffices to prove that \ssf{1a}$\IMP$\ssf{1b} and \ssf{2a}$\IMP$\ssf{2b}.
% We only prove the first.
% The second follows because $\neg\grB$ and $\neg\mrA$ are approximated by $\neg\phi({\mr x}\,;{\gr z})$ and  $\neg\phi({\mr x}\,;{\gr z})^*$, respectively.

% Assume \ssf{1a} and negate \ssf{1b} for a contradiction.
% Then \ssf{2b} holds.
% Pick ${\mr a_0}$ arbitrarily, then recursively find ${\gr b_i}\notin\grB$ and  ${\mr a_i}\in\mrA$ such that

% {\def\medrel#1{\parbox[t]{12ex}{$\displaystyle\kern2ex #1$}}

% \ceq{\hfill\phi(\U\,;{\gr b_i})}{ =_{{\mr a_0},\dots,{\mr a_{i-1}}}}{\mrA} 

% \ceq{\hfill\phi({\mr a_i}\,;\U)}{ =_{{\gr b_0},\dots,{\gr b_i}}}{\grB}
% }

% Then

% {\def\medrel#1{\parbox[t]{6ex}{$\displaystyle\kern2ex #1$}}

% \ceq{\hfill i< j}{\IMP}{\phantom{\neg}\phi({\mr a_i}\,;{\gr b_j})}

% \ceq{\hfill j\le i}{\IMP}{\neg\phi({\mr a_i}\,;{\gr b_j})}

% }

% Which contradicts the stability of $\phi({\mr x}\,;{\gr z})$.
% \end{proof}
% \end{exercise}




\section{Stable theories}
\label{stable_theories}

We say that $T$ is a \emph{stable theory\/} if every formula is stable. By Theorem~\ref{thm_def_stable_formula2} this is equivalent to requiring that all externally definable sets are definable.

If $p({\mr x})\in S(\U)$ is a global type, a \emph{canonical base\/} of $p({\mr x})$ is a definably closed set $\emph{$\Cb(p)$}\subseteq \U^\eq$ such that an automorphism $f\in\Aut(\U)$ fixes $p({\mr x})$ if and only if it fixes $\Cb(p)$ pointwise. When they exist, canonical bases are unique, see Exercise~\ref{ex_Cb}.

Clearly, all definable types (Definition~\ref{def_defble_type}) have a canonical base, namely

\ceq{\hfill\Cb(p)}{=}{{\dcl}^\eq\Big(\big\{{\gr\D_{p,\phi}}\ :\ \phi({\mr x}\,;{\gr z})\in L\big\}\Big).}

Therefore if $T$ is stable, all global types have a canonical base.

We now turn to Lascar invariance. Quite interestingly when $T$ is stable this reduces to a more manageable kind of invariance.

\begin{proposition}\label{prop_type_over_acl2} Let $T$ be stable and let $p({\mr x})\in S(\U)$. Then the following are equivalent
\begin{itemize}
\item[1.] $p({\mr x})$ is Lascar invariant over $A$;
\item[2.] $p({\mr x})$ is definable over $\acl^\eq\!A$;
\item[3.] ${\gr\D_{p,\phi}}\in \acl^\eq\!A$ for all $\phi({\mr x}\,;{\gr z})\in L$.
\end{itemize}
\end{proposition}
\begin{proof}
 \ssf{3}$\IMP$\ssf{2}$\IMP$\ssf{1} are clear (stability is not required).
 
 \ssf{1}$\IMP$\ssf{3} The sets ${\gr\D_{p,\phi}}$ are externally definable therefore, by Theorem~\ref{thm_def_stable_formula2}, definable (over $\U$).
 As $p({\mr x})$ is Lascar invariant over $A$, so are the sets ${\gr\D_{p,\phi}}$.
 Hence they belong to $\acl^\eq\!A$ by Theorem~\ref{thm_Galois_alg=alg}.
\end{proof}
% 
% \begin{proposition}\label{prop_type_over_dcl} Let $T$ be stable and let $p({\mr x})\in S(\U)$ be Lascar invariant over $A$. Then for every formula $\phi({\mr x}\,;{\gr z})\in L$ the set ${\gr\D_{p,\phi}}$ belongs to $\dcl^\eq(A,c)$ for every $c\models p_{\restriction A}(x)$.
% \end{proposition}
% \begin{proof}
% Without loss of generality we can assume $p({\mr x})\in S(\U^\eq)$. 
% \end{proof}

A type $q({\mr x})\subseteq L(A)$ is \emph{stationary\/} if it has a unique global extension that is Lascar invariant over $A$.
The following proposition says that in a stable theory with elimination of imaginaries types over algebraically closed sets are stationary.

\begin{proposition}\label{prop_type_over_acl_stationary} If $T$ is stable then every type $q({\mr x})\in S\big(\acl^\eq\!A\big)$ is stationary.
\end{proposition}

\begin{proof}
Let $p_i({\mr x})\in S(\U^\eq)$, for $i=1,2$, be two global types that extend $q({\mr x})$ and are invariant over $\acl^\eq\!A$.
To prove that $p_1({\mr x}) =p_2({\mr x})$, it suffices to show that for every formula $\phi({\mr x}\,;{\gr z})\in L$

\ceq{\#\hfill {\gr\D_{p_1,\,\phi}}}{=}{{\gr\D_{p_2,\,\phi}}}

Note that, by Proposition~\ref{prop_type_over_acl2} both sets belong to  $\acl^\eq\!A$.
Clearly, for $i=1,2$, the formula $\A{\gr z}\,\big[\phi({\mr x}\,;{\gr z})\iff {\gr z}\in {\gr\D_{p_i,\,\phi}}\big]$ belongs to $p_i({\mr x})$.
Then both formulas belong to $q({\mr x})$ and $\#$ follows.
\end{proof}

\begin{corollary}
If $T$ is stable then the following are equivalent
\begin{itemize}
\item[1.] ${\mr a}\equivL_A{\mr b}$, see Definition~\ref{def_Lascar_type};
\item[2.] ${\mr a}\equivSh_A{\mr b}$, see Definition~\ref{def_Sh_strong_type}.
\end{itemize}
\end{corollary}
\begin{proof}\ssf{1}$\IMP$\ssf{2}.
  This is left as an exercise to the reader (stability is not required).

  \ssf{2}$\IMP$\ssf{1}.
  Assume ${\mr a}\equivSh_A{\mr b}$.
  By Proposition~\ref{prop_Shelah_strong_types} this is equivalent to ${\mr a}\equiv_{\acl^\eq\!A}{\mr b}$.
  Let $q({\mr x})=\tp({\mr a}/{\acl}^\eq\!A)=\tp({\mr b}/{\acl}^\eq\!A)$. Let $p({\mr x})\in S(\U^\eq)$ be the unique global type that is invariant over $\acl^\eq\! A$ and extends $q({\mr x})$ which we obtain from by Proposition~\ref{prop_type_over_acl_stationary}.
  Let ${\mr\bar c}=\<{\mr c_i}:i<\omega\>$ be such that ${\mr c_i}\models p{\restriction \acl^\eq(A),\,{\mr a},\,{\mr b},\,{\mr c_{\restriction i}}}$.
  Then ${\mr a},{\mr\bar c}$ and ${\mr b},{\mr\bar c}$ are $A$-indiscernible sequences, which proves \ssf{1}, see Exercise~\ref{ex_Lstp_indiscernibles}.
\end{proof}

We end this section with a characterization of stability which is not directly related with the properties discussed above. 

\begin{proposition} The following are equivalent
\begin{itemize}
\item[1.] $T$ is stable;
\item[2.] every $A$-indiscernible sequence is totally $A$-indiscernible.
\end{itemize}
\end{proposition}

\begin{proof}
\ssf{2}$\IMP$\ssf{1}. Assume $\neg$\ssf{1} and let $\phi({\mr x}\,;{\gr z})$ be an unstable formula witnessed by the ladder sequence $\<{\mr a_i}\,;{\gr b_i}:i<\omega\>$. Let $\<{\mr a'_i}\,;{\gr b'_i} : i<\omega\>$ be 
indiscernible sequence that models the EM-type of $\<{\mr a_i}\,;{\gr b_i} : i<\omega\>$. This is not totally indiscernible because $\phi({\mr a'_i}\,;{\gr b'_j})$ if and if $i< j$.

\ssf{1}$\IMP$\ssf{2}. Assume $\neg$\ssf{2} and let $\<a_i : i<\omega\>$ be an $A$-indiscernible sequence, which is not totally $A$-in\-dis\-cern\-ible. Then there is a formula $\phi(x, y)\in L(A)$ and some $i<j$ such that $\phi(a_i , a_j )\wedge\neg\phi(a_j , a_i) $. By indiscernibility  $\phi(a_i, a_j)\vee a_i=a_j$ holds if and only if $i\le j$. Hence $\phi(x\,; y)\vee x=y$ is not stable.
\end{proof}

\begin{exercise}\label{ex_stable_orderproperty}
Prove that the following are equivalent
\begin{itemize}
\item[1.] $T$ is stable;
\item[2.] there is an infinite set $A\subseteq\U^{|{\mr x}|}$ and a formula  $\psi({\mr x}\,;{\mr y})$ such that $A$ is linearly ordered by the relation ${\mr a}<{\mr b}\iff\psi({\mr a}\,;{\mr b})$.
\end{itemize}
Hint: suppose $\phi({\mr x}\,;{\gr z})$ is unstable and let $\<{\mr a_i}\,;{\gr b_i} : i<\omega\>$ be an infinite ladder sequence, then $A=\{{\mr a_i}\,{\gr b_i} : i<\omega\}$ is linearly ordered.\QED
\end{exercise}


\begin{exercise}
Prove that if every formula $\phi({\mr x}\,;{\gr z})\in L$ with $|{\mr x}|=1$ is stable then $T$ is stable. Hint: by compactness, if all sets approximable by $\phi({\mr x}\,;y,{\gr z})$ are definable, so are the sets approximable by $\phi({\mr x},y\,;{\gr z})$.\QED
\end{exercise}


\begin{exercise}
Prove that strongly minimal theories are stable.\QED
\end{exercise}


\begin{exercise}\label{ex_Cb}
Let $p({\mr x})\in S(\U)$. Prove that there is at most one definably closed set $A\subseteq\U^\eq$ such that $\Aut(\U/A)$ is the set of automorphisms that fix $p({\mr x})$.\QED
\end{exercise}

\section{Stability and the number of types}

The following proposition highlights the connection between stability and the cardinality of types. 

Binary trees of formulas have been introduced in Definition~\ref{def_tree_formulas}.
Here we restrict to trees of a particular form.
Namely, $\<\psi_s:s\in 2^{<\omega}\>$  where $\psi_\0=\top$ and for $s\in 2^{<\omega}$ and $i\in 2$ we have $\psi_{s^\frown i}({\mr x})=\neg^i\phi({\mr x}\,;{\gr b_s})$.

In general, we write \emph{$\neg^i$\/} for $\neg\!\stackrel{i\ \rm times}{\dots\dots\dots}\!\neg$.

% Set the overall layout of the tree
\tikzstyle{level 1}=[level distance=3.5cm, sibling distance=2.5cm]
\tikzstyle{level 2}=[level distance=3.5cm, sibling distance=1.2cm]
\tikzstyle{level 3}=[level distance=2.5cm, sibling distance=0.5cm]
\tikzstyle{level 4}=[level distance=0.5cm, sibling distance=0.5cm]

% Define styles for bags and leafs
\tikzstyle{bag0} = [text width=1.5ex, align=left]
\tikzstyle{bag} = [text width=7.5ex, align=right]
%\tikzstyle{end} = [circle, minimum width=3pt,fill, inner sep=0pt]

\def\leaf{...}

\quad
\begin{tikzpicture}[grow=right]
\node[bag0] {{$\top$}}
    child {
        node[bag] {$\phi(x;b_{\0}\!)$}      
            child {
                node[bag] {$\phi(x;b_0\!)$}
                    child {
                       node[bag] {\footnotesize$\phi(x;b_{00})$\rlap{$\ \cdots$}}
                       edge from parent
                    }    
                    child {
                       node[bag] {\footnotesize$\llap{$\neg$}\phi(x;b_{00})$\rlap{$\ \cdots$}}
                       edge from parent
                    }  
                 edge from parent
            }
            child {
                node[bag] {\llap{$\neg$}$\phi(x;b_0\!)$}
                edge from parent
                    child {
                       node[bag] {\footnotesize$\phi(x;b_{01})$\rlap{$\ \cdots$}}
                       edge from parent
                    }    
                    child {
                       node[bag] {\footnotesize\llap{$\neg$}$\phi(x;b_{01})$\rlap{$\ \cdots$}}
                       edge from parent
                    }  
                 edge from parent
            }
       edge from parent 
    }
    child {
        node[bag] {\llap{$\neg$}$\phi(x;b_{\0}\!)$}         
            child {
                node[bag] {$\phi(x;b_1\!)$}
                    child {
                       node[bag] {\footnotesize$\phi(x;b_{10})$\rlap{$\ \cdots$}}
                       edge from parent
                    }    
                    child {
                       node[bag] {\footnotesize\llap{$\neg$}$\phi(x;b_{10})$\rlap{$\ \cdots$}}
                       edge from parent
                    }  
                 edge from parent
            }
            child {
                node[bag] {\llap{$\neg$}$\phi(x;b_1\!)$}
                edge from parent
                    child {
                       node[bag] {\footnotesize$\phi(x;b_{11})$\rlap{$\ \cdots$}}
                       edge from parent
                    }    
                    child {
                       node[bag] {\footnotesize\llap{$\neg$}$\phi(x;b_{11})$\rlap{$\ \cdots$}}
                       edge from parent
                    }  
                 edge from parent
            } 
        edge from parent
    };
\end{tikzpicture}

\medskip
When a binary tree of this form exists, we say that $\phi({\mr x}\,;{\gr z})$ has the \emph{binary tree property}.

\begin{theorem}\label{thm_count_types}
  The following are equivalent
  \begin{itemize}
  \item[1.] $\phi({\mr x}\,;{\gr z})$ is not stable;
  \item[2.] $\phi({\mr x}\,;{\gr z})$ has the binary tree property.
  \end{itemize}
\end{theorem}

\begin{proof}
  \ssf{1}$\IMP$\ssf{2}.
  From $\ssf{1}$, by compactness, there is a ladder sequence  $\<{\mr a_s}\,;{\gr b_s} : s\in2^{<\omega}\>$, where $2^{<\omega}$ is ordered lexicografically.
  We claim that for every $r\in2^\omega$ the type $p_r({\mr x})=\big\{\phi({\mr x}\,;{\gr b_s})\iff r<s : s\in2^{<\omega}\big\}$ is consistent. 
  In fact, $\phi({\mr a_{r\restriction n}}\,;{\gr b_s})$ holds for all $s\in2^{<n}$.
  Then consistency follows by compactness.

  For $s\in2^{<\omega}$ let $\psi_s({\mr x})$ be as above with ${\gr b_{s^\frown1}}$ for ${\gr b_s}$.
  We claim that $\<\psi_s({\mr x}):s\in 2^{<\omega}\>$ is a binary tree.
  We need to prove that the branches are consistent.
  It suffices to show that $\psi_{r\restriction (n+1)}({\mr x})\in p_r$.

  First, suppose that $r(n)=0$. 
  Then $\psi_{r\restriction (n+1)}({\mr x})=\phi({\mr x}\,;{\gr b_{r\restriction n^\frown1}})$ belongs to $p_r({\mr x})$ because $r<r{\restriction} n^\frown1$.
  Otherwise $r(n)=1$. 
  Then $\psi_{r\restriction (n+1)}({\mr x})=\neg\phi({\mr x}\,;{\gr b_{r\restriction(n+1)}})$ belongs to $p_r({\mr x})$ because $r\not< r{\restriction}(n+1)$.

  \ssf{2}$\IMP$\ssf{1}. From $\ssf{2}$, by compactess, there is a binary tree of height $\kappa$. 
  Hence there are $2^\kappa$ sets that are externally definable by $\phi({\mr x}\,;{\gr z})$.
  Therefore, by Therorem~\ref{thm_def_stable_formula2}, $\phi({\mr x}\,;{\gr z})$ is not stable.
\end{proof}

\begin{corollary}\label{corol_count_types}
The following are equivalent
\begin{itemize}
\item[1.] $\phi({\mr x}\,;{\gr z})$ is a stable formula;
\item[2.] $\big|S_{\phi}(A)\big|\le|A|$ for all countable sets $A$;
\item[3.] $\big|S_{\phi}(A)\big|<2^{|A|}$ for all countable sets $A$.
\end{itemize}
\end{corollary}

\begin{proof}
  The corollary follows immediately from Lemma~\ref{lem_bin_tree} and Theorem~\ref{thm_count_types}.
\end{proof}

% \begin{theorem}
% The following are equivalent
% \begin{itemize}
% \item[1.] $T$ is stable;
% \item[2.] $|S(A)|\le|A|$ for some infinite cardinal $\lambda$, and all sets $A$ of cardinality $\le\lambda$;
% \item[3.] $|S(A)|\le|A|$ for every set $A$ such that $|L|<\cf(A)$.
% \end{itemize}
% \end{theorem}
% \begin{proof}
% \ssf{2}$\IMP$\ssf{1}.
% Suppose a formula $\phi({\mr x}\,;{\gr z})$ is unstable.
% Let $\<{\gr z_s}:s\in2^{<\lambda}\>$ be a sequence variables of length $|{\gr z}|$.
% Let $p({\gr z_s}:s\in2^{<\lambda})$ be the type that says that $\big\<{\gr z_s}:i<2^{<\lambda}\big\>$ witnesses a binary tree of height $\lambda$.
% As  $\phi({\mr x}\,;{\gr z})$ is unstable, $p$ is finitely consistent.
% As $\lambda$ is an arbitrary infinite cardinal, this contradicts \ssf{2}.

% \ssf{3}$\IMP$\ssf{2}.
% Trivial.

% \ssf{1}$\IMP$\ssf{3}.
% Proposition~\ref{thm_count_types} implies that $|S(A)|\le |A|^{|L|}$. Therefore, when $|L|<\cf(A)$, we obtain $|S(A)|\le |A|$.
% \end{proof}

\end{document}