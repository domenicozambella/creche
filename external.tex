% !TEX root = creche.tex
\documentclass[creche.tex]{subfiles}
\begin{document}
\chapter{Externally definable sets}
\label{external}

\def\medrel#1{\parbox[t]{6ex}{$\displaystyle\hfil #1$}}
\def\ceq#1#2#3{\parbox{25ex}{$\displaystyle #1$}\medrel{#2}$\displaystyle  #3$}




%%%%%%%%%%%%%%%%%%%%%%%%%%%%%
%%%%%%%%%%%%%%%%%%%%%%%%%%%%%
%%%%%%%%%%%%%%%%%%%%%%%%%%%%%
%%%%%%%%%%%%%%%%%%%%%%%%%%%%%
\section{Approximable sets}
\label{approximable}

\def\ceq#1#2#3{\parbox{25ex}{$\displaystyle #1$}\medrel{#2}$\displaystyle  #3$}

Let $\grC,\grD\subseteq\U^{|{\gr z}|}$.
The set $\grD\cap A^{|{\gr z}|}$ is called the \emph{trace\/} of $\grD$ over $A$.
We write $\grC=_A\grD$ if  $\grC$ and $\grD$ have the same trace on $A$.

We say that $\grD$ is \emph{externally definable\/} if it is of the form ${\gr\D_{p,\phi}}$ for some global type $p\in S_{\mr x}(\U)$ and some formula $\phi({\mr x}\,;{\gr z})\in L$, see Section~\hyperref[invariant_sets]{\ref*{invariant}.\ref*{invariant_sets}}. We may say that $\grD$ is externally definable \emph{by $p$}. Equivalently, a set $\grD$ is externally definable if it is the trace over $\U$ of a set which is definable in some elementary extension of $\U$. Precisely, $\grD$ is the trace on $\U$ of a set of the form $\phi({\mr b'}\,;{\gr\U'})$ where $\U'$ is elementary extension of $\U$ and ${\mr b'}\in\U'^{|{\mr x}|}$. The latter interpretation explains the terminology.

\noindent\llap{\textcolor{red}{\Large\danger}\kern1.5ex}We prefer to deal with external definability in a different, though equivalent, way.
This is not the most common approach.

\begin{definition}\label{def_epprox}
We say that $\grD$ is \emph{approximated\/} by the formula $\phi({\mr x}\,;{\gr z})$ if for every finite $B$ there is a ${\mr b}\in\U^{|{\mr x}|}$ such that $\phi({\mr b}\,;{\gr\U})=_B\grD$.
If in addition we have that $\phi({\mr b}\,;{\gr\U})\subseteq\grD$, we say that  $\grD$ is \emph{approximated from below}.
If  $\grD\subseteq\phi({\mr b}\,;{\gr\U})$ we say that  $\grD$ is \emph{approximated from above}.
We may call $\phi({\mr x}\,;{\gr z})$ the \emph{sort} of $\grD$.\QED
\end{definition} 
 
The following proposition is clear by compactness.

\begin{proposition}\label{lem_approx=external}
For every $\grD$ the following are equivalent
\begin{itemize}
\item[1.] $\grD$ is approximable;
\item[2.] $\grD$ is externally definable.\QED 
\end{itemize}
\end{proposition}

Approximability from below is an adaptation to our context of the notion of \textit{having an honest definition} in \cite{CS}.
\begin{definition}
We say that the global type $p\in S_{{\mr x}}(\U)$ is \emph{honestly definable\/} if for every $\phi({\mr x}\,;{\gr z})\in L$ the set ${\gr\D_{p,\phi}}$ is approximated from below (by some formula). We say that $p$ is a \emph{definable\/} if the sets ${\gr\D_{p,\phi}}$ are all definable (over $\U$).
\end{definition}

Note that definable implies honestly definable, but not vice versa.

\begin{example}
Every definable set is trivially approximable.
Clearly, only by the right formulas.
Let $T=T_{\rm dlo}$.
Then $\D=\{z\in\U:a\le z\le b\}$ is approximable both from below and from above by the formula $x_1<z<x_2$ though it is not definable by this formula.

Now, let $T=T_{\rm rg}$.
Then every $\D\subseteq\U$ is approximable and, when $\D$ has small infinite cardinality, it is approximable from above but not from below, see Exercise~\ref{ex_rg_small_def_set}.\QED
\end{example}

In Definition~\ref{def_epprox}, the sort $\phi({\mr x}\,;{\gr z})$ is fixed (otherwise any set would be approximable) but this requirement of uniformity may be dropped if we allow $B$ to have larger cardinality.

\begin{proposition}\label{lem_approx_nonunif}
For every $\grD$ the following are equivalent
\begin{itemize}
\item[1.] $\grD$ is approximable;
\item[2.] for every $C\subseteq\U$ of cardinality $\le|T|$ there is $\psi({\gr z})\in L(\U)$ such that $\psi({\gr\U})=_C\grD$.
\end{itemize}
Similarly, the following are equivalent:
\begin{itemize}
\item[3.] $\grD$ is approximable from below;
\item[4.]  for every ${\gr C}\subseteq\grD$ of cardinality $\le|T|$ there is $\psi({\gr z})\in L(\U)$ such that ${\gr C}\subseteq \psi({\gr\U})\subseteq\grD$.
\end{itemize}
\end{proposition}

\begin{proof}
To prove \ssf{2}$\IMP$\ssf{1} assume \ssf{2} and negate \ssf{1} for a contradiction.
For each formula $\psi({\mr x}\,;{\gr z})\in L$ choose a finite set $B$ such that $\psi({\mr b}\,;{\gr\U})\neq_B\grD$ for every ${\mr b}\in\U^{|{\mr x}|}$.
Let $C$ be the union of all these finite sets.
Clearly $|C|\le|T|$.
By \ssf{2} there are a formula $\phi({\mr x}\,;{\gr z})$ and a tuple ${\mr c}$ such that $\phi({\mr c}\,;{\gr\U})=_C\grD$, contradicting the definition of $C$.

Implication \ssf{1}$\IMP$\ssf{2} is obtained by compactness and the equivalence \ssf{3}$\IFF$\ssf{4} is proved similarly.
\end{proof}



\begin{remark}\label{prop_approx_el_eq}
Let $\grC\subseteq\U^{|{\gr z}|}$.
If $\grD$ is approximated by $\phi({\mr x}\,;{\gr z})$ then so is any $\grC$ such that $\grC\equiv\grD$, see Section~\hyperref[expansions]{\ref*{invariantL}.\ref*{expansions}} for the definition.
In fact, if the set $\grD$ is approximable by $\phi({\mr x}\,;{\gr z})$ then for every $n$

\hfil$\displaystyle\A {\gr z_1},\dots,{\gr z_n}\;\E {\mr x}\ \bigwedge^n_{i=1}\big[\phi({\mr x}\,;{\gr z_i})\ \iff\ {\gr z_i}\in{\gr \D}\big]$.


So the same holds for any $\grC\equiv\grD$.
A similar remark apply to approximability from below and from above.
For approximability from below, add the conjunct $\A {\gr z}\,\big[\phi({\mr x}\,;{\gr z})\imp {\gr z}\in\grD\big]$ to the formula above.
Similarly for approximability from above.\QED
\end{remark}

From the following easy observation of Chernikov and Simon~\cite{CS} we obtain an interesting quantifier elimination result.

% \begin{proposition}
% Let $\C\subseteq\U^{|{\gr z},w|}$ be approximated from below by the formula $\phi({\mr x}\,;{\gr z},w)$. Then $\grD=\big\{{\gr z}:\E w\ \big({\gr z}\,w\in\C\big)\big\}$ is approximated from below by the formula $\E w\,\phi({\mr x}\,;{\gr z}\,w)$.
% \end{proposition}
% 
% \begin{proof}
% Let $B\subseteq\U$ be finite. 
% We want ${\mr a}\in\U^{|{\mr x}|}$ such that 
% 
% \ceq{\ssf{a.}\hfill \E w\ \big({\gr b}\,w\in\C\big)}{\iff}{\E w\,\phi({\mr a}\,;{\gr b}\,w)}\hfill for every ${\gr b}\in B^{|{\gr z}|}$
% 
% \ceq{\ssf{b.}\hfill \A {\gr z}\;\Big[\E w\,\phi({\mr a}\,;{\gr z}\,w)}{\imp}{\E w\ \big({\gr z}\,w\in\C\big)\Big]}
% 
% Let $C\subseteq\U$ be a finite set such that 
% 
% \ceq{\ssf{c.}\hfill \E w\in C^{|w|}\ \big({\gr b}\,w\in\C\big)}{\iff}{\E w\ \big({\gr b}\,w\in\C\big)}\hfill for every ${\gr b}\in B^{|{\gr z}|}$
% 
% As $\C$ is approximable from below, there is an ${\mr a}$ such that
% 
% \ceq{\ssf{a'.}\hfill {\gr b}\,c\in\C}{\iff}{\phi({\mr a}\,;{\gr b}\,c)}\hfill for every ${\gr b}\,c\in \big(B\cup C\big)^{|{\gr z}\,w|}$
% 
% \ceq{\ssf{b'.}\hfill \A {\gr z}\,w\ \Big[\phi({\mr a}\,;{\gr z}\,w)}{\imp}{{\gr z}\,w\in\C\Big]}
% 
% We obtain \ssf{b} from \ssf{b'} simply by logic. 
% Implication $\imp$ in \ssf{a} follows from \ssf{a'} and \ssf{c}. 
% Implication $\pmi$ follows from \ssf{b}.
% \end{proof}

\begin{proposition}
Let $\C\subseteq\U^{|y\,{\gr z}|}$ be approximated from below by the formula $\phi({\mr x}\,;y\,{\gr z})$. Then $\grD=\big\{{\gr z}:\E y\ \big(y\,{\gr z}\in\C\big)\big\}$ is approximated from below by the formula $\E y\,\phi({\mr x}\,;y\,{\gr z})$.
\end{proposition}

\begin{proof}
\def\vl{\gr}
Let $B\subseteq\U$ be finite. 
We want ${\mr a}\in\U^{|{\mr x}|}$ such that 

\ceq{\ssf{a.}\hfill \E y\ \big(y\,{\vl b}\in\C\big)}{\iff}{\E y\,\phi({\mr a}\,;y\,{\vl b})}\hfill for every ${\vl b}\in B^{|{\vl z}|}$

\ceq{\ssf{b.}\hfill \A {\vl z}\;\Big[\E y\,\phi({\mr a}\,;y\,{\vl z})}{\imp}{\E y\ \big(y\,{\vl z}\in\C\big)\Big]}

Let $D\subseteq\U$ be a finite set such that 

\ceq{\ssf{c.}\hfill \E y\in D^{|y|}\ \big(y\,{\vl b}\in\C\big)}{\iff}{\E y\ \big(y\,{\vl b}\in\C\big)}\hfill for every ${\vl b}\in B^{|{\vl z}|}$

As $\C$ is approximable from below, there is an ${\mr a}$ such that

\ceq{\ssf{a'.}\hfill d\,{\vl b}\in\C}{\iff}{\phi({\mr a}\,;d\,{\vl b})}\hfill for every $d\,{\vl b}\in \big(D\cup B\big)^{|y\,{\vl z}|}$

\ceq{\ssf{b'.}\hfill \A y\,{\vl z}\ \Big[\phi({\mr a}\,;y\,{\vl z})}{\imp}{y\,{\vl z}\in\C\Big]}

We obtain \ssf{b} from \ssf{b'} simply by logic. 
Implication $\imp$ in \ssf{a} follows from \ssf{a'} and \ssf{c}. 
Implication $\pmi$ follows from \ssf{b}.
\end{proof}



\begin{corollary}
If $p\in S_{{\mr x}}(\U)$ is honestly definable then the family of sets externally definable by $p$ is closed under quantifiers and Boolean combinations. 
\end{corollary}

\begin{proof}
The sets externally definable by $p$ are always closed under Boolean operations. By the proposition above, they are closed under quantifiers. 
\end{proof}

% Let ${\gr z}=\<z_i:i<\lambda\>$ and let $\grD\subseteq\U^{|{\gr z}|}$. 
% For $I\subseteq\lambda$ we write $\E z_{\restriction I}\,\grD$ for the projection of $\grD$ to the $\lambda\sm I$ coordinates, i.e.\@ the set $\{b_{\restriction\lambda\sm I}: {\gr b}\in\grD\}$. 
% Expand the language with a symbol for $\E z_{\restriction I}\,\grD$ for every finite $I\subseteq\lambda$. 
% Then the theory of $\U$ in the expanded language has positive $\Delta$-quantifier elimination, for $\Delta=L\cup\big\{\E z_{\restriction I}\,\grD: I\subseteq\lambda\textrm{ finite}\big\}$, see Section~\hyperref[eliminazionequantificatoricriterio]{\ref*{eliminazione}.\ref*{eliminazionequantificatoricriterio}}.

%%%%%%%%%%%%%%%%%%%%%%%%%%%%%%%%%%%%%%%%%%%%%%%%%%%%%%%%%%%%%%%%%%%%%%%%%%%%%%%%%%%%%%%
\section{Ladder sequences and definability}

The attentive reader may have noticed that by \textit{formula\/} we often understand a pair $\phi(x)$ that consists of a formula and a tuple of variables.
In the following we deal with two sorts of variables: those that are placeholder for parameters, and those used to define a set.
To make things more complicated, here and there we shall invert the roles of the two sorts.

\begin{definition}\label{def_partitioned-fla}
A \emph{partitioned formula\/}  (strictly speaking, we should say a \emph{$2$-partitioned\/} formula) is a triple $\phi({\mr x}\,;{\gr z})$ consisting of a formula and two tuples of variables. The variables actually occurring in $\phi$ are all among ${\mr x},{\gr z}$. We write $\phi({\mr x}\,;{\gr z})^*$ for the partitioned formula where the order of the two tuples is inverted (i.e.\@ it is the opposite of what displayed).
\end{definition}

Let $\phi({\mr x}\,;{\gr z})\in L(\U)$ be a partitioned formula. We say that $\<{\mr a_i}\,;{\gr b_i} : i<\alpha\>$ is a \emph{ladder sequence\/} for $\phi({\mr x}\,;{\gr z})$ if 

\ceq{\#\hfill i< j}{\IFF}{\phantom{\neg}\phi({\mr a_i}\,;{\gr b_j})}\hfill for all $0\le i,j\le\alpha$

We say that $\phi({\mr x}\,;{\gr z})$ is \emph{stable\/} if all ladders have length $\alpha<\omega$. Otherwise we say it is \emph{unstable}. Note that by compactness the ladder sequences of a stable formula cannot have unbounded finite length. 

It is evident that $\phi({\mr x}\,;{\gr z})$ is stable if and only if $\phi({\mr x}\,;{\gr z})^*$ is stable.

Suppose $\phi({\mr x}\,;{\gr z})$ be unstable as witnessed by $\<{\mr a_i}\,;{\gr b_i} : i<\omega\>$. By padding redundant variables, we can read $\phi({\mr x}\,;{\gr z})$ as a formula $\phi({\mr x}\,{\gr z}\,;{\mr y}\,{\gr w})$. Then $\phi({\mr x}\,{\gr z}\,;{\mr y}\,{\gr w})$ defines the order of the sequence $\<{\mr a_i}\,{\gr b_i} : i<\omega\>$. For this reason it is also common to say that $\phi({\mr x}\,;{\gr z})$ has the \emph{order property}. It simply means that it is unstable. 

The following is arguably one of the most important fact about stable formulas.

\begin{theorem}\label{thm_def_stable_formula}
Any $\grD\subseteq\U^{|{\gr z}|}$ approximated by a stable formula is definable.
Precisely, if $\phi({\mr x}\,;{\gr z})$ is a stable formula that approximates $\grD$ then there are ${\mr a_{1,1}},\dots,{\mr a_{n,m}}\in\U^{|{\mr x}|}$ such that 

\ceq{\hfill {\gr z}\in\grD}{\iff}{\bigvee^n_{i=1}\bigwedge^m_{j=1}\phi({\mr a_{i,j}}\,;{\gr z})}
\end{theorem}

\begin{proof}
The theorem follows immediately from the the three lemmas below.
\end{proof}

\begin{remark}
Theorem~\ref{thm_def_stable_formula} is often stated in the following  apparently more general form.
For every $\Aa\subseteq\U$ there are ${\mr a_{1,1}},\dots,{\mr a_{n,m}}\in \Aa^{|{\mr x}|}$ such that

\ceq{\hfill {\gr z}\in\grD}{\iff}{\bigvee^n_{i=1}\bigwedge^m_{j=1}\phi({\mr a_{i,j}}\,;{\gr z})}\hfill for every ${\gr z}\in\Aa^{|{\gr z}|}$ 

Note that the equivalence above is restricted to ${\gr\Aa}$.

The proof is exactly the same, in fact elementarity and saturation are never used (to a great extent, no model theory is used either, just finite combinatorics).\QED
\end{remark}


\begin{lemma}
If $\grD$ is approximated from below by a stable formula $\phi({\mr x}\,;{\gr z})$ then there are ${\mr a_1},\dots,{\mr a_n}\in\U^{|{\mr x}|}$ such that 

\ceq{\hfill {\gr z}\in\grD}{\iff}{\bigvee^n_{i=1}\phi({\mr a_i}\,;{\gr z})}

\end{lemma}

\begin{proof}
Choose ${\mr a_n}\in\U^{|{\mr x}|}$ such that ${\gr b_0},\dots,{\gr b_{n-1}}\in\phi({\mr a_n}\,;{\gr\U})\subseteq\grD$.
Then, if possible, choose ${\gr b_n}$ such that

\ceq{\hfill{\gr b_n}}{\in}{\grD\sm\bigvee^n_{i=0}\phi({\mr a_i}\,;{\gr\U})}

and iterate the procedure.
By stability, the procedure has to stop as some $n$, yielding the required  ${\mr a_1},\dots,{\mr a_n}$.
\end{proof}


\begin{lemma}
If $\grD$ is approximated by a stable formula $\phi({\mr x}\,;{\gr z})$ with ladder order $m$.
Then the formula 

\ceq{\hfill\psi({\mr x_1},\dots,{\mr x_m}\,;{\gr z})}{=}{\bigwedge^m_{j=1}\phi({\mr x_j}\,;{\gr z})}

approximates $\grD$ from below.
\end{lemma}

\begin{proof}
Let $B\subseteq\grD$ be finite.
It suffices to prove that there are some ${\mr a_1},\dots,{\mr a_m}$ such that $B^{|{\gr z}|}\subseteq\psi({\mr a_1},\dots,{\mr a_m}\,;{\gr\U})\subseteq\grD$. Choose ${\mr a_n}\in\U^{|{\mr x}|}$ such that 

\hfil$B^{|{\gr z}|}\ \subseteq\ \phi({\mr a_n}\,;{\gr\U})\ \subseteq\ \U^{|{\gr z}|}\sm\{{\gr b_0},\dots,{\gr b_{n-1}}\}$.

Then, if possible, let ${\gr b_n}$ such that

\ceq{\hfill{\gr b_n}}{\in}{\bigwedge^n_{i=0}\phi({\mr a_i},{\gr\U})\sm\grD}

and iterate the procedure.
The procedure has to stop at some $n\le m$.
Hence the required parameters are ${\mr a_1},\dots,{\mr a_n}={\mr a_{n+1}}=\dots={\mr a_{m}}$.
\end{proof}

\begin{lemma}
If $\phi({\mr x}\,;{\gr z})$ is a stable formula then for every $m$ the formula 

\ceq{\hfill \psi({\mr x_1},\dots,{\mr x_m}\,;{\gr z})}{\iff}{\bigwedge^m_{j=1}\phi({\mr x_j}\,;{\gr z})}

is stable.
\end{lemma}

\begin{proof}
For legibility, we only prove that $\phi({\mr x_1}\,;{\gr z})\wedge\phi({\mr x_2}\,;{\gr z})$ is stable.
Suppose not and let ${\mr a^1_i},{\mr a^2_i}\in\U^{|{\mr x}|}$ and ${\gr b_i}\in\U^{|{\gr z}|}$ be such that 

\ceq{\hfill i\le j}{\IFF}{\phantom{\neg}\phi({\mr a^1_i}\,;{\gr b_j})\wedge\phi({\mr a^2_i}\,;{\gr b_j})}\hfill for all $i,j<\omega$

For $n=1,2$ let $H_n\subseteq\omega^{[2]}$ contain those pairs $j<i$ such that $\neg\phi({\mr a^n_i}\,;{\gr b_j})$. 
By the equivalence above $H_1\cup H_2=\omega^{[2]}$. 
By the Ramsey Theorem there is an infinite set $H$ such that $H^{[2]}\subseteq H_n$ for at least one of $n=1,2$. Suppose $H_1$ for definiteness. So, we obtain an infinite sequence $a^1_i$, $b_i$ such that

\ceq{\hfill j<i}{\IFF}{\neg\phi({\mr a^1_i}\,;{\gr b_j})}\hfill for all $i,j<\omega$

which contradicts the stability of $\phi({\mr x}\,;{\gr z})$.
\end{proof}

This last lemma concludes the proof of Theorem~\ref{thm_def_stable_formula}. 

The following is an important consequence.

\begin{corollary}
For every set $\D$ approximable by a stable formula the following are equivalent
\begin{itemize}
\item[1.] $\D$ is Lascar invariant over $A$;
\item[2.] $\D\in\acl^\eq(A)$.
\end{itemize}
\end{corollary}

\begin{proof}
Only \ssf{2}$\IMP$\ssf{3} requires a proof. By \ssf{1} the orbit of $\D$ over $A$ is small. Therefore, as $\D$ is definable by Theorem~\ref{thm_def_stable_formula}, its orbit is finite. Hence $\D\in\acl^\eq(A)$.
\end{proof}


\section{Stability and the number of types}

The following proposition highlight the connection between stability and the cardinality of types.

\begin{proposition}The following are equivalent
\begin{itemize}
\item[1.] $\phi({\mr x}\,;{\gr z})$ is a stable formula;
\item[2.] for is some countable set $A$, such that $\big|S_{\phi}(A)\big|=2^\omega$;
\item[2'.] for is some countable set $A$, such that $\big|S_{\phi}(A)\big|>\omega$;
\item[3.] there is binary tree of $\pmDelta$-formulas, for $\Delta=\{\phi({\mr x}\,;{\gr b}):{\gr b}\in\U^{|{\gr z}|}\}$, see Definition~\ref{def_tree_formulas}.
\end{itemize}
\end{proposition}

\def\ceq#1#2#3{\parbox[t]{15ex}{$\displaystyle #1$}\medrel{#2}$\displaystyle  #3$}

\begin{proof}\ssf{1}$\IMP$\ssf{2}\quad
Let $\I,<_\I$ be a dense linear order of cardinality $\kappa$. Note that we can choose $\I$ such that there are $2^\kappa$ Dedekind's cut in $\I$. Let ${\mr x_i}$ and ${\gr z_i}$, for $i\in\I$, be a tuples of variables. The type that says 

\ceq{\hfill i< j}{\IFF}{\phantom{\neg}\phi({\mr x_i}\,;{\gr z_j})}\hfill for all $i,j\in\I$

is finitely consistent. If ${\mr a_i}$, ${\gr b_i}$ is a realization, then for every Dedekind cut  $\C\subseteq\I$ the type that says

\ceq{\hfill i< j}{\IFF}{\phantom{\neg}\phi({\mr x}\,;{\gr b_j})\ \iff\ j\in\C}

is a finitely consistent and can be extended to some $p({\mr x})\in S_{\phi}(\U)$. Any two such types are mutually inconsistent and \ssf{3} follows.

\ssf{2}$\IMP$\ssf{3}\quad As in the proof of implication \ssf{3}$\IMP$\ssf{1} in~\ref{prop_small_equivalents}.

\ssf{4}$\IMP$\ssf{1} Fix an arbitrary $n$. For $i=0,\dots, n$ abbreviate $b_{1^i}$ by $b_i$.

\ceq{\hfill {\mr a_j}}{\models}{\displaystyle\bigwedge^{j-1}_{i=0}\phi({\mr x}\,;{\gr b_i})\wedge \neg\phi({\mr x}\,;{\gr b_j})}

Then 

\ceq{\hfill i<j}{\IFF}{\phi({\mr a_i}\,;{\gr b_j})}


We show that there is a ladder sequence of length $n$. For each $s\in2^n$ let $a_s$ be a solution of $\phi_s({\mr x})$. Then for all $r,s\in2^n$ the formula $\phi({\mr a_r}\,;{\gr b_s})$ holds exactly when $r<s$ in the lexicographic order.
\end{proof}


\section{Stable theories}

We say that $T$ is a \emph{stable theory\/} if every formula is stable.

The following is arguably one of the most important results that holds when the whole theory is stable. 

\begin{proposition} Let $T$ be stable and let $q({\mr x})\in S\big(\acl^\eq(A)\big)$. Then there is a unique global type $p({\mr x})\in S(\U)$ that is consistent with $q({\mr x})$ and Lascar invariant over $A$.
\end{proposition}

\begin{proof}
Let $p_i({\mr x})\in S(\U)$, for $i=1,2$, be two global types invariant over $A$ and consistent with $q({\mr x})$. To prove that $p_1({\mr x}) =p_2({\mr x})$, it suffices to show that for every formula $\phi({\mr x}\,;{\gr z})\in L$

\ceq{\#\hfill {\gr\D_{p_1,\phi}}}{=}{{\gr\D_{p_2,\phi}}}

Note that, by Theorem~\ref{thm_def_stable_formula} both sets are definable, hence in $\U^\eq$.

Let $p'_i({\mr x})\in S(\U^\eq)$ denote the unique extension of $p_i({\mr x})$. Clearly the formulas $\A{\gr z}\,\big[\phi({\mr x}\,;{\gr z})\iff {\gr z}\in {\gr\D_{p_i,\phi}}\big]$ belong to $p'_i({\mr x})$ respectively. 

As $p_i({\mr x})$ are invariant over $A$, so are the sets ${\gr\D_{p_i,\phi}}$. Hence they belong to $\acl^\eq(A)$. Then both formulas $\A{\gr z}\,\big[\phi({\mr x}\,;{\gr z})\iff {\gr z}\in {\gr\D_{p_i,\phi}}\big]$ belong to $q({\mr x})$ and $\#$ follows.
\end{proof}


\begin{corollary}
If $T$ is stable then the following are equivalent
\begin{itemize}
\item[1.] $a\equivL_A b$, see Definition~\ref{def_Lascar_type};
\item[2.] $a\equivSh_A b$, see Definition~\ref{def_Sh_strong_type}.
\end{itemize}
\end{corollary}
\begin{proof}\ssf{1}$\IMP$\ssf{2}\quad This clearly holds for every theory $T$.

\ssf{2}$\IMP$\ssf{1}\quad Exercise.%It suffices to show that if $a\equivSh_A b$ then $a\equiv_M b$ for some model $M$ containing $A$.
\end{proof}

\begin{proposition} The following are equivalent
\begin{itemize}
\item[1.] $T$ is stable;
\item[2.] every sequence of $A$-indiscernible is totally $A$-indiscernible.
\end{itemize}
\end{proposition}

\begin{proof}
Negate \ssf{1} and let $\phi({\mr x}\,;{\gr z})$ be an unstable formula witnessed by the ladder sequence $\<{\mr a_i}\,;{\gr b_i}:i<\omega\>$. Let $\<{\mr a'_i}\,;{\gr b'_i} : i<\omega\>$ be 
indiscernible sequence with the same Ehren\-feucht-Mostowski type. Then $\phi({\mr a'_i}\,;{\gr b'_j})$ if and if $i< j$. 

Conversely, assume that $\<a_i : i<\omega\>$ which is $A$-indiscernible, but not totally. Then there is a formula $\phi(x, y)\in L(A)$ and some $i<j$ such that $\phi(a_i , a_j )\wedge\neg\phi(a_j , a_i ) $. By indiscernibility  $\phi(a_i , a_j )$ holds if and only if $i \le j$. Hence $\phi(x, y)$ is not stable.
\end{proof}


\section{Notes and references}
\begin{biblist}[]\normalsize



\bib{CS}{article}{
   author={Chernikov, Artem},
   author={Simon, Pierre},
   title={Externally definable sets and dependent pairs},
   status={\href{https://arxiv.org/abs/1007.4468}{arXiv:1007.4468}},
   journal={Israel J. Math.},
   volume={194},
   date={2013},
   number={1},
   pages={409--425},
   %note={},
   %issn={0021-2172},
   %doi={10.1007/s11856-012-0061-9},
}

\bib{TZ}{book}{
   author={Tent, Katrin},
   author={Ziegler, Martin},
   title={A course in model theory},
   series={Lecture Notes in Logic},
   volume={40},
   publisher={Association for Symbolic Logic, Cambridge University Press},
   date={2012},
   pages={x+248},
   %isbn={978-0-521-76324-0},
   %doi={10.1017/CBO9781139015417},
}



\end{biblist}


\end{document}


