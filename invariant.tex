% !TEX root = creche.tex
\documentclass[creche.tex]{subfiles}
\begin{document}

\chapter{Invariant sets}
\label{invariant}

\def\medrel#1{\parbox[t]{6ex}{$\displaystyle\hfil #1$}}
\def\ceq#1#2#3{\parbox[t]{18ex}{$\displaystyle #1$}\medrel{#2}{$\displaystyle #3$}}

In this chapter we fix a signature $L$, a complete theory $T$ without finite models, and a saturated model $\U$ of inaccessible cardinality $\kappa$ strictly larger than $|L|$.
Notation and implicit assumptions are as presented in Section~\hyperref[monster]{\ref*{saturation}.\ref*{monster}}.

%%%%%%%%%%%%%%%%%%%%%%%%%%%%%%%%%%%%
%%%%%%%%%%%%%%%%%%%%%%%%%%%%%%%%%%%%
%%%%%%%%%%%%%%%%%%%%%%%%%%%%%%%%%%%%
\section{Invariant sets and types}\label{invariant_sets}

Let $\grD\subseteq\U^{|{\gr z}|}$, where ${\gr z}$ is a tuple of length $<\kappa$.
We say that $\grD$ is an \emph{$A$-invariant set} if it is fixed (set-wise) by $A$-automorphisms.
That is, $f[\grD]=\grD$ for every automorphism $f\in\Aut(\U/A)$ or, yet in other words,

\ceq{\ssf{is1.}\hfill {\gr a}\in\grD}{\iff}{f{\gr a}\in\grD}\hfill for every ${\gr a}\in\U^{|{\gr z}|}$ and every $f\in\Aut(\U/A)$,

which, by homogeneity is equivalent to,

\ceq{\ssf{is2.}\hfill {\gr a}\in\grD}{\iff}{{\gr b}\in\grD}\hfill for all ${\gr a},{\gr b}\in\U^{|{\gr z}|}$ such that ${\gr a}\equiv_A{\gr b}$.

This yields the following bound on the number of invariant sets

\begin{proposition}\label{prop_numberIS}
Let $\lambda=|L_{\gr z}(A)|$.
There are at most $2^{2^{\lambda}}$ sets $\grD\subseteq\U^{|{\gr z}|}$ that are invariant over $A$.
\end{proposition}
\begin{proof}
By \ssf{is2}, sets that are invariant over $A$ are union of equivalence classes of the relation $\equiv_A$, that is, union of sets of the form $p({\gr\U})$ where $p({\gr z})\in S(A)$.
Then the number of $A\jj$invariant sets is $2^{|S_{\gr z}(A)|}$.
Clearly $|S_{\gr z}(A)|\le 2^\lambda$.
\end{proof} 

We say that $\grD$ is an \emph{invariant set\/} if it is invariant over some (small) set $A$.
As we require $\kappa$ to be inaccessible, there are exactly $\kappa$ invariant sets.

In this chapter we work with $\Delta$-types where $\Delta$ may be either $L(\Aa)$ or the set of Boolean combinations of $\phi({\mr x}\,;{\gr b})$ for ${\gr b}\in\Aa^{|{\gr z}|}$ and some fixed $\phi({\mr x}\,;{\gr z})\in L$.
In the latter case  $\Delta$-types are called a \emph{$\phi$-types}.
We denote by \emph{$S_{\phi}(\Aa)$\/} the set of complete $\phi$-types with parameters in $\Aa$.
Typically, $\Aa$ is either the whole of $\U$ or some small set $A\subseteq\U$.
Types in \emph{$S_{\phi}(\U)$\/} are called \emph{global $\phi$-types}.

% \begin{proposition}
% Suppose $M$ realizes every consistent type $p({\gr z})\subseteq L(A)$.
% Let $D\subseteq M^{|{\gr z}|}$ be such that \ssf{is2} holds for $D$ when restricted to ${\gr a},{\gr b}\in M^{|{\gr z}|}$.
% Then $D$ is the trace on $M$ of some unique $A\jj$invariant set $\grD$.
% \end{proposition}
% 
% \begin{proof}
% The set $\grD$ is the union, for $p({\gr z})\in S(A)$, of the sets $p(\U)$ that intersect $D$.
% \end{proof}



Let $p({\mr x})\subseteq L(\U)$ be a consistent type.
For every formula $\phi({\mr x}\,;{\gr z})\in L$ we define

\ceq{\hfill\emph{$\gr\D_{p,\phi}$}}{=}{\Big\{{\gr a}\in\U^{|{\gr z}|}\ :\ \phi({\mr x}\,;{\gr a})\in p\Big\}}

We can read the notation in two ways.
Either the tuple ${\gr z}$ has infinite length and is the same for all formulas, or it is finite and depends on $\phi$.
This is possible because adding or erasing dummy variables to the second tuple of $\phi({\mr x}\,;{\gr z})$ does not change ${\gr\D_{p,\phi}}$ in any relevant way, in particular invariance is preserved.

Let $p({\mr x})\subseteq L(\U)$ be a consistent type.
We say that $p({\mr x})$ is an \emph{$A$-invariant type\/}, if for every formula $\phi({\mr x}\,;{\gr z})\in L$, 

\ceq{\ssf{it1.}\hfill \phi({\mr x}\,;{\gr a})\in p}{\IFF}{\phi({\mr x}\,;f{\gr a})\in p}\hfill for every ${\gr a}\in\U^{|{\gr z}|}$ and every $f\in\Aut(\U/A)$.

By \emph{invariant type\/} we mean invariant over some (small) set $A$.

A global $\phi$-type can be identified with ${\gr\D_{p,\phi}}$ and a global type can be identified with the collection of all these sets.
The notions of invariance for types and sets coincide.

We say that the type $p({\mr x})\subseteq L(\U)$ \emph{does not split\/} over $A$ if

\ceq{\ssf{it2.}\hfill  {\gr a}\equiv_A{\gr b}}{\IMP}{\Big(\phi({\mr x}\,;{\gr a})\in p\,\IFF\,\phi({\mr x}\,;{\gr b})\in p\Big)}\hfill for all ${\gr a},{\gr b}\in\U^{|{\gr z}|}$

for every formula $\phi({\mr x}\,;{\gr z})\in L$.
For global type (or a global $\phi$-types  when $\phi$ is fixed) \ssf{it2} is equivalent to

\ceq{\ssf{it2'.}\hfill  {\gr a}\equiv_A{\gr b}}{\IMP}{\phi({\mr x}\,;{\gr a})\iff\phi({\mr x}\,;{\gr b})\,\in\, p}

By homogeneity, non splitting is equivalent to invariance.

The following is yet another important equivalent of invariance over $A$ that easily follows from \ssf{it2'}

\ceq{\ssf{it3.}\hfill {\gr a}\equiv_A{\gr b}}{\IMP}{{\gr a}\equiv_{A,{\mr c}}{\gr b}}\hfill for all ${\gr a},{\gr b}\in\U^{|{\gr z}|}$ and for all ${\mr c}\models p_{\restriction A,{\gr a},{\gr b}}$.

Note that \ssf{it3} applies to global types $p({\mr x})\in S(\U)$ but not to $\phi$-types.

%%%%%%%%%%%%%%%%%%%%%%
%%%%%%%%%%%%%%%%%%%%%%
%%%%%%%%%%%%%%%%%%%%%%
%%%%%%%%%%%%%%%%%%%%%%
\section{Invariance from the dual perspective}

\noindent\llap{\textcolor{red}{\Large\danger}\kern1.5ex}The following terminology in not standard.
We say that the set $\mrB\subseteq\U^{|{\mr x}|}$, typically a definable set, is \emph{quasi-invariant\/} over $A$ if for any finitely many automorphisms $f_1,\dots,f_n\in\Aut(\U/A)$ the sets $f_i[\mrB]$ have non-empty intersection.

We say that the type $p({\mr x})\subseteq L(\U)$, typically a global type, is \emph{quasi-invariant\/} over $A$ if $\phi({\mr\U})$ is quasi-invariant over $A$  for every $\phi({\mr x})$ conjunction of formulas in $p({\mr x})$.
For types of small cardinality, this is equivalent to requiring that $p({\mr\U})$ is quasi-invariant.% So, the type $p({\mr x})$ is quasi-invariant if the set $p({\mr\V})$ is invariant, where $\V$ is an large saturated extension of $\U$, see Exercise~\ref{ex_quasi_inv_type}.

For global types quasi-invariance coincides with invariance, in fact we have the following.


\begin{proposition}\label{prop_invariance-quasi-invariance}
Let $p({\mr x})\in S_{\phi}(\U)$ be a global $\phi$-type, then the following are equivalent
\begin{itemize}
\item[1.] $p({\mr x})$ is invariant over $A$;
\item[2.] $p({\mr x})$ is quasi-invariant over $A$.
\end{itemize}
\end{proposition}
\begin{proof}
\ssf{1}$\IMP$\ssf{2}.
Assume $p({\mr x})$ is invariant and let $\psi({\mr x}\,;{\gr b})\,\in\, p$ and some ${\gr b}\in\U^{|{\gr z}|}$.
Then $\psi({\mr x}\,;f{\gr b})\,\in\, p$ for every $f\in\Aut(\U/A)$, so \ssf{2} follows from the finite consistency of $p({\mr x})$.

\ssf{2}$\IMP$\ssf{1}.
Assume $p({\mr x})$ is not invariant.
Then there is ${\gr b}\in\U^{|{\gr z}|}$ such that $\phi({\mr x}\,;{\gr b})\in p$ and $\phi({\mr x}\,;f{\gr b})\notin p$ for some  $f\in\Aut(\U/A)$.
By completeness, $p({\mr x})$ contains the formula $\phi({\mr x}\,;{\gr b})\wedge\neg\phi({\mr x}\,;f{\gr b})$ which clearly is not consistent with its $f$-translate.
\end{proof}

\begin{exercise}
We say that $\mrB$ \textit{strongly\/} quasi-invariant if for every definable set $\mrD$ at least one of $\mrB\cap\mrD$ and $\mrB\cap\neg\mrD$ is quasi-invariant.
Strongly quasi invariant types are defined in analogy to the definition of quasi-invariant type.
Note parenthetically that for global types the two notions coincide.
Prove that every strongly quasi-invariant type has an extension to a global invariant type.

Hint: it may help to prove that if $\mrB$ is strongly quasi-invariant then for every definable $\mrD$ either $\mrB\cap\mrD$ or $\mrB\cap\neg\mrD$ is strongly quasi-invariant.
Then the maximal strongly quasi-invariant set is the required global extension.\QED
\end{exercise}

%Quasi-invariance makes sense also for incomplete types.
Unfortunately it is not true in general that every quasi-invariant type can be extended to a global one.
In next section we consider a stronger notion, \textit{coheirs}, which has this desirable property, see Proposition~\ref{prop_exisntence_coheirs}.
In the next chapters we introduce a weaker notion, \textit{non-forking}, that also allows to extend types to global types.

\begin{comment}
We say that $\mrB$ is \emph{drifting over $A$\/} if for every finitely many $f_1,\dots,f_n\in\Aut(\U/A)$ there is a $g\in\Aut(\U/A)$ such that $g[\mrB]$ is disjoint from all $f_i[\mrB]$.
%We say that  $\mrB$ is \emph{piecewise drifting\/} if it is the union of finitely many drifting sets.

\begin{lemma}\label{lem_newelski}
A quasi-invariant set is not the union of finitely many drifting sets.
\end{lemma}

\begin{proof}
% \def\medrel#1{\parbox[t]{6ex}{$\displaystyle\hfil #1$}}
% \def\ceq#1#2#3{\noindent\parbox[t]{30ex}{$\displaystyle #1$}\medrel{#2}{$\displaystyle #3$}}

It is convenient to prove an apparently more general claim.
If ${\mr\B_1},\dots,{\mr\B_n}$ are all drifting and $\Ll$ is such that for some finite $F\subseteq G$\medskip 

\ceq{\sharp\hfill\Ll}{\subseteq}{\bigcup_{f\in F}f[{\mr\B_1}\cup\dots\cup{\mr\B_n}],} 

then ${\mr\Ll}$ is not quasi-invariant.
The claim is vacuously true for $n=0$.
Now, assume $n$ is positive and that the claim holds for $n-1$.
Define $\mrC={\mr\B_1}\cup\dots\cup{\mr\B_{n-1}}$ and rewrite $\sharp$ as follows

\ceq{\hfill{\mr\Ll}}{\subseteq}{\bigcup_{f\in F}f[\mrC]\ \cup\ \bigcup_{h\in F}h[{\mr\B_n}]}

Since ${\mr\B_n}$ is drifting there is a $g\in G$ such that $g[{\mr\B_n}]$ is disjoint from $h[{\mr\B_n}]$ for every $h\in F$, which implies that\medskip 

\ceq{\hfill{\mr\Ll}\cap g[{\mr\B_n}]}{\subseteq}{\bigcup_{f\in F}f[\mrC].} 

Hence for every $h$ there holds\medskip 

\ceq{\hfill h g^{-1}[{\mr\Ll}]\cap h[{\mr\B_n}]}{\subseteq}{\bigcup_{f\in F}hg^{-1}f[\mrC]} 

So, from $\sharp$ we obtain\medskip 

\ceq{\hfill{\mr\Ll}\cap\bigcap_{h\in F}h g^{-1}[{\mr\Ll}]}{\subseteq}{\bigcup_{f\in F}f[\mrC]\ \cup\ \bigcup_{h\in F}\bigcup_{f\in F}hg^{-1}f[\mrC].}\smallskip


\noindent By the induction hypothesis, the set on the r.h.s.\@ is not quasi-invariant.
Hence neither is ${\mr\Ll}$, proving the claim and with it the lemma.
\end{proof}

Every non piecewise drifting type extends to a non drifting global type.

\label{coheirs}
\end{comment}

\section{Heirs and coheirs}

The easiest way to obtain types that are invariant over a set $A$ is via types that are finitely satisfiable in $A$.
We say that a type $p({\mr x})$ is  \emph{finitely satisfiable\/} in $A$ if every conjunction of formulas in $p({\mr x})$ has a solution in $A^{|{\mr x}|}$.

\begin{proposition}\label{prop_coeredi_quasiinvarienti}
Every type $p({\mr x})\subseteq L(\U)$ finitely satisfied in $A$ is quasi-invariant over $A$.

\end{proposition}

\begin{proof}
Clearly, the very same ${\mr a}\in A^{|{\mr x}|}$ that satisfies $\phi({\mr x})$ satisfies also every $\Aut(\U/A)$-translates of $\phi({\mr x})$.
\end{proof}

The proposition above and Propositions~\ref{prop_invariance-quasi-invariance} yield the following

\begin{proposition}\label{prop_coeredi_invarienti}
Let $p({\mr x})\in S_{\phi}(\U)$ be a global $\phi$-type finitely satisfiable in $A$.
Then $p({\mr x})$ is $A\jj$in\-vari\-ant.\QED
\end{proposition}

\begin{proposition}\label{prop_exisntence_coheirs}
Every type $q({\mr x})\subseteq L(\U)$ finitely satisfiable in $A$ has an extension to a global type finitely satisfiable in $A$.
\end{proposition}

\begin{proof} 
Let $p({\mr x})\subseteq L(\U)$ be maximal among the types containing $q({\mr x})$ and finitely satisfiable in $A$.
We prove that $p({\mr x})$ is complete.
If for a contradiction that $p({\mr x})$ contains neither $\psi({\mr x})$ nor $\neg\psi({\mr x})$.
Then  neither $p({\mr x})\cup\big\{\psi({\mr x})\big\}$ nor $p({\mr x})\cup\big\{\neg\psi({\mr x})\big\}$ is finitely satisfiable in $A$.
This contradicts the finite satisfiability of $p({\mr x})$.
\end{proof}

Note that by elementarity every type over a model is finitely satisfiable in the model, that is, if $p({\mr x})\subseteq L(M)$ for some model $M$, then $p({\mr x})$ is finitely satisfiable in $M$.
For this reason the notion is mainly applied to models.

If $p({\mr x})$ is a type finitely satisfied in some set $A$, then any extension of $p({\mr x})$ that is also finitely satisfied in $A$ is a \emph{coheir\/} of $p({\mr x})$.

In many cases it is useful to focus on elements instead of types.
We introduce the following notation to express that $\tp({\mr b}/A,{\gr a})$ is finitely satisfied in $A$

\begin{definition}\label{def_coheir_idepencence} For every ${\gr a}\in\U^{|{\gr z}|}$ and ${\mr b}\in\U^{|{\mr x}|}$ we define

\noindent\llap{\textcolor{red}{\Large\danger}\kern1.5ex}%
\ceq{\hfill\emph{${\gr a}\nonforkc_A{\mr b}$}}{\IFF}{\phi({\gr a}\,;{\mr b}) \textrm{ for all }\phi({\gr z}\,;{\mr x})\in L(A)\textrm{ such that }A^{|{\mr x}|}\subseteq\phi({\gr a}\,;{\mr \U})}

\ceq{\hfill}{\IFF}{\phi({\gr a}\,;{\mr\U})\cap A^{|{\mr x}|}\neq\0 \textrm{ for all }\phi({\gr z}\,;{\mr x})\in L(A)\textrm{ such that }\phi({\gr a}\,;{\mr b})}

We call this the \emph{heir/coheir} relation.
The symbol $\nonforkc$ is not standard.
More properly one says that $\tp({\gr a}/A,{\mr b})$ is the heir of $\tp({\gr a}/A)$ or that $\tp({\mr b}/A,{\gr a})$ is the coheir of $\tp({\mr b}/A)$.

We also write

\ceq{\hfill\emph{${\gr a}\nonforkc_A{\mr x}$}}{=}{\Big\{\phi({\gr a}\,;{\mr x})\ :\ \phi({\gr z}\,;{\mr x})\in L(A)\textrm{ and }A\subseteq\phi({\gr a}\,;{\mr \U})\Big\}}

and we will use the symbol \emph{${\gr a}\nonforkc_A{\mr x}\equiv_A{\mr b}$} with the obvious meaning.\QED
\end{definition}

We imagine ${\gr a}\nonforkc_A{\mr b}$ as saying that ${\mr b}$ is \emph{independent\/} of ${\gr a}$ over $A$.
This is the strongest form of independence we will consider.
In general it is not symmetric, that is ${\gr a}\nonforkc_A{\mr b}$ is not always equivalent to ${\mr b}\nonforkc_A{\gr a}$ (symmetry being equivalent to stability).

We shall use, sometimes without reference, the following easy lemma

\begin{lemma}\label{lem_coheir_independence}
The following properties hold for all $A,{\gr a},{\mr b}$, and $c$
\begin{itemize}
\item[1.] ${\gr a}\nonforkc_A{\mr b}\ \ \IMP\ \ f{\gr a}\nonforkc_Af{\mr b}$ \ \ for every $f\in\Aut(\U/A)$\hfill \textit{invariance}
\item[2.] ${\gr a}\nonforkc_A{\mr b}\ \ \IFF\ \ {\gr a_0}\nonforkc_{A_0}{\mr b_0}$\ for all finite $A_0\subseteq A, {\gr a_0}\subseteq{\gr a}, {\mr b_0}\subseteq{\mr b}$\hfill \textit{finite character}
% \item[3$\,'$\!.] ${\gr a},c\nonforkc_A{\mr b}\ \ \IMP\ \ {\gr a}\nonforkc_{A,\,c}{\mr b}$\hfill \textit{transitivity$'$}
% \item[3$\,''$\!.] ${\gr a}\nonforkc_{A,\,c}{\mr b}$ \ and \ ${\gr a}\nonforkc_Ac\ \ \IMP\ \ {\gr a}\nonforkc_Ac,{\mr b}$\hfill \textit{transitivity$''$}
\item[3.] ${\gr a},c\nonforkc_{A}{\mr b}$ \ and \ ${\gr a}\nonforkc_Ac\ \ \IMP\ \ {\gr a}\nonforkc_Ac,{\mr b}$\hfill \textit{right transitivity}
\item[4.] ${\gr a}\nonforkc_A{\mr b}\ \ \IMP\ \ $ there exists ${\mr b'}\equiv_{A,\,{\gr a}}{\mr b}$ such that ${\gr a},c\nonforkc_A{\mr b'}$\hfill \textit{right extension}

\end{itemize}
\end{lemma}
\begin{proof}
Properties \ssf{1}-\ssf{3} follow immediately from Definition~\ref{def_coheir_idepencence}.
We prove \ssf{4}.
Let $p({\mr x})$ be a global coheir of $\tp({\mr b}/A,{\gr a})$, which exists by Proposition~\ref{prop_exisntence_coheirs}.
Then any ${\mr b'}\models p_{\restriction A,\,{\gr a},\,c}({\mr x})$ proves the lemma.
\end{proof}


Transitivity has dual version that holds under stronger assumptions.


The type ${\gr a}\nonforkc_A{\mr x}\equiv_A{\mr b}$ defined above is the intersection of all coheirs of $\tp({\mr b}/A)$.
In general coheirs of $\tp({\mr b}/A)$ may not exist i.e.,  ${\gr a}\nonforkc_A{\mr x}\equiv_A{\mr b}$ is inconsistent.
However, it is always consistent when $A$ is a model.
Even when consistent it need not be a complete.
In fact, completeness  of the type ${\gr a}\nonforkc_A{\mr x}\equiv_A{\mr b}$ is a very strong property.
An example is given in Lemma~\ref{lem_Hindman}.

\begin{definition}\label{def_coheir_stationary}We say that $\nonforkc_A$ is \emph{stationary\/} if ${\gr a}\nonforkc_A{\mr x}\equiv_A{\mr b}$ is a complete type (over $A,{\gr a}$) for every ${\gr a}\in\U^{<\omega}$, every ${\mr b}\in\U^{|{\mr x}|}$, and every tuple of variables ${\mr x}$.
We say \emph{$n$-stationary\/} if this is restricted to $|{\mr x}|=n$.\QED
\end{definition}

Stationarity of $\nonforkc_A$ is often ensured by the following property that will receive the due attention in Section~\hyperref[stable_teories]{\ref*{external}.\ref*{stable_teories}}.

\begin{proposition}
Fix a tuple of variable ${\mr x}$ of length $n$.
If for every $\phi({\mr x})\in L(\U)$ there is a formula $\psi({\mr x})\in L(A)$ such that $\phi(\U)=_A\psi(\U)$ then $\nonforkc_A$ is $n$-stationary.
\end{proposition}

\begin{proof}
Fix ${\gr a}\in\U^{|{\gr z}|}$ and  ${\mr b_1},{\mr b_2}\in\U^{|{\mr x}|}$ such that ${\gr a}\nonforkc_A{\mr b_i}$ and ${\mr b_1}\equiv_A{\mr b_2}$.
We claim that ${\mr b_1}\equiv_{A,\,{\gr a}}{\mr b_2}$.
We need to prove that $\phi({\gr a},\,{\mr b_1})\iff\phi({\gr a},\,{\mr b_2})$ for every  $\phi({\gr z},\,{\mr x})\in L(A)$.
Let $\psi({\mr x})\in L(A)$ be such that $\phi({\gr a},\,{\mr \U})=_A\psi({\mr \U})$.
From ${\gr a}\nonforkc_A{\mr b_i}$ we obtain that  $\phi({\gr a},\,{\mr b_i})\iff\psi({\mr b_i})$.
Finally, the claim follows because ${\mr b_1}\equiv_A{\mr b_2}$.
\end{proof}

% \begin{lemma}\label{lem_coheirext}
% If ${\gr a}\nonforkc_A{\mr c}$ then for every ${\gr b}$ there is ${\mr c'}$ such that ${\gr a},\,{\gr b}\nonforkc_A{\mr c'}\equiv_{A,\,{\gr a}}{\mr c}$.
% \end{lemma}

% \begin{exercise}.
% Prove that every invariant type $q({\mr x})\subseteq L(\U)$ can be extended to a global invariant type $p({\mr x})\in S(\U)$.(?)\QED
% \end{exercise}

% We say that $p({\mr x})\in S(\U)$ is a \emph{global heir\/} of $p_{\restriction M}({\mr x})$ if  
% 
% \ceq{\hfill{\gr\D_{p,\phi}}\cap M^{|{\gr z}|}\neq\0}{\IFF}{{\gr\D_{p,\phi}}\neq\0}  for every formula $\phi({\mr x}\,;{\gr z})\in L$.

%Every type over a model has an extension to a global coheir. 
%A satisfactory generalization of this notions to types over sets should guarantee this existence.
%Unfortunately, this is not possible in general and the definition Section~\hyperref[coheirs_sets]{\ref*{invariantL}.\ref*{coheirs_sets}} is as good as it gets.

\begin{remark}\label{rk_coheir_stationary}
Stationarity of $\nonforkc_A$ over every set $A$, or just over every model, is equivalent to the stability of $T$, see Section~\hyperref[stable_teories]{\ref*{external}.\ref*{stable_teories}}.
However, there are many more theories where the assumption holds for some particular set $A$.
For example, if every subset of $A^n$ is the trace of a definable set, $\nonforkc_A$ is clearly $n$-stationary.
This simple observation will be of help in the proof of Theorem~\ref{thm_Hindman}.
For natural example let $T=T_{\rm dlo}$ and let $A\subseteq\U$ have the order-type of $\RR$.
By quantifier elimination every definable of $\U$ union of finitely many intervals.
By Dedekind completeness, the trace on $A$ of any interval of $\U$ coincides with that of an $A$-definable interval. 
\end{remark}

%%%%%%%%%%%%%%%%%%%%%%%%%%%%%%
%%%%%%%%%%%%%%%%%%%%%%%%%%%%%%
%%%%%%%%%%%%%%%%%%%%%%%%%%%%%%
%%%%%%%%%%%%%%%%%%%%%%%%%%%%%%
%%%%%%%%%%%%%%%%%%%%%%%%%%%%%%
\section{Morley sequences and indiscernibles}


In the following $\alpha$ is some ordinal $\le\kappa$, typically $\omega$, and ${\mr x}$ is a tuple of variables of length $<\kappa$.
Let \mbox{$p({\mr x})\in S(\U)$} be a global type.
We say that ${\mr\bar c}=\<{\mr c_i}:i<\alpha\>$ is a \emph{Morley sequence\/} of $p({\mr x})$ over $A$ if for every $i<\alpha$

\ceq{\ssf{Ms.}\hfill {\mr c_i}}{\models}{p_{\restriction A,\,{\mr c_{\restriction i}}}({\mr x})}.

We will usually require that $p({\mr x})$ is invariant over $A$.
In particular, when $p({\mr x})$ is finitely satisfiable in $A$, we say that ${\mr\bar c}$ is a \emph{coheir sequence\/} over $A$.

The following is a convenient characterization of coheir sequences.

\begin{lemma}\label{lem_coheir_property}
The following are equivalent
\begin{itemize}
\item[1.] ${\mr\bar c}=\<{\mr c_n}:n<\alpha\>$ is a coheir sequence over $A$;
\item[2.] ${\mr c_{\restriction n}}\nonforkc_A{\mr c_n}$ and ${\mr c_{n+1}}\equiv_{A,\,{\mr c_{\restriction n}}}{\mr c_n}$ per ogni $n<\alpha$.
\end{itemize}
\end{lemma}

\begin{proof}
\ssf{1}$\IMP$\ssf{2}.
Assume \ssf{1} and let $p({\mr x})\in S(\U)$ be a global type finitely satisfiable in $A$ and such that ${\mr c_i}\models p_{\restriction A,{\mr c_{\restriction i}}}({\mr x})$.
The requirement ${\mr c_{n+1}}\equiv_{A,{\mr c_{\restriction n}}} {\mr c_n}$ is clear.
Now, suppose $\phi({\mr c_{n+1}})$ for some $\phi({\mr x})\in L(A,{\mr c_{\restriction n+1}})$.
Then $\phi({\mr x})$ belongs to $p({\mr x})$ hence $\phi({\mr\U})\cap A^{|{\mr x}|}\neq\0$ because $p({\mr x})$ is finitely satisfiable in $A$.

\ssf{2}$\IMP$\ssf{1}.
Assume \ssf{2}.
Then it is immediate that ${\mr\bar c}$ is a coheir sequence of any global type that extends ${\mr\bar c}\nonforkc_A{\mr x}$.
\end{proof}

Let $I,<_I$ be a linear order.
We call a function ${\mr\bar a}:I\to\U^{|{\mr x}|}$ an \emph{$I$-sequence}, or simply a \emph{sequence\/} when $I$ is clear.
Often we introduce an $I\jj$sequence by writing ${\mr\bar a}=\<{\mr a_i}: i\in I\>$.


If $I_0\subseteq I$ we call ${\mr a_{\restriction I_0}}$ a \emph{subsequence\/} of ${\mr\bar a}$.
Most relevant are the subsets $I_0\subseteq I$ that are well-ordered by $<_I$ and, in particular, the finite ones.
When $I_0$ has order-type $\alpha$, an ordinal, we identify ${\mr a_{\restriction I_0}}$ with a tuple of length $\alpha$.
 
\begin{definition}
Let $I,<_I$ be an infinite linear order and let ${\mr\bar a}$ be an $I\jj$sequence.
We say that $a$ is a \emph{sequence of indiscernibles\/} over $A$ or, a sequence of \emph{$A\jj$indiscernibles}, if ${\mr a_{\restriction I_0}}\equiv_A {\mr a_{\restriction I_1}}$ for every $I_0,I_1\in [I]^n$ and $n<\omega$.
Recall that \emph{$[I]^n$} denotes that the set of \emph{$n$-subsets\/} of $I$,  i.e.\@ the subsets of $I$ of cardinality $n$.
It is also common to write\smallskip

\ceq{\hfill\emph{$\displaystyle{I\choose n}$}}{=}{[I]^n}.\QED
\end{definition}

The condition above can be formulated in a number of equivalent ways.
We can require that for every formula $\phi(x_1,\dots,x_n)\in L(A)$ and every pairs of tuples in $I^n$ such that $i_0<\dots<i_n$ and $j_0<\dots<j_n$.


\ceq{\hfill\phi(a_{i_0},\dots,a_{i_n})}{\iff}{\phi(a_{j_0},\dots,a_{j_n})}

Or we can simply say that for every $i_0,\dots,i_n\in I$ the type $\tp(a_{i_0},\dots,a_{i_n}/A)$ only depends on the order type of $i_0,\dots,i_n$.

\begin{proposition}
Let $p({\mr x})\in S(\U)$ be a global $A\jj$invariant type and let ${\mr\bar c}=\<{\mr c_i}:i<\alpha\>$ be a Morley sequence of $p({\mr x})$ over $A$.
Then ${\mr\bar c}$ is a sequence of indiscernibles over $A$.
\end{proposition}

\begin{proof}

%\def\ceq#1#2#3{\parbox[t]{20ex}{$\displaystyle #1$}{\hspace*{1ex}$\displaystyle #2$\hspace*{1ex}}{$\displaystyle #3$}}

We prove by induction on $n<\omega$ that

\ceq{\sharp\hfill {\mr c_{\restriction n}}}{\ \equiv_A}{\mr c_{\restriction I_0}}\ \ \ for every $I_0\subseteq\alpha$ of cardinality $n$.

For $n=0$ the claim is trivial, then we assume $\,\sharp\,$ above is true and prove that

\ceq{\hfill {\mr c_{\restriction n}},{\mr c_n}}{\equiv_A}{{\mr c_{\restriction I_0}},{\mr c_i}}\ \ \ for every $I_0<i<\alpha$.

As $c$ is  Morley sequence, ${\mr c_n}\equiv_{A,{\mr c_{\restriction n}}} {\mr c_i}$ whenever $n<i$.
Hence we can equivalently prove that

\ceq{\hfill {\mr c_{\restriction n}},{\mr c_i}}{\equiv_A}{{\mr c_{\restriction I_0}},{\mr c_i},}

which is equivalent to

\ceq{\hfill {\mr c_{\restriction n}}}{\equiv_{A,\,{\mr c_i}}}{{\mr c_{\restriction I_0}}.}

The latter holds by induction hypothesis $\,\sharp\,$ and the invariance of $p({\mr x})$ as formulated in \ssf{it3} of Section~\ref{invariant}.
\end{proof}

%Let $I,<_I$ and $J,<_J$ be two infinite linear orders and let $a$ and $b$ be an $I\jj$sequence, respectively a $J\jj$sequence
%Note that the expression $a\equiv_A b$ is meaningless unless there is a unique isomorphism between $I,<_I$ and $J,<_J$. However it always make sense, even for non isomorphic orders, when $a$ and $b$ are indiscernibles over $A$. In fact, we agree that \emph{$a\equiv_A b$\/} means $a_{\restriction I_0}\equiv_A b_{\restriction J_0}$ for every finite $I_0\subseteq I$ and $J_0\subseteq J$ of equal cardinality.% That being said, the following proposition is immediate.


% \begin{proposition}\label{prop_embedding_indsc_seq}
% Let $J,<_J$ be an infinite linear orders of cardinality $<\kappa$ and let $a$ be an $A\jj$in\-dis\-cern\-i\-ble $J\jj$sequence.

% Let $I,<_I$ be another infinite linear order, and $b$ and  $A\jj$in\-dis\-cern\-i\-ble $I\jj$sequence such that $a\equiv_Ab$. Let $f:J\to I$ be an embedding of linear orders. Then $h(a_i)=b_{f(i)}$ for some $h\in\Aut(\U/A)$.\QED
% \end{proposition}


% 
% \begin{exercise}
% Let $c=\<{\mr c_i}:i<\alpha \>$ be a constant sequence, i.e.\@ ${\mr c_0}={\mr c_i}$ for all $i$. Prove that $c$ is a Morley sequence over $A$ if and only if ${\mr c_0}\in(\dcl A)^{|{\mr x}|}$ and that $c$ is a coheir sequence over $A$ if and only if  ${\mr c_0}\in A^{|{\mr x}|}$.\QED 
% \end{exercise}



% \begin{exercise}
% The following are equivalent
% \begin{itemize}
%  \item[1.] $\grD$ is invariant over $M$;
%  \item[2.] $c_0\in\grD\iff c_1\in\grD$ for every $M$-indiscernible sequence $\<c_i:i<\omega\>$ 
% \end{itemize}
% 
% \end{exercise}



\end{document}
