% !TEX root = creche.tex
\chapter{Invariant sets}
\label{invariant}

\def\medrel#1{\parbox[t]{6ex}{$\displaystyle\hfil #1$}}
\def\ceq#1#2#3{\parbox[t]{16ex}{$\displaystyle #1$}\medrel{#2}{$\displaystyle #3$}}

In this chapter, $L$ is a signature, $T$ is a complete theory without finite models, and $\U$ is a saturated model of inaccessible cardinality $\kappa$ strictly larger than $|L|$.
We use the same notation and make the same implicit assumptions as in Section~\ref{monster}.

%%%%%%%%%%%%%%%%%%%%%%%%%%%%%%%%%%%%
%%%%%%%%%%%%%%%%%%%%%%%%%%%%%%%%%%%%
%%%%%%%%%%%%%%%%%%%%%%%%%%%%%%%%%%%%
\section{Invariant sets and types}\label{invariant_sets}

Let $G$ be a group that acts on $\U^{\gr z}$.
In this chapter we only consider $G=\Aut(\U/A)$, but for later applcation we introduce a more general notation.
Let $\grD\subseteq\U^{\gr z}$.
We say that $\grD$ is an \emph{invariant\/} under the action of $G$, or \emph{$G$-invariant,} if it is fixed setwise by $G$.
That is, $g\grD=\grD$ for every $g\in G$.
Yet in other words, if

\ceq{\ssf{is1.}\hfill {\gr a}\in\grD}{\iff}{g{\gr a}\in\grD}\hfill for every ${\gr a}\in\U^{\gr z}$ and every $g\in G$.

% Suppose that $\grD=\phi({\mr a}\,;\U^{\gr z})$ for some $\phi({\mr x}\,;{\gr z})\in L$ and that $\phi({\mr\U}\,;\U^{\gr z})$ is invariant (this is always the case when $G$ acts by automorphisms).
% Then \ssf{is1} is aquivalent to

% \ceq{\ssf{is2.}\hfill\phi({\mr a}\,;{\gr z})}{\iff}{\phi(g{\mr a}\,;{\gr z})}\hfill for every ${\mr a}\in\U^{\mr x}$ and every $g\in G$.

The definition of invariant type is less straightforward.
We need to introduce some notation.
We are mainly interested in $\Delta$-types where either $\Delta=L(\Aa)$ for some $\Aa\subseteq\U$ or 

\ceq{\hfill\Delta}{=}{\Big\{\phi({\mr x}\,;{\gr b}),\ \neg\phi({\mr x}\,;{\gr b})\ :\ {\gr b}\in\Aa^{\gr z}\Big\}} 

for a given $\phi({\mr x}\,;{\gr z})\in L$.
In the latter case $\Delta$-types are called \emph{$\phi$-types.}
We denote by \emph{$S_\phi(\Aa)$\/} the set of maximal $\phi$-types with parameters in $\Aa$.
Typically, $\Aa$ is either $\U$ or some small set $A\subseteq\U$.
Types in \emph{$S_{\phi}(\U)$\/} are called \emph{global $\phi$-types.} 
They may be identified with subsets of $\Aa^{\gr z}$.
Namely, $p({\mr x})\in S_\phi(\Aa)$ is identified with
   
\ceq{\hfill\grD}{=}{\Big\{{\gr b}\in\Aa^{\gr z}\ :\ \phi({\mr x}\,;{\gr b})\in p\Big\}.}

% \begin{proposition}
% Suppose $M$ realizes every consistent type $p({\gr z})\subseteq L(A)$.
% Let $D\subseteq M^{\gr z}$ be such that \ssf{is2} holds for $D$ when restricted to ${\gr a},{\gr b}\in M^{\gr z}$.
% Then $D$ is the trace on $M$ of some unique $A$-invariant set $\grD$.
% \end{proposition}
% 
% \begin{proof}
% The set $\grD$ is the union, for $p({\gr z})\in S(A)$, of the sets $p(\U)$ that intersect $D$.
% \end{proof}

Let $p({\mr x})\subseteq L(\U)$ be a consistent type.
For every formula $\phi({\mr x}\,;{\gr z})\in L$ we define

\ceq{\hfill\emph{$\gr\D_{p,\phi}$}}{=}{\Big\{{\gr a}\in\U^{\gr z}\ :\ \phi({\mr x}\,;{\gr a})\in p\Big\}.}

We can read the notation in two ways: either the tuple ${\gr z}$ has length $\omega$ and is the same for all formulas, or it is finite and depends on $\phi$.
This is possible because adding or erasing dummy variables to the tuple of ${\gr z}$ does not change ${\gr\D_{p,\phi}}$ in any relevant way; in particular, invariance is preserved.

A $\phi$-type $p({\mr x})$ can be identified with the set ${\gr\D_{p,\phi}}$.
Therefore any type $p({\mr x})\subseteq L(\U)$ can be identified with the collection of the sets ${\gr\D_{p,\phi}}$, where $\phi({\mr x}\,;{\gr z})$ ranges over $L$.

Let $p({\mr x})\subseteq L(\U)$ be a consistent type.
We say that $p({\mr x})$ is an \emph{invariant\/} under the action of $G$ if, for every formula $\phi({\mr x}\,;{\gr z})\in L$ 

\ceq{\ssf{it1.}\hfill\phi({\mr x}\,;{\gr a})\in p}{\IFF}{\phi({\mr x}\,;g{\gr a})\in p}\hfill for every ${\gr a}\in\U^{\gr z}$ and every $g\in G$.

Hence $p({\mr x})$ is invariant exactly when all the sets $\gr\D_{p,\phi}$ are.

%%%%%%%%%%%%%%%%%%%%%%
%%%%%%%%%%%%%%%%%%%%%%
%%%%%%%%%%%%%%%%%%%%%%
%%%%%%%%%%%%%%%%%%%%%%
\section{Invariance over a set of parameters}

We now consider the case $G=\Aut(\U/A)$.
We say \emph{invariant over $A$\/} for invariant under the action of $\Aut(\U/A)$ and we write \emph{$A$-invariant\/} for $\Aut(\U/A)$-invariant.

Note that by homogeneity \ssf{is1} con be restated as

\ceq{\ssf{is3.}\hfill{\gr a}\equiv_A{\gr b}}{\IMP}{{\gr a}\in\grD\iff{\gr b}\in\grD}\hfill for all ${\gr a},{\gr b}\in\U^{\gr z}$.

This gives a bound on the number of sets invariant over $A$.

\begin{proposition}\label{prop_numberIS}
Let $\lambda=|L_{\gr z}(A)|$.
There are at most $2^{2^{\lambda}}$ sets $\grD\subseteq\U^{\gr z}$ that are invariant over $A$.
\end{proposition}
\begin{proof}
By \ssf{is3}, sets that are invariant over $A$ are unions of equivalence classes of the relation $\equiv_A$, that is, unions of sets of the form $p(\U^{\gr z})$ where $p({\gr z})\in S\/(A)$.
Then the number of $A$-invariant sets is at most $2^{|S_{\gr z}(A)|}$.
Clearly $|S_{\gr z}(A)|\le 2^\lambda$.
\end{proof} 

The invariance of types can be rephrased in a similar way

\ceq{\ssf{it2.}\hfill  {\gr a}\equiv_A{\gr b}}{\IMP}{\Big(\phi({\mr x}\,;{\gr a})\in p\,\IFF\,\phi({\mr x}\,;{\gr b})\in p\Big)}\hfill for all ${\gr a},{\gr b}\in\U^{\gr z}$

% In words we say that $p({\mr x})\subseteq L(\U)$ \emph{does not split\/} over $A$.

If $p(x)$ in a global $\phi$-types types \ssf{it2} is equivalent to

\ceq{\ssf{it2$^\prime$.}\hfill  {\gr a}\equiv_A{\gr b}}{\IMP}{p({\mr x})\,\proves\,\phi({\mr x}\,;{\gr a})\iff\phi({\mr x}\,;{\gr b})}

Finally, the following is another very useful characterization of invariance over $A$ of a global type $p({\mr x})\in S(\U)$ that follows easily from \ssf{it2}

\ceq{\ssf{it3.}\hfill {\gr a}\equiv_A{\gr b}}{\IMP}{{\gr a}\equiv_{A,\,{\mr c}}{\gr b}}\hfill for all ${\gr a},{\gr b}\in\U^{\gr z}$ and for all ${\mr c}\models p_{\restriction A,{\gr a},{\gr b}}$.

Note that \ssf{it3} applies to complete types but not to $\phi$-types.

%%%%%%%%%%%%%%%%%%%%%%
%%%%%%%%%%%%%%%%%%%%%%
%%%%%%%%%%%%%%%%%%%%%%
%%%%%%%%%%%%%%%%%%%%%%
\section{Invariance of types from the dual perspective}

In the previous sections the invariance of $p({\mr x})$ has been definend using the sets ${\gr\D_{p,\phi}}$.
Here we examine invariance using the formulas in $p({\mr x})$.

Let $G$ be a group that acts on $\U^{\mr x}$ and $\U^{\gr z}$.
Let $\phi({\mr x}\,;{\gr z})\in L$ be such that $\phi({\mr\U}\,;\U^{\gr z})$ is invariant under the action of $G$.
Note that this assumpion is redundant when $G$ acts by automorphisms.% -- the generality is only required in Section~\ref{}.
% The terms invariant, orbit, and the few other indroduced below all refer to the action of $G$ on $\U$.
Let $\mrX\subseteq\U^{\mr x}$ be a set that is type-definable over some small set of parameters and invariant.
At the first reading one may assume that $\mrX=\U^{\mr x}$.% as, again, more generality is required only in Section~\ref{}.

A set $\mrD\subseteq\U^{\mr x}$ is \emph{$G$-generic\/} in $\mrX$ if finitely many $G$-translations of $\mrD$ cover $\mrX$; we say \emph{$n$-$G$-generic\/} if $\le n$ translates suffices.
Dually, we say that $\mrD$ is \emph{$G$-persistent\/} in $\mrX$ if the intersection of any finitely many $G$-translations of $\mrD$ has a solution in $\mrX$; we say \emph{$n$-$G$-persistent\/} when the request is limited to $\le n$ translates.
We may drop reference to $G$ and $\mrX$ when these are clear from the context.

The same properties may be attributed to formulas (as these are identified with the set they define).
When the properties above are attributed to a type, it is understood that every conjunction of formulas in the type has the property.

\noindent\llap{\textcolor{red}{\Large\warning}\kern1.5ex}\ignorespaces
The terminology above is non-standard.
In the leterature some distinguished authors write \textit{quasi-non-dividing\/} for \textit{persistent}.
Their terminology has good motivations, but it is a mouthful.

\begin{fact}\label{fact_fip}
  Let $G$ and $\mrX$ be given.
  The following are equivalent
  \begin{itemize}
    \item[1.] $\mrD$ is not $n$-generic in $\mrX$
    \item[2.] $\neg\mrD$ is $n$-persistent in $\mrX$.
  \end{itemize}
\end{fact}

\begin{proof}
  Immediate by spelling out the definitions\smallskip
  \begin{itemize}
    \item[1.] there are no ${\mr\D_1},\dots,{\mr\D_n}$ translations of $\mrD$ such that \smash{$\displaystyle\mrX\ \subseteq\ \bigcup_{i=1}^n{\mr\D_i}$}
    \item[2.]  $\displaystyle\0\ \neq\ \mrX\,\cap\,\bigcap_{i=1}^n\neg{\mr\D_i}$ for every ${\mr\D_1},\dots,{\mr\D_n}$ translations of $\mrD$.\qedhere
  \end{itemize} 
\end{proof}

A \emph{$\phi$-formula\/} is a formula $\theta({\mr x}\,;{\gr\bar z})$, where ${\gr\bar z}={\gr z_1},\dots,{\gr z_n}$, that is a Boolean combination of formuas of the form $\phi({\mr x}\,;{\gr z_i})$.
A set is \emph{$\phi$-definable\/} if it is of the form $\theta({\mr\U}\,;{\gr\bar b})$ for some $\phi$-formula $\theta({\mr x}\,;{\gr\bar z})$ and ${\gr\bar b}={\gr b_1},\dots,{\gr b_n}$.

Our assumptions on $\phi$ ensure that $g[\theta({\mr\U}\,;{\gr\bar b})]=\theta({\mr\U}\,;g{\gr\bar b})$ for every $g\in G$.
Clearly global $\phi$-types decide all $\phi$-formulas.
Therefore $p({\mr x})\in S_\phi(\U)$ is invariant if 

\ceq{\hfill p({\mr x})\proves{\mr x}\in\mrD}{\IFF}{p({\mr x})\proves{\mr x}\in g\mrD}\hfill for every $\phi$-definable $\mrD\subseteq\U^{\mr x}$ and $g\in G$.

\begin{theorem}\label{thm_generic_invariant}
  Let $G$,  $\mrX$, and $\phi({\mr x}\,;{\gr z})\in L$ be given.
  Let $p({\mr x})\in S_\phi(\U)$ be finitely consistent in $\mrX$.
  Then the following are equivalent 
  \begin{itemize}
    \item[1.] $p({\mr x})$ is invariant
    \item[2.] $p({\mr x})\proves{\mr x}\in\mrD$ for every generic $\phi$-definable set $\mrD$
    \item[3.] $p({\mr x})\proves{\mr x}\in\mrD$ for every 2-generic $\phi$-definable set $\mrD$
    \item[4.] $p({\mr x})$ is 2-persistent
    \item[5.] $p({\mr x})$ is persistent.
  \end{itemize}
\end{theorem}

\begin{proof}
  \ssf1$\IMP$\ssf2.
  Let ${\mr\D_1},\dots,{\mr\D_n}$ be translations of $\mrD$ that cover $\mrX$.
  If $p({\mr x})\not\proves {\mr x}\in\mrD$, by the remark above

  \ceq{\hfill p({\mr x})}{\proves}{{\mr x}\notin\bigcup_{i=1}^n\mrD.}

  This contradicts the consistency of $p({\mr x})$ in $\mrX$.

  \ssf2$\IMP$\ssf3.
  Trivial.
  
  \ssf3$\IMP$\ssf4.
  Let $\mrD$ be defined by a conjunction of formulas in $p({\mr x})$.
  If $\mrD$ is not 2-persistent then, by Fact~\ref{fact_fip}, $\neg\mrD$ is 2-generic contradicting the consistency of $p({\mr x})$.

  \ssf4$\IMP$\ssf1.
  % uppose $p({\mr x})$ is not $(n+2)$-persistent and let $n$ be minimal.
  % Let $g_1,\dots,g_{n+2}\in G$ and $\theta({\mr x})\,;{\gr\bar b}$ witness this.
  % Then 
  If $p({\mr x})$ is not invariant then, by completeness, $p({\mr x})\proves\phi({\mr x}\,;{\gr b})\wedge\neg\phi({\mr x}\,;g{\gr b})$ for some $g\in G$.
  Clearly $\phi({\mr x}\,;{\gr b})\wedge\neg\phi({\mr x}\,;g{\gr b})$ is not 2-persistent as it is inconsistent with its $g$-translation.

  \ssf5$\IMP$\ssf2.
  Trivial.

  \ssf2$\IMP$\ssf5.
  By the same argument as in \ssf3$\IMP$\ssf4.
\end{proof}

% \begin{corollary}
%   Let $G$,  $\mrX$, and $\phi({\mr x}\,;{\gr z})\in L$ be given.
%   Assume there exists an invariant type $p({\mr x})\in S_\phi(\U)$ finitely consistent in $\mrX$.
%   Let $\mrD$ is a generic $\phi$-definable set.
%   Then $\neg\mrD$ is not generic.
% \end{corollary}

Theorem~\ref{thm_generic_invariant} gives a condition for a global $\phi$-type to be invariant.
The following theorem gives a condition for the existence of invariant types.
Ideally, we would like to prove that every persistent formula extends to invariant/persistent type.
Unfortunately this is not true and we need a stronger property which we might call \textit{hereditary persistency\/} 
or, with the terminology of the authors mentioned above, \textit{quasi-non-forking}.

\begin{theorem}\label{thm_generic_invariant2}
  Let $G$,  $\mrX$, and $\phi({\mr x}\,;{\gr z})\in L$ be given.
  Then for every $\phi$-definable set $\mrD\subseteq\U^{\mr x}$ the following are equivalent 
  \begin{itemize}
    \item[1.] $p({\mr x})\proves{\mr x}\in\mrD$ for some invariant type $p({\mr x})\in S_\phi(\U)$ finitely consistent in $\mrX$
    \item[2.] every finite cover of $\mrD\cap\mrX$ by $\phi$-definable sets contains a persistent set
    \item[3.] every finite cover of $\mrD\cap\mrX$ by $\phi$-definable sets contains a 2-persistent set.
  \end{itemize}
\end{theorem}

\begin{proof}
  \ssf1$\IMP$\ssf2.
  Suppose ${\mr\C_1},\dots,{\mr\C_n}$ cover $\mrD\cap\mrX$ and pick $p({\mr x})$ as in \ssf1.
  By completeness, $p({\mr x})\proves {\mr x}\in{\mr\C_i}$ for some $i$.
  Then, by Theorem~\ref{thm_generic_invariant}, $\neg{\mr\C_i}$ is not generic.
  Therefore, by Fact~\ref{fact_fip}, ${\mr\C_i}$ is persistent.
  
  \ssf2$\IMP$\ssf3.
  Trivial.

  \ssf3$\IMP$\ssf1.
  Let $p({\mr x})$ be maximal among the types that contain ${\mr x}\in\mrD$ and are such that $\theta({\mr\U})$ satisfies \ssf3 for every $\theta({\mr x})\in p$.
  We claim that $p$ is a complete type.
  Suppose for a contradiction that $\theta({\mr x}),\neg\theta({\mr x})\notin p$.
  By maximality there is some $\psi({\mr x})\in p$ and some ${\mr\C_1},\dots,{\mr\C_n}$ that cover both $\psi({\mr\U})\cap\theta({\mr\U})$ and $\psi({\mr\U})\smallsetminus\theta({\mr\U})$ and are such that no ${\mr\C_i}$ is persistent.
  This is a contradiction because ${\mr\C_1},\dots,{\mr\C_n}$ cover $\psi({\mr\U})$ which satisfies \ssf3.
  It is only left to show that $p({\mr x})$ is finitely consitent in $\mrX$ and invariant.
  Finite consistency is clear.
  Negate invariance.
  Then, by Theorem~\ref{thm_generic_invariant} and completeness, $p({\mr x})\proves {\mr x}\in\neg\mrB$ for some 2-generic set $\mrB$.
  As $\neg\mrB$ is not 2-persistent by Fact~\ref{fact_fip}, this contradics \ssf3.
\end{proof}


%%%%%%%%%%%%%%%%%%%%%%%%%%%%%
%%%%%%%%%%%%%%%%%%%%%%%%%%%%%
%%%%%%%%%%%%%%%%%%%%%%%%%%%%%
%%%%%%%%%%%%%%%%%%%%%%%%%%%%%
%%%%%%%%%%%%%%%%%%%%%%%%%%%%%
%%%%%%%%%%%%%%%%%%%%%%%%%%%%%
\section{Heirs and coheirs}
\label{coheirs}

The easiest way to obtain types that are invariant over a model $M$ is via types that are finitely satisfiable in $M$.
We say that a type $p({\mr x})$ is \emph{finitely satisfiable\/} in $A$ if every conjunction of formulas in $p({\mr x})$ has a solution in $A^{\mr x}$.

\begin{proposition}\label{prop_coeredi_quasiinvarienti}
Every type $p({\mr x})\subseteq L(\U)$ that is finitely satisfiable in $A$ is invariant over $A$.
\end{proposition}

\begin{proof}
  Note that $p({\mr x})$ is $\Aut(\U/A)$-persistent, in fact the same ${\mr a}\in A^{\mr x}$ that satisfies $\phi({\mr x})$ also satisfies every $\Aut(\U/A)$-translate of $\phi({\mr x})$.
  Then the proposition follows from Theorem~\ref{thm_generic_invariant}.
  (But a direct proof is also easy.)  
\end{proof}

\begin{proposition}\label{prop_exisntence_coheirs}
Every type $q({\mr x})\subseteq L(\U)$ that is finitely satisfiable in $A$ has an extension to a global type that is finitely satisfiable in $A$.
\end{proposition}

\begin{proof} 
Let $p({\mr x})\subseteq L(\U)$ be maximal among the types that contain $q({\mr x})$ and are finitely satisfiable in $A$.
We prove that $p({\mr x})$ is complete.
If for a contradiction $p({\mr x})$ contains neither $\psi({\mr x})$ nor $\neg\psi({\mr x})$, then  neither $p({\mr x})\cup\big\{\psi({\mr x})\big\}$ nor $p({\mr x})\cup\big\{\neg\psi({\mr x})\big\}$ are finitely satisfiable in $A$.
This contradicts the finite satisfiability of $p({\mr x})$.
\end{proof}

In most cases we work with types that are finitely satisfiable over a model.
The reason is explained by the next proposition, which is clear by elementarity.

\begin{proposition}\label{prop_coher_over_model}
    Every consistent type over a model is finitely satisfiable in that model, that is, whenever $p({\mr x})\subseteq L(M)$ is consistent, $p({\mr x})$ is finitely satisfiable in $M$.
\end{proposition}

\begin{definition}\label{def_choeir_uno} A type $p({\mr x}) \subseteq L(\U)$ that is finitely satisfiable in $M$ is said to be a \emph{coheir\/} of $p_{\restriction M}({\mr x})$.
\end{definition}

%A type that extends $p({\mr x})$ and is finitely satisfied in $M$ is called a \emph{coheir\/} of $p({\mr x})$ over~$M$.

In many cases it is useful to focus on elements instead of types.
We introduce the following notation to express that $\tp({\mr a}/M,{\gr b})$ is finitely satisfied in $M$.

\begin{definition}\label{def_coheir_idepencence} 
  For every ${\mr a}\in\U^{\mr x}$ and ${\gr b}\in\U^{\gr z}$ we define

  \noindent\llap{\textcolor{red}{\Large\danger}\kern1.5ex}
  %
  \ceq{\hfill\emph{${\mr a}\cnonfork_M{\gr b}$}}
  {\IFF}
  {\phi(M^{\mr x},{\gr b})\neq\0
  \textrm{ for all }\phi({\mr x}\,;{\gr z})\in L(M) 
  \textrm{ such that }\phi({\mr a}\,;{\gr b})}.

  %We call this the \emph{coheir-heir} relation.
  %
  We say that $\tp({\mr a}/M,{\gr b})$ is a \emph{coheir} of $\tp({\mr a}/M)$, or, equivalently, that $\tp({\mr b}/M,{\gr a})$ is a \emph{heir} of $\tp({\mr b}/M)$.
  We define the type

  \ceq{\hfill\emph{${\mr x}\cnonfork_M{\gr b}$}}
  {=}
  {\Big\{\neg\phi({\mr x}\,;{\gr b})
  \ :\ 
  \phi({\mr x}\,;{\gr b})\in L(M)
  \textrm{ and } \phi(M^{\mr x}\,;{\gr b})=\0\Big\}.}

    % 
  We will write \emph{${\mr a}\equiv_M{\mr x}\cnonfork_M{\gr b}$} 
  for the union of the types ${\mr x}\cnonfork_M{\gr b}$ and 
  $\tp({\mr a}/M)$.
\end{definition}
 The tuples ${\mr a}$ realizing ${\mr x}\cnonfork_M{\gr b}$ are exactly those such that ${\mr a}\cnonfork_M{\gr b}$.Note that $\tp({\mr a}/M,{\gr b})$ is a coheir of $\tp({\mr a}/M)$ according to Definition~\ref{def_choeir_uno}, so the terminology is consistent.

We think of  ${\mr a}\cnonfork_M{\gr b}$ as saying that 
${\mr a}$ is \emph{independent\/} from ${\gr b}$ over $M$.
%
%This is a strong notion of independence.
%
In general %it is not symmetric, that is 
${\mr b}\cnonfork_M{\gr a}$ is not 
equivalent to ${\gr a}\cnonfork_M{\mr b}$.
%(symmetry is equivalent to stability).

We shall use, sometimes without explicit reference, the following easy lemma.

\begin{lemma}\label{lem_coheir_independence}
    The following properties hold for all $M,{\mr a},{\mr b}$, and $c$
    \begin{itemize}
    \item[1.] ${\mr a}\cnonfork_M{\gr b}\ \ \IMP\ \ f{\mr a}\cnonfork_Mf{\gr b}$ \ \ 
              for every $f\in\Aut(\U/M)$\hfill \textit{invariance}
    \item[2.] ${\mr a}\cnonfork_M{\gr b}\ \ \IFF\ \ {\mr a_0}\cnonfork_M{\gr b_0}$
              \ for all finite ${\mr a_0}\subseteq{\mr a}$ and 
              ${\gr b_0}\subseteq{\gr b}$ \hfill\textit{finite character}
    \item[3.] ${\mr a}\cnonfork_Mc,{\gr b}$ \ and \ 
              $c\cnonfork_M{\gr b}\ \ \IMP\ \ {\mr a},c\cnonfork_M{\gr b}$
              \hfill\hfill\hfill\textit{transitivity}
    \item[4.] ${\mr a}\cnonfork_M{\gr b}\ \ \IMP\ \ $ 
              there exists ${\mr a'}\equiv_{M,\,{\gr b}}{\mr a}$ such that 
              ${\mr a'}\cnonfork_M{\gr b},c$
              \hspace{\stretch{20}}\textit{coheir extension}
    \item[5.] ${\mr a}\cnonfork_M{\gr b_1},{\gr b_2}$ \ and \ 
    ${\gr b_1}\equiv_M{\gr b_2}\ \ \IMP\ \ {\gr b_1}\equiv_{M,\,{\mr a}}{\gr b_2}$
              \hspace{\stretch{20}}\textit{non-splitting}
    \end{itemize}
    \end{lemma}
\begin{proof}

Properties \ssf{1}-\ssf{3} follow immediately from Definition~\ref{def_coheir_idepencence}.
We prove \ssf{4}.
Assume ${\mr a}\cnonfork_M{\gr b}$, that is,  $\tp({\mr a}/M,{\gr b})$ is finitely satisfiable in $M$.
By Proposition~\ref{prop_exisntence_coheirs} $\tp({\mr a}/M,{\gr b})$ extends to a global type $p({\mr x})$ that is finitely satisfiable in $M$.
Then any ${\mr a'}\models p_{\restriction M,\,{\gr b},\,c}({\mr x})$ proves the lemma.
The proof of \ssf{5} is left to the reader.
\end{proof}

\begin{proposition}\label{prop_saturate_heir}
  Let ${\mr a}\cnonfork_M{\gr b}$ then there is $\V\preceq\U$ that is isomorphic to $\U$ and such that ${\gr b}\in\V^{\gr z}$ and ${\mr a}\cnonfork_M\V$.
\end{proposition}

\begin{proof}
  Let $c$ be an enumeration of $\U$.
  Let $p(w,{\gr z})=\tp(c,{\gr b}/M)$.
  We can take $\V$ to be the structure enumerated by the tuple that realizes symultaneously $p(w,{\gr b}) $ and the following type 

  \ceq{\hfill\Big\{\neg\phi({\mr a},w,{\gr b})}{:}{\phi({\mr x},w,{\gr z})\in L(M),\ \ p(w,{\gr b})\imp\phi(M^{\mr x},w,{\gr b})=\0\Big\}.}

  We only need to prove consistency.
  If inconsistent, then 

  \ceq{\hfill\theta(w,{\gr b})}{\imp}{\bigvee_{i=1}^n\phi_i({\mr a},w,{\gr b})}
  
for some $\theta(w,{\gr b})\in p$ and some $\phi_i({\mr x},w,{\gr b})$ such that $p(w,{\gr b})\imp\phi_i(M,w,{\gr b})=\0$.
We obtain a contradiction because by elementarity we can replace ${\mr a}$ by some ${\mr a'}\in M^{\mr x}$.
\end{proof}

%Transitivity has dual version that holds under stronger assumptions.

The type ${\mr a}\equiv_M{\mr x}\cnonfork_M{\gr b}$ in Definition~\ref{def_coheir_idepencence} is the intersection of all global coheirs of $\tp({\mr a}/M)$.
%
Its consistency is guaranteed by the fact that $M$ is a model (see Proposition~\ref{prop_coher_over_model}).
%
However, in general it need not be a complete type over $M,{\gr b}$.
%
In fact, its completeness of this type is a property with important consequences.

\begin{definition}\label{def_coheir_stationary} We say that $\cnonfork_M$ is \emph{stationary\/} if ${\mr a}\equiv_M{\mr x}\cnonfork_M{\gr b}$ is a complete type over $M,{\gr b}$ for all finite tuples ${\gr b}$ and ${\mr a}$.
We say \emph{$n$-stationary\/} if we limit the request to tuples ${\mr a}$ of length $n$.
\end{definition}

An application of stationarity is given in Section~\ref{semigroups}.
Stationarity is often ensured by the following property, which will receive due attention in Section~\ref{stable_theories}.

\begin{proposition}
Let ${\mr x}$ be a tuple of variables of length $n$.
If for every $\phi({\mr x})\in L(\U)$ there is a formula $\psi({\mr x})\in L(M)$ such that $\phi(M^{\mr x})=\psi(M^{\mr x})$ then $\cnonfork_M$ is $n$-stationary.
\end{proposition}

\begin{proof}
Let ${\gr b}\in\U^{\gr z}$ and ${\mr a_1},{\mr a_2}\in\U^{\mr x}$ be such that ${\mr a_i}\cnonfork_M{\gr b}$ and ${\mr a_1}\equiv_M{\mr a_2}$.
We claim that ${\mr a_1}\equiv_{M,\,{\gr b}}{\mr a_2}$.
We need to prove that $\phi({\mr a_1}\,;{\gr b})\iff\phi({\mr a_2}\,;{\gr b})$ for every  $\phi({\mr x}\,;{\gr z})\in L(M)$.
Let $\psi({\mr x})\in L(M)$ be such that $\phi({\mr\U}\,;{\gr b})=\psi({\mr \U})$.
From ${\mr a_i}\cnonfork_M{\gr b}$ we obtain that  $\phi({\mr a_i}\,;{\gr b})\iff\psi({\mr a_i})$.
Finally, the claim follows because ${\mr a_1}\equiv_M{\mr a_2}$.
\end{proof}

% \begin{lemma}\label{lem_coheirext}
% If ${\gr a}\nonforkc_A{\mr c}$ then for every ${\gr b}$ there is ${\mr c'}$ such that ${\gr a},\,{\gr b}\nonforkc_A{\mr c'}\equiv_{A,\,{\gr a}}{\mr c}$.
% \end{lemma}

% \begin{exercise}.
% Prove that every invariant type $q({\mr x})\subseteq L(\U)$ can be extended to a global invariant type $p({\mr x})\in S(\U)$.(?)
% \end{exercise}

% We say that $p({\mr x})\in S(\U)$ is a \emph{global heir\/} of $p_{\restriction M}({\mr x})$ if  
% 
% \ceq{\hfill{\gr\D_{p,\phi}}\cap M^{\gr z}\neq\0}{\IFF}{{\gr\D_{p,\phi}}\neq\0}  for every formula $\phi({\mr x}\,;{\gr z})\in L$.

%Every type over a model has an extension to a global coheir.
%A satisfactory generalization of this notions to types over sets should guarantee this existence.
%Unfortunately, this is not possible in general and the definition Section~\hyperref[coheirs_sets]{\ref*{invariantL}.\ref*{coheirs_sets}} is as good as it gets.

% \begin{remark}\label{rk_coheir_stationary}
% %The stationarity of $\cnonfork_A$ over every set $A$, or just over every model, is equivalent to the stability of $T$, see Section~\hyperref[stable_teories]{\ref*{external}.\ref*{stable_teories}}.
% There are theories where the stationarity of $\cnonfork_M$ holds for some particular $M$.
% For example, if every subset of $M^n$ is $M$-definable then $\cnonfork_M$ is clearly $n$-stationary.
% This simple observation will help in the proof of Theorem~\ref{thm_Hindman}.
% For a natural example, let $T=T_{\rm dlo}$ and let $M\subseteq\U$ have the order type of $\RR$.
% By quantifier elimination every definable subset of $\U$ is a union of finitely many intervals.
% By Dedekind completeness, the trace on $M$ of any interval of $\U$ coincides with that of an $M$-definable interval.
% \end{remark}

%%%%%%%%%%%%%%%%%%%%%%%%%%%%%%
%%%%%%%%%%%%%%%%%%%%%%%%%%%%%%
%%%%%%%%%%%%%%%%%%%%%%%%%%%%%%
%%%%%%%%%%%%%%%%%%%%%%%%%%%%%%
%%%%%%%%%%%%%%%%%%%%%%%%%%%%%%
\section{Morley sequences and indiscernibles}

In what follows $\alpha$ is some ordinal $\le\kappa$, typically $\omega$, and ${\mr x}$ is a tuple of variables of length $<\kappa$.

Let \mbox{$p({\mr x})\in S(\U)$} be a global type.
We say that ${\mr\bar c}=\<{\mr c_i}:i<\alpha\>$ is a \emph{Morley sequence\/} of $p({\mr x})$ over $A$ if for every $i<\alpha$

\ceq{\ssf{Ms.}\hfill {\mr c_i}}{\models}{p_{\restriction  A,\,{\mr c_{\restriction i}}}({\mr x})}.

When $p({\mr x})$ is finitely satisfiable in $A$, we say that ${\mr\bar c}$ is a \emph{coheir sequence\/} of $p({\mr x})$ over $A$.

When we say that ${\mr\bar c}$ is a coheir sequence over $A$ (with no explicit reference to a global type), we mean that \textit{there is\/} a type $p({\mr x})\in S(\U)$ that is finitely satisfiable in $A$ such that ${\mr\bar c}$ is a \emph{coheir sequence\/} of $p({\mr x})$.

The following is a convenient characterization of coheir sequences.

\begin{lemma}\label{lem_coheir_property}
The following are equivalent
\begin{itemize}
\item[1.] ${\mr\bar c}=\<{\mr c_n}:n<\omega\>$ is a coheir sequence over $M$
\item[2.] ${\mr c_n}\cnonfork_M{\mr c_{\restriction n}}$ and ${\mr c_{n+1}}\equiv_{M,\,{\mr c_{\restriction n}}}{\mr c_n}$ for every $n<\omega$.
\end{itemize}
\end{lemma}

\begin{proof}
\ssf{1}$\IMP$\ssf{2}.
Assume \ssf{1} and let $p({\mr x})\in S(\U)$ be a global type that is finitely satisfiable in $M$ and such that ${\mr c_i}\models p_{\restriction M,{\mr c_{\restriction i}}}({\mr x})$.
The requirement ${\mr c_{n+1}}\equiv_{M,{\mr c_{\restriction n}}} {\mr c_n}$ is clear.
Now, suppose $\phi({\mr c_{n+1}})$ for some $\phi({\mr x})\in L(M,{\mr c_{\restriction n+1}})$.
Then $\phi({\mr x})$ belongs to $p({\mr x})$, so $\phi({\mr\U})\cap M^{\mr x}\neq\0$ because $p({\mr x})$ is finitely satisfiable in $M$.
This proves ${\mr c_n}\cnonfork_M{\mr c_{\restriction n}}$.

\ssf{2}$\IMP$\ssf{1}.
Let $q({\mr x})=\{\phi({\mr x})\in L(M, {\mr\bar c})\,:\, \phi({\mr c_n}) \textrm{ holds for cofinitely many } n\}$.
%
We claim that $q({\mr x})$ is finitely satisfiable in $M$.
%
Let $\phi({\mr x}\,;{\mr z})\in L(M)$ be such that $\phi({\mr x}\,;{\mr c_{\restriction n}})\in q$.
%
By the definition of $q({\mr x})$, the formula $\phi({\mr c_m}\,;{\mr c_{\restriction n}})$ holds for all sufficiently large $m$.
%
Hence, from \ssf{2} we infer ${\mr c_m}\cnonfork_M{\mr c_{\restriction n}}$ and conclude that $\phi({\mr x}\,;{\mr c_{\restriction n}})$ is satisfied in $M$.

Let $p({\mr x})$ be any global extension of $q({\mr x})$ finitely saisfied in $M$.
%
We prove that ${\mr\bar c}$ is a Morley sequence of $p({\mr x})$ over $M$.
%
By \ssf{2} either ${\mr c_m}\models p_{\restriction  M,\,{\mr c_{\restriction n}}}({\mr x})$ for all $m\ge n$ or ${\mr c_m}\notmodels p_{\restriction  M,\,{\mr c_{\restriction n}}}({\mr x})$ for all $m\ge n$.
%
As $p({\mr x})$ extends $q({\mr x})$, the latter cannot occur.
\end{proof}

Let $(I,<_I)$ be a linear order.
A function ${\mr\bar a}:I\to\U^{\mr x}$ is said to be an \emph{$I$-sequence}, or simply a \emph{sequence\/} when $I$ is clear.
We will often introduce an $I$-sequence as ${\mr\bar a}=\<{\mr a_i}: i\in I\>$.

If $I_0\subseteq I$ we call ${\mr a_{\restriction I_0}}$ a \emph{subsequence\/} of ${\mr\bar a}$.
The subsets $I_0\subseteq I$ that are well-ordered by $<_I$, in particular the finite ones, are especially relevant.
When $I_0$ has order type $\alpha$, an ordinal, we identify ${\mr a_{\restriction I_0}}$ with a tuple of length $\alpha$.

Recall that \emph{$I^{(n)}$} denotes that the set of \emph{$n$-subsets\/} of $I$,  i.e.\@ the subsets of $I$ of cardinality $n$.
The notation \smallskip

\ceq{\hfill\emph{$\displaystyle\binom{I}{n}$}}{=}{I^{(n)}}

is also common.
\begin{definition}
Let $(I,<_I)$ be an infinite linear order and let ${\mr\bar a}$ be an $I$-sequence.
We say that ${\mr\bar a}$ is a \emph{sequence of indiscernibles\/} over $A$ or, an \emph{$A$-indiscernible sequence\/}, if ${\mr a_{\restriction I_0}}\equiv_A {\mr a_{\restriction I_1}}$ for every $I_0,I_1\in I^{(n)}$ and $n<\omega$.

\end{definition}

The indiscernibility condition can be formulated in a number of equivalent ways.
For example, we can require that, for every formula $\phi(x_1,\dots,x_n)\in L(A)$ and every pair of tuples in $I^n$ such that $i_0<\dots<i_n$ and $j_0<\dots<j_n$,


\ceq{\hfill\phi(a_{i_0},\dots,a_{i_n})}{\iff}{\phi(a_{j_0},\dots,a_{j_n})}

Alternatively, we can simply say that for all $i_0,\dots,i_n\in I$ the type $\tp(a_{i_0},\dots,a_{i_n}/A)$ only depends on the order type of $i_0,\dots,i_n$.

\begin{proposition}
Let $p({\mr x})\in S(\U)$ be a global $A$-invariant type and let ${\mr\bar c}=\<{\mr c_i}:i<\alpha\>$ be a Morley sequence of $p({\mr x})$ over $A$.
Then ${\mr\bar c}$ is an $A$-indiscernibles sequence.
\end{proposition}

\begin{proof}

%\def\ceq#1#2#3{\parbox[t]{20ex}{$\displaystyle #1$}{\hspace*{1ex}$\displaystyle #2$\hspace*{1ex}}{$\displaystyle #3$}}

We prove by induction on $n<\omega$ that

\ceq{\sharp\hfill {\mr c_{\restriction n}}}{\ \equiv_A}{\mr c_{\restriction I_0}}\ \ \ for every $I_0\subseteq\alpha$ of cardinality $n$.

For $n=0$ the claim is trivial.
We assume inductively that $\,\sharp\,$ above is true and prove that

\ceq{\hfill {\mr c_{\restriction n}},{\mr c_n}}{\equiv_A}{{\mr c_{\restriction I_0}},{\mr c_i}}\ \ \ for every $I_0<i<\alpha$.

As ${\mr\bar c}$ is  Morley sequence, ${\mr c_n}\equiv_{A,{\mr c_{\restriction n}}} {\mr c_i}$ whenever $n<i$.
Hence we can equivalently prove that

\ceq{\hfill {\mr c_{\restriction n}},{\mr c_i}}{\equiv_A}{{\mr c_{\restriction I_0}},{\mr c_i},}

which is equivalent to

\ceq{\hfill {\mr c_{\restriction n}}}{\equiv_{A,\,{\mr c_i}}}{{\mr c_{\restriction I_0}}.}

The latter holds by induction hypothesis $\,\sharp\,$ and the invariance of $p({\mr x})$ as formulated in \ssf{it3} of Section~\ref{invariant_sets}.
\end{proof}

%Let $I,<_I$ and $J,<_J$ be two infinite linear orders and let $a$ and $b$ be an $I$-sequence, respectively a $J$-sequence
%Note that the expression $a\equiv_A b$ is meaningless unless there is a unique isomorphism between $I,<_I$ and $J,<_J$.However it always make sense, even for non isomorphic orders, when $a$ and $b$ are indiscernibles over $A$.In fact, we agree that \emph{$a\equiv_A b$\/} means $a_{\restriction I_0}\equiv_A b_{\restriction J_0}$ for every finite $I_0\subseteq I$ and $J_0\subseteq J$ of equal cardinality.% That being said, the following proposition is immediate.


% \begin{proposition}\label{prop_embedding_indsc_seq}
% Let $J,<_J$ be an infinite linear orders of cardinality $<\kappa$ and let $a$ be an $A$-in\-dis\-cern\-i\-ble $J$-sequence.

% Let $I,<_I$ be another infinite linear order, and $b$ and  $A$-in\-dis\-cern\-i\-ble $I$-sequence such that $a\equiv_Ab$.Let $f:J\to I$ be an embedding of linear orders.Then $h(a_i)=b_{f(i)}$ for some $h\in\Aut(\U/A)$.
% \end{proposition}


% 
% \begin{exercise}
% Let $c=\<{\mr c_i}:i<\alpha \>$ be a constant sequence, i.e.\@ ${\mr c_0}={\mr c_i}$ for all $i$.Prove that $c$ is a Morley sequence over $A$ if and only if ${\mr c_0}\in(\dcl A)^{\mr x}$ and that $c$ is a coheir sequence over $A$ if and only if  ${\mr c_0}\in A^{\mr x}$.
% \end{exercise}



% \begin{exercise}
% The following are equivalent
% \begin{itemize}
%  \item[1.] $\grD$ is invariant over $M$;
%  \item[2.] $c_0\in\grD\iff c_1\in\grD$ for every $M$-indiscernible sequence $\<c_i:i<\omega\>$ 
% \end{itemize}
% 
% \end{exercise}


\begin{exercise}
  Let $\bar a$ be a sequence such that $a_{\restriction I_0}\equiv a_{\restriction I_1}$ for every $I_0,I_1$ such that $I_0<I_1$.
  Prove that $\bar a$ is a sequence of indiscernibles.
  \end{exercise}