% !TEX root = creche.tex
\documentclass[creche.tex]{subfiles}
\begin{document}
\chapter{Stable formulas (tentative, a scribbling block)}
\label{stability}

\def\mrA{{\mr\Aa}}
\def\grB{{\gr\B}}
\def\grC{{\gr\C}}
\def\grD{{\gr\D}}

\def\medrel#1{\parbox[t]{6ex}{$\displaystyle\hfil #1$}}
\def\ceq#1#2#3{\parbox{25ex}{$\displaystyle #1$}\medrel{#2}$\displaystyle  #3$}

%%%%%%%%%%%%%%%%%%%%%%%%%%%%%
%%%%%%%%%%%%%%%%%%%%%%%%%%%%%
%%%%%%%%%%%%%%%%%%%%%%%%%%%%%
%%%%%%%%%%%%%%%%%%%%%%%%%%%%%
\section{Externally definable sets}
\label{external}


Let $\grC,\grD\subseteq\U^{|{\gr z}|}$. The set $\grD\cap A^{|{\gr z}|}$ is called the \emph{trace\/} of $\grD$ over $A$. We write $\grC=_A\grD$ if  $\grC$ and $\grD$ have the same trace on $A$.

We say that $\grD$ is \emph{externally definable\/} if there are a global type $p\in S_{\mr x}(\U)$ and a formula $\phi({\mr x}\,;{\gr z})$ such that $\grD=\{{\gr a}\,:\,\phi({\mr x}\,;{\gr a})\in p\}$. Equivalently, a set $\grD$ is externally definable if it is the trace over $\U$ of a set which is definable in some elementary extension of $\U$. Precisely, $\grD$ is the trace on $\U$ of a set of he form $\phi({\mr b'}\,;{\gr\U'})$ where $\<\U',{\gr\D'}\>$ is elementary extension of $\<\U,\grD\>$ and ${\mr b'}\in\U'^{|{\mr x}|}$. This explains the terminology.

\noindent\llap{\textcolor{red}{\Large\danger}\kern1.5ex}We prefer to deal with external definability in a different, though equivalent, way. This is not the most commond approach.

\begin{definition}\label{def_epprox}
We say that $\grD$ is \emph{approximable\/} by the formula $\phi({\mr x}\,;{\gr z})$ if for every finite $B$ there is a ${\mr b}\in\U^{|{\mr x}|}$ such that $\phi({\mr b}\,;{\gr\U})=_B\grD$. If in addition we have that $\phi({\mr b}\,;{\gr\U})\subseteq\grD$, we say that  $\grD$ is \emph{approximable from below}. If  $\grD\subseteq\phi({\mr b}\,;{\gr\U})$ we say that  $\grD$ is \emph{approximable from above}. We may call $\phi({\mr x}\,;{\gr z})$ the \emph{sort} of $\grD$.\QED
\end{definition} 
 
Approximability from below is an adaptation to our context of the notion of \textit{having an honest definition} in \cite{CS}.  It will be discussed in Chapter~\textbf{??}.

The following proposition is clear by compactness.

\begin{proposition}\label{lem_approx=external}
For every $\grD$ the following are equivalent:
\begin{itemize}
\item[1.] $\grD$ is approximable;
\item[2.] $\grD$ is externally definable.\QED 
\end{itemize}
\end{proposition}

\begin{example}
Every definable set is trivially approximale. However if we fix the sort, this is not true any more. Let $T=T_{\rm dlo}$. Then $\D=\{z\in\U:a\le z\le b\}$ is approximable both from below and from above by the formula $x_1<z<x_2$ though it is not definable by this formula.

Now, let $T=T_{\rm rg}$. Then every $\D\subseteq\U$ is approximable and, when $\D$ has small infinite cardinality, it is approximable from above but not from below, see Exercise~\ref{ex_rg_small_def_set}.\QED
\end{example}

In Definition~\ref{def_epprox}, the sort $\phi({\mr x}\,;{\gr z})$ is fixed (otherwise any set would be approximable) but this requirement of uniformity may be dropped if the sets $B$ are allowed to be infinite.

\begin{proposition}\label{lem_approx_nonunif}
For every $\grD$ the following are equivalent:
\begin{itemize}
\item[1.] $\grD$ is approximable;
\item[2.] for every $B\subseteq\U$ of cardinality $\le|T|$ there is $\psi({\gr z})\in L(\U)$ such that $\psi({\gr\U})=_B\grD$.
\end{itemize}
Similarly, the following are equivalent:
\begin{itemize}
\item[3.] $\grD$ is approximable from below;
\item[4.]  for every ${\gr B}\subseteq\grD$ of cardinality $\le|T|$ there is $\psi({\gr z})\in L(\U)$ such that ${\gr B}\subseteq \psi({\gr\U})\subseteq\grD$.
\end{itemize}
\end{proposition}

\begin{proof}
To prove \ssf{2}$\IMP$\ssf{1} assume \ssf{2} and negate \ssf{1} for a contradiction. For each formula $\psi({\mr x}\,;{\gr z})\in L$ choose a finite set $B$ such that $\psi({\mr b}\,;{\gr\U})\neq_B\grD$ for every ${\mr b}\in\U^{|{\mr x}|}$. Let $C$ be the union of all these finite sets. Clearly $|C|\le|T|$. By \ssf{2} there are a formula $\phi({\mr x}\,;{\gr z})$ and a tuple ${\mr c}$ such that $\phi({\mr c}\,;{\gr\U})=_C\grD$, contradicting the definition of $C$.

The implication \ssf{1}$\IMP$\ssf{2} is obtained by compactness and the equivalence \ssf{3}$\IFF$\ssf{4} is proved similarly. 
\end{proof}



\begin{remark}\label{prop_approx_el_eq}
Let $\grC\subseteq\U^{|{\gr z}|}$. If $\grD$ is approximated by $\phi({\mr x}\,;{\gr z})$ then so is any $\grC$ such that $\grC\equiv\grD$, see Section~\hyperref[expansions]{\ref*{invariant}.\ref*{expansions}} for the definition. In fact, if the set $\grD$ is approximable by $\phi({\mr x}\,;{\gr z})$ then for every $n$

\hfil$\displaystyle\A {\gr z_1},\dots,{\gr z_n}\;\E {\mr x}\ \bigwedge^n_{i=1}\big[\phi({\mr x}\,;{\gr z_i})\ \iff\ {\gr z_i}\in{\gr \D}\big]$. 

So the same holds for any $\grC\equiv\grD$. A similar remark apply to approximability from below and from above. For approximability from below, add the conjunct $\A {\gr z}\,\big[\phi({\mr x}\,;{\gr z})\imp {\gr z}\in\grD\big]$ to the formula above, and similarly for approximability from above.\QED
\end{remark}

%We say that a set $\grD$ approximable by $\phi({\mr x}\,;{\gr z})$ is stationary over $A$ if it is invariant over $A$ and for any other such set $\grC$ if $\grC=_A\grD$ then $\grC=\grD$. Note that the notion of stationary depends only on $\phi({\mr x}\,;{\gr z})$ and $A^{|{\gr z}|}\cap\grD$. 




%%%%%%%%%%%%%%%%%%%%%%%%%%%%%
%%%%%%%%%%%%%%%%%%%%%%%%%%%%%
%%%%%%%%%%%%%%%%%%%%%%%%%%%%%
%%%%%%%%%%%%%%%%%%%%%%%%%%%%%
\section{The ladder property}

The attentive reader may have noticed that by \textit{formula\/} we often understand a pair $\phi(x)$ that consists of a formula and a tuple of variables. In the following we need to attach to formulas two two sorts of variables: those that are placeholder for parameters, and those used to define a set. To make things more complicated, here and there we shall invert the roles of the two sorts.

\begin{definition}\label{def_parted-fla}
A \emph{2-parted formula\/} is a triple $\phi({\mr x}\,;{\gr z})$ consistig of a formula and two tuples of variables. The variables actually occurring in $\phi$ are all among ${\mr x},{\gr z}$. We write $\phi({\mr x}\,;{\gr z})^*$ for the parted formula where the order of the two tuples is switched.
\end{definition}


The \emph{ladder order\/} of a parted formula $\phi({\mr x}\,;{\gr z})$ is the largest $n\in\omega$ such that for some ${\mr a_i}\in\U^{|{\mr x}|}$ and ${\gr b_i}\in\U^{|{\gr z}|}$ 

\ceq{\hfill i< j}{\IFF}{\phantom{\neg}\phi({\mr a_i}\,;{\gr b_j})}\hfill for all $i,j<n$

If such an $n$ exists, we say that  $\phi({\mr x}\,;{\gr z})$ is \emph{stalbe}, otherwise we say it is \emph{instable}. Note that $\phi({\mr x}\,;{\gr z})$ is stable if and only if $\phi({\mr x}\,;{\gr z})^*$ is stable.

The following is arguably the most important fact about stable formulas.

\begin{theorem}\label{thm_def_stable_formula}
Any $\grD\subseteq\U^{|{\gr z}|}$ approximated by a stable formula is definable. Precisely, if $\phi({\mr x}\,;{\gr z})$ is a stable formula that approximates $\grD$ then there are ${\mr a_{1,1}},\dots,{\mr a_{n,m}}\in\U^{|{\mr x}|}$ such that 


\ceq{\hfill {\gr z}\in\grD}{\iff}{\bigvee^n_{i=1}\bigwedge^m_{j=1}\phi({\mr a_{i,j}}\,;{\gr z})}


\end{theorem}

\begin{proof}
The theorem follows immediately from the the following three lemmas.
\end{proof}





\begin{lemma}
If $\grD$ is approximated from below by a stable formula $\phi({\mr x}\,;{\gr z})$ then there are ${\mr a_1},\dots,{\mr a_n}\in\U^{|{\mr x}|}$ such that 


\ceq{\hfill {\gr z}\in\grD}{\iff}{\bigvee^n_{i=1}\phi({\mr a_i}\,;{\gr z})}


\end{lemma}

\begin{proof}
Choose ${\mr a_n}\in\U^{|{\mr x}|}$ such that ${\gr b_0},\dots,{\gr b_{n-1}}\in\phi({\mr a_n}\,;{\gr\U})\subseteq\grD$. Then, if possible, choose ${\gr b_n}$ such that

\ceq{\hfill{\gr b_n}}{\in}{\grD\sm\bigvee^n_{i=0}\phi({\mr a_i}\,;{\gr\U})}

and iterate the procedure. By stability, the procedure has to stop as some $n$, yielding the required  ${\mr a_1},\dots,{\mr a_n}$.
\end{proof}






\begin{lemma}
If $\grD$ is approximated by a stable formula $\phi({\mr x}\,;{\gr z})$ with ladder order $m$. Then the formula 


\ceq{\hfill\psi({\mr x_1},\dots,{\mr x_m}\,;{\gr z})}{\iff}{\bigwedge^m_{j=1}\phi({\mr x_j}\,;{\gr z})}

approximates $\grD$ from below.
\end{lemma}

\begin{proof}
Let $B\subseteq\grD$ be finite. It suffices to prove that there are some ${\mr a_1},\dots,{\mr a_m}$ such that $B^{|{\gr z}|}\subseteq\psi({\mr a_1},\dots,{\mr a_m}\,;{\gr\U})\subseteq\grD$. 

Choose ${\mr a_n}\in\U^{|{\mr x}|}$ such that $B^{|{\gr z}|}\subseteq\phi({\mr a_n}\,;{\gr\U})\subseteq\U^{|{\gr z}|}\sm\{{\gr b_0},\dots,{\gr b_{n-1}}\}$. Then, if possible, let ${\gr b_n}$ such that

\ceq{\hfill{\gr b_n}}{\in}{\bigwedge^n_{i=0}\phi({\mr a_i},{\gr\U})\sm\grD}

and iterate the procedure. The procedure has to stop at some $n\le m$. Hence the required parameters are ${\mr a_1},\dots,{\mr a_n}=\dots={\mr a_{m}}$.
\end{proof}



\begin{lemma}
If $\phi({\mr x}\,;{\gr z})$ is a stable formula then for every $m$ the formula 

\ceq{\hfill \psi({\mr x_1},\dots,{\mr x_m}\,;{\gr z})}{\iff}{\bigwedge^m_{j=1}\phi({\mr x_j}\,;{\gr z})}


is stable.
\end{lemma}

\begin{proof}
For legibility, we only prove that $\phi({\mr x_1}\,;{\gr z})\wedge\phi({\mr x_2}\,;{\gr z})$ is stable. Suppose not and let ${\mr a_i},{\mr c_i}\in\U^{|{\mr x}|}$ and ${\gr b_i}\in\U^{|{\gr z}|}$ be such that 


\ceq{\hfill i< j}{\IFF}{\phantom{\neg}\phi({\mr a_i}\,;{\gr b_j})\wedge\phi({\mr c_i}\,;{\gr b_j})}\hfill for all $i,j<\omega$


For every pair $j<i<\omega$
\end{proof}


\begin{remark}
Theorem~\ref{thm_def_stable_formula} is often stated in the following  apparently more general form. For every $\Aa\subseteq\U$ there are ${\mr a_{1,1}},\dots,{\mr a_{n,m}}\in \Aa^{|{\mr x}|}$ such that

\ceq{\hfill {\gr z}\in\grD}{\iff}{\bigvee^n_{i=1}\bigwedge^m_{j=1}\phi({\mr a_{i,j}}\,;{\gr z})}\hfill for every ${\gr z}\in\Aa^{|{\gr z}|}$

The proof is exactly the same. Note that the equivalence above may not be extended outside of $\Aa$.\QED 
\end{remark}


{
\def\grB{{\gr{\EuScript B}}}
\def\mrA{{\mr{\EuScript A}}}
The following is a version of Harrinton's mysterious 
Lemma (cfr.~\cite[Lemma 8.3.4]{TZ}).
\begin{lemma}\label{lem_harrington}
Let $\phi({\mr x}\,;{\gr z})\in L$ be a stable formula and suppose $\grB\subseteq\U^{|{\gr z}|}$ and $\mrA\subseteq\U^{|{\mr x}|}$ are approximated by $\phi({\mr x}\,;{\gr z})$ and  $\phi({\mr x}\,;{\gr z})^*$, respectively. Then one of the following occurs

\begin{itemize}
\item[1.]
\ssf{a.}\kern2ex$\phi({\mr x}\,;{\gr z})^{\phantom{*}}\wedge\ {\mr x}\in{\mrA}$ approximates $\grB$ and\\
\ssf{b.}\kern2ex$\phi({\mr x}\,;{\gr z})^*\wedge\ {\gr z}\in{\grB}$ approximates $\mrA$. 

\item[2.]
\ssf{a.}\kern2ex$\phi({\mr x}\,;{\gr z})^{\phantom{*}}\wedge\ {\mr x}\notin{\mrA}$ approximates $\grB$ and\\
\ssf{b.}\kern2ex$\phi({\mr x}\,;{\gr z})^*\wedge\ {\gr z}\notin{\grB}$ approximates $\mrA$. 
\end{itemize}

\end{lemma}
\begin{proof}
Note that at least one of \ssf{1a} or \ssf{2a} occurs. Hence it sufficies to prove that 
\ssf{1a}$\IMP$\ssf{1b} and 
\ssf{2a}$\IMP$\ssf{2b}. We only prove the first. The second follows because $\neg\grB$ and $\neg\mrA$ are approximated by $\neg\phi({\mr x}\,;{\gr z})$ and  $\neg\phi({\mr x}\,;{\gr z})^*$, respectively.


Assume \ssf{1a} and negate \ssf{1b} for a contradiction.  Then \ssf{2b} holds. Pick ${\mr a_0}$ arbitrarily, then recursively find ${\gr b_i}\notin\grB$ and  ${\mr a_i}\in\mrA$ such that


{\def\medrel#1{\parbox[t]{12ex}{$\displaystyle\kern2ex #1$}}

\ceq{\hfill\phi(\U\,;{\gr b_i})}{ =_{{\mr a_0},\dots,{\mr a_{i-1}}}}{\mrA} 


\ceq{\hfill\phi({\mr a_i}\,;\U)}{ =_{{\gr b_0},\dots,{\gr b_i}}}{\grB}
}

Then

{\def\medrel#1{\parbox[t]{6ex}{$\displaystyle\kern2ex #1$}}

\ceq{\hfill i< j}{\IMP}{\phantom{\neg}\phi({\mr a_i}\,;{\gr b_j})}

\ceq{\hfill j\le i}{\IMP}{\neg\phi({\mr a_i}\,;{\gr b_j})}

}

Which contradicts the stability of $\phi({\mr x}\,;{\gr z})$.
\end{proof}



\def\mrsA{{\mr\Aa'}}
\def\grsB{{\gr\B'}}
\begin{corollary}
Under the same assumptions as in Lemma~\ref{lem_harrington}. Let $\<\U',\mrsA,\grsB\>$ be a saturated  elementary extension of  $\<\U,\mrA,\grB\>$ of larger cardinality. Then there are an  ${\mr a'}\in\U^{|{\mr x}|}$ and a ${\gr b'}\in\U^{|{\gr z}|}$ such that $\phi({\mr\U'}\,;{\gr b'})=_{\U,\, {\mr a'}}\mrsA$ and $\phi({\mr a'}\,;{\gr\U'})=_{\U,\,{\gr b'}}\grsB$.
\end{corollary}
\begin{proof}
Note that \ssf{1} in the lemma is equivalent to

\begin{itemize}
\item[1'.]
\ssf{a.}\kern2ex there is an ${\mr a'}\in{\mrA'}$  such that $\phi({\mr a'}\,;{\gr\U'})=_{\U}\grsB$ and\\
\ssf{b.}\kern2ex   there is a\phantom{n} ${\gr b'}\in\grsB$ such that $\phi({\mr\U'}\,;{\gr b'})=_{\U}\mrsA$. 
\end{itemize}

If in the above we replace $\in$ by $\notin$, we obtain a property equivalent to \ssf{2}.

Let  ${\mr a'}\in\U^{|{\mr x}|}$ be such that $\phi({\mr a'}\,;{\gr\U'})=_{\U}\grsB$. Suppose first that ${\mr a'}\in\mrsA$. Then \ssf{1'a} holds and by the lemma we obtain \ssf{1b}. By compactness there is ${\gr b'}\in\grsB$ such that $\phi({\mr\U'}\,;{\gr b'})=_{\U,\,{\mr a'}}\mrsA$. Then $\phi({\mr a'}\,;{\gr b'})$ holds, so $\phi({\mr a'}\,;{\gr\U'})=_{\U,\,{\gr b'}}\grsB$ follows. A similar argument apply if  ${\mr a'}\notin\mrsA$.
\end{proof}
\begin{proposition}
Suppose $\grD$ is approximated by $\phi({\mr x}\,;{\gr z})\in L(A)$, a stable formula, and Lascar invariant over $A$.  Then $\grC\ \equiv_{\acl^\eq A}\grD$ implies $\grC=\grD$. 
\end{proposition}

\begin{proof}
\def\grsC{{\gr\C'}}
\def\grsD{{\gr\D'}}
\def\mrA{{\mr\Aa}}
\def\mrsA{{\mr\Aa'}}
Let ${\gr b}\in\U^{|{\gr z}|}$ be arbitrary, we prove that ${\gr b}\in\grC\iff{\gr b}\in\grD$. Let $\phi({\mr\U}\,;{\gr b})=\mrA$. Let $\<\U',\mrsA,\grsC,\grsD\>$ be a saturated elementary extension of $\<\U,\mrA, \grC,\grD\>$ of larger cardinality. 

Let ${\mr c'}\in\U^{|{\mr x}|}$ be such that $\phi({\mr c'}\,;{\gr\U'})\cap\U^{|{\gr z}|}=\grC$. By saturation and elementarity, there is a ${\mr d'}\equiv_A{\mr c'}$ such that $\phi({\mr d'}\,;{\gr\U'})\cap\U^{|{\gr z}|}=\grD$. Then there is an 

or some 
\end{proof}


We say that $\D\subseteq\U^{|{\gr z}|}$ is \emph{approximable\/} by $\phi({\mr x}\,;{\gr z})$ if for every finite $B\subseteq\U^{|{\gr z}|}$ there is a ${\mr a}\in\U^{|{\mr x}|}$ such that $\phi({\mr a}\,;B)=\D\cap B$. For $A\subseteq\U$, we say approximable \emph{over $A$\/} if we can further require that ${\mr a}\in A^{|{\mr x}|}$. We say \emph{almost over $A$\/} if  ${\mr a}\in M^{|{\mr x}|}$ for every model $M$ containing $A$.

\begin{proposition}
If\/ $\D$ is\/ approximable from $A$, then it is invariant over $A$. If it is\/ ap\-prox\-i\-ma\-ble almost over $A$, then it is Lascar invariant over $A$.
\end{proposition}

\begin{proof}
Suppose for a contradiction that $f[\D]\neq\D$ for some $f\in\Aut(\U/A)$. Pick some ${\gr b}$ such that ${\gr b}\in\D\niff f{\gr b}\in\D$. Clearly $\big\{{\gr b},f{\gr b}\big\}\cap\D$ cannot be the trace on $\big\{{\gr b},f{\gr b}\big\}$ of a set definable over $A$. The same argument apply to prove Lascar invariance.
\end{proof}

Tentativo

}

\begin{proposition}
Let $A\subseteq\U$ and $\D\subseteq\U^{|{\gr z}|}$. Suppose that some stable formula $\phi({\mr x}\,;{\gr z})\in L$ approximates $\D\subseteq\U^{|{\gr z}|}$. Then the following are equivalent:
\begin{itemize}
\item[1.] $\D$ is invariant over $\acl^\eq A$;
\item[2.] $\D$ is definable over $\acl^\eq A$.
\end{itemize}
\end{proposition}

\begin{proof}
Only \ssf{1}$\IMP$\ssf{2} requires a proof. Assume \ssf{1}. By stability, $\D$ is definable. By invariance, $\D\in M^\eq$ for every model $M$ containing $A$ and \ssf{2} follows.
\end{proof}


\begin{proposition}
If $\grC\equiv_{\acl^\eq A}\grD$ then there is an $M$ containing $A$, such that  $\grC\equiv_M\gr\D$. 
\end{proposition}

\begin{proof}

\end{proof}

Let $p(x)\subseteq L(\U\,;\grC,\grD)$ be a type which is finitely satisfiable in every model containing $A$. 



\section{Notes and references}
\begin{biblist}[]\normalsize



\bib{CS}{article}{
   author={Chernikov, Artem},
   author={Simon, Pierre},
   title={Externally definable sets and dependent pairs},
   journal={Israel J. Math.},
   volume={194},
   date={2013},
   number={1},
   pages={409--425},
   note={\href{https://arxiv.org/abs/1007.4468}{ArXiv:1007.4468}},
   %issn={0021-2172},
   %doi={10.1007/s11856-012-0061-9},
}

\bib{TZ}{book}{
   author={Tent, Katrin},
   author={Ziegler, Martin},
   title={A course in model theory},
   series={Lecture Notes in Logic},
   volume={40},
   publisher={Association for Symbolic Logic, Cambridge
   University Press},
   date={2012},
   pages={x+248},
   %isbn={978-0-521-76324-0},
   %doi={10.1017/CBO9781139015417},
}



\end{biblist}


\end{document}


