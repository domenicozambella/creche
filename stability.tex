% !TEX root = creche.tex
\chapter{Stability}
\label{stability}

\def\medrel#1{\parbox[t]{6ex}{$\displaystyle\hfil #1$}}
\def\ceq#1#2#3{\parbox{25ex}{$\displaystyle #1$}\medrel{#2}$\displaystyle  #3$}

\noindent\llap{\textcolor{red}{\Large\warning}\kern1.5ex}\ignorespaces
Chapter under revision.

In this chapter we fix a signature $L$, a complete theory $T$ without finite models, and a saturated model $\U$ of inaccessible cardinality $\kappa>|L|$.
The notation and implicit assumptions are as in Section~\ref{monster}.

%%%%%%%%%%%%%%%%%%%%%%%%%%%%%
%%%%%%%%%%%%%%%%%%%%%%%%%%%%%
%%%%%%%%%%%%%%%%%%%%%%%%%%%%%
%%%%%%%%%%%%%%%%%%%%%%%%%%%%%
\section{Externally definable sets}
\label{externally}

\def\ceq#1#2#3{\parbox{25ex}{$\displaystyle #1$}\medrel{#2}$\displaystyle  #3$}

Let $\grC,\grD\subseteq\U^{\gr z}$.
The set $\grD\cap A^{\gr z}$ is called the \emph{trace\/} of $\grD$ over $A$.
We write $\grC=_A\grD$ if  $\grC$ and $\grD$ have the same trace on $A$.

Let $p({\mr x})\subseteq L(\U)$ be a consistent type.
Recall from Section~\ref{invariant_sets} that for every formula $\phi({\mr x}\,;{\gr z})\in L$ we define

\ceq{\hfill\emph{$\gr\D_{p,\phi}$}}{=}{\Big\{{\gr a}\in\U^{\gr z}\ :\ \phi({\mr x}\,;{\gr a})\in p\Big\}.}

We say that $\grD$ is \emph{externally definable\/} if it is of the form ${\gr\D_{p,\phi}}$ for a type $p(x)$ in $S(\U)$ or $S_\phi(\U)$.
We say that $\grD$ is externally definable \emph{by $p({\mr x})$ and $\phi({\mr x}\,;{\gr z})$}.

Equivalently, a set $\grD$ is externally definable if it is the trace over $\U$ of a set which is definable in some elementary extension of $\U$.
More precisely, $\grD$ is the trace on $\U$ of a set of the form $\phi({\mr ^{*\kern-.3ex}b}\,;{\gr ^{*\kern-.2ex}\U})$ where $ ^{*\kern-.2ex}\U$ is elementary extension of $\U$ and ${\mr  ^{*\kern-.3ex}b}\in {}^{*\kern-.2ex}\U^{\mr x}$.
The latter interpretation explains the terminology.

\noindent\llap{\textcolor{red}{\Large\warning}\kern1.5ex}%
We prefer to deal with external definability in a different, though equivalent, way.
This is not the most common approach.

\begin{definition}\label{def_approx}
We say that $\grD$ is \emph{approximated\/} by the formula $\phi({\mr x}\,;{\gr z})$ if for every finite $B$ there is a tuple ${\mr a}\in\U^{\mr x}$ such that $\phi({\mr a}\,;B^{\gr z})=\grD\cap B^{\gr z}$.
We call $\phi({\mr x}\,;{\gr z})$ the \emph{sort} of $\grD$.
If in addition $\phi({\mr a}\,;\U^{\gr z})\subseteq\grD$, we say that  $\grD$ is \emph{approximated from below}.
Equivalently, we say that  $\grD$ is approximated from below if for every finite $B^{\gr z}\subseteq\grD$ there is a tuple ${\mr a}\in\U^{\mr x}$ such that  $B^{\gr z}\subseteq\phi({\mr a}\,;\U^{\gr z})\subseteq\grD$.
The dual notion of \emph{approximation from above\/} is defined as expected (and coincides with $\neg\grD$ being approximated by $\neg\phi({\mr x}\,;{\gr z})$ from below).
\end{definition} 

The following proposition is clear by compactness.

\begin{proposition}\label{prop_approx=external}
   For every $\grD$ the following are equivalent
   \begin{itemize}
   \item[1.] $\grD$ is approximated by $\phi({\mr x}\,;{\gr z})$
   \item[2.] $\grD$ is externally definable by $\phi({\mr x}\,;{\gr z})$.
   \end{itemize}
\end{proposition}

The rest of this section is only required in Chapter~\ref{vc}.

Approximability from below is an adaptation to our context of the notion of \textit{having an honest definition} in \cite{CS}.
% We say that the global type $p\in S_{{\mr x}}(\U)$ is \textit{honestly definable\/} if for every $\phi({\mr x}\,;{\gr z})\in L$ the set ${\gr\D_{p,\phi}}$ is approximated from below (by some formula).
% We say that $p$ is \emph{definable\/} if the sets ${\gr\D_{p,\phi}}$ are all definable (over $\U$).
% Note that the terminology is misleading: honestly definable is weaker than definable.

\begin{example}
Every definable set is trivially approximable.
Sets may be approximable by different formulas.
For instance, if $T=T_{\rm dlo}$, then $\D=\{z\in\U:a\le z\le b\}$ is approximable both from below and from above by the formula $x_1<z<x_2$ though it is not definable by this formula.

Now, let $T=T_{\rm rg}$.
Then every $\D\subseteq\U$ is approximable and, when $\D$ has small infinite cardinality, it is approximable from above but not from below.
\end{example}

In Definition~\ref{def_approx}, the sort $\phi({\mr x}\,;{\gr z})$ is fixed (otherwise any set would be approximable) but this requirement of uniformity may be dropped if we allow $B$ to have larger cardinality.

\begin{proposition}\label{lem_approx_nonunif}
For every $\grD$ the following are equivalent
\begin{itemize}
\item[1.] $\grD$ is approximable
\item[2.] for every $C$ of cardinality $\le|L|$ there is $\psi({\gr z})\in L(\U)$ such that $\psi(C^{\gr z})=\grD\cap C^{\gr z}$.
\end{itemize}
\end{proposition}

\begin{proof}
To prove \ssf{2}$\IMP$\ssf{1} assume \ssf{2} and negate \ssf{1} for a contradiction.
For each formula $\psi({\mr x}\,;{\gr z})\in L$ choose a finite set $B$ such that $\psi({\mr b}\,;\U^{\gr z})\neq_B\grD$ for every ${\mr b}\in\U^{\mr x}$.
Let $C$ be the union of all these finite sets.
Clearly $|C|\le|L|$.
By \ssf{2} there are a formula $\phi({\mr x}\,;{\gr z})$ and a tuple ${\mr c}$ such that $\phi({\mr c}\,;C^{\gr z})=\grD\cap C^{\gr z}$, contradicting the definition of $C$.
\end{proof}

\begin{remark}\label{prop_approx_el_eq}
If $\grD\subseteq\U^{\gr z}$ is approximated by $\phi({\mr x}\,;{\gr z})$ then so is any $\grC$ such that $\grC\equiv\grD$, see Section~\ref{expansions} for the notation.
In fact, if the set $\grD$ is approximable by $\phi({\mr x}\,;{\gr z})$ then for every $n$

\hfil$\displaystyle\A {\gr z_1},\dots,{\gr z_n}\;\E {\mr x}\ \bigwedge^n_{i=1}\big[\phi({\mr x}\,;{\gr z_i})\ \iff\ {\gr z_i}\in{\gr \D}\big]$.


So the same holds for any $\grC\equiv\grD$.
A similar remark apply to approximability from below and from above (e.g.\@ for approximability from below, add the conjunct $\A {\gr z}\,\big[\phi({\mr x}\,;{\gr z})\imp {\gr z}\in\grD\big]$ to the formula above).
\end{remark}

From the following easy observation of Chernikov and Simon~\cite{CS} we obtain an interesting (and misterious) quantifier elimination result originally due to Shelah, see Corollary~\ref{corol_sh_exp_qe} below.

% \begin{proposition}
% Let $\C\subseteq\U^{|{\gr z},w|}$ be approximated from below by the formula $\phi({\mr x}\,;{\gr z},w)$. Then $\grD=\big\{{\gr z}:\E w\ \big({\gr z}\,w\in\C\big)\big\}$ is approximated from below by the formula $\E w\,\phi({\mr x}\,;{\gr z}\,w)$.
% \end{proposition}
% 
% \begin{proof}
% Let $B\subseteq\U$ be finite. 
% We want ${\mr a}\in\U^{\mr x}$ such that 
% 
% \ceq{\ssf{a.}\hfill \E w\ \big({\gr b}\,w\in\C\big)}{\iff}{\E w\,\phi({\mr a}\,;{\gr b}\,w)}\hfill for every ${\gr b}\in B^{\gr z}$
% 
% \ceq{\ssf{b.}\hfill \A {\gr z}\;\Big[\E w\,\phi({\mr a}\,;{\gr z}\,w)}{\imp}{\E w\ \big({\gr z}\,w\in\C\big)\Big]}
% 
% Let $C\subseteq\U$ be a finite set such that 
% 
% \ceq{\ssf{c.}\hfill \E w\in C^{|w|}\ \big({\gr b}\,w\in\C\big)}{\iff}{\E w\ \big({\gr b}\,w\in\C\big)}\hfill for every ${\gr b}\in B^{\gr z}$
% 
% As $\C$ is approximable from below, there is an ${\mr a}$ such that
% 
% \ceq{\ssf{a'.}\hfill {\gr b}\,c\in\C}{\iff}{\phi({\mr a}\,;{\gr b}\,c)}\hfill for every ${\gr b}\,c\in \big(B\cup C\big)^{|{\gr z}\,w|}$
% 
% \ceq{\ssf{b'.}\hfill \A {\gr z}\,w\ \Big[\phi({\mr a}\,;{\gr z}\,w)}{\imp}{{\gr z}\,w\in\C\Big]}
% 
% We obtain \ssf{b} from \ssf{b'} simply by logic. 
% Implication $\imp$ in \ssf{a} follows from \ssf{a'} and \ssf{c}. 
% Implication $\pmi$ follows from \ssf{b}.
% \end{proof}

\begin{proposition}\label{prop_sh_exp_qe}
Let $\C\subseteq\U^{y\,{\gr z}}$ be approximated from below by the formula $\phi({\mr x}\,;y\,{\gr z})$.
Then $\grD=\big\{{\gr z}:\E y\ \big(y\,{\gr z}\in\C\big)\big\}$ is approximated from below by $\E y\,\phi({\mr x}\,;y\,{\gr z})$.
\end{proposition}

\begin{proof}
Let $B\subseteq\U$ be finite.
We want ${\mr a}\in\U^{\mr x}$ such that

\ceq{\ssf{a.}\hfill \E y\ \big(y\,{\gr b}\in\C\big)}{\iff}{\E y\,\phi({\mr a}\,;y\,{\gr b})}\hfill for every ${\gr b}\in B^{\gr z}$

\ceq{\ssf{b.}\hfill \A {\gr z}\;\Big[\E y\,\phi({\mr a}\,;y\,{\gr z})}{\imp}{\E y\ \big(y\,{\gr z}\in\C\big)\Big]}

Let $D\subseteq\U$ be a finite set such that 

\ceq{\ssf{c.}\hfill \E y\in D^y\ \big(y\,{\gr b}\in\C\big)}{\iff}{\E y\ \big(y\,{\gr b}\in\C\big)}\hfill for every ${\gr b}\in B^{\gr z}$

As $\C$ is approximable from below, there is an ${\mr a}$ such that

\ceq{\ssf{a$'$.}\hfill d\,{\gr b}\in\C}{\iff}{\phi({\mr a}\,;d\,{\gr b})}\hfill for every $d\,{\gr b}\in \big(D\cup B\big)^{y\,{\gr z}}$

\ceq{\ssf{b$'$.}\hfill \A y\,{\gr z}\ \Big[\phi({\mr a}\,;y\,{\gr z})}{\imp}{y\,{\gr z}\in\C\Big]}

We obtain \ssf{b} from \ssf{b$'$} simply by logic. 
Implication $\imp$ in \ssf{a} follows from \ssf{a$'$} and \ssf{c}. 
Implication $\pmi$ follows from \ssf{b}.
\end{proof}

\begin{corollary}
If $p\in S_{{\mr x}}(\U)$ is honestly definable then the family of sets externally definable by $p$ is closed under quantifiers and Boolean combinations. 
\end{corollary}

\begin{proof}
The sets externally definable by $p({\mr x})$ are always closed under Boolean operations. By the proposition above, they are closed under quantifiers. 
\end{proof}

% Let ${\gr z}=\<z_i:i<\lambda\>$ and let $\grD\subseteq\U^{\gr z}$. 
% For $I\subseteq\lambda$ we write $\E z_{\restriction I}\,\grD$ for the projection of $\grD$ to the $\lambda\sm I$ coordinates, i.e.\@ the set $\{b_{\restriction\lambda\sm I}: {\gr b}\in\grD\}$. 
% Expand the language with a symbol for $\E z_{\restriction I}\,\grD$ for every finite $I\subseteq\lambda$. 
% Then the theory of $\U$ in the expanded language has positive $\Delta$-quantifier elimination, for $\Delta=L\cup\big\{\E z_{\restriction I}\,\grD: I\subseteq\lambda\textrm{ finite}\big\}$, see Section~\hyperref[eliminazionequantificatoricriterio]{\ref*{elimination}.\ref*{eliminazionequantificatoricriterio}}.

%%%%%%%%%%%%%%%%%%%%%%%%%%%%%%%%%%
%%%%%%%%%%%%%%%%%%%%%%%%%%%%%%%%%%
%%%%%%%%%%%%%%%%%%%%%%%%%%%%%%%%%%
\section{Ladders and definability}


%Here we write \emph{$\phi({\mr x}\,;{\gr z})^*$\/} for the partitioned formula where the order of the two tuples is inverted (i.e.\@ it is the opposite of what displayed).

Let $\R\subseteq\U^{\mr x}\times\U^{\gr z}$ be a binary relation invariant over $A$.
We denote by $\R({\mr x}\,;{\gr z})$ the preditate associated to $\R$.
We say that $\<{\mr a_i}\,;{\gr b_i} : i<\alpha\>$ is a \emph{ladder\/} of length $\alpha$ for $\R({\mr x}\,;{\gr z})$ if for every $i<j<\alpha$


\ceq{\hfill\R({\mr a_i}\,;{\gr b_j})}{\wedge}{\neg\R({\mr a_j}\,;{\gr b_i})}

We say that $\R({\mr x}\,;{\gr z})$ is \emph{stable\/} if there is no ladder of length $\omega$. Otherwise we say it is \emph{unstable} or that it has the \emph{order property}.
We will mainly deal with relations that are definable or type-definable.
We will say that a formula or a type is stable accordingly.
The proof of the following easy fact is left to the reader.

\begin{fact}\label{}
  The following are equivalent for every $p({\mr x}\,;{\gr z})\subseteq L(A)$
  \begin{itemize}
    \item [1.] $p({\mr x}\,;{\gr z})$ is stable
    \item [1.] for some finite $n$, there is no ladder of length $n$ for $p({\mr x}\,;{\gr z})$
    \item [2.] there is $q({\mr x}\,;{\gr z})\subseteq L(A)$ equivalent to $p({\mr x}\,;{\gr z})$ containing only stable formulas.
  \end{itemize}
\end{fact}

We can required that the ladder is an indiscernible sequence. 

\begin{theorem}\label{thm_sability_indiscernibility}
  Let $\R({\mr x}\,;{\gr z})$ be invariant over $A$.
  Then the following are equivalent
  \begin{itemize}
    \item[1.] $\R({\mr x}\,;{\gr z})$ is stable
    \item[2.] $\R({\mr a_0}\,;{\gr b_1})\imp\R({\mr a_1}\,;{\gr b_0})$ for every $A$-indiscernible sequence $\<{\mr a_i}\,;{\gr b_i} : i<\omega\>$.
  \end{itemize}
\end{theorem}

\begin{proof}
  (2$\IMP$1) \ Assume $\R({\mr x}\,;{\gr z})$ is unstable and that this is witnessed by $\<{\mr a_i}\,;{\gr b_i} : i<\omega\>$.
  Let $\<{\mr a'_i}\,;{\gr b'_i} : i<\omega\>$ be a sequence of $A$-indiscernibles with the same EM-type as $\<{\mr a_i}\,;{\gr b_i} : i<\omega\>$.
  By invariance $\R({\mr a'_0}\,;{\gr b'_1})$ and $\neg\R({\mr a'_1}\,;{\gr b'_0})$.

  (2$\IMP$1) Immediate by indiscernibility.
\end{proof}
%It is clear that $p({\mr x}\,;{\gr z})$ is stable if and only if $p({\mr x}\,;{\gr z})^*$ is stable.

%Suppose $p({\mr x}\,;{\gr z})$ is unstable and let $\<{\mr a_i}\,;{\gr b_i} : i<\omega\>$ be a ladder sequence. By padding redundant variables, we can read $p({\mr x}\,;{\gr z})$ as a formula $p({\mr x}\,{\gr z}\,;{\mr y}\,{\gr w})$. Then $p({\mr x}\,{\gr z}\,;{\mr y}\,{\gr w})$ defines the order of the sequence $\<{\mr a_i}\,{\gr b_i} : i<\omega\>$. For this reason it is also common to say that $p({\mr x}\,;{\gr z})$ has the \emph{order property}, this simply means that it is unstable. 

\begin{lemma}\label{lem_stab_Boole}
  Every Boolean combination of stable relations is stable.
  Moreover, if $\R({\mr x}\,;{\gr z})$ then  $\R^{-1}({\mr x}\,;{\gr z})$ and $\R({\mr x},x'\,;{\gr z},z')$ are also stable formula (in the latter $x'$ and $z'$ are dummy variables).
\end{lemma}

\begin{proof}
  It suffices to consider conjunction and negation.
  Let $\R_i({\mr x}\,;{\gr z})$, for $i=1,2$,  be stable relations that are invariant over $A$.
  By Theorem~\ref{thm_sability_indiscernibility}, it is immediate that $\R_1({\mr x}\,;{\gr z})\wedge\R_2({\mr x}\,;{\gr z})$ is stable.
  As for negation, let $\<{\mr a_i}\,;{\gr b_i} : i<\omega\>$ be a ladder for $\R({\mr x}\,;{\gr z})$ that is indiscernible over $A$.
  By Exercise~\ref{ex_symmetry_ind}, there is a $\<{\mr a'_i}\,;{\gr b'_i} : i<\omega\>$ of $A$-indiscernibles such that ${\mr a'_0},{\gr b'_0}={\mr a_1}{\gr b_1}$ and ${\mr a'_1},{\gr b'_1}={\mr a_0}{\gr b_0}$.
  This sequence witnesses the instability of $\neg\R({\mr x}\,;{\gr z})$.

  The second claim is immediate.
\end{proof}

The following theorem claims what is arguably one of the most important properties of stable formulas: any set that is externally definable by a stable formula is definable (by a related formula).

\begin{theorem}\label{thm_def_stable_formula}
Any $\grD\subseteq\U^{\gr z}$ approximated by a stable formula is definable.
Precisely, if $\phi({\mr x}\,;{\gr z})$ is a stable formula that approximates $\grD$ then there are ${\mr a_{i,j}}\in\U^{\mr x}$ such that 

\ceq{\hfill {\gr z}\in\grD}{\iff}{\bigvee^n_{i=1}\bigwedge^m_{j=1}\phi({\mr a_{i,j}}\,;{\gr z})}
\end{theorem}

Theorem~\ref{thm_def_stable_formula2} will prove the converse: if every set approxiamted by $\phi({\mr x}\,;{\gr z})$ is definable then $\phi({\mr x}\,;{\gr z})$ is stable.

\begin{proof}
  The theorem follows immediately from the two lemmas below.
\end{proof}

\begin{lemma}
If $\grD$ is approximated from below by a stable formula $\phi({\mr x}\,;{\gr z})$ then

\ceq{\hfill {\gr z}\in\grD}{\iff}{\bigvee^n_{i=0}\phi({\mr a_i}\,;{\gr z})}

for some ${\mr a_0},\dots,{\mr a_n}\in\U^{\mr x}$. 
\end{lemma}

\begin{proof}
The elements ${\mr a_0},\dots,{\mr a_n}$ are defined recursively together with some auxiliary elements ${\gr b_0},\dots,{\gr b_{n-1}}\in\grD$.

Suppose ${\gr b_0},\dots,{\gr b_{n-1}}$ have been defined (this assumption is empty if $n=0$).
We first define ${\mr a_n}$, then ${\gr b_n}$. 
Choose ${\mr a_n}\in\U^{\mr x}$ such that ${\gr b_0},\dots,{\gr b_{n-1}}\in\phi({\mr a_n}\,;\U^{\gr z})\subseteq\grD$.
This is possible because $\grD$ is approximated from below.
Now, if possible, choose ${\gr b_n}$ such that

\ceq{\hfill{\gr b_n}}{\in}{\grD\sm\bigcup^n_{i=0}\phi({\mr a_i}\,;\U^{\gr z})}.

Then $\<{\mr a_i}\,;{\gr b_i} : i\le n\>$ is a ladder sequence. 
By stability, for some $n$, the tuple ${\gr b_n}$ does not exist.
This yields the required ${\mr a_0},\dots,{\mr a_n}$.
\end{proof}

\begin{lemma}\label{lem_stab_approx_below}
If $\grD$ is approximated by a stable formula $\phi({\mr x}\,;{\gr z})$.
Then, for some $m$, the formula 

\ceq{\hfill\psi({\mr x_0},\dots,{\mr x_m}\,;{\gr z})}{=}{\bigwedge^m_{j=0}\phi({\mr x_j}\,;{\gr z})}

approximates $\grD$ from below.
\end{lemma}

\begin{proof}
Let $m$ be such that there is no ladder sequence for $\phi({\mr x}\,;{\gr z})$ of length greater then $m$.
Let $C^{\gr z}\subseteq\grD$ be finite.
We prove that there are some ${\mr a_0},\dots,{\mr a_m}$ such that $C^{\gr z}\subseteq\psi({\mr a_0},\dots,{\mr a_m}\,;\U^{\gr z})\subseteq\grD$.
As in the proof above, we define by recursion a ladder sequence for $\phi({\mr x}\,;{\gr z})$.
Suppose that ${\mr a_0},\dots,{\mr a_{n-1}}$ and ${\gr b_0},\dots,{\gr b_{n-1}}\notin\grD$ have been defined.
We first define ${\mr a_n}$, then ${\gr b_n}$. 
Choose ${\mr a_n}\in\U^{\mr x}$ such that 

\hfil$C^{\gr z}\ \subseteq\ \phi({\mr a_n}\,;\U^{\gr z})\ \subseteq\ \U^{\gr z}\sm\{{\gr b_0},\dots,{\gr b_{n-1}}\}$.

This ${\mr a_n}$ exists, because $\grD$ is approximated by $\phi({\mr x}\,;{\gr z})$.
(Apply Definition~\ref{def_approx} with any $B$ such that $C^{\gr z}\cup\{{\gr b_0},\dots,{\gr b_{n-1}}\}\subseteq B^{\gr z}$.)
Then, if possible, let ${\gr b_n}$ such that

\ceq{\hfill{\gr b_n}}{\in}{\bigcap^n_{i=0}\phi({\mr a_i},\U^{\gr z})\sm\grD}

This procedure has to stop at some $n\le m$.
Hence the required parameters are ${\mr a_1},\dots,{\mr a_n}={\mr a_{n+1}}=\dots={\mr a_{m}}$.
\end{proof}

% \begin{lemma}\label{lem_stable3}
% If $\phi({\mr x}\,;{\gr z})$ is a stable formula then for every $m$ the formula $\psi({\mr x_0},\dots,{\mr x_m}\,;{\gr z})$ defined above is stable.
% \end{lemma}

% \begin{proof}
% It suffices to prove that if $\phi_1({\mr x_1}\,;{\gr z})\wedge\phi_2({\mr x_2}\,;{\gr z})$ is unstable then one of the formulas $\phi_n({\mr x_i}\,;{\gr z})$ is unstable. For simplicity, we use that instability implies the existence of an infinite ladder (this uses compactness, apparently contradicting Remark~\ref{rem_sability_no_compactness}). We leave to the reader to adapt the argument so that compactness is not required.

% Let ${\mr a^1_i},{\mr a^2_i}\in\U^{\mr x}$ and ${\gr b_i}\in\U^{\gr z}$ be such that 

% \ceq{\hfill i\le j}{\IFF}{\phantom{\neg}\phi_1({\mr a^1_i}\,;{\gr b_j})\wedge\phi_2({\mr a^2_i}\,;{\gr b_j})}\hfill for all $i,j<\omega$

% For $n=1,2$ let $H_n\subseteq{\omega\choose 2}$ contain those pairs $j<i$ such that $\neg\phi_n({\mr a^n_i}\,;{\gr b_j})$. 
% By the equivalence above $H_1\cup H_2={\omega\choose 2}$. 
% By the Ramsey Theorem there is an infinite set $H$ such that ${H\choose 2}\subseteq H_n$ for at least one of $n=1,2$. Suppose $H_1$ for definiteness. So, we obtain an infinite sequence $a^1_i$, $b_i$ such that

% \ceq{\hfill j<i}{\IFF}{\neg\phi_1({\mr a^1_i}\,;{\gr b_j})}\hfill for all $i,j<\omega$

% hence $\phi_1({\mr x_1}\,;{\gr z})$ is unstable.
% \end{proof}

By Lemma~\ref{lem_stab_Boole} the formula $\psi({\mr x_0},\dots,{\mr x_m}\,;{\gr z})$ is stable therefore this concludes the proof of Theorem~\ref{thm_def_stable_formula}. 

\begin{remark}\label{rem_sability_no_compactness}
  Theorem~\ref{thm_def_stable_formula} is often casted in the following  apparently more general form.
  Let $\mrA\subseteq\U^{\mr x}$ and $\grB\subseteq\U^{\gr z}$.
  We say that $\phi({\mr x}\,;{\gr z})$ is stable between $\mrA$ and $\grB$ if for some $n$ no ladder of length $n$ exists with ${\mr a_i}\in\mrA$ and ${\gr b_i}\in\grB$.

  Let $\grD\subseteq\grB$ be approximable by $\phi({\mr x}\,;{\gr z})\wedge {\mr x}\in\mrA$.
  Then for any such $\grD$ there are ${\mr a_{1,1}},\dots,{\mr a_{n,m}}\in \mrA$ such that 
  
  \ceq{\hfill {\gr b}\in\grD}{\iff}{\bigvee^n_{i=1}\bigwedge^m_{j=1}\phi({\mr a_{i,j}}\,;{\gr b}).}
  
  The proof is essentially the same. 
  In fact, elementarity and saturation as only been used to prove that conjunctions of stable formulas is stable.
  Exercise~\ref{ex_stability_conjunction} suggests a finitary proof of this fact.
\end{remark} 

\begin{exercise}
  Prove Fact~\ref{fact_stability_compactness}.
\end{exercise}

\begin{exercise}
  Prove that if $p({\mr x}\,;{\gr z})$ admits ladder sequences of arbitrary finite length, then it admits a ladder sequence of infinite length.
\end{exercise}

\begin{exercise}
  Let $\phi(x,y)\in  L$, where $|x|=|y|=1$.
  Suppose there is an infinite set $A\subseteq\U$ such that $\phi(a,b)\niff\phi(b,a)$ for every two distinct $a,b\in A$.
  Prove that $\phi(x\,;y)$ is unstable.
\end{exercise}


% \begin{exercise}\label{ex_harrington}
%   Prove the following claim (a version of Harrington's mysterious Lemma cfr.~\cite[Lemma 8.3.4]{TZ}).
%   Let $\phi({\mr x}\,;{\gr z})\in L$ be a stable formula and suppose $\grB\subseteq\U^{\gr z}$ and $\mrA\subseteq\U^{\mr x}$ are approximated by $\phi({\mr x}\,;{\gr z})$ and  $\phi({\mr x}\,;{\gr z})^{\rm op}$, respectively.
%   Then at least one of the conditions \ssf{1} and \ssf{2} below occurs
  
%   \begin{itemize}
%   \item[1.]
%   \ssf{a.}\kern2ex$\phi({\mr x}\,;{\gr z})^{\phantom{{\rm op}}}\wedge\ {\mr x}\in{\mrA}$ approximates $\grB$ and\\
%   \ssf{b.}\kern2ex$\phi({\mr x}\,;{\gr z})^{\rm op}\wedge\ {\gr z}\in{\grB}$ approximates $\mrA$.
  
%   \item[2.]
%   \ssf{a.}\kern2ex$\phi({\mr x}\,;{\gr z})^{\phantom{{\rm op}}}\wedge\ {\mr x}\notin{\mrA}$ approximates $\grB$ and\\
%   \ssf{b.}\kern2ex$\phi({\mr x}\,;{\gr z})^{\rm op}\wedge\ {\gr z}\notin{\grB}$ approximates $\mrA$.
%   \end{itemize}
%   Hint: Note that at least one of \ssf{1a} or \ssf{2a} occurs.
%   Hence it suffices to prove that \ssf{1a}$\IMP$\ssf{1b} and \ssf{2a}$\IMP$\ssf{2b}. The two implications are essentially equivalent.
%   \begin{proof}
%   Note that at least one of \ssf{1a} or \ssf{2a} occurs.
%   Hence it suffices to prove that \ssf{1a}$\IMP$\ssf{1b} and \ssf{2a}$\IMP$\ssf{2b}.
%   We only prove the first.
%   The second follows because $\neg\grB$ and $\neg\mrA$ are approximated by $\neg\phi({\mr x}\,;{\gr z})$ and  $\neg\phi({\mr x}\,;{\gr z})^*$, respectively.
  
%   Assume \ssf{1a} and negate \ssf{1b} for a contradiction.
%   Then \ssf{2b} holds.
%   Pick ${\mr a_0}$ arbitrarily, then recursively find ${\gr b_i}\notin\grB$ and  ${\mr a_i}\in\mrA$ such that
  
%   {\def\medrel#1{\parbox[t]{12ex}{$\displaystyle\kern2ex #1$}}
  
%   \ceq{\hfill\phi(\U\,;{\gr b_i})}{ =_{{\mr a_0},\dots,{\mr a_{i-1}}}}{\mrA} 
  
%   \ceq{\hfill\phi({\mr a_i}\,;\U)}{ =_{{\gr b_0},\dots,{\gr b_i}}}{\grB}
%   }
  
%   Then
  
%   {\def\medrel#1{\parbox[t]{6ex}{$\displaystyle\kern2ex #1$}}
  
%   \ceq{\hfill i< j}{\IMP}{\phantom{\neg}\phi({\mr a_i}\,;{\gr b_j})}
  
%   \ceq{\hfill j\le i}{\IMP}{\neg\phi({\mr a_i}\,;{\gr b_j})}
  
%   }
  
%   Which contradicts the stability of $\phi({\mr x}\,;{\gr z})$.
%   \end{proof}
%   \end{exercise} 

  \begin{exercise}\label{ex_stability_conjunction}
    Prove that if $p({\mr x}\,;{\gr z})$ and $q({\mr x}\,;{\gr z})$ are stable between $\mrA$ and $\grB$, then so is $p({\mr x}\,;{\gr z})\cup q({\mr x}\,;{\gr z})$.
  \end{exercise}

%%%%%%%%%%%%%%%%%%%%%%
%%%%%%%%%%%%%%%%%%%%%%
%%%%%%%%%%%%%%%%%%%%%%
%%%%%%%%%%%%%%%%%%%%%%
%%%%%%%%%%%%%%%%%%%%%%
\section{Stability and the number of types}


The following proposition highlights the connection between the stability of $\phi({\mr x}\,;{\gr z})$ and the cardinality of $S_\phi(\U)$, equivalently the number of sets externally definable by $\phi({\mr x}\,;{\gr z})$. 

\begin{theorem}\label{thm_def_stable_formula2}
   The following are equivalent
   \begin{itemize}
     \item[1.] $\phi({\mr x}\,;{\gr z})$ is stable
     \item[2.] every subset of $\U^{\gr z}$ that is externally definable by $\phi({\mr x}\,;{\gr z})$ is definable
     \item[3.] there are $\le\kappa$ subsets of $\U^{\gr z}$ that are externally definable by $\phi({\mr x}\,;{\gr z})$
     \item[4.] there are $<2^\kappa$ subsets of $\U^{\gr z}$ that are externally definable by $\phi({\mr x}\,;{\gr z})$.
   \end{itemize}
 \end{theorem}
 
 \begin{proof}
 \ssf{1}$\IMP$\ssf{2} Clear by Proposition~\ref{prop_approx=external} and Theorem~\ref{thm_def_stable_formula}.
 
 \ssf{2}$\IMP$\ssf{3}$\IMP$\ssf{4} Obvious.
 
 \ssf{4}$\IMP$\ssf{1} Suppose that $\phi({\mr x}\,;{\gr z})$ is not stable.
 By compactness there is a ladder sequence  $\<{\mr a_i}\,;{\gr b_i} : i\in I\>$ where $I,<_I$ a dense linear order of cardinality $\kappa$ with $2^\kappa$ cuts, where by \textit{cut\/} we mean a subset $c\subseteq I$ that is closed downward.
 For every such $c\subseteq I$ we pick a global type
 
 \ceq{\hfill p_c({\mr x} ) }{\supseteq}{\big\{\phi({\mr x}\,;{\gr b_i})\iff i\in c\ :\ i\in I\big\}.}
 
 Clearly the sets ${\gr\D_{p_c,\,\phi} }$ are all distinct.
 \end{proof}


Binary trees of formulas have been introduced in Definition~\ref{def_tree_formulas}.
Here we restrict to trees of a particular form.
Namely, $\<\psi_s:s\in 2^{<\omega}\>$  where $\psi_\0=\top$ and for $s\in 2^{<\omega}$ and $i\in 2$ we have $\psi_{s^\frown 0}({\mr x})=\neg\phi({\mr x}\,;{\gr b_s})$ and $\psi_{s^\frown 1}({\mr x})=\phi({\mr x}\,;{\gr b_s})$.
If we define $\phi^0=\neg\phi$ and $\phi^1=\phi$ the condition of concistency becomes for every $s\in 2^\omega$ the type $\{\phi^{s_n}({\mr x}\,;{\gr b_{s\restriction n}})\ :\ n<\omega\}$.


% Set the overall layout of the tree
\tikzstyle{level 1}=[level distance=3.5cm, sibling distance=2.5cm]
\tikzstyle{level 2}=[level distance=3.5cm, sibling distance=1.2cm]
\tikzstyle{level 3}=[level distance=2.5cm, sibling distance=0.5cm]
\tikzstyle{level 4}=[level distance=0.5cm, sibling distance=0.5cm]

% Define styles for bags and leafs
\tikzstyle{bag0} = [text width=1.5ex, align=left]
\tikzstyle{bag} = [text width=7.5ex, align=right]
%\tikzstyle{end} = [circle, minimum width=3pt,fill, inner sep=0pt]

\def\leaf{...}

\quad
\begin{tikzpicture}[grow=right]
\node[bag0] {{$\top$}}
    child {
        node[bag] {\llap{$\neg$}$\phi(x;b_{\0}\!)$}      
            child {
                node[bag] {\llap{$\neg$}$\phi(x;b_0\!)$}
                    child {
                       node[bag] {\footnotesize\llap{$\neg$}$\phi(x;b_{00})$\rlap{$\ \cdots$}}
                       edge from parent
                    }    
                    child {
                       node[bag] {\footnotesize$\phi(x;b_{00})$\rlap{$\ \cdots$}}
                       edge from parent
                    }  
                 edge from parent
            }
            child {
                node[bag] {$\phi(x;b_0\!)$}
                edge from parent
                    child {
                       node[bag] {\footnotesize\llap{$\neg$}$\phi(x;b_{01})$\rlap{$\ \cdots$}}
                       edge from parent
                    }    
                    child {
                       node[bag] {\footnotesize$\phi(x;b_{01})$\rlap{$\ \cdots$}}
                       edge from parent
                    }  
                 edge from parent
            }
       edge from parent 
    }
    child {
        node[bag] {$\phi(x;b_{\0}\!)$}         
            child {
                node[bag] {\llap{$\neg$}$\phi(x;b_1\!)$}
                    child {
                       node[bag] {\footnotesize\llap{$\neg$}$\phi(x;b_{10})$\rlap{$\ \cdots$}}
                       edge from parent
                    }    
                    child {
                       node[bag] {\footnotesize$\phi(x;b_{10})$\rlap{$\ \cdots$}}
                       edge from parent
                    }  
                 edge from parent
            }
            child {
                node[bag] {$\phi(x;b_1\!)$}
                edge from parent
                    child {
                       node[bag] {\footnotesize\llap{$\neg$}$\phi(x;b_{11})$\rlap{$\ \cdots$}}
                       edge from parent
                    }    
                    child {
                       node[bag] {\footnotesize$\phi(x;b_{11})$\rlap{$\ \cdots$}}
                       edge from parent
                    }  
                 edge from parent
            } 
        edge from parent
    };
\end{tikzpicture}

\medskip
When a binary tree of this form exists, we say that $\phi({\mr x}\,;{\gr z})$ has the \emph{binary tree property}.

\begin{theorem}\label{thm_count_types}
  The following are equivalent
  \begin{itemize}
  \item[1.] $\phi({\mr x}\,;{\gr z})$ is unstable
  \item[2.] $\phi({\mr x}\,;{\gr z})$ has the binary tree property.
  \end{itemize}
\end{theorem}

\begin{proof}
  \ssf{1}$\IMP$\ssf{2}.
  The argument is the same as in the proof of Lemma~\ref{lem_bin_tree}.
  Assume \ssf1. 
  By Theorem~\ref{thm_def_stable_formula2} there are $2^\kappa$ sets externally definable by $\phi({\mr x}\,;{\gr z})$.
  Then there is ${\gr b_\0}\in{\U^z}$ such that there are $2^\kappa$ sets ${\gr\D}$ externally definable by $\phi({\mr x}\,;{\gr z})$ and such that ${\gr b_\0}\in{\gr\D}$ and $2^\kappa$ sets such that ${\gr b_\0}\notin{\gr\D}$.

  Assume inductively that ${\gr b}:2^n\to{\gr\U^z}$ is such that for all $s\in2^n$ and $r\in2^{n+1}$ there are $2^\kappa$ sets ${\gr\D}$ externally definable by $\phi({\mr x}\,;{\gr z})$ and such that ${\gr b_{s\restriction i}}\in\D\iff r(i)=1$.
  Reasoning as above we can extend ${\gr b}$ to a map ${\gr b'}:2^{n+1}\to{\gr\U^z}$ with the same property.

  % We show that if there is a ladder of length $m=2^n$, say $a_1,\dots,a_{m-1}$ and $b_0,\dots,b_{m-1}$, then there is a branching tree $\bar a'$ of height $n$.
  
  % The branching tree $\bar b'=\<b'_r\, :\, r\in{}^{<n}2\>$ is defined as follows

  % \quad $a'_r=b_h$\quad  where $h$ is obtained reading $r^\frown1^\frown0^{n-|r|-1}$ as an $n$-digit binary number.

  % To verify \ssf{2r} we define for $s\in{}^n2$ 
  
  % \quad $a'_s=b_k$\quad  where $k$ is obtained reading $s$ as an $n$-digit binary number.

  % Then it is easy to verify that for all pairs $r\subset s\in{}^n2$

  % \ceq{\hfill\phi(b'_r\,;a'_s)}
  % {\IFF}
  % {\phi(a_h\,;b_k)}\hfill where $h$ and $k$ are like above

  % \ceq{}
  % {\IFF}
  % {h\le k}

  % \ceq{}
  % {\IFF}
  % {r^\frown1^\frown0^{n-|r|-1}\ \le\ s}\hfill  as $n$-digit binary numbers

  % \ceq{}
  % {\IFF}
  % {r^\frown1\ \subseteq\ s}

  \ssf{2}$\IMP$\ssf{1}. From $\ssf{2}$, by compactness, there is a binary tree of height $\kappa$. 
  Hence there are $2^\kappa$ sets that are externally definable by $\phi({\mr x}\,;{\gr z})$.
  Therefore, by Theorem~\ref{thm_def_stable_formula2}, $\phi({\mr x}\,;{\gr z})$ is not stable.
\end{proof}

\begin{corollary}\label{corol_count_types}
The following are equivalent
\begin{itemize}
\item[1.] $\phi({\mr x}\,;{\gr z})$ is a stable formula
\item[2.] $\big|S_{\phi}(A)\big|\le|A|$ for all countable sets $A$
\item[3.] $\big|S_{\phi}(A)\big|<2^{|A|}$ for all countable sets $A$.
\end{itemize}
\end{corollary}

\begin{proof}
  The corollary follows immediately from Lemma~\ref{lem_bin_tree} and Theorem~\ref{thm_count_types}.
\end{proof}

We use binary trees to prove the following fundamental lemma (attributed by Harnik and Harrington~\cite{HH} to Martin Ziegler).

\begin{lemma}\label{lem_ziegler}
   Let $\phi({\mr x}\,;{\gr z}),\psi({\mr x}\,;{\gr z}) \in L(A)$.
   Let ${\gr b}\in\U^{\gr z}$.
   Assume that $\phi({\mr x}\,;{\gr z})$ is stable and that $\phi({\mr x}\,;{\gr b})\vee\psi({\mr x}\,;{\gr b})$ is finitely satisfied in every $M\supseteq A$.
   Then $\phi({\mr x}\,;{\gr b})$ or $\psi({\mr x}\,;{\gr b})$ is finitely satisfied in every $M\supseteq A$.
\end{lemma} 

\begin{proof}
  Negate the theorem. 
  Then there are two models containing $A$, the first omitting $\phi({\mr x}\,;{\gr b})$, the second omitting $\psi({\mr x}\,;{\gr b})$.
  It is easy to see that we can expand these models to two substructures $\V_0$ and $\V_1$ that are $A$-isomorphic to $\U$ and such that $\V_0=\neg\phi(\V_0^{\mr x};{\gr b})$ and $\V_1=\neg\psi(\V_1^{\mr x};{\gr b})$.

  Let $f_i:\U\to\V_i$ be the $A$-isomorphisms mentioned above. 
  Let $\V_\0=\U$ and ${\gr b_\0}={\gr b}$.
  Define inductively for every $s\in 2^{<\omega}$ 

  \hfil$\V_{s^\frown i}=f_i[\V_s]$\ \ and\ \ ${\gr b_{s\frown i}}=f_i({\gr b_s})$.

  Then from the inclusion above we obtain

  \#\hfil$\V_{s\frown0}^{{\mr x}}\subseteq\neg\phi(\V_{s\frown0}^{\mr x};{\gr b_s})$ \ \ and \ \  $\V_{s\frown1}^{\mr x}\subseteq\neg\psi(\V_{s\frown1}^{\mr x};{\gr b_s})$.

  We prove that the branches of the tree $\<\phi({\mr x}\,;{\gr b_s})\ :\  s\in2^{<\omega}\>$ are (finitely) consistent. 
  By Theorem~\ref{thm_count_types}, this contradicts the stability of $\phi({\mr x}\,;{\gr z})$.
  Assume provisionally that there is a formula $\theta({\mr x})\in L(A)$ such that
  
  \ceq{\#\#\hfill\theta({\mr x})}{\imp}{\phi({\mr x}\,;{\gr b})\vee\psi({\mr x}\,;{\gr b})}
  
  Let $s\in2^n$ be given.
  Let ${\mr a}\in\theta(\V_s^{\mr x})$ be arbitrary. We claim that ${\mr a}$ witnesses the concistency of $\{\phi^{s_i}({\mr x}\,;{\gr b_{s\restriction i}})\ :\ i<n\}$.
  As $\V_s\preceq\V_{s\restriction i}$, we have that ${\mr a}\in\theta(\V_{s\restriction i}^{\mr x})$.
  We prove that $\phi^{s_i}({\mr a}\,;{\gr b_{s\restriction i}})$.
  If $s_i=0$ then from \# we obtain $\V_{s\restriction(i+1)}^{{\mr x}}\subseteq\phi^0(\V_{s\restriction(i+1)}^{\mr x};{\gr b_{s\restriction i}})$ and $\phi^0({\mr a}\,;{\gr b_{s\restriction i}})$ follows by elementarity.
  If $s_i=1$, from the second inclusion in \# we obtain $\V_{s\restriction(i+1)}^{{\mr x}}\subseteq\neg\psi(\V_{s\restriction(i+1)}^{\mr x};{\gr b_{s\restriction i}})$.
  Therefore, from $\theta({\mr a})$ and \#\#, we obtain that $\V_{s\restriction(i+1)}^{{\mr x}}\subseteq\phi(\V_{s\restriction(i+1)}^{\mr x};{\gr b_{s\restriction i}})$ and $\phi^1({\mr a}\,;{\gr b_{s\restriction i}})$ follows.

  We are left with proving that the provisional assumption is redundant.
  By Exercise~\ref{ex_almost_satisfied} there is a formula $\theta({\mr\bar x})\in L(A)$ such that

  \ceq{\hfill\theta({\mr\bar x})}{\imp}{\phi'({\mr\bar x}\,;{\gr b})\vee\psi'({\mr\bar x}\,;{\gr b})}

  where ${\mr\bar x}={\mr x_1},\dots,{\mr x_n}$ and 

  \ceq{\hfill\phi'({\mr\bar x}\,;{\gr b})}{=}{\bigvee_{i=1}^n\phi({\mr x_i}\,;{\gr b})}

  \ceq{\hfill\psi'({\mr\bar x}\,;{\gr b})}{=}{\bigvee_{i=1}^n\psi({\mr x_i}\,;{\gr b})}

  Note that $\phi'({\mr\bar x}\,;{\gr z})$ is a stable formula.
  Therefore we can apply what proved above to the formulas $\phi'({\mr\bar x}\,;{\gr b})$ and $\psi'({\mr\bar x}\,;{\gr b})$ and note that these formulas are satisfied in every model $M\supseteq A$ if and only if $\phi({\mr x}\,;{\gr b})$, respectively $\psi({\mr x}\,;{\gr b})$, are.
\end{proof}

% \begin{theorem}
% The following are equivalent
% \begin{itemize}
% \item[1.] $T$ is stable
% \item[2.] $|S(A)|\le|A|$ for some infinite cardinal $\lambda$, and all sets $A$ of cardinality $\le\lambda$
% \item[3.] $|S(A)|\le|A|$ for every set $A$ such that $|L|<\cf(A)$.
% \end{itemize}
% \end{theorem}
% \begin{proof}
% \ssf{2}$\IMP$\ssf{1}.
% Suppose a formula $\phi({\mr x}\,;{\gr z})$ is unstable.
% Let $\<{\gr z_s}:s\in2^{<\lambda}\>$ be a sequence variables of length $|{\gr z}|$.
% Let $p({\gr z_s}:s\in2^{<\lambda})$ be the type that says that $\big\<{\gr z_s}:i<2^{<\lambda}\big\>$ witnesses a binary tree of height $\lambda$.
% As  $\phi({\mr x}\,;{\gr z})$ is unstable, $p$ is finitely consistent.
% As $\lambduse the fact that a$ is an arbitrary infinite cardinal, this contradicts \ssf{2}.

% \ssf{3}$\IMP$\ssf{2}.
% Trivial.

% \ssf{1}$\IMP$\ssf{3}.
% Proposition~\ref{thm_count_types} implies that $|S(A)|\le |A|^{|L|}$. Therefore, when $|L|<\cf(A)$, we obtain $|S(A)|\le |A|$.
% \end{proof}
\section{Symmetry and stationarity}
\label{stationarity}

\def\ceq#1#2#3{\parbox{20ex}{$\displaystyle #1$}\medrel{#2}$\displaystyle  #3$}

We say that $T$ is a \emph{stable theory\/} if every formula is stable.
Most theorems in this section have a formulation that only requires the stability of a given formula $\phi({\mr x}\,;{\gr z})\in L$.
This is sometimes called \textit{local\/} stability.
When the global case does not trivially follows from the local one, we use a set $\Delta\subseteq L_{{\mr x}\,{\gr z}}$.
We say that that $\Delta$ is stable if all formulas in $\Delta$ are stable.
When $|{\mr x}|=|{\gr z}|=\omega$ and $\Delta=L_{{\mr x}\,{\gr z}}$ the stability of $\Delta$ is tantamount to the stability of $T$.

The following corollary of Theorem~\ref{lem_ziegler} is a fundamental tool.

\begin{corollary}\label{corol_stable_coheir_over_models}
  Let  $\Delta\subseteq L_{{\mr x}\,{\gr z}}$ be stable.
  Let $q({\mr x})\subseteq L(\U)$ be finitely satisfiable in every $M\supseteq A$.
  Then there is a type $p({\mr x})\in S_\Delta(\U)$ such that $q({\mr x})\cup p({\mr x})$ is finitely satisfiable in every $M\supseteq A$.
\end{corollary}

\begin{proof}
  Let $p({\mr x})\subseteq \pmDelta(\U)$ be maximal such that $q({\mr x})\cup p({\mr x})$ is finitely satisfiable in every $M\supseteq A$.
  We prove that $p({\mr x})$ is complete.
  Suppose not than there are a conjuction $\alpha({\mr x})$ of formulas in $q({\mr x})$, a conjuction $\theta({\mr x}\,;{\gr b_1},\dots,{\gr b_n})$ of formulas in $p({\mr x})$, and a formula $\phi({\mr x}\,;{\gr b})$ for some $\phi({\mr x}\,;{\gr z})\in\Delta$ such that 

  \hfil$\alpha({\mr x})\wedge\theta({\mr x}\,;{\gr b_1},\dots,{\gr b_n})\wedge\phi({\mr x}\,;{\gr b})$ \ \ and\ \   $\alpha({\mr x})\wedge\theta({\mr x}\,;{\gr b_1},\dots,{\gr b_n})\wedge\neg\phi({\mr x}\,;{\gr b})$ 
 
  are not satisfied in every $M\supseteq A$.
  The disjunction of the two formulas is satisfied in every $M\supseteq A$ by the definition of $p({\mr x})$.
  Moreover the formulas obtained replacing ${\gr b},{\gr b_1},\dots,{\gr b_n}$ by ${\gr z},{\gr z_1},\dots,{\gr z_n}$ are stable.
  This contradict Theorem~\ref{lem_ziegler}
\end{proof}

The following theorem shows that in stable theories the heir-coheir relation is symmetric.
In fact, when $T$ is stable, it implies that

\ceq{\hfill{\mr a}\cnonfork_M{\gr b}}{\IMP}{{\gr b}\cnonfork_M{\mr a}.}

\begin{theorem}[ (Symmetry)]\label{thm_symmetry}
  Let $\phi({\mr x}\,;{\gr z})\in L(M)$ be stable.
  Let ${\mr a}\in\U^{\mr x}$ and ${\gr b}\in\U^{\gr z}$ be such that ${\mr a}\cnonfork_M{\gr b}$.
  Then

  \ceq{\hfill\phi({\mr a}\,;{\gr b})}{\IMP}{\phi({\mr a}\,;M^{\gr z})\neq\0.}
\end{theorem}

\vspace*{-\parskip}
\begin{proof}
  Assume $\phi({\mr a}\,;{\gr b})$.
  Let $\V\preceq\U$ be isomorphic to $\U$ over $M$ and such that ${\gr b}\in\V^{\gr z}$ and ${\mr a}\cnonfork_M\V$. Such $\V$ exists by Proposition~\ref{prop_saturate_heir}.
  By stability, there is $\psi({\gr z})\in L(\V)$ such that $\psi(\V^{\gr z})=\phi({\mr a},\V^{\gr z})$.
  Recall that by Lemma~\ref{lem_coheir_independence} (non-splitting) if ${\gr b'}\equiv_M{\gr b''}$ are in $\V$ then ${\gr b'}\equiv_{M,\kern.2ex{\mr a}}{\gr b''}$.
  Then $\psi(\V^{\gr z})$ is invariant under $\Aut(\V/M)$, so we can assume that $\psi({\gr z})\in L(M)$.
  Therefore $\psi({\gr z})$ is satisfied in $M$ and so is $\phi({\mr a},{\gr z})$.
\end{proof}

We deduce a version of Harrington's mysterious Lemma cfr.~\cite{TZ}*{Lemma 8.3.4}.

\begin{corollary}\label{corol_harrington0}
  Let $\phi({\mr x}\,;{\gr z})\in L$ be stable.
  Let ${\mr a}\in\U^{\mr x}$ and ${\gr b}\in\U^{\gr z}$.
  Let $\theta({\mr x}),\psi({\gr z})\in L(M)$ be such that $\theta(M^{\mr x})=\phi(M^{\mr x};{\gr b})$ and $\psi(M^{\gr z})=\phi({\mr a}\,;M^{\gr z})$.
  Such formulas exist by stability.
  Then $\theta({\mr a})\iff\psi({\gr b})$.
\end{corollary}

\begin{proof}
  We can assume that ${\mr a}\cnonfork_M{\gr b}$.
  First, assume $\phi({\mr a}\,;{\gr b})$.
  Then $\theta({\mr a})$ otherwise, by ${\mr a}\cnonfork_M{\gr b}$, we would contradict $\theta(M^{\mr x})=\phi(M^{\mr x};{\gr b})$.
  We claim that $\psi({\gr b})$.
  Suppose not.
  Then $\phi({\mr a}\,;{\gr b})\wedge\neg\psi({\gr b})$.
  As $\phi({\mr x}\,;{\gr z})\wedge\neg\psi({\gr z})$ is stable, from Theorem~\ref{thm_symmetry} we obtain $\phi({\mr a}\,;M^{\gr z})\sm\theta(M^{\gr z})\neq\varnothing$ which contradicts  $\psi(M^{\gr z})=\phi({\mr a}\,;M^{\gr z})$.
  Assuming $\neg\phi({\mr a}\,;{\gr b})$ we obtain $\neg\theta({\mr a})$ and $\neg\psi({\gr b})$ by a symilar argument.
  Then $\theta({\mr a})\iff\psi({\gr b})$ as required.
\end{proof}

Now we show that under the assumption of stability Lascar invariance reduces to a tamer form of invariance.

\begin{proposition}\label{prop_type_over_acl2}
  Let $\phi({\mr x}\,;{\gr z})\in L$ be stable.
  Let $p({\mr x})\in S_\phi(\U)$. 
  Then the following are equivalent
  \begin{itemize}
  \item[1.] ${\gr\D_{p,\phi}}\in \acl^\eq\!A$ -- recall that, by stability, ${\gr\D_{p,\phi}}\in\U^\eq$
  \item[2.] $p({\mr x})$ is finitely satisfied in every $M\supseteq A$
  \item[3.] $p({\mr x})$ is invariant over $\acl^\eq\!A$
  \item[4.] $p({\mr x})$ is Lascar invariant over $A$.
  \end{itemize}
\end{proposition}
\begin{proof}
  \ssf{2}$\IMP$\ssf{3}$\IMP$\ssf{4} (and even \ssf{1}$\IMP$\ssf{3}) are clear~--~stability is not required.

  \ssf{1}$\IMP$\ssf{2}
  Let $M\supseteq A$ be given.
  First, for simplicity, we prove that any $\phi({\mr x}\,;{\gr b})\in p$ is satisfied in $M$.
  Let $\psi({\gr z})\in L(\acl^\eq\!A)$ be a formula defining ${\gr\D_{p,\phi}}$.
  Pick any ${\mr a}\models p({\mr x}){\restriction} M$ such that ${\gr b}\cnonfork_M{\mr a}$.
  As $\phi({\mr a}\,;M^{\gr z})=\psi(M^{\gr z})$ we have that $\phi({\mr a}\,;{\gr b})=\psi({\gr b})$.
  But $\psi({\gr b})$ holds because $\phi({\mr x}\,;{\gr b})\in p$, hence $\phi({\mr a}\,;{\gr b})$ and, from Theorem~\ref{thm_symmetry} we obtain $\phi(M^{\mr x};{\gr b})\neq\0$.
  The same argument extends easily to the conjunction of formulas of the form $\phi({\mr x}\,;{\gr b})$ and negation thereof (use that the complement and the Cartesian product of sets in $\acl^\eq\!A$ is again $\acl^\eq\!A$).
 
  \ssf{3}$\IMP$\ssf{1} By Theorem~\ref{thm_def_stable_formula2} the sets ${\gr\D_{p,\phi}}$ are definable (over $\U$).
  As $p({\mr x})$ is Lascar invariant over $A$, so are the sets ${\gr\D_{p,\phi}}$.
  By Corollary~\ref{coroll_defble_Lascar_inv}, they belong to $\acl^\eq\!A$.
\end{proof}

A type $q({\mr x})\subseteq L(A)$ is \emph{stationary\/} if it has a unique global extension that is Lascar invariant over $A$.
The following proposition says that in a stable theory with elimination of imaginaries types over algebraically closed sets are stationary (the actual statement is a local version of this property).

\begin{theorem}[ (Stationarity)]\label{thm_stationarity}
  Let $\Delta\subseteq L_{{\mr x}\,{\gr z}}$ be stable.
  Let $q({\mr x})$ be a complete $\GDelta(\acl^\eq\!A)$-type.
  Then there is a unique type $p({\mr x})\in S_\Delta(\U)$ containing $q({\mr x})$ that is invariant over $\acl^\eq A$.
\end{theorem}

\begin{proof}
  Existence follows from Corollary~\ref{corol_stable_coheir_over_models} and Proposition~\ref{prop_type_over_acl2}.
  Suppose $p_i({\mr x})\in S_\Delta(\U)$, for $i=1,2$, are two types that extend $q({\mr x})$ and are invariant over $\acl^\eq\!A$.
  It sufficies to prove that ${\gr\D_{p_1,\phi}}={\gr\D_{p_2,\phi}}$ for every $\phi({\mr x}\,;{\gr z})\in\Delta$.
  Let $\psi_i({\gr z})\in L(\acl^\eq\!A)$ define ${\gr\D_{p_i,\phi}}$.
  These formulas exist by stability.
  We claim that $\psi_1({\gr b})\iff\psi_2({\gr b})$, for all ${\gr b}\in\U^{\gr z}$.
  Let $p'({\gr z})\in S_\Delta(\U)$ be invariant over $\acl^\eq\!A$ and consistent with $\tp({\gr b}/\acl^\eq\!A)$.
  Such a type exists, again, by Corollary~\ref{corol_stable_coheir_over_models} and Proposition~\ref{prop_type_over_acl2}.
  By stability there is a formula $\theta({\mr x})\in\BDelta(\U)$ that defines ${\mr\D_{t,\phi^{\rm op}}}$.
  But ${\mr\D_{t,\phi^{\rm op}}}$ is invariant over $\acl^\eq\!A$, then $\theta({\mr x})\in\GDelta(\acl^\eq\!A)$.
  Let $M$ be any model containing $A$.
  Let ${\gr b'}\models p'({\gr z}){\restriction}M$ and ${\mr a_i}\models p_i({\mr x}){\restriction}M$.
  Then $\theta(M^{\mr x})=\phi(M^{\mr x};{\gr b'})$ and $\psi_i(M^{\gr z})=\phi({\mr a_i}\,;M^{\gr z})$.
  Corollary~\ref{corol_harrington0} yelds $\psi_i({\gr b'})\iff\theta({\mr a_i})$.
  As ${\mr a_1},{\mr a_2}\models q({\mr x})$, by completeness they both satisfy $\theta({\mr x})$. 
  Hence we obtain $\psi_1({\gr b'})\iff\psi_2({\gr b'})$.
  As ${\gr b}\equiv_{\acl^\eq\!A}{\gr b'}$, the claim follows and with it the theorem.
\end{proof}

% If $p({\mr x})\in S(\U)$ is a global type, a \emph{canonical base\/} of $p({\mr x})$ is a definably closed set $\emph{$\Cb(p)$}\subseteq \U^\eq$ such that an automorphism $f\in\Aut(\U)$ fixes $p({\mr x})$ if and only if it fixes $\Cb(p)$ pointwise. When they exist, canonical bases are unique, see Exercise~\ref{ex_Cb}.
% Clearly, all definable types (Definition~\ref{def_defble_type}) have a canonical base, namely

% \ceq{\hfill\Cb(p)}{=}{{\dcl}^\eq\Big(\big\{{\gr\D_{p,\phi}}\ :\ \phi({\mr x}\,;{\gr z})\in L\big\}\Big).}

% Therefore, when $T$ is stable all global types have a canonical base.

% \begin{proposition}\label{prop_type_over_dcl} Let $T$ be stable and let $p({\mr x})\in S(\U)$ be Lascar invariant over $A$. Then for every formula $\phi({\mr x}\,;{\gr z})\in L$ the set ${\gr\D_{p,\phi}}$ belongs to $\dcl^\eq(A,c)$ for every $c\models p_{\restriction A}(x)$.
% \end{proposition}
% \begin{proof}
% Without loss of generality we can assume $p({\mr x})\in S(\U^\eq)$. 
% \end{proof}

\begin{corollary}
($T$ stable)
The following are equivalent
\begin{itemize}
\item[1.] ${\mr a}\equivL_A{\mr b}$, see Definition~\ref{def_Lascar_type}
\item[2.] ${\mr a}\equivSh_A{\mr b}$, see Definition~\ref{def_Sh_strong_type}
% \item[3.] ${\mr a}\,\equiv_{\acl^\eq\! A}{\mr b}$.
\end{itemize}
\end{corollary}
\begin{proof}\ssf{1}$\IMP$\ssf{2}.
  This is left as an exercise to the reader~--~stability is not required.

  \ssf{2}$\IMP$\ssf{1}.
  Assume \ssf2 which, by Proposition~\ref{prop_Shelah_strong_types}, is equivalent to ${\mr a}\equiv_{\acl^\eq\!A}{\mr b}$.
  Write $q({\mr x})$ for $\tp({\mr a}/{\acl}^\eq\!A)=\tp({\mr b}/{\acl}^\eq\!A)$. Let $p({\mr x})\in S(\U^\eq)$ be the unique global type that is invariant over $\acl^\eq\!A$ and extends $q({\mr x})$ which we obtain from Theorem~\ref{thm_stationarity}.
  Let ${\mr\bar c}=\<{\mr c_i}:i<\omega\>$ be such that ${\mr c_i}\models p({\mr x}){\restriction}\acl^\eq\!A,\,{\mr a},\,{\mr b},\,{\mr c_{\restriction i}}$.
  Then ${\mr a},{\mr\bar c}$ and ${\mr b},{\mr\bar c}$ are $A$-indiscernible sequences, which proves \ssf{1}, see Exercise~\ref{ex_Lstp_indiscernibles}.
\end{proof}

% \begin{proposition}
%   ($T$ stable)
%   Let $p({\mr x})\in S(\U)$.
%   Then the following are equivalent
%   \begin{itemize}
%   \item[1.] $p({\mr x})\in S(\U)$ invariant over $\acl^\eq\!A$
%   \item[2.] $p({\mr x})$ is almost finitely satisfiable in $A$. 
%   \end{itemize}
% \end{proposition}
% \begin{proof}
%   (\ssf2$\IMP$\ssf1) As \ssf2 implies that $p({\mr x})$ is Lascar invariant over $A$, \ssf1 follows from Proposition~\ref{prop_type_over_acl2}

%   (\ssf1$\IMP$\ssf2) Let $M\supseteq A$. 
%   By Corollary~\ref{corol_stable_coheir_over_models}, there exists a global type $q({\mr x})$ that extends $p({\mr x})\restriction\acl^\eq\!A$ and is almost finitely satisfied in $A$.
%   As $q({\mr x})$ is Lascar invariant over $A$, by stationarity it coincides with $p({\mr x})$.
% \end{proof}

The following theorem presents a property of definability that seems to go in the opposite direction.
It has an application in the proof of Corollary~\ref{corol_persistent_finsat}.

\begin{theorem}\label{thm_stability_definable_rovescio1}
  Let $\phi({\mr x}\,;{\gr z})\in L(\U)$ be stable.
  Let $M$ and ${\gr b}\in\U^{\gr z}$ be arbitrary.
  Then, for some ${\gr b_i}\equiv_M{\gr b}$, where $i=1,\dots,n$, some positive Boolean combination of the sets $\phi(\U^{\mr x};{\gr b_i})$ is definable over $M$.
\end{theorem}

\begin{proof}
  Let $\V\preceq\U$ be isomorphic to $\U$ over $M$.
  By Theorem~\ref{thm_def_stable_formula} with $\V$ for $\U$ and the role of the two sorts reversed, there are  some  ${\gr b_i}\equiv_M{\gr b}$ in $\V$, such that $\phi(\V^{\mr x}\,;{\gr b})$ is a positive Bolean combination of the sets $\phi(\V^{\mr x}\,;{\gr b_i})$.
  Let $\theta(\V^{\mr x}\,;{\gr b_1},\dots,{\gr b_n})$ be such a Boolean combination.
  Now, by Proposition~\ref{prop_saturate_heir}, we can choose $\V$ such that ${\gr b}\cnonfork_M\V$.
  By non-splitting, i.e.\@ reasoning as in Theorem~\ref{thm_symmetry}, there is a formula $\psi({\mr x})\in L(M)$ such that $\psi(\V^{\mr x})=\phi(\V^{\mr x}\,;{\gr b})$.
  Finally, by elementarity, we obtain that $\psi(\U^{\mr x})=\theta(\U^{\mr x};{\gr b_1},\dots,{\gr b_n})$.
\end{proof}

A more general version of the theorem holds which requires a different proof.

\begin{theorem}\label{thm_stability_definable_rovescio2}
  Let $\phi({\mr x}\,;{\gr z})\in L(\U)$ be stable.
  Let $A$ and ${\gr b}\in\U^{\gr z}$ be arbitrary.
  Then, for some ${\gr b_i}\equiv_{\acl^\eq\!A}{\gr b}$, where $i=1,\dots,n$, some positive Boolean combination of the sets $\phi(\U^{\mr x};{\gr b_i})$ is definable over $A$.
\end{theorem}

\begin{proof}
  Let $q({\gr z})=\tp({\gr b}/\acl^\eq\!A)$.
  By Corollary~\ref{corol_stable_coheir_over_models} there is a type $p({\gr z})\in S_\phi(\U)$ consistent with $q({\gr z})$ and invariant over $\acl^\eq\!A$.
  By Theorem~\ref{thm_def_stable_formula}, with the role of the two sorts reversed, there are some ${\gr b_1},\dots,{\gr b_n}\models q({\gr z})$ such that ${\mr\D_{p,\phi^{\rm op}}}$ is equivalent to a positive Bolean combination of the sets $\phi(\U^{\mr x};{\gr b_i})$.
  Let $\theta(\U^{\mr x};{\gr b_1},\dots,{\gr b_n})$ be such a Boolean combination.
  As ${\mr\D_{p,\phi^{\rm op}}}\in\acl^\eq\!A$ by Proposition~\ref{prop_type_over_acl2}, this prove the theorem when $A=\acl^\eq\!A$.

  For the general case, reason as follows.
  Let $f_1,\dots,f_m\in\Aut(\U/A)$ be such that the sets $f_i{\mr\D_{p,\phi^{\rm op}}}$ span the whole orbit of ${\mr\D_{p,\phi^{\rm op}}}$ over $A$.
  Clearly

  \ceq{\hfill\bigcup_{i=1}^m f_i{\mr\D_{p,\phi^{\rm op}}}}{=}{\bigvee_{i=1}^m\theta(\U^{\mr x};f_i{\gr b_1},\dots,f_i{\gr b_n}).}

  The set on the l.h.s.\@ is invariant (hence definable) over $A$.
  Therefore the formula on the r.h.s.\@ is the required positive Boolean combination.
\end{proof}

%%%%%%%%%%%%%%%%%%%%%%%%%%%%
%%%%%%%%%%%%%%%%%%%%%%%%%%%%
%%%%%%%%%%%%%%%%%%%%%%%%%%%%
%%%%%%%%%%%%%%%%%%%%%%%%%%%%
\section{The action of the Lascar group}

\def\medrel#1{\parbox[t]{6ex}{$\displaystyle\hfil #1$}}
\def\ceq#1#2#3{\parbox{10ex}{$\displaystyle #1$}\medrel{#2}$\displaystyle  #3$}

We adopt the notation and terminology of Chapter~\ref{actions} with $G=\Autf(\U/A)$ and $\grZ=\U^{\gr z}$.
The set $\mrX$ will specified in the context.
We say \emph{Lascar generic\/} over $A$ for generic under the action of $\Autf(\U/A)$.
We define \emph{Lascar persistent\/} similarly.

\begin{corollary}\label{corol_persistent_finsat}
  Let $\Delta\subseteq L_{{\mr x}\,{\gr z}}$ be stable.
  Let $\mrX= q(\U^{\mr x})$ for some $q({\mr x})\subseteq L(\acl^\eq\!A)$.
  Let $\mrD$ be a $\BDelta(\U)$-definable set.
  Then the following are equivalent 
  \begin{itemize}
    \item[1.] $\mrD$ is Lascar persistent over $A$
    \item[2.] $\{{\mr x}\in\mrD\}\cup q({\mr x})$ is finitely satisfied every in $M\supseteq A$.
  \end{itemize}
\end{corollary}
\begin{proof}
  As Boolean combinations of stable formulas are stable, we can assume that $\mrD$ is defined by the formula $\phi({\mr x}\,;{\gr b})$ for some $\phi({\mr x}\,;{\gr z})\in\Delta$.

  \ssf1$\IMP$\ssf2.
  Let $M\supseteq A$.
  Let $\psi({\mr x})\in q$.
  If $\phi({\mr x}\,;{\gr b})$ is Lascar persistent, then it is also persistent under the action of $\Aut(\U/M)$.
  Then the formula $\theta({\mr x}\,;{\gr b_1},\dots,{\gr b_n})$ in the proof of Theorem~\ref{thm_stability_definable_rovescio1} is consistent with $\psi({\mr x})$.
  As $\theta({\mr x}\,;{\gr b_1},\dots,{\gr b_n})$ is equivalent to a formula in $L(M)$, by elementarity it satisfiable in $\psi(M^{\mr x})$.
  Hence so is $\phi({\mr x}\,;{\gr b_i})$ for some $i$.
  As ${\gr b_i}\equiv_M{\gr b}$, also $\phi({\mr x}\,;{\gr b})$ is satisfiable in $\psi(M^{\mr x})$.
  
  \ssf2$\IMP$\ssf1.
  From \ssf2 and Corollary~\ref{corol_stable_coheir_over_models} it follows that $\phi({\mr x}\,;{\gr b})$ is contained in a type $p({\mr x})\in S_\Delta(\U)$ that is finitely satisfied in every $M\supseteq A$.
  By Proposition~\ref{prop_type_over_acl2}, $p({\mr x})$ is Lascar invariant.
  Then, by Theorem~\ref{thm_generic_invariant}, $p({\mr x})$ and in particular $\phi({\mr x}\,;{\gr b})$ is Lascar persistent.
\end{proof}

\begin{proposition}\label{prop_stable_lanscape}
  Let $\Delta\subseteq L_{{\mr x}\,{\gr z}}$ be stable.
  Let $\mrX= q(\U^{\mr x})$ for some $q({\mr x})\subseteq L(\acl^\eq\!A)$.
  Then the equivalent conditions of Theorem~\ref{thm_coalesce} are satisfied.
\end{proposition}

\begin{proof}
  Let $\mrD$ be a $\BDelta(\U)$-definable set that is weakly Lascar persistent.
  We prove that $\mrD$ is Lascar persistent.
  Pick a finite $H\subseteq\Autf(\U/A)$ such that $\cup\,H\,\mrD$ is Lascar persistent.
  By Corollary~\ref{corol_persistent_finsat}, the type $\{{\mr x}\in\cup\,H\,\mrD\}\cup q({\mr x})$ is finitely satisfiable in every $M\supseteq A$.
  Then, by Corollary~\ref{corol_stable_coheir_over_models} there is a Lascar invariant type $p({\mr x})\in S_\Delta(\U)$ containing ${\mr x}\in\cup\,H\,\mrD$ and finitely consistent with $q({\mr x})$.
  By completeness and invariance ${\mr x}\in\mrD$ is in $p({\mr x})$.
  Then $\mrD$ is persistent by Theorem~\ref{thm_generic_invariant}.
\end{proof}

\begin{proposition}\label{prop_stable_genericiffpersistent}
  Let $\Delta\subseteq L_{{\mr x}\,{\gr z}}$ be stable.
  Let $\mrX=q(\U^{\mr x})$ for some $q({\mr x})\in S(\acl^\eq\!A)$.
  Then for every $\BDelta(\U)$-definable set $\mrD$, the following are equivalent
  \begin{itemize}
    \item[1.] $\mrD$ is Lascar generic over $A$.
    \item[2.] $\mrD$ is Lascar persistent over $A$.
  \end{itemize}
\end{proposition}

\begin{proof}
  \ssf1$\IMP$\ssf2.
  By Theorem~\ref{thm_stationarity} there is a type $p({\mr x})\in S_\Delta(\U)$ consistent with $q({\mr x})$ that is Lascar invariant.
  Therefore, from \ssf1 and Corollary~\ref{corol_def_mu} we infer that $\mrD$ is Lascar persistent.

  \ssf2$\IMP$\ssf1.
  Assume $\mrD$ is not Lascar generic.
  Then $\neg\mrD$ is Lascar persistent.
  Assume for a contradiction that also $\mrD$ is Lascar persistent.
  Let $\theta({\mr x}\,;{\gr b_1},\dots,{\gr b_n})$ be the $\BDelta(\U)$-formula that defines $\mrD$.
  As $\theta({\mr x}\,;{\gr z_1},\dots,{\gr z_n})$ is stable, by Corollary~\ref{corol_persistent_finsat} both $\theta({\mr x}\,;{\gr b_1},\dots,{\gr b_n})$ and its negation are finitely satisfied in every model $M\supseteq A$.
  Then there are two distinct types $p_i({\mr x})\in S_\Delta(\U)$ finitely consistent with $q({\mr x})$ and finitely satisfied in every $M\supseteq A$.
  This contradicts stationarity, see Proposition~\ref{prop_type_over_acl2} and Theorem~\ref{thm_stationarity}.
\end{proof}

\begin{corollary}\label{corol_stable_generic}
  Let $\Delta\subseteq L_{{\mr x}\,{\gr z}}$ be stable.
  Let $\mrX=q(\U^{\mr x})$ for some $q({\mr x})\in S(\acl^\eq\!A)$.
  Then for every $\BDelta(\U)$-definable set $\mrD$, either $\mrD$ or $\neg\mrD$ is Lascar generic over $A$.
\end{corollary}

\begin{proof}
  By Theorem~\ref{thm_stationarity} there is a type $p({\mr x})\in S_\Delta(\U)$ finitely consistent with $q({\mr x})$ that is Lascar invariant and therefore persistent.
  Hence, by Corollary~\ref{corol_def_mu}, $\mrD$ and $\neg\mrD$ are not both generic.
  If $\mrD$ is not generic then $\neg\mrD$ is persistent and, by Proposition~\ref{prop_stable_genericiffpersistent}, generic.
\end{proof}

% \begin{exercise}\label{ex_Cb}
%   Let $p({\mr x})\in S(\U)$.
%   Prove that there is at most one definably closed set $A\subseteq\U^\eq$ such that $\Aut(\U/A)$ is the set of automorphisms that fix $p({\mr x})$.
% \end{exercise}
\begin{proposition}
  Let $\Delta\subseteq L_{{\mr x}\,{\gr z}}$.
  Let $p_0({\mr x})$ be complete $\GDelta(A)$-type.
  Let 
  
  \ceq{\hfill Q}{=}{\big\{q({\mr x})\subseteq\GDelta(\acl^\eq\!A)\ :\ q({\mr x})\supseteq p_0({\mr x})\textrm{ and is complete}\big\}}.
  
  Then $\Aut(\U/A)$ acts on $Q$ transistively, i.e.\@ any two types in $Q$ are conjugated.
\end{proposition}

\begin{proof}
  We prove that any two types $q_1,q_2\in Q$ are realized be some $a_1\equiv_Aa_2$.
  The claim follows because any automorphism in $\Aut(\U/A)$ that takes $a_1$ to $a_2$ takes also $q_1$ in $q_2$.
  It suffices to prove that every type $p({\mr x})\in S(A)$ is consistent with every $q\in Q$.
  Suppose not, then $p({\mr x})\imp\neg\phi({\mr x})$ for some $\phi({\mr x})\in q$.
  Let $\psi({\mr x})$ be the disjunction of all the (finitely many) conjugates of $\phi({\mr x})$ over $A$.
  Then $p({\mr x})\imp\neg\psi({\mr x})$.
  But $\psi({\mr x})\in\BDelta(\acl^\eq\!A)$ and is invariant over $A$.
  Therefore $\psi({\mr x})\in\GDelta(A)$.
  Then $\psi({\mr x})\in p$, a contradiction.
\end{proof}

\begin{theorem}
  Let $\Delta\subseteq L_{{\mr x}\,{\gr z}}$ be stable and finite.
  Let $p_0({\mr x})$ be complete $\GDelta(A)$-type.
  Let 
  
  \ceq{\hfill P}{=}{\big\{p({\mr x})\in S_\Delta(\U)\ :\ p({\mr x})\proves p_0({\mr x})\textrm{ and is Lascar invariant over }A\big\}}.
  
  Then the following hold
  \begin{itemize}
    \item [1.] $\Aut(\U/A)$ acts on $P$ transistively.
    \item [2.] $P$ is finite
    \item [3.] there is a finite equivalence relation $\varepsilon({\mr x}\,;{\mr y})\in\GDelta(A)$ such that for all $p_1,p_2\in P$ 
    
    \ceq{\hfill p_1=p_2}{\IFF}{p_i({\mr x})\cup p_j({\mr y}) \textrm{ is consistent with }\epsilon({\mr x},{\mr y})}.
    
  \end{itemize}
\end{theorem}

\begin{proof}
  \ssf1.
  Let $p_i$, where $i=1,2$, be two elements of $P$.
  Let $q_i({\mr x})$ be some complete $\GDelta(\acl^\eq\!A)$-type consistent with $p_i({\mr x})$.
  By the proposition above $f\,q_1=q_2$ for some $f\in\Aut(\U/A)$.
  Then $f\,p_1$ is consistent with $q_2$.
  By stationarity $f\,p_1=p_2$.

  \ssf2.
  Identify a type $p({\mr x})\in S_\Delta(\U)$ with the sets $\<{\gr\D_{p,\phi}}:\phi({\mr x}\,;{\gr z})\in\Delta\>$.
  For any $f\in\Aut(\U/A)$, we have that ${\gr\D_{fp,\phi}}=f[{\gr\D_{p,\phi}}]$.
  Therefore the orbit of $p({\mr x})$ corresponds to the orbit of $\<{\gr\D_{p,\phi}}:\phi({\mr x}\,;{\gr z})\in\Delta\>$.
  As $p\in P$ is Lascar invariant over $A$ the sets ${\gr\D_{p,\phi}}$ are in $\acl^\eq\!A$.
  Hence have a finite orbit.
  As $\Delta$ is finite the orbit of $\<{\gr\D_{p,\phi}}:\phi({\mr x}\,;{\gr z})\in\Delta\>$ is finite, hence so is the orbit of $p$.
  By \ssf1, $P$ is finite.

  \ssf3. 
  Let $p_1,\dots,p_n$ be an enumeration without repetitions of $P$.
  Let $q_1({\mr x})$ be some complete $\GDelta(\acl^\eq\!A)$-type consistent with $p_1$.
  By \ssf2 there are some $g_1,\dots,g_n\in\Aut(\U/A)$ be such that $p_i=g_ip_1$.
  Let $q_i=g_iq_1$.
  Clearly $q_i$ is a complete $\GDelta(\acl^\eq\!A)$-type consistent with $p_i$.
  By stationarity the $q_i$ are mutually inconsistent.
  Pick $n$ mutually inconsistent formulas $\theta_i({\mr x})\in q_i$ and define

  \ceq{\hfill\epsilon({\mr x},{\mr y})}{=}{\bigwedge_{i=1}^n\big[\theta_i({\mr x})\iff\theta_i({\mr y})\big]}
  
  This shows that $p_i=p_j$ if and only if $p_i({\mr x})\cup p_j({\mr y})$ is consistent with $\epsilon({\mr x},{\mr y})$.
  From the definition of $\epsilon({\mr x},{\mr y})$ we know that it belongs to $\GDelta(\acl^\eq\!A)$, but it is easily seen that $\epsilon({\mr x},{\mr y})$ is invariant over $A$, hence \ssf3 follows.
\end{proof}

%%%%%%%%%%%%%%%%%%%%%%%%
%%%%%%%%%%%%%%%%%%%%%%%%
%%%%%%%%%%%%%%%%%%%%%%%%
%%%%%%%%%%%%%%%%%%%%%%%%
%%%%%%%%%%%%%%%%%%%%%%%%
%%%%%%%%%%%%%%%%%%%%%%%%
\section{Stable groups}

In this section we adopt the setting of Section~\ref{definablegroups}.
Here we also assume that the action is transitive, that is, $G\,{\mr a}=\mrX$ for some/any ${\mr a}\in\mrX$.
We also assume that the formula $\phi({\mr x}\,;{\gr z})$ defined in Section~\ref{definablegroups} is stable.

We remark that there is another group with a natural action on $\mrX$ and $\grZ$: the Lascar group $\Autf(\U/A)$.
For obvious reasons, if $\phi({\mr x}\,;{\gr g})$ is Lascar generic then it is $G$-generic and if it is $G$-persistent then it is Lascar persistent.

\begin{theorem}\label{thm_Ggeneric_persistent}
  Let $\phi({\mr x}\,;{\gr z})$ be stable.
  Then the following are equivalent
  \begin{itemize}
    \item [1.] $\phi({\mr x}\,;{\gr g})$ is Lascar persistent over $A$ for every ${\gr g}\in\grZ$
    \item [2.] $\phi({\mr x}\,;{\gr 1})$ is $G$-generic.
  \end{itemize}
\end{theorem}

\begin{proof}
  Let ${\mr a}\in\mrX$ and ${\gr g}\in\grZ$ be such that $\phi({\mr a}\,;{\gr g})$.
  Let $q({\mr x})=\tp({\mr a}/\acl^\eq\!A)$.
  Assume \ssf1 then, by Proposition~\ref{prop_stable_genericiffpersistent}, $\phi({\mr x}\,;{\gr g})$ is Lascar generic relative to $q(\U^{\mr x})$, i.e.\@ with $q(\U^{\mr x})$ for $\mrX$.
  A fortiori $\phi({\mr x}\,;{\gr g})$ is $G$-generic relative to $q(\U^{\mr x})$.
  The same reasoning applies if we replace ${\mr a}$ and ${\gr g}$ with  ${\gr h}\,{\mr a}$ and ${\gr h}\,{\gr g}$ for any ${\gr h}\in\grZ$.
  By the transitivity of the action, ${\gr h}\,{\mr a}$ ranges the whole of $\mrX$.
  We conclude that there is a set $H\subseteq\grZ$ of cardinality $<2^{|L(A)|}$ such that $\bigcup_{h\in H}\phi(\mrX\,;{\gr h})$ covers $\mrX$.
  By compactness, we can assume $H$ finite, so $G$-genericity follows.

  \ssf2$\IMP$\ssf1.
  Consider the 2-sorted model $\U'=\big\<\mrX\,;\grZ\big\>$ whose signature $L'$ contains only a relation symbol $r'({\mr x}\,;{\gr z})$ that is interpreted as $\phi({\mr x}\,;{\gr z})$.
  To each ${\gr h}\in\grZ$ we associate the $L'$-automorphism $h:\<{\mr a}\,;{\gr g}\>\mapsto\<{\gr h}\,{\mr a}\,;{\gr h}\,{\gr g}\>$.
  As $\U'$ is a stable structure, there is a global $r'$-type $p'({\mr x})$ that is definable over $L'\mbox{-}\acl^\eq\0$.
  Now, let ${\gr g}$ be given.
  If $\phi({\mr x}\,;{\gr g})$ is $G$-generic, $p'({\mr x})\proves r'({\mr x}\,;{\gr h}\,{\gr g})$ for some ${\gr h}\in\grZ$.
  Then $r'({\mr x}\,;{\gr g})\in{h^{-1}}p'$.
  We claim that ${h^{-1}}p'({\mr x})$ is also definable over $L'\mbox{-}\acl^\eq\0$.
  Indeed, it suffices to note that ${\gr\D_{{h^{-1}}p',r'}}=h^{-1}\,[{\gr\D_{p',r'}}]$ and that if 
  ${\gr\D_{p',r'}}\in L'\mbox{-}\acl^\eq\0$ then also $h^{-1}\,{\gr\D_{p,r'}}\in L'\mbox{-}\acl^\eq\0$.
  Now let $p({\mr x})\in S_\phi(\U)$ a a type finitely satisfiable in ${\mr x}\in\mrX$ and containing ${h^{-1}}p'({\mr x})$ with $r'$ replaced by $\phi$.
  Note that $\phi({\mr x}\,;{\gr g})\in p$ and that $p({\mr x})$ is Lascar invariant over $A$ because every automorphism of $\U$ over $A$ induces an $L'$-automorphism of $\U'$.
  This shows that there is a Lascar invariant type, finitely satisfiable in $\mrX$, containing $\phi({\mr x}\,;{\gr g})$.
  Then \ssf1 follows.
  % \ssf3$\IMP$\ssf1.
  % Let ${\gr g}$ be such that $\phi({\mr x}\,;{\gr g})$ is not Lascar persistent.
  % Then $\neg\phi({\mr x}\,;{\gr g})$ is Lascar generic.
  % Then it is also  $G$-generic.
  % The $G$-genericity of $\neg\phi({\mr x}\,;{\gr 1})$ follows.
\end{proof}

\begin{corollary}
  Let $\phi({\mr x}\,;{\gr z})$ be stable.
  Then $\phi({\mr x}\,;{\gr 1})$ or $\neg\phi({\mr x}\,;{\gr 1})$ is $G$-generic.
\end{corollary}

\begin{proof}
  Suppose $\phi({\mr x}\,;{\gr 1})$ is not $G$-generic.
  Then $\phi({\mr x}\,;{\gr g})$ is not Lascar persistent for some ${\gr g}$.
  Then $\neg\phi({\mr x}\,;{\gr g})$ is Lascar generic.
  Then it is also $G$-generic.
  The $G$-genericity of $\neg\phi({\mr x}\,;{\gr 1})$ follows.
\end{proof}


%%%%%%%%%%%%%%%%%%%%%%%%%
%%%%%%%%%%%%%%%%%%%%%%%%%
%%%%%%%%%%%%%%%%%%%%%%%%%
%%%%%%%%%%%%%%%%%%%%%%%%%
%%%%%%%%%%%%%%%%%%%%%%%%%
\section{Stable theories}
\label{stable_theories}

\begin{theorem}[ (Pierre Simon)]
  If every formula $\phi(x\,;y)\in L(\U)$, where $|x|=|y|=1$, is stable then $T$ is stable.
\end{theorem} 

\begin{proof}
  Suppose $\phi(x\,;y,z)\in L(\U)$ is not stable.
  We prove that there is a formula $\psi(x\,;y)\in L(\U)$ that is not stable.
  Let $\<a_i\,;b_i,c_i\ :\ i\in\QQ\>$ is a sequence of indiscernibles such that 

  \ceq{\hfill i< j}{\IFF}{\phi(a_i\,;b_j,c_j)}\hfill for all $i,j\in\QQ$.

  Let $\QQ^*=\QQ\sm\{0\}$.
  Assume first that the sequence $\<a_i\ :\ i\in\QQ^*\>$ is indiscernible over $c_0$.
  Then for every $k\in\QQ^*$ the type below is consistent

  \ceq{\hfill p_k(y)}{=}{\{\phi(a_i\;y,c_0)\iff i<k\ \ :\ i\in\QQ^*\sm\{k\}\}.}

  In fact, by indiscernibility, $b_0$ witnesses the consistency of all finite subsets of $p_k(y)$.
  Let $b'_k\models p_k(y)$.
  Then

  \ceq{\hfill i< k}{\IFF}{\phi(a_i\,;b'_k,c_0)}\hfill for all $i,j\in\QQ^*, i\neq j$

  From this the instability of $\psi(x\,;y)=\phi(x\,;y,c_0)$ follows easily.

  Now, assume instead that $\<a_i\ :\ i\in\QQ^*\>$ is not indiscernible over $c_0$.
  Note that the sequences $\<a_i\ :\ i<0\>$ and $\<a_i\ :\ i>0\>$ are mutually indiscernible over $c_0$.
  Then there is a maximal $n$ such that

  \ceq{\hfill a_{\restriction\{-1,\dots,-n\}}}{\equiv_{c_0}}{a_{\restriction\{1,\dots,n\}}.}

  Let $A=a\restriction\{\pm1,\dots,\pm n\}$.
  By maximality, $a_i\nequiv_{c_0,A}a_j$ for every $-1<i<0<j<1$.
  Let $\psi(x\;y)$ be a formula such that $\psi(a_i\,;c_0)$ and $\neg\psi(a_j\,;c_0)$.
  We claim that for every $k\in(-1,0)\cup(0,1)$ the type below is consistent

  \ceq{\hfill q_k(z)}{=}{\{\psi(a_i\;y)\iff i<k\ \ :\ i\in (-1,0)\cup(0,1)\}.}

  In fact as $\<a_i\ :\ i\in(-1,0)\cup(0,1)\>$ is indiscernible over $A$, all finite subsets of $q_k(z)$ are realized by $c_0$.
  Finally let $c'_k\models p_k(z)$.
  Then $\<a_i,c'_i\ :\ i\in(-1,0)\cup(0,1)\>$ witness the instability of $\psi(x\;z)$.
\end{proof}

\begin{exercise}
  Prove that if every formula $\phi(x\,;z)\in L$, where $|x|=1$, is stable then $T$ is stable.
\end{exercise}

\begin{exercise} 
  A sequence $\<a_i:i<\omega\>$ is totally indiscernible if $a_1,\dots,a_n\equiv a_{i_1},\dots,a_{i_n}$ for every distinct $i_1,\dots,i_n$.
  Prove that the following are equivalent
  \begin{itemize}
  \item[1.] $T$ is stable
  \item[2.] every indiscernible sequence is totally indiscernible.
  \end{itemize}
\end{exercise}

\begin{exercise}\label{ex_stable_orderproperty}
Prove that the following are equivalent
\begin{itemize}
\item[1.] $T$ is unstable
\item[2.] there is an infinite set $A\subseteq\U^n$ and a formula  $\psi(x\,;y)$, with $|x|=|y|=n$ such that $A$ is linearly ordered by the relation $a<b\iff\psi(a\,;b)$.
\end{itemize}
\end{exercise}

\begin{exercise}
Prove that strongly minimal theories are stable.
\end{exercise}

%%%%%%%%%%%%%%%%%%%%%%%
%%%%%%%%%%%%%%%%%%%%%%%
%%%%%%%%%%%%%%%%%%%%%%%
% \section{Stable groups}

% In this section $\U$ has two type-definable sets $\grG\subseteq\U^{\gr z}$ and $\mrX\subseteq\U^{\mr x}$ where $\grG$ is a group that acts from the left on $\mrX$.
% The group operations and the group action are definable.
% We use the symbol $\,\cdot\,$ for both the group multiplication and the group action.

% Let $\D\subseteq\X$ be a relatively definable set.
% Let $L'$ be a two-sorted language.
% The sorts are called $\G$ and $\X$.
% There is only one binary relation symbol $r(x,y)$ that has sort $\G\times\X$.
% We write $\langle\G, \X\, \D\rangle$ for the $L'$-structure that has domain $\langle\G, \X\rangle$ and interpretes $r(x,y)$ as follows 

% \ceq{\hfill\langle\G,\X, \D\rangle\models r(g,a)}{\Leftrightarrow}{a\in g\cdot\D}

% Note that $\D$ is definable in $\langle\G,\X, \D\rangle$ by the formula $r(1,y)$.
% The action of $\G$ on $\U$ defined by left multiplication on $\G$ and $\X$ is an $L'$-automorphisms.
% Therefore we may identify $\G$ with a common subgroup of the groups $L'\mbox{-Aut}\U,{\D})$ as $\D$ ranges over the definable subsets of $\X$.

% \begin{fact}
%   The structure $\langle\G,\X, \D\rangle$ is saturated.
% \end{fact}

% \begin{proof}
%   First note that the sets defined by $r(x,y)$ and by $\neg r(x,y)$ in $\langle\G,\X, \D\rangle$ are a type-definable subsets of $\U$.
%   It follows that the sets defined by $L'$-formulas are type-definables subsets of $\U$.
%   Therefore the saturation of $\langle\G,\X, \D\rangle$ follows from the saturation of $\U$.
% \end{proof}

% %A set $\EuScript X\subseteq\U$ is generic if there are some finitely many $g_1,\dots,g_n\in\U$ such that the sets $g_i\EuScript X$ cover $\U$.

% \begin{fact}\label{fact_generic_G_L}
% Let $\D\subseteq\X$ be a definable set.
% Write $G_{\D}$ for $L'$-$\Aut(\G,\X,{\D})$.
% Then the following are equivalent
% \begin{itemize}
%   \item[1.] $\D$ is $G_{\D}$-generic
%   \item[2.] $\D$ is $\G$-generic.
% \end{itemize}
% \end{fact}

% \begin{proof}
%   The equivalence follows from the fact that the orbits of $\D$ under the action of $G_{\D}$ and $\G$ coincide.
%   In fact, for $f\in L'\mbox{-Aut}(\G,\X,\D)$ let $g=f1$.
%   Then $f{\D}=g\cdot\D$.
% \end{proof}

% \begin{fact}
%   Assume $\X=\G$, where $\G$ acts on itself by translation. 
%   Let $\mu$ be S1 and invariant under $\G$-translation.
%   If $\D\subseteq\G$ is $\mu$-wide then $\D\cdot\D^{-1}$ is $\G$-generic.
% \end{fact}

% \begin{proof}
%   Let $A\subseteq\G$ be maximal such that $a\cdot\D\cap b\cdot\D=\varnothing$ for every distinct $a,b \in A$.
%   By maximality for every $b\in\G$ there is an $a\in A$ such that $b\cdot\D\cap a\cdot\D\neq\varnothing$.
%   That is, there are $d_1,d_2\in\D$ such that $b=a\cdot d_2\cdot d_1^{-1}$.
%   This shows that the $A$-translates of $\D\cdot\D^{-1}$ cover $\G$.
%   It suffices to prove that $A$ is finite.
%   Suppose not, then we can find a indiscernible sequence $\langle a_i:i<\omega\rangle$ such that $a_0\cdot\D\cap a_1\cdot\D=\varnothing$.
%   Then $a_0\cdot\D$ is not wide and, by tranlation invariance, $\D$ is not wide, contrary to our assumption.
% \end{proof}

% \begin{theorem}
%   Let $\mrD$ be a definable set.
%   Assume that $\phi({\mr x}\,;{\gr z})={\mr x}\in{\gr z}\cdot\mrD$ is stable.
%   Then for every ${\mr a}\in\mrX$ either $\D$ or $\G\cdot{\mr a}\smallsetminus\D$ is $\G$-generic in $\G\cdot{\mr a}$.
% \end{theorem}

% \begin{proof}
%   Assume $\G\cdot{\mr a}\smallsetminus\D$ is not generic.
%   Then $\D$ is persistent.
%   We prove that $\D$ is generic.
%   By Theorem~\ref{thm_generic_invariant2} there is a invariant type $p(x)\in S_\phi(\U)$ containing $x\in\D$.
%   By Theorem~\ref{thm_stability_definable_rovescio}, there is a positive Boolean combination of sets in the $G_{\D}$-orbit of $\D$ that is $L'$-definable over $\varnothing$.
%   By the transitivity of the action, $\X$ is the only set that is $L'$-definable over $\varnothing$.
%   Therefore some disjunction of sets in the $G_{\D}$-orbit of $\D$ cover $\X$.
%   As the $G_{\D}$-orbit and the $\G$-orbit of $\D$ coincide, the claim follows.
% \end{proof}

% Let $p(x)$ be a global type containing $x\in\X$ we define

% \ceq{\hfill g\cdot p(x)}{=}{\big\{\varphi(g^{-1}\cdot x)\ :\ \varphi(x)\in p\big\}}

% and

% \ceq{\hfill{\rm Stab}(p)}{=}{\big\{g\in\G\ :\ g\cdot p(x)=p(x)\big\}.}



\section{Notes and references}
\begin{biblist}[]\normalsize
\bib{CS}{article}{
  author={Chernikov, Artem},
  author={Simon, Pierre},
  title={Externally definable sets and dependent pairs},
  note={\href{https://arxiv.org/abs/1007.4468}{ArXiv:1007.4468}},
  journal={Israel J. Math.},
  volume={194},
  date={2013},
  number={1},
  pages={409--425},
  %note={},
  %issn={0021-2172},
  %doi={10.1007/s11856-012-0061-9},
}\smallskip
  \bib{HH}{article}{
   author={Harnik, Victor},
   author={Harrington, Leo},
   title={Fundamentals of forking},
   journal={Ann. Pure Appl. Logic},
   volume={26},
   date={1984},
   number={3},
   pages={245--286},
  %  issn={0168-0072},
  %  review={\MR{0747686}},
  %  doi={10.1016/0168-0072(84)90005-8},
}\smallskip
\bib{TZ}{book}{
  author={Tent, Katrin},
  author={Ziegler, Martin},
  title={A course in model theory},
  series={Lecture Notes in Logic},
  volume={40},
  publisher={Association for Symbolic Logic, Cambridge University Press},
  date={2012},
  pages={x+248},
  %isbn={978-0-521-76324-0},
  %doi={10.1017/CBO9781139015417},
}\smallskip
\end{biblist}