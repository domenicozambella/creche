% !TEX root = creche.tex
\documentclass[creche.tex]{subfiles}
\begin{document}

\chapter{Fra\"issé limits}
\label{fraisse} 

\def\ceq#1#2#3{\parbox[b]{20ex}{$\displaystyle #1$}\parbox[b]{4ex}{\hfil$#2$}$\displaystyle #3$}

We introduce \textit{Fra\"iss\'e limits}, aleas \textit{homogeneous-universal\/} or \textit{generic\/} structures, which here we call \textit{rich\/} models, after Poizat. Rich models generalize the examples in Chapters~\ref{relational} and the many more to come. We also prove \textit{elimination of quantifiers\/} for the theories introduced in Chapters~\ref{relational}.

\section{Rich models.}\label{rich}
We now define \textit{categories of models and partial morphisms}. These are example of concrete categories as intended in category theory. However, apart from the name, in what follows we dispense with all notions of category theory as they would make the exposition less basic than intended (and not provide much more technical instruments). We refer the interested reader to the references at the end of the chapter.

\noindent\llap{\textcolor{red}{\Large\danger}\kern1.5ex}Warning: the terminology introduced in this section is not standard. In the literature many details are left implicit. This is generally safe when the category used is fixed. Here we prefer to be more explicit because, when discussing saturation and quantifier elimination, it helps to compare different categories.

A \emph{category (of models and partial morphisms)\/} is a class $\M$ which is disjoint union of two classes: \emph{$\M_{\textrm{ob}}$} and \emph{$\M_{\textrm{ar}}$}. The first is the class of \emph{objects\/} and contains structures with a common signature $L$ which we call \emph{models}. The second is the class of \emph{arrows\/} and contains (partial) maps between models which we call \emph{morphisms}. We require that the identity maps are morphisms and that composition of two morphism is again morphism. This makes $\M$ a well-defined category. 

For example, $\M_{\rm ob}$ could consist of all models of some theory $T_0$ and $\M_{\rm ar}$ of all partial isomorphisms between these. Alternatively, as morphisms we could take elementary maps between models. At a first reading the reader may assume $\M$ is as in one of these two examples. In the general case we need to make some assumptions on $\M$.

\begin{definition}\label{def_com_c}For ease of reference we list together all properties required below
\begin{itemize} 
\item[c0.] between any two models $M$ and $N$ there is some morphism;
\item[c1.] the identity map $\id_A:M\to M$ is a morphism, for any $A\subseteq M$;
\item[c2.] morphisms are invertible maps and the inverse of a morphism is a morphism;
\item[c3.] morphisms preserve the truth of $\atL$\,-formulas;
\item[c4.] if $M$ is a model, every elementary map $k:M\to N$ is morphism (and, $N$ is a model);
\item[c5.] if $k_i:M\to N$ is a chain of morphisms, then \smash{$\bigcup_{i<\lambda} k_i:M\to N$} is a morphism.
\item[c6.] if $k':M\to N$ is a morphism for every finite $k'\subseteq k$, then $k:M\to N$ is a morphism.\QED
\end{itemize}
\end{definition}
 
Note that \ssf{c0} says that the category is \emph{connected} and that \ssf{c1} implies that any restriction of a morphism is a morphism. Given \ssf{c1}, we can rephrase \ssf{c0} by saying that $\0:M\to N$ is a morphism for any pair of models $M$ and $N$. 

A consequence of \ssf{c4} is that the class of models is closed under elementary equivalence. In fact, if $N\equiv M$ then $\0:M\to N$ is elementary, hence it is a morphism and, in particular, and $N$ is a model. 

We call \ssf{c6} the \emph{finite character of morphisms}. Note that it implies \ssf{5}. We shall never use \ssf{6} in these notes, but most natural categories of models and partial morphisms have this property.

The following two definitions assume \ssf{c2}. The generalization to non injective morphisms is not straightforward (in fact, there are two generalizations: \textit{projective\/} and \textit{inductive\/}). These generalizations are not very common and will not be considered here. 

\begin{definition}
Assume that $\M$ satisfies \ssf{c0-c2} of Definition~\ref{def_com_c}. We say that a model $N$ is \emph{$\lambda\jj$rich\/} if for every model $M$ of cardinality $\le\lambda$ and every morphism $k:M\imp N$ of cardinality $<\lambda$ there is a total morphism $h:M\imp N$ that extends $k$. We say that $N$ is \emph{rich\/} tout court  if it is $\lambda\jj$rich for $\lambda=|N|$. When $\M_{\rm ob}=\Mod(T_0)$ for some theory $T_0$ and $\M_{\rm ar}$ is clear from the context, we say \emph{rich model of $T_0$}.\QED
\end{definition}

Rich models are also called \textit{Fra\"iss\'e limits} or \textit{homogeneous-universal\/} for a reason that will soon be clear; they are also called \textit{generic}. Unfortunately these names are either too long or too generic, so we opt for the less common term \textit{rich\/} proposed by Poizat. 

The following two notions are deeply connected with richness.

\begin{definition}\label{def_omogenea_universale}
Assume that $\M$ satisfies \ssf{c0-c2} of Definition~\ref{def_com_c}. We say that a model $N$ is \emph{$\lambda\jj$universal\/} if for every model $M$ of cardinality $\le\lambda$ there is a total morphism $k:M\to N$. We say that a model $N$ is \emph{$\lambda\jj$homogeneous\/} if every $k:N\to N$ of cardinality $<\lambda$ extends to a bijective morphism $h:N\to N$ (an automorphism when \ssf{c3} below holds).

Note that the larger $\M_{\rm ar}$, the stronger notion of homogeneity. When  $\M_{\rm ar}$ contains all partial isomorphisms between models (the largest class of morphisms considered here), it is common to say \emph{$\lambda$-ultra\-homo\-geneous\/} for $\lambda\jj$homogeneous. 

As above, when $\lambda=|N|$ we say \emph{universal}, respectively \emph{homogeneous} and \emph{ultrahomogeneous}.\QED
\end{definition}


In Section~\ref{relational}.\ref{dlo} we implicitly used $\M_{\rm ob}=\Mod(T_{\rm lo})$ and partial isomorphisms as $\M_{\rm ar}$. In Section~\ref{relational}.\ref{randomgraph} we used $\M_{\rm ob}=\Mod(T_{\rm gph})$ and again partial isomorphisms as $\M_{\rm ar}$. Corollary~\ref{coroll_ordinericco} proves that every model of $T_{\rm dlo}$ is $\omega\jj$rich.  Corollary~\ref{coroll_graforicco} claims the analogous fact for $T_{\rm rg}$.


In the following we fequently work under the following assumption (even when not all properties are strictly necessary).
 
\begin{assumption}\label{ass_c0-5}
Assume  $|L|\le\lambda$ and suppose that $\M$ satisfies \ssf{c0-c5} of Definition~\ref{def_com_c}
\end{assumption} 
 
The assumption on the cardinality $L$ is only necessary to apply the downward L\"owenheim-Skolem Theorem when required.

\begin{proposition}\label{prop_6.1}
(Assume~\ref{ass_c0-5})  \  The following are equivalent
\begin{itemize}
\item[1.] $N$ is a $\lambda\jj$rich model;
\item[2.]for every morphism $k:M\imp N$ of cardinality $<\lambda$ and every $b\in M$ there is a morphism $h:M\imp N$ that extends $k$ and is defined in $b$.
\end{itemize}
\end{proposition}
\begin{proof}
Closure under union of chains of morphisms, which is ensured by \ssf{c5}, immediately yields \ssf{2}$\IMP$\ssf{1}. As for implication \ssf{1}$\IMP$\ssf{2} we only need to consider the case $\lambda<|M|$.  By the downward L\"owenheim-Skolem theorem there is an  $M'\preceq M$ of cardinality $\lambda$ containing $b$. Let $h:M'\to N$ be the total morphism obtained from \ssf{1}. By \ssf{c4}, the map  $h:M\to N$ is a composition of morphisms, hence a morphism.
\end{proof}

The following theorem subsumes both Theorem~\ref{zigzagcantor} and Theorem~\ref{gaomegacat}.

\begin{theorem}\label{thm_riccozigzag}
(Assume~\ref{ass_c0-5})  \  Let $M$ and $N$ be two rich models of the same cardinality $\lambda$. Then every morphism $k:M\imp N$ of cardinality $<\lambda$ extends to an isomorphism.
\end{theorem}

\begin{proof}
When $\lambda=\omega$, we can take the proof of Theorem~\ref{zigzagcantor} and replace \textit{partial isomorphism\/} by \textit{morphism\/} and the references to Lemma~\ref{lem_ordinericco} by references to Proposition~\ref{prop_6.1}. As for uncountable $\lambda$, we only need to extend the construction through limit stages. By \ssf{c5} we can simply take the union.
\end{proof}


\begin{corollary}\label{coroll_riccozigzag}
(Assume~\ref{ass_c0-5})  \  Then all rich models of cardinality $\lambda$ are isomorphic.\QED
\end{corollary}

It is obvious that $\lambda\jj$rich models are $\lambda\jj$universals and, by Theorem~\ref{thm_riccozigzag}, they are $\lambda\jj$ho\-mo\-ge\-ne\-ous. These two notions are weaker than richness. For instance, when $\M$ is as in Section~\ref{relational}.\ref{randomgraph}, the countable graph that has no arc is trivially ultrahomogeneous but it is not universal and a fortiori not rich. On the other hand if we add to a countable random graph an isolated point we obtain an universal graph which is not ultrahomogeneous. However, when taken together, these two properties are equivalent to richness.

\begin{theorem}\label{ricco<->universaleomogeneo}
(Assume~\ref{ass_c0-5}) \ The following are equivalent:
\begin{itemize}
\item[1.] $N$ is rich;
\item[2.] $N$ is homogeneous and universal.
\end{itemize}
\end{theorem}
\begin{proof} 
Implication \ssf{1}$\IMP$\ssf{2} is clear as noted above, so we prove \ssf{2}$\IMP$\ssf{1}. Let $k:M\imp N$ be a morphism of cardinality $<|N|$ and let $b\in M$. As $N$ is universal, there is a morphism \hbox{$f:M\imp N$} defined on $\dom k\cup\big\{b\big\}$. By \ssf{c2} the map $k\circ f^{-1}:N\imp N$ is a morphism of cardinality $<|N|$. By homogeneity it has an extension to a bijective morphism $h:N\imp N$. It is immediate that $h\circ f:M\imp N$ is the required extension of $k$.
\end{proof}


\begin{exercise}
Let $N$ be the structure obtained by adding to a countable random graph an isolated point. Show that this graph is homogeneous if $\M_{\rm ar}$ is the class of elementary maps but not if $\M_{\rm ar}$ is the class of partial isomorphisms.\QED
\end{exercise}

Next theorem is one of the most important facts about $\lambda$-rich models

\begin{theorem}\label{thm_morphism_rich_elementary}
(Assume~\ref{ass_c0-5})  \  Then every morphism between $\lambda\jj$rich models is elementary. In particular, $\lambda\jj$rich models are elementary equivalent.
\end{theorem}

\begin{proof}
Let $k:M\to N$ be a morphism between rich models. It suffices to prove that every finite restriction of $k$ is elementary. By \ssf{c1}, we may as well assume that $k$ itself is finite. It suffices to construct $M'\preceq M$ and  $N'\preceq N$ together with a morphism $h:M\to N$ that extends $k$ and maps $M'$ bijectively to $N'$. Then by \ssf{c4} the map $h:M'\to N'$ is also a morphism because it is the composition of morphisms. Finally, by \ssf{c3}, it is an isomorphism, in particular an elementary map.

We construct simultaneously a chain $\<A_i:i{<}\lambda\>$ of subsets of $M$, a chain $\<B_i:i{<}\lambda\>$ of subsets of $N$, and a chain functions $\<h_i:i{<}\lambda\>$ such that $h_i:M\to N$ are mor\-phisms, $\dom h_i\subseteq A_i$ and $\range h_i\subseteq B_i$. In the end we set

\hfil $\displaystyle M'\ \ =\ \ \bigcup_{i<\lambda}A_i$,\hfil $\displaystyle N'\ \ =\ \ \bigcup_{i<\lambda}B_i$,\hfil  $\displaystyle h\ \ =\ \ \bigcup_{i<\lambda}h_i$.

The chains start with $A_0=\dom k$, $B_0=\range k$ and $h_0=k$. As usual at limit stages we take the union. Now we consider successor stages. Fix some enumerations of $M$ and $N$. At stage $i+1$ we also fix some enumerations of $A_i$, $B_i$, $L_x(A_i)$, and $L_x(B_i)$, where $|x|=1$. Let $\<i_1,i_2\>$ be the $i\jj$th pair of ordinals $<\lambda$. If the $i_2\jj$th formula in $L_x(A_{i_1})$ is consistent in $M$, pick any solution $a$. Also, if the $i_2\jj$th formula in $L_x(B_{i_1})$ is consistent in $N$, pick any solution $b$. Let $a'$ be the $i_2\jj$th element of $A_{i_1}$ and let $b'$ be the $i_2\jj$th element of $B_{i_1}$. 

Let $h_{i+1}:M\to N$ be an extension of $h_i$ such that $\dom h_{i+1}=\dom h_i\cup\big\{a'\big\}$ and $\range h_{i+1}=\range h_i\cup\big\{b'\big\}$. Note that the latter requirement is met applying Proposition~\ref{prop_6.1} to $h_i^{-1}:N\to M$. Define $A_{i+1}=A_i\cup\big\{a,h_{i+1}^{-1}b'\big\}$ and $B_{i+1}=B_i\cup\big\{b,h_{i+1}a'\big\}$.

The elements $a$ and $b$, added to $M'$ and $N'$, ensure that $M'\preceq M$ and $N'\preceq N$, by the Tarski-Vaught test. In fact every formula in $L(M')$ belongs to $L(A_{i_1})$ for some $i_1$ and it occurs as $i_2\jj$ formula in the enumeration that has been fixed along with the construction. Therefore at stage $i=\<i_1,i_2\>$ a witness is added to $M'$. The same happens for formulas in $L(N')$. 

A similar argument shows that the elements $a'$ and $b'$, added to the domain and range of $h$, ensure that the map $h:M'\to N'$ is both total and surjective. 
\end{proof}

By the theorem below  we can unambiguisly speak of \emph{the theory of rich models}. In fact it immediately follows that if $M$ is $\mu$-rich and $N$ is $\lambda$-rich then $M\equiv N$.

\begin{exercise}
Let $L$ be the language of strict orders. Prove that $\QQ\preceq\RR$.\QED
\end{exercise}

Assume for simplicity that $L$ is countable. Then all $\omega\jj$rich models belong to $\Mod(T)$ for some complete theory $T$. It is interesting to ask if the converse is true: if $T$ is the theory of the $\omega\jj$rich models of $\M$, do all models in $\M\cap \Mod(T)$ are rich?  The answer is affirmative when $T$ is either $T_{\rm dlo}$ or $T_{\rm rg}$, but this is not always the case, see Example~\ref{ex_infiniti_colori} for an easy counterexample. An affirmative answer implies that the theory of rich models is $\omega\jj$categorical. An interesting variant of this question is considered in Exercise~\ref{ricchezza_saturazione_EQ}.

For convenience of the reader, we instantiate Theorem~\ref{thm_morphism_rich_elementary} with the two categories used in Chapter~\ref{relational}.

\begin{corollary}\label{coroll_tutteleimmersionisonoelementari=eliminazioneqantificatori}
Partial isomorphisms between models of $T_{\rm dlo}$ and between models of $T_{\rm rg}$ are elementary maps.\QED
\end{corollary}

When partial isomorphism between models of a given theory $T$ coincide with elementary maps, it is always by a fundamental reason. Let us introduce some terminology. Let $T$ be a consistent theory. We say that \emph{$T$ has (or admits) elimination of quantifiers\/} if for every formula $\phi(x)\in L$ there is a quantifier-free formula $\psi(x)\in L_{\rm qf}$ such that 

\ceq{\hfill T}{\proves}{\psi(x)\iff\phi(x).}

We will discuss general criteria for elimination of quantifiers in Chapter~\ref{eliminazione}, here we report without proof the following theorem.

\begin{theorem}\label{thm_tutteleimmersionisonoelementari=eliminazioneqantificatori}
Let $T$ be a theory that decides the characteristic of its models. The following are equivalent
\begin{itemize}
\item[1.] $T$ has elimination of quantifiers;
\item[2.] every partial isomorphism between models $T$ is an elementary map.\QED
\end{itemize}
\end{theorem}

For the time being when saying that \textit{$T$ has quantifier elimination\/} we intend \ssf{2} of the theorem above.

In the next chapter we introduce important examples of $\omega\jj$rich models that do not have an $\omega\jj$categorical theory. These are algebraic structures (groups, fields etc.) hence more complex than pure relational structures. So, we conclude this section with an example of this phenomenon in almost trivial context. 

\begin{example}\label{ex_infiniti_colori}
Let $L$ contain a unary predicate $r_n$ for every positive integer $n$. The theory $T_0$ contains the axioms $\neg\E x\, \big[r_n(x)\wedge r_m(x)\big]$ for $n\neq m$ and $\E^{\le n}x\,r_n(x)$ for every $n$. Work in the the category of models of $T_0$ and partial isomorphisms. Let $T$ be the theory that extends $T_0$ with the axioms $\E^{= n} x\, r_n(x)$ for every  $n$. Let $q(x)$ be the type $\big\{\neg r_n(x):n\in\omega\big\}$. There are models of $T$ that do not realize $q(x)$, hence $T$ is not $\omega$-categorical. It is easy to verify that the following are equivalent.
\begin{itemize}
 \item[1.] $N$ is $\omega$-rich model;
 \item[2.] $N\models T$ and  $q(N)$ is infinite.
\end{itemize}
The reader may use Theorem~\ref{thm_morphism_rich_elementary} to prove that $T$ is complete and has elimination of quantifiers. It is also easy to verify that every uncountable model of $T$ is rich and consequently that $T$ is uncountably categorical.\QED
\end{example}

\begin{exercise}\label{ex_finiti_colori}
Let $T_0$ and $\M$ be as in Example~\ref{ex_infiniti_colori} except that we restrict the language to the relations $r_0,\dots,r_n$ for a fixed $n$. Do $\omega\jj$rich models of $T_0$ exist? Is their theory $\omega\jj$categorical? What if we add to $T_0$ the axiom $r_0(x)\vee\dots\vee r_n(x)$~?\QED
\end{exercise}

\begin{exercise}
The language contains only the binary relations $<$ and $e$. The theory $T_0$ says that $<$ is a strict linear order and that $e$ is an equivalence relation. Let $\M_{\rm ob}=\Mod(T_0)$ and let $\M_{\rm ar}$  be the class of partial isomorphisms between models.   Do rich models exist? Can we axiomatize their theory? Is it $\omega\jj$categorical?\QED
\end{exercise}

\begin{exercise}
The language contains only two binary relations. The theory $T_0$ says that they are equivalence relations. Let $\M_{\rm ob}=\Mod(T_0)$ and let $\M_{\rm ar}$ be the class of partial isomorphisms between models. Do rich models exist? Can we axiomatize their theory? Is it $\omega\jj$categorical?\QED
\end{exercise}

\begin{exercise}
In the language of graphs let $T_0$ say that there are no cycles (equivalently, there is at most one path between any two nodes). In combinatorics these graphs are called \textit{forests}, their connected components, \textit{trees}. Let $\M_{\rm ob}=\Mod(T_0)$ and let $\M_{\rm ar}$  be the class of partial isomorphisms between models. Do rich models exist? Can we axiomatize their theory? Is it $\omega\jj$categorical?\QED
\end{exercise}

%%%%%%%%%%%%%%%%%%%%%%%%%%%%%%%%%%
%%%%%%%%%%%%%%%%%%%%%%%%%%%%%%%%%%
%%%%%%%%%%%%%%%%%%%%%%%%%%%%%%%%%%
%%%%%%%%%%%%%%%%%%%%%%%%%%%%%%%%%%
\section{Weaker notions of universality and homogeneity}\label{weak}

We want to extend the equivalence in Theorem~\ref{ricco<->universaleomogeneo} to $\lambda\jj$rich models. For that we need to weaken the notions of $\lambda\jj$universality and  $\lambda\jj$homogeneity. 

\begin{definition}
We say that a structure $N$ is \emph{weakly $\lambda$-homogeneous\/} if for every $b\in N$ every morphism $k:N\imp N$ of cardinality $<\lambda$ extends to one defined in $b$. The term \emph{back-and-forth $\lambda$-homogeneous\/} is also used.\QED
\end{definition}

The following easy exercise on back-and-forth is required in the sequel.

\begin{exercise}\label{omogeneo=debolmenteaomogeneo}
Assume that $\M$ satisfies \ssf{c0-c5} of Definition~\ref{def_com_c}. Prove that any weakly $\lambda\jj$ho\-mo\-ge\-ne\-ous structure of cardinality $\lambda$ is homogeneous.\QED
\end{exercise}

\begin{definition}\label{def_weakly_universal}
We say that a structure $N$ is \emph{weakly $\lambda$-universal\/} if for every model $M$ and every $A\subseteq M$ of cardinality $<\lambda$ there is a morphism $k:M\to N$ such that $A\subseteq\dom k$.\QED
\end{definition}

\begin{lemma}\label{debolmenteomogeneoandirivieni}
(Assume~\ref{ass_c0-5})  \  Let $N$ be a weakly $\lambda\jj$ho\-mo\-ge\-ne\-ous model. Let $A\subseteq N$ have cardinality $\le\lambda$ and let $k:N\to N$ be a morphism of cardinality $<\lambda$. Then there is a model  $M\preceq N$ containing $A$ and an automorphism $h:M\to M$ that extends $k$.
\end{lemma}

\begin{proof}
Similar to the proof of Theorem~\ref{thm_morphism_rich_elementary}. We shall construct simultaneously a chain $\<A_i:i<\lambda\>$ of subsets of $N$ and a chain functions $\<h_i:i<\lambda\>$, such that $h_i:N\to N$ are morphisms. In the end we will set

\hfil $\displaystyle M\ \ =\ \ \bigcup_{i<\lambda}A_i$\hfil  and\hfil  $\displaystyle h\ \ =\ \ \bigcup_{i<\lambda}h_i$ 

The chains start with $A_0=A\cup\dom k\cup\range k$ and $h_0=k$. As usual, at limit stages we take the union. Now we consider successor stages. At stage $i$ we fix some enumerations of $A_i$ and of $L_x(A_i)$, where $|x|=1$. Let $\<i_1,i_2\>$ be the $i\jj$th pair of ordinals $<\lambda$. If the $i_2\jj$th formula in $L_x(A_{i_1})$ is consistent in $N$, let $a$ be any of its solutions. Also let $b$ be the $i_2\jj$th element of $A_{i_1}$. Let $h_{i+1}:N\to N$ be a minimal morphism that extends $h_i$ and is such that $b\in\dom h_{i+1}\cap\range h_{i+1}$. Define $A_{i+1}=A_i\cup\big\{a,\ h_{i+1}b,\  h_{i+1}^{-1}b\big\}$.
\end{proof}



\begin{theorem}\label{ricco=universaledebolmenteomogeneo}
(Assume~\ref{ass_c0-5})  \  For every model $N$ the following are equivalent
\begin{itemize}
\item[1.] $N$ is $\lambda$-rich;
\item[2.] $N$ is  weakly $\lambda\jj$universal and weakly $\lambda\jj$homogeneous.
\end{itemize}
\end{theorem}

\begin{proof} 
Implication \ssf{1}$\IMP$\ssf{2} is clear. To prove \ssf{2}$\IMP$\ssf{1} we generalize the proof of Theorem~\ref{ricco<->universaleomogeneo}. We assume \ssf{2} fix some morphism $k:M\imp N$ of cardinality $<\lambda$ and let $b\in M$. By weak $\lambda\jj$universality there is a morphism $f:M\imp N$ with domain of definition $\dom k\cup\big\{b\big\}$. The map $f\circ k^{-1}:N\imp N$ has cardinality $<\lambda$ and, by Lemma~\ref{debolmenteomogeneoandirivieni}, it has an extension to an automorphism $h:N'\to N'$ for some $N'\preceq N$ containing $\range k\cup\range f$. Then $h\circ f:M\imp N$ extends $k$ and is defined on $b$.
\end{proof}



\section{The amalgamation property}

We say that $\M$ has the \emph{amalgamation property\/} if for every pair of morphisms $f_1:M\to M_1$ and $f_2:M\to M_2$ there are some total morphism $g_1:M_1\to N$ and $g_2:M_2\to N$ such that $g_1\circ f_1\, (a) = g_2\circ f_2\, (a)$ for every $a\;\in\; \dom (g_1\circ f_1)\; \cap\; \dom (g_2\circ f_2)$.

We write \emph{$M\le N$\/} for $M\subseteq N$ and $\id_M:M\to N$ is a morphism. We say that $\<M_i:i<\lambda\>$ is a \emph{$\le\jj$chain\/} if $M_i\le M_j$ for all $i<j<\lambda$.

We say that $h:M'\to N'$ \emph{extends\/} $k:M\to N$ if $k\subseteq h$, $M\le M'$, and  $N\le N'$. As we assume that morphisms are invertible, we may express the amalgamation property in the following convenient form: every morphism extends to a total one.

For the following theorem to hold we need the following property:
\begin{definition}
We say that $\M$ is \emph{closed under union of $\le\jj$chains\/} if
\begin{itemize}
\item[c7.] if $\<M_i:i<\lambda\>$ is a $\le\jj$chain,  then \smash{$\displaystyle M_i\le\bigcup_{j<\lambda}M_j$} \ for all $i<\lambda$.\QED
\end{itemize}
\end{definition}

The following is a general existence theorem for rich models.


\begin{theorem} Let $\lambda$ be a cardinal such that $\lambda=\lambda^{<\lambda}$. (Assume~\ref{ass_c0-5})  \  Assume further that $\M$ has the amalgamation property and is closed under union of $\le$ chains. Then there is a rich model $N$ of cardinality $\lambda$.
\end{theorem}

\begin{proof} We construct $N$ as union of a $\le\jj$chain of models $\langle N_i:i < \lambda\rangle$ such that $|N_i| = \lambda$ for all $i<\lambda$. Let $N_0$ be any model of cardinality $\lambda$. At stage $i+1$, let $f:M\imp N_i$ be the least morphism (in a well-ordering that we specify below) such that $|f|<|M|\le\lambda$ and $f$ has no extension to an embedding $f':M\imp N_i$. Apply the amalgamation property to obtain an total morphism $f':M\imp N'$ that extends $f:M\imp N_i$. By the downword L\"owenheim-Skolem Theorem we may assume $|N'|=\lambda$. Let $N_{i+1}=N'$. At limit stages take the union.

The well-ordering mentioned needs to be chosen so that in the end we forget nobody. So, first at each stage we well-order the isomorphism-type of the morphisms $f:M\imp N_i$ such that  $f<|M|\le\lambda$. Then the required well-ordering is obtained by dovetailing all these well-orderings.  The length of this enumeration is at most $2^\lambda\cdot\lambda^{<\lambda}$, which is $\lambda$ by hypothesis.

We check that $N$ is rich. Let $f:M\imp N$ be a morphism and $|f|<|M|\le\lambda$. The cofinality of $\lambda$ is larger than $|f|$, hence $\range f\subseteq N_i$ for some $i<\lambda$. So $f:M\imp N_i$ is a morphism and at some stage $j$ we have ensured the existence of a total morphism of $f':M\imp N_{j+1}$ that extends $f$.
\end{proof}



\section{Notes}
\begin{biblist}[]\normalsize

 \bib{LR}{article}{
   author={Lieberman, Michael},
   author={Rosick\'y, Jir\'i},
   title={Classification theory for accessible categories},
   eprint={https://arxiv.org/abs/1404.2528},
   %eprint={arXiv:1404.2528},
   date={2014}
}  
   

\end{biblist}


\begin{comment}


%%%%%%%%%%%%%%%%%%%%%%%%%%%%%%%%%%%%
%%%%%%%%%%%%%%%%%%%%%%%%%%%%%%%%%%%%
%%%%%%%%%%%%%%%%%%%%%%%%%%%%%%%%%%%%
%%%%%%%%%%%%%%%%%%%%%%%%%%%%%%%%%%%%
\section{A non $\omega$-categorical example}\label{eserciziorisolto}

In the next chapter we introduce important examples of $\omega\jj$rich models that do not have an $\omega\jj$categorical theory. They are algebraic structures (groups, fields etc.) hence more complex than pure relational structures. So, it may be useful to observe the phenomenon in a simpler context. (It is not the simplest one, see Exercise~\ref{ex_infiniti_colori}.)

Let $L$ be the language that contains a binary relation $<$ and a unary predicate $r_n$ for every $n\in\omega$. The theory  $T_0$ contains $T_{\rm lo}$ and the axioms $\neg\E x\, \big[r_n(x)\wedge r_m(x)\big]$ for $n\neq m$. Let $T$ be the theory that contains $T_0$ and the axioms
\begin{itemize}
\item[1.] $\E y,z\ (y<x<z)$;
\item[d$_n$] $x<y\ \imp\ \E z\,\big[\,x<z<y\ \wedge\ r_n(z)\big]$;
\end{itemize}
We claim that $T$ is consistent. In fact, $\QQ$ is a model of $T$ if we interpret $r_n$ in the set of rationals of the form $k\cdot p_n^{-i}$, where $p_n$ is the $n$-th prime, $i\in\ZZ$ is positive, and $k\in\ZZ$ is coprime with $p_n$.

For the rest of this section $\M$ is the c.o.m.\@ with $\M_{\rm ob}=\Mod(T_0)$ and $\M_{\rm ar}$ the class of partial isomorphisms between models.

\begin{proposition}
For every model $N$ and $a,b\in N$ define 

\ceq{\hfill p_{a,b}(x)}{=}{\Big\{a<b\ \imp\ a<x<b\wedge\neg r_n(x)\ :\ n\in\omega\Big\}}

Then the following are equivalent
\begin{itemize}
 \item[1.] $N$ is an $\omega$-rich model;
 \item[2.] $N\models T$ and  $N\models \E x\, p_{a,b}(x)$ for every $a,b\in N$.
\end{itemize}
\end{proposition}

Note that it is easy to find a countable model of $T$ that does not realizes all the types in \ssf{2}. For instance, we could take the model described above and throw away all rationals that are not in any of the sets $r_n(\QQ)$.

\begin{proof}
The proof of implication \ssf{1}$\IMP$\ssf{2} is left to the reader. (It is similar to Exercise~\ref{ex_ricco->dlo}.)

As for \ssf{2}$\IMP$\ssf{1}, we adapt the proof of Lemma~\ref{lem_ordinericco}. We assume for simplicity that $A\neq\0\neq B$, the reader may easily adapt the argument to the general case. Suppose first that $M\models r_n(a_i)$ for some $n$. As the $r_n$ are mutually exclusive, for $h_{i+1}$ to be a partial isomorphism it sufficient to find $c\in N$ such that $h_i[A]<c<h_i[B]$ and $N\models r_n(c)$. This we can do by axiom \ssf{d$_{n}$}. It could also be the case that  $M\models \neg r_n(a_i)$ for every $n$. In this case the required $c$ is any witness of $M\models \E x\, p_{a,b}(x)$ with $h_i[A]=a$ and $h_i[B]=b$
\end{proof}

We sum up: $\Mod(T)$ contains countable models that not rich, hence $T$ is not $\omega$-categorical; still, by Theorem~\ref{thm_morphism_rich_elementary}, $T$ is complete and has elimination of quantifiers.
\end{comment}


\end{document}
